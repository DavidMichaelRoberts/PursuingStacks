%% Anti-Copyright 2015 - the scrivener

\chapter{Test categories and test functors}
\label{ch:II}

\centerline{\itshape Reflections on homotopical algebra}\pspage{1}

\bigbreak

\presectionfill\ondate{27.2.}83.\par

The following notes are the continuation of the reflection started in
my letter to Daniel Quillen written previous week (19.2 -- 23.2),
which I will cite by \hyperref[ch:I]{(L)} (``letter''). I begin with
some corrections and comments to this letter.

% 14
\hangsection[The unnoticed failure \dots]{The unnoticed failure
  \texorpdfstring{\textup(}{(}of the present foundations of topology
  for expressing topological intuition\texorpdfstring{\textup)}{)}.}%
\label{sec:14}%
Homotopical algebra can be viewed as being concerned mainly with the
study of spaces of continuous maps between spaces and the algebraic
analogons of these, with a special emphasis on homotopies between such
homotopies between homotopies etc. The kind of restrictive properties
imposed on the maps under consideration is exemplified by the typical
example when demanding that the maps should be extensions, or
liftings, of a given map. Homotopical algebra is not directly suited
for the study of spaces of homeomorphisms, spaces of immersions, of
embeddings, of fibrations, etc.\ -- and it seems that the study of
such spaces has not really yet taken off the ground. Maybe the main
obstacle here lies in the wildness phenomena, which however, one
feels, makes a wholly artificial difficulty, stemming from the
particular way by which topological intuition has been mathematically
formalized, in terms of the basic notion of topological spaces and
continuous maps between them. This transcription, while adequate for
the homotopical point of view, and partly adequate too for the use of
analysts, is rather coarsely inadequate, it seems to me, in most other
geometrico-topological contexts, and particularly so when it comes to
studying spaces of homeomorphisms, immersions, etc.\ (in all those
questions when ``isotopy'' is replacing the rather coarse homotopy
relation), as well as for a study of stratified structures, when it
becomes indispensable to give intrinsic and precise meaning to such
notions as tubular neighborhoods, etc.  For a structure theory of
stratifications, it turns out (somewhat surprisingly maybe) that even
the somewhat cumbersome context of topoi and pretopoi is better suited
than topological spaces, and moreover directly applicable to
unconventional contexts such as \'etale topology of schemes, where the
conventional transcription of topological intuition in terms of
topological spaces is quite evidently breaking down. To emphasize one
point I was making in (\hyperref[ch:I]{L}, p.~\ref{p:L.1}), it
seems to me that this breakdown is almost as evident in isotopy
questions or for the needs of a structure theory of stratified
``spaces'' (whatever we mean by ``spaces'' \ldots). It is a matter of
amazement to me that this breakdown has not been clearly noticed, and
still less overcome by working out a more suitable transcription of
topological objects and\pspage{2} topological intuition, by the people
primarily concerned, namely the topologists. The need of eliminating
wildness phenomena has of course been felt repeatedly, and (by lack of
anything better maybe, or rather by lack of any attempt of a
systematic reflection on what \emph{was} needed) it was supposed to be
met by the notion of piecewise linear structures. This however was
falling from one extreme into another -- from a structure species with
vastly too many maps between ``spaces'', like a coat vastly too wide
and floating around in a million wild wrinkles, to one with so few
(not even a quadratic map from $\bR$ to $\bR$ is allowed!) that it
feels like too narrow a coat, bursting apart on all edges and
ends. The main defect here, technically speaking, seems to me the fact
that numerical piece-wise linear functions are not stable under
multiplication, and as a geometric consequence of this, that when
contracting a compact p.l.\ subspace of a (compact, say) piecewise
linear space into one point, we do not get on the quotient space a
natural p.l.\ structure. This alone should have sufficed, one might
think, to eliminate the piecewise linear structure species as a
reasonable candidate for ``doing topology'' without wildness
impediments -- but strangely enough, it seems to be hanging around
till this very day!

% 15
\hangsection{Overall review of standard descriptions of homotopy
  types.}\label{sec:15}%
But my aim here is not to give an outline of foundations on ``tame
topology'', but rather to fill some foundational gaps in homotopy
theory, more specifically in homotopical algebra. The relative success
in the homotopical approach to topology is probably closely tied to
the well known fact (Brouwer's starting point as a matter of fact,
when he introduced the systematics of triangulations) that every
continuous map between triangulated spaces can be approximated by
simplicial maps. This gave rise, rather naturally, to the hope
expressed in the ``Hauptvermutung'' -- that two homeomorphic
triangulated spaces admit isomorphic subdivisions, a hope that finally
proved a delusion. With a distance of two or three generations, I
would comment on this by saying that this negative result was the one
to be expected, once it has become clear that neither of the two
structure species one was comparing, namely topological spaces and
triangulated spaces, was adequate for expressing what one is really
after -- namely an accurate mathematical transcription, in terms of
``spaces'' of some kind or other, of some vast and deep and misty and
ever transforming mass of intuitions in our psyche, which we are
referring to as ``topological'' intuition. There \emph{is} something
positive though, definitely, which can be viewed as an extremely
weakened version of the Hauptvermutung, namely the fact that
topological spaces on the one hand, and semi-simplicial sets on the
other, give rise, by a suitable ``localization''\pspage{3} process
(formally analogous to the passage from categories of chain complexes
to the corresponding derived categories), to
eventually\scrcommentinline{?}  ``the same'' (up to equivalence)
``homotopy category''. One way of describing it is via topological
spaces which are not ``too wild'' as objects (the CW-spaces),
morphisms being homotopy classes of continuous maps. The other is via
semi-simplicial sets, taking for instance Kan complexes as objects,
and again homotopy classes of ``maps'' as morphisms. The first
description is the one most adapted to direct topological intuition,
as long as least as no more adequate notion than ``topological
spaces'' is at hand. The second has the advantage of being a purely
algebraic description, with rather amazing conceptual simplicity
moreover. In terms of the two basic sets of algebraic invariants of a
space which has turned up so far, namely cohomology (or homology) on
the one hand, and homotopy groups on the other, it can be said that
the description via topological spaces is adequate for direct
description of neither cohomology nor homotopy groups, whereas the
description via semi-simplicial sets is fairly adequate for
description of cohomology groups (taking simply the abelianization of
the semi-simplicial set, which turns out to be a chain complex, and
taking its homology and cohomology groups). The same can be said for
the alternative algebraic description of homotopy types, using cubical
complexes instead of semi-simplicial ones, which were introduced by
Serre as they were better suited, it seems, for the study of
fibrations and of the homology and cohomology spectral sequences
relative to these. One somewhat surprising common feature of those two
standard algebraic descriptions of homotopy types, is that neither is
any better adapted for a direct description of homotopy groups than
the objects we started with, namely topological spaces. This is all
the more remarkable as it is the homotopy groups really, rather than
the cohomology groups, which are commonly viewed as \emph{the} basic
invariants in the homotopy point of view, sufficient, e.g., for test
whether a given map is a ``weak equivalence'', namely gives rise to an
isomorphism in the homotopy category. It is here of course that the
point of view of ``stacks'' (``champs'' in French) of
\hyperref[ch:I]{(L)} (previously called ``\oo-groupoids'' in the
beginning of the reflections of \hyperref[ch:I]{(L)}) sets in. These
presumably give rise to a ``category of models'' and
\scrcommentinline{?} there, to the usual homotopy category, in much
the same way as topological spaces or simplicial (or cubical)
complexes, thus yielding a third \scrcommentinline{?}  description of
homotopy types, and corresponding wealth of algebraico-geometric
intuitions. Moreover, stacks are ideally suited for expressing the
homotopy groups, in an even more direct way than simplicial
complexes\pspage{4} allow description of homology and cohomology
groups. As a matter of fact, the description is formally analogous,
and nearly identical, to the description of the homology groups of a
chain complex -- and it would seem therefore that that stacks (more
specifically, Gr-stacks) are in a sense the closest possible
non-commutative generalization of chain complexes, the homology groups
of the chain complex becoming the homotopy groups of the
``non-commutative chain complex'' or stack.

It is well understood, since Dold-Puppe, that chain complexes form a
category equivalent to the category of abelian group objects in the
category of semi-simplicial sets, or equivalently, to the category of
semi-simplicial abelian groups. By this equivalence, the homology
groups of the chain complex are identified with the \emph{homotopy}
groups of the corresponding homotopy type. As for the homology and
cohomology groups of this homotopy type, their description in terms of
the chain complex we started with is kind of delicate (I forgot all
about it I am afraid!). A fortiori, when a homotopy type is described
in terms of a stack, i.e., a ``non-commutative chain complex'', there
is no immediate way for describing its homology or cohomology groups
in terms of this structure.\footnote{\alsoondate{3.5.} This is an
  error -- it appeared to me after that the cohomology can be
  expressed much in the same way as for simplicial or ``cubical
  complexes, using a \scrcommentinline{?}  the ``source'' and
  ``target'' structure of the stack (i.e., part of the primitive
  structure).} What probably should be done, is to define first a
\emph{nerve} functor from stacks to semi-simplicial sets (generalizing
the familiar nerve functor defined on the category of categories), and
define homology invariants of a stack via those of the associated
semi-simplicial set (directly suited for calculating these).

% 16
\hangsection[Stacks over topoi as unifying concept for homotopical and
\dots]{Stacks over topoi as unifying concept for homotopical and
  cohomological algebra.}\label{sec:16}%
These reflections on the proper place of the notion of a stack which in
standard homotopy algebra are largely a posteriori -- the clues they
give are surely not so strong as to give an imperative feeling for the
need of developing this new approach to the homotopy category. Rather,
the imperative feeling comes from the intuitions tied up with
non-commutative cohomological algebra over topological spaces, and
more generally over topoi, in the spirit of Giraud's thesis,\scrcomment{\cite{Giraud1971}} where a
suitable formalism for non-commutative $K^i$'s for $i=0$, $1$ or $2$
is developed. He develops in extenso the notion of stacks, we should
rather say now $1$-stacks, over a topos, constantly alluding (and for
very understandable reasons!) to the notion of a $2$-stack, appearing
closely on the heels of the $1$-stacks. Keeping in mind that
$0$-stacks are just ordinary sheaves of sets, on the space or the
topos considered, the hierarchy of increasingly higher and more sophisticated notions of
$0$-stacks, $1$-stacks, $2$-stacks, etc., which will have to be
developed over an arbitrary topos, just parallels the hierarchy of
corresponding notions over the one-point\pspage{5} space, namely
sets (= $0$-stacks), categories (or $1$-stacks), $2$-categories, etc.
Among these structures, those generalizing groupoids among categories,
namely Gr-stacks of various orders $n$, play a significant role,
especially for the description of homotopy types, but equally for a
non-commutative ``geometric'' interpretation of the cohomology groups
$\mathrm H^i(X,F)$ of arbitrary dimension (or ``order''), of a topos $X$ with
coefficients in an abelian sheaf $F$. The reflections in \hyperref[ch:I]{(L)} therefore
were directly aimed at getting a grasp on a definition of such
Gr-stacks, and whereas it seems to me to have come to a concrete
starting point for such a definition, a similar reflection for
defining just stacks rather than Gr-stacks is still lacking. This is
one among the manifold things I have in mind while sitting down on the
present reflections.

Thus $n$-stacks, relativized over a topos to ``$n$-stacks over $X$'',
are viewed primarily as the natural ``coefficients'' in order to do
(co)homological algebra of dimension $\le n$ over $X$. The
``integration'' of such coefficients, in much the same spirit as
taking objects $\RGamma_*$ (with $\RGamma$ the derived functor of the
sections functor $\Gamma$) for complexes of abelian sheaves $F_j$ on
$X$, is here merely the trivial operation of taking sections, namely
the ``value'' of the $n$-stack on the final object of $X$ (or of a
representative site of $X$, if $X$ is described in terms of a
site). The result of integration is again an $n$-stack, whose homotopy
groups (with a dimension shift of $n$) should be viewed as the
cohomology invariants $\mathrm H^i(X,F_*)$, where $F_*$ now stands for the
$n$-stack rather than for a complex of abelian sheaves. In my letters\scrcomment{\textcite{Grothendieck1975}}
(two or three) to Larry Breen in 1975, I develop some heuristics along
this point of view, with constant reference to various geometric
situations (mainly from algebraic geometry), providing the
motivations. The one motivation maybe which was the strongest, was the
realization that the classical Lefschetz theorem about comparison of
homology and homotopy invariants of a projective variety, and a
hyperplane section -- once it was reformulated suitably so as to get
rid of non-singularity assumptions, replaced by suitable assumptions
on cohomological ``depth'' -- could be viewed as comparison statements
of ``cohomology'' with coefficients in more or less arbitrary
stacks. This is carries through completely, within the then existing
conceptual framework restricted to $1$-stacks, in the thesis\scrcomment{\textcite{Raynaud1975}} of Mme
Raynaud, a beautiful piece of work. There seems to me to be
overwhelming evidence that her results (maybe her method of proof
too?) should generalize in the context of non-commutative
cohomological algebra of arbitrary dimension, with a suitable property
of ind-finiteness as the unique\pspage{6} restriction on the
coefficient stacks under consideration.

Technically speaking, \oo-stacks are the common denominator of
$n$-stacks for arbitrary $n$, in much the same way as $n$-stacks
appear both as the next-step generalization of $(n-1)$-stacks (the
former forming a category which admits the category of $(n-1)$-stacks
as a full subcategory), and as the most natural ``higher order
structure'' appearing on the category of all $(n-1)$-stacks (and on
various analogous categories whose objects are $(n-1)$-stacks subject
to some restrictions or endowed with some extra structure). I'll have
to make this explicit in due course. For the time being, when speaking
of ``stacks'' or ``Gr-stacks'', it will be understood (unless
otherwise specified) that we are dealing with the infinite order
notions, which encompass the finite ones.

Working out a theory of stacks over topoi, as the natural foundation
of non-commutative cohomological algebra, would amount among others to
write Giraud's book within this considerably wider framework. Of
course, this mere prospect wouldn't be particularly exciting by
itself, if it did not appear as something more than grinding through
an unending exercise of rephrasing and reproving known things,
replacing everywhere $n=0$, $1$ or $2$ by arbitrary $n$. I am
convinced however that there is a lot more to it -- namely the
fascination of gradually discovering and naming and getting acquainted
with presently still unknown, unnamed, mysterious structures. As is
the case so often when making a big step backwards for gaining new
perspective, there is not merely a quantitative change (from $n\le2$
to arbitrary $n$ say), but a qualitative change in scope and depth of
vision. One such step was already taken I feel by Daniel Quillen and
others, when realizing that homotopy constructions make sense not only
in the usual homotopy category, or in one or the other categories of
models which give rise to it, but in more or less arbitrary
categories, by working with semi-simplicial objects in these. The step
I am proposing is of a somewhat different type. The notion of a
stack here appears as the unifying concept for a synthesis of
homotopical algebra and non-commutative cohomological algebra. This (rather than merely furnishing us
with still another description of homotopy types, more convenient for
expression of the homotopy groups) seems to me the real ``raison
d'\^etre'' of the notion of a stack, and the main motivation for
pushing ahead a theory of stacks.\pspage{7}

% 16bis
\DontChangeNextSectionNumber
\renewcommand{\thesection}{\arabic{section}bis}
\hangsection[Categories as models for homotopy types. First glimpse
\dots]{Categories as models for homotopy types. First glimpse
  upon an ``impressive bunch'' \texorpdfstring{\textup(}{(}of
  modelizers\texorpdfstring{\textup)}{)}.}\label{sec:16bis}%
One last comment before diving into more technical
matters. Without even climbing up the ladder of increasing
sophistication, leading up to \scrcommentinline{stacks?}, there is on the very first
step, namely with just usual categories, the possibility of describing
\scrcommentinline{?}. Namely, there are two natural, well-known ways to associate to a
category $C$ (I mean here a ``small'' category, belonging to the given
universe we are working in) some kind of topological object, and hence
a homotopy type. One is by associating to $C$ the topos $\Chat$
or \scrcommentinline{?}, namely arbitrary contravariant functors from $C$ to
\Sets. The other is through the \scrcommentinline{nerve?} functor, associating to $C$
a semi-simplicial set -- and hence, if this suits us better, a
topological space, by taking the geometrical realization. By a
construction of Verdier, any topos and therefore $\mathrm{Top}(C)$
gives rise canonically to a \scrcommentinline{pro-object?} in the category of semi-simplicial
sets, and hence by ``localization'' to a pro-object in the homotopy
category (namely a ``pro-homotopy type'' in the terminology of
Artin-Mazur). In the same way, the nerve $\cst N(C)$ gives rise to a
homotopy type -- and of course \scrcommentinline{?} and may be called \emph{the}
homotopy type of $C$. When $C$ is a groupoid, we get merely a
$1$-truncated homotopy type, namely with homotopy groups $\pi_n$ which
vanish for $n\ge2$, or equivalently, with connected components
(corresponding of course to connected components of $C$) $K(\pi,1)$
spaces. This had led me at one moment in the late sixties to
hastily surmise that even for arbitrary $C$, we got merely such sums
of $K(\pi,1)$ spaces (namely, that the homotopy type of $C$ does not
change when replacing $C$ by the universal enveloping groupoid,
deduced from $C$ by making formally invertible all its arrows).
As Quillen pointed out to me, this is definitely not so -- indeed,
using categories, we get (up to isomorphism) arbitrary homotopy
types. This is achieved, I guess, using the left adjoint functor $\cst
N'$ from the inclusion functor
\[ \begin{tikzcd} \Cat \ar[r, hook, "\cst N"] & \Sssets\end{tikzcd}\]
which is fully faithful (the adjoint functor being therefore a
localization functor), and showing that for a semisimplicial set $K$,
the natural map
\[ K \to \cst N\,\cst N' (K) \]
is a weak equivalence; or what amounts to the same, that the set of
arrows in \Sssets{} by which we localize in order to get \Cat\ (namely
those transformed into invertible arrows by $\cst N'$) is made up with
weak equivalences only.%
\footnote{\alsoondate{5.3.} This is false, see \S\ref{sec:24} below
  (p.~\ref{p:21}--\ref{p:23}).} This would imply
that we may reconstruct the usual homotopy category, up to
equivalence, in terms of \Cat, by\pspage{8} just localizing \Cat{}
with respect to weak isomorphism, namely functors $C \to C'$ inducing
an isomorphism between the corresponding homotopy types. Pushing a
little further in this direction, one may conjecture that \Cat{} is a
``closed model category'', whose weak equivalences are the functors
just specified, and whose cofibrations are just functors which are
injective on objects and injective on arrows.

The fact that \scrcommentinline{?} and one moreover it seems which has
not found its way still into the minds of topologists or homotopists,
with only few exceptions I guess. These objects are extremely simply
and familiar to most mathematicians; what is somewhat more
sophisticated is the process of localization towards homotopy types,
or equivalently, the explicit description of weak equivalences, within
the framework of usual category theory. This would amount more or less
to the same as describing the homotopy groups of a category, which
does not seem any simpler than the same task for its nerve. As for the
cohomology invariants, which can be interpreted as the left derived
functors of the $\varprojlim_C$ functor, or rather its values on
particular presheaves (for instance constant presheaves), they are of
course known to be significant, independently of any particular
topological interpretation, but they are not expressible in direct
terms. (The most common computation for these is again via the nerve
of $C$.)

This situation suggests that for any natural integer $n\ge1$, the
category of $n$-stacks can be used as a category of models for the
usual homotopy category, in particular any $n$-stack gives rise to a
homotopy type, and up to equivalence we should get any homotopy type
in this way (for instance, through the $n$-category canonically
associated to any $1$-category giving rise to this homotopy type). The
homotopy types coming from $n$-Gr-stacks, however, should be merely
the $n$-truncated ones, namely those whose homotopy groups in
dimension $>n$ are zero. Moreover, $n$-Gr-stacks appear as the most
adequate algebraic structures for expressing $n$-truncated homotopy
types, the latter being deduced from the former, presumably, by the
same process of localization by weak equivalences. Moreover, in the
context of $n$-Gr-stacks, the notions of homotopy groups and of weak
equivalences are described in a particularly obvious way. Thus,
passing to the limit case $n=\oo$, it is
\scrcommentinline{\oo-Gr-stacks?} rather than general \oo-stacks which
appear as the neatest \scrcommentinline{model for homotopy types?}.

These\pspage{9} reflections suggest that there should be a rather
impressive bunch of algebraic structures, each giving rise to a model
category for the usual homotopy category, or in any case yielding this
category by localization with respect to a suitable notion of ``weak
equivalences''. The ``bunch'' is all the more impressive, if we
remember that the notion of stack (dropping now the qualification $n$,
namely assuming $n=\oo$) is not really a uniquely defined one, but
depends on the choice of a ``coherator'', namely (mainly) a category
$C$ satisfying certain requirements, which can be met in a vast
variety of ways, presumably. The construction of coherators is
achieved in terms of universal algebra, which seems here the
indispensable Ariadne's thread not to get lost in overwhelming
messiness. The natural question which arises here (and which do not
feel though like pursuing) is to give in terms of universal algebra
some kind of characterization, among all algebraic structures, of
those which give rise in some specified way (including the known
cases) to a category of models say for usual homotopy theory.

% 17
\renewcommand{\thesection}{\arabic{section}} \hangsection[The
Artin-Mazur cohomological criterion for weak \dots]{The Artin-Mazur
  cohomological criterion for weak equivalence.}\label{sec:17}%
When referring (p.\ \ref{p:5}) to the notion of a stack as a unifying
concept for homotopical algebra and non-commutative cohomological
algebra, I forgot to mention one significant observation of
Artin-Mazur along those lines (messy unification), namely that (for
ordinary homotopy types) weak equivalences (namely maps inducing
isomorphisms for all homotopy groups) can be characterized as being
those which induce isomorphisms on cohomology groups of the spaces
considered not only for constant coefficients, but also for arbitrary
twisted coefficients on the target space, including also the
non-commutative $\mathrm H^0$ and $\mathrm H^1$ for twisted
(non-commutative) group coefficients. This is indeed the basic
technical result enabling them, from known results on \'etale
cohomology of schemes (including non-commutative $\mathrm H^1$'s) to
deduce corresponding information on homotopy types. Maybe however that
the observation has acted rather as a dissuasion for developing higher
non-commutative cohomological algebra, as it seemingly says that the
non-commutative $\mathrm H^1$, plus the commutative $\mathrm H^i$'s,
was all that was needed to recover stringent information about
homotopy types. In other words, there wasn't too little in Giraud's
book, but rather, too much!

\bigbreak

\presectionfill\ondate{28.2.}\pspage{10}\par

% 18
\hangsection[Corrections and contents to letter. Bénabou's lonely
\dots]{Corrections and contents to letter. Bénabou's lonely
  approach.}\label{sec:18}%
I still have to correct a number of ``\'etourderies'' of
\hyperref[ch:I]{(L)}. The most persistent one, ever since page~\ref{p:L.8} of
that letter, is about fiber products in the coherator $\bC_\oo$, or,
equivalently, amalgamated sum in the dual category $\bB_\oo$. The
``correction'' I added in the last PS (p.~\ref{p:L.12}) is
still incorrect, namely it is not true even in the subcategory $\bB_0$
of $\bB$ that arbitrary amalgamated sums exist. I was mislead by the
interpretation of elements of $\bB_0$ in terms of (contractible)
spaces, obtained inductively by gluing together discs $D_n$
($n\in\bN$) via subdiscs, corresponding to the cellular subdivisions
of the discs $D_n$ considered p.~\ref{p:L.7}. I was implicitly
thinking of amalgamated sums of the type
\[ K \amalg_L M,\]
where $L\to K$ and $M\to K$ are \emph{monomorphisms}, corresponding to
the geometric vision of embeddings -- in which case the usual
amalgamated sum in the category of topological spaces is indeed
contractible, which was enough to make me happy. But I overlooked the
existence of morphisms in $\bB_0$ which are visibly no
monomorphisms, such as $K \amalg_L K \to K$ the codiagonal map, when
$L \to K$ is a strict inclusion of discs. In any case, I will have to
come back upon the description of the categories $\bB_0$ and its
dual $\bC_0$ and upon the definition of coherators, after the
heuristic introduction \hyperref[ch:I]{(L)}. It will be time then too to correct the
mistaken description of $\bC_{n+1}$ in terms of $\bC_n$, which I
propose on p.~\ref{p:L.11prime}, yielding probably much too big a subcategory of
$\bC_\oo$ -- the correct definition should make explicit use of the
total set $A$ of ``new'' arrows, by which $\bC_\oo$ is described in
terms of universal algebra via $\bC_0$. Another \'etourdie the day
before, p.~\ref{p:L.7prime}, is the statement that in a standard amalgamated sum,
any intersection of two maximal subcells is a subcell -- which is seen
to be false in the standard example.\footnote{Drawing of two
  hemispheres intersecting in a sphere.}

Of lesser import is the misstatement on p.~\ref{p:L.2prime},
stating that the associativity relation for the operation $*_\ell$
should by replaced by a homotopy arrow
$(\lambda * \mu) * \nu \to \lambda * (\mu * \nu)$. This is OK for
$\ell=1$ (primary composition), but already for $\ell=2$ does not make
sense as stated, because \scrcommentinline{highlighted, maybe: because
  the two sides do not have the same source and target?}. Here the
statement should be replaced by one making sense, with a homotopy
``making commutative'' a certain square, and accordingly for cubes,
etc.\ for higher order compositions $\lambda *_\ell \mu$, to give
reasonable meaning to associativity. Anyhow, such painstaking
explicitations of particular coherence properties (rather, coherence
homotopies) is kind of ruled out by the sweeping axiomatic description
of the kind of structure species we want for a ``stack'', at least, I
guess, in\pspage{11} a large part of the development of the theory of
Gr-stacks. A systematic study of particular sets of homotopies is
closely connected of course to an investigation into irredundancy
conditions which can be figured out for a coherator $\bC$. This is
indeed an interesting topic, but I decided not to get involved in
this, unless I am really forced to!

The basic notion which has been peeling out in the reflection \hyperref[ch:I]{(L)} is
of course the notion of a coherator. Concerning terminology, it
occurred to me that the dual category $\bB_\oo$ to $\bC_\oo$ is
more suggestive in some cases, for instance because of the topological
interpretation attached to its objects, and (more technically) because
of formal analogy of the role of this category, for developing
homotopical algebra, with the category of the standard (ordered)
simplices. Both mutually dual objects $\bB_\oo$, $\bC_\oo$ seem
to me to merit a name, I suggest to call them respectively \emph{left}
and \emph{right coherators}, or simply \emph{coherators} of course
when for a while it is understood on which side of the mirror we are
playing the game.

One last comment still before taking off for a heuristic voyage of
discovery of stacks! I just had a glance at Bénabou's\scrcomment{\textcite{Benabou1967}} exposé in
1967 of what he calls ``bicategories'' (Springer Lecture Notes
n\textsuperscript{\b o}~47, p.~1--77). These are none else, it appears,
than \emph{non-associative} $2$-categories, namely $2$-stacks in the
terminology I am proposing (but not $2$-Gr-stacks -- namely it is a
particular case of a general notion of \oo-stack which has still to be
developed). The most interesting feature of this expos\'e, it seems to
me, is the systematic reference to topological intuition, notably of
the structure of various diagrams. His terminology, referring to
elements of $F_0$, $F_1$, $F_2$ respectively as $0$-cells, $1$-cells
and $2$-cells, is quite suggestive of an idea of topological
realization of a $2$-stack -- it is not clear from this expos\'e
whether B\'enabou has worked out this idea, nor whether he has made a
connection with Quillen's ideas on axiomatics of homotopy theory,
which appeared the same year in the same series. In any case, in the
last section of his expos\'e, he deals with his bicategories formally
as with topological spaces, much in the same spirit as the one I was
contemplating since around 1975, and which is now motivating the
present notes. While there is no mention of B\'enabou's ideas in my
letters to Larry Breen, it is quite possible that on the unconscious
level, the little I had heard of his approach on one or two casual
occasions, had entered into reaction with my own intuitions, coming
mainly from geometry and cohomological algebra, and finally resulted
in the program outlined in those letters.

% 19
\hangsection{Beginning of a provisional itinerary
  \texorpdfstring{\textup(}{(}through
  stacks\texorpdfstring{\textup)}{)}.}\label{sec:19}%
I\pspage{12} would like now to write down a provision itinerary of
the voyage ahead -- namely to make a list of those main features of a
theory of stacks which are in my mind these days. I will write them
down in the order in which they occur to me -- which will be no
obligation upon me to follow this order, when coming back on those
features separately to elaborate somewhat on them. This I expect to
do, mainly as a way to check whether the main notions and intuitions
introduced are sound indeed, and otherwise, to see how to correct
them.

\begin{enumerate}[label=\arabic*\textsuperscript{\b{o}})]
\item Definition of the categories $\bB_0$ and its dual $\bC_0$, and
  formal definition of \scrcommentinline{?}. This definition will
  still be a provisional one, and will presumably have to be adjusted
  somewhat to allow for the various structures we are looking for in
  the corresponding category of Gr-stacks.
\item Relation between the category of Gr-stacks and the category of
  topological spaces, via two adjoint functors, the ``topological
  realization functor'' $F_* \mapsto \abs{F_*}$, and the ``singular
  stack functor'' $X \mapsto \bF_*(X)$. The situation should be
  formally analogous to the corresponding situation for
  semi-simplicial sets versus topological spaces, the role of the
  category $\cst S_*$ of standard ordered simplices being taken by the
  left coherator $\bB$ we are working with. The main technical
  difference here is that the category of Gr-stacks is not just the
  category of presheaves on $\bB$, but the full subcategory defined
  by the requirement that the presheaves considered should transform
  ``standard'' amalgamated sums into fibered products. As a matter of
  fact, the topological realization functor $F_* \mapsto \abs{F_*}$
  could be defined in a standard way on the whole of $\bB\uphat$, in
  terms of any functor
  \begin{equation}
    \label{eq:19.star}
    \bB \mapsto \Spaces \tag{*}
  \end{equation}
  (by the requirement that the extension of this functor to $B\uphat$
  commutes with arbitrary direct limits). A second difference with
  the simplicial situation lies in the fact that the only really
  compelling choice for the functor \eqref{eq:19.star} is it's
  restriction to $\bB_0$, in terms of the cells $\cst D_n$ and the
  standard ``half-hemisphere maps'' between these (\hyperref[ch:I]{L},
  p.~\ref{p:L.6}--\ref{p:L.7}). The
  extension of this functor to $\bB$ is always possible, due to the
  inductive construction of $\bB$ and to the interpretation of
  elements of $\bB_0$ as \emph{contractible} spaces, via the
  functor \eqref{eq:19.star}$_0$; but it depends on a bunch of
  arbitrary choices. To give precise meaning to the intuition that
  these choices don't really make a difference, and that the choice of
  coherator we are starting with doesn't make much of a difference
  either, will need some elaboration on the notion of equivalences
  between\pspage{13} Gr-stacks, which will have to be developed at
  a later stage.
\end{enumerate}

The idea just comes to my mind whether the exactness condition
implying standard amalgamated sums, defining the subcategory of stacks
within $\bB\uphat$, cannot be interpreted in terms of some more
or less obvious topology on $\bB$ turning $\bB$ into a site, as
the subcategory of corresponding \emph{sheaves}. This would mean that
the category (Gr-stacks) is in fact a topos, with the host of
categorical information and topological intuition that goes with such
a situation. In this connection, it is timely to recall that the
related categories \Cat{} of ``all'' categories, and (Groupoids) of
``all'' groupoids, are definitely \emph{not} topoi (if my recollection
is correct -- it isn't immediately clear to me why they are not). This
seems to suggest that, granting that (Gr-stacks) is indeed a topos,
that this would be a rather special feature of the structure species
of infinite order we are working with (as one ward so to say, among a
heap of others, for conceptual sophistication!), in contrast to the
categories (Gr-$n$-stacks) with finite $n$, which presumably are not
topoi. (For a definition of Gr-$n$-stacks in terms of Gr-stacks,
namely Gr-\oo-stacks, see below.)

A related question is whether the category (Gr-stacks) is a model
category for the usual homotopy category, the pair of adjoint functors
considered before satisfying moreover the conditions of Quillen's
comparison theorem. The obvious idea that comes to mind here, in order
to define the model structure on (Gr-stacks), is to take as ``weak
equivalences'' the maps which are transformed into weak equivalences
by the topological realization functor (which should be readily
expressible in algebraic terms), for cofibrations the monomorphisms,
and defining fibrations by the Serre-Quillen lifting property with
respect to cofibrations which are weak equivalences (with the
expectation that we even get a ``closed model category'' in the sense
of Quillen). Here it doesn't look too unreasonable to expect the same
constructions to work in each of the categories (Gr-$n$-stacks),
$n\ge1$, as well as in the categories ($n$-stacks) without Gr, which
are still to be defined though.

Coming back upon the question of a suitable topology on $\bB$, the
idea that comes to mind immediately of course is to define covering
families of an object $K$ of $\bB$, i.e., of $\bB_0$, in terms
of the components which occur in the description of $K$ as iterated
amalgamated sum of cells $D_n$. A quick glance (too quick a glance?)
seems to show this is indeed a topology, and that the sheaves for this
topology are what we expect.

\bigbreak

\presectionfill\ondate{5.3.}\pspage{14}\par

% 20
\hangsection{Are model categories sites?}\label{sec:20}%
I had barely stopped writing last Tuesday, when it became clear that
the ``quick glance'' had been too quick indeed. As a matter of fact,
the ``topology'' I was contemplating on $\bB$ in terms of covering
families does not satisfy the conditions for a ``site'' -- that
something was fishy first occurred to me through the heuristic
consequence, that the functors ``$n$-th component''
\[F_* \mapsto F_n\]
from stacks to sets are fiber functors, or equivalently, that direct
limits in the category of Gr-stacks can be computed componentwise --
now this is definitely false, for much the same reasons why it is
false already in the category \Cat{} of categories. However, it
occurred to me that the latter category \Cat{}, although not a topos
(for instance because, according to Giraud's paper on descent theory
of 1965,\scrcomment{\cite{Giraud1964}} the implications for epimorphisms
\[
\begin{tikzcd}[row sep=tiny,column sep=huge,arrows=Rightarrow]
  & \text{effective} \ar[dr] & \\
  \text{effective universal} \ar[ur] \ar[dr] & & \text{just epi} \\
  & \text{universal} \ar[ur]
\end{tikzcd}
\]
are strict), there \emph{is} a very natural topology, turning it into
a site, namely the one where a family of morphisms namely functors
\[ A_i \to A\]
is covering if and only if the corresponding family in \Sssets
\[ \text{Nerve}(A_i) \to \text{Nerve}(A)\]
is covering, i.e., if{f} every sequence of composable arrows in $A$
\[ a_0 \to a_1 \to a_2 \to a_3 \to \dots \to a_n\]
can be lifted to one among the $A_i$'s. As a matter of fact, this
condition (where it suffices to take $n=2$, visibly) is equivalent
(according to Giraud) to the condition that the family be
``universally effectively epimorphic'', i.e., covering with respect to
the ``canonical topology'' of \Cat{}.\footnote{\alsoondate{5.3.} This is definitely
  false, see \S24 below.} This suggests a third ward for associating
to a category $A$ a topology-like structure, namely the topos of all
sheaves over $\Cat_{/A}$, the site of all categories over $A$,
endowed with the topology induced by the canonical topology of \Cat{}
(which is indeed the canonical topology of $\Cat_{/A}$). It should be
an easy exercise in terms of nerves to check that the homotopy type (a
priori, a pro-homotopy type) associated to this site is just ``the''
homotopy type of $A$, defined either as the homotopy type of
Nerve$(A)$, or as the (pro)homotopy type of the topos $\Ahat$ of
all presheaves over $A$. This of course parallels the similar familiar
fact in the category \Sssets{} of all semisimplicial sets, namely that
for such a ss~set $K$, the homotopy type of $K$ can be viewed also as
the homotopy type of the induced topos $\Sssets_{/K}$ of all ss~sets
over $K$. The difference in the two cases is that in the second
case\pspage{15} the category \Sssets{} (and hence any induced
category) is already a topos, namely equivalent to the category of
sheaves on the same for the canonical topology, whereas on the
category \Cat{} this is not so.

These reflections suggest that in most if not all categories of models
encountered for describing usual homotopy types, there is a natural
structure of a site on the model category $M$
(\emph{presumably}\footnote{\alsoondate{5.3.} this assumption is false actually.} the
one corresponding to the canonical topology), with the property that
the homotopy type of any object $X$ in $M$ is canonically isomorphic
to the (pro)homotopy type of the induced site $M_{/X}$, or what
amounts to the same of the corresponding topos
\[ (M_{/X})^\sim = M^\sim_{/X}.\]
I would expect definitely this to be the case for each of the
categories (Gr-stacks), ($n$-Gr-stacks) (although this is not really a
model category in the sense of Quillen, as it gives rise only to
$n$-truncated homotopy types \ldots), (stacks) and ($n$-stacks),
corresponding to an arbitrary choice of coherator, defining the notion
of a (Gr-stack), or of a stack (for the latter and relations between
the two, see below).

% 21
\hangsection[Further glimpse upon the ``bunch'' of possible model
\dots]{Further glimpse upon the ``bunch'' of possible model categories
  and a relation between \texorpdfstring{$n$}{n}-complexes and
  \texorpdfstring{$n$}{n}-stacks.}\label{sec:21}%
I would like to digress a little more, to emphasize still about the
vast variety of algebraic structures giving rise to model categories
for the usual homotopy category, or at any rate suitable for
expressing more or less arbitrary homotopy types. Apart from stacks,
where everything is still heuristics for the time being, we have
noticed so far three examples of such structures, namely
\scrcommentinline{cubical?  and semisimplicial complexes? and
  topological spaces?}. There are a few familiar variants of the two
former, such as the ``simplicial complexes'' (in contract to
semi-simplicial ones), namely presheaves on the category of non-empty
finite sets, which can be interpreted as ss~complexes enriched with
symmetry operations on each component -- and there is the
corresponding variant in the cubical case. It would be rather
surprising that there were not just as good model categories, as the
more habitual ones -- all the more as the singular complex (simplicial
or cubical) is naturally endowed with this extra structure, which one
generally chooses to forget. More interesting variants are the
$n$-multicomplexes (simplicial or cubical, with or without
symmetries), defined by contravariant functors to sets in $n$
arguments rather than in just one, where $n\ge1$. These complexes are
familiar mainly, it seems, because of their connections with product
spaces and the K\"unneth-Eilenberg-Zilber type relations. It is
generally understood that to such a multicomplex is
associated\pspage{16} the corresponding ``diagonal'' complex, which is
just a usual complex and adequately describes the ``homotopy type'' of
the multicomplex. So why bother with relatively messy kinds of models,
when just usual complexes suffice! Here however the point is not to
get the handiest possible model categories (whatever our criteria of
``handyness''), but rather to get an idea of the variety of algebraic
structures suitable for defining homotopy types, and perhaps to come
to a clue of what is common to all these. Moreover, I feel the
relation between ordinary complexes and $n$-multicomplexes, is of much
the same nature as the relation between just ordinary categories,
perfectly sufficient for describing homotopy types, and $n$-categories
or $n$-stacks. This reminds of course of Quillen's idea of
\scrcommentinline{?}, in much the same way as categories can be
defined (via the Nerve functor) in terms of usual ss~complexes. I hit
again upon multisimplices (without symmetry), when trying to reduce to
a minimum the category $\bB_0$ of what should be called ``standard''
amalgamated sums of the cells $D_n$, where my tendency initially
(\hyperref[ch:I]L, p.~\ref{p:L.7}) had rather been to be as generous
as possible, in order to be as stringent as imaginable for the
completion condition \ref{it:8.A} of (\hyperref[ch:I]L, p.~%
\ref{p:L.8}).  Now it turns out that the coherence relations which
seem to have been written down so far (and the like of which
presumably will suffice to imply full completeness of coherence
relations, in the sense of \ref{it:8.A}) make use only of very
restricted types of such amalgamated sums, expressible precisely in
terms of multisimplices. This I check for instance on the full list of
data and axioms for B\'enabou's ``bicategories'' namely $2$-stacks, in
his 1967 Midwest Category Seminar expos\'e (already referred to). I'll
have to come back upon this point with some care, which gives also a
pretty natural way for getting Quillen's functor from $n$-stacks to
$n$-ss~complexes.

% 22
\hangsection{Oriented sets as models for homotopy types.}\label{sec:22}%
These examples of possible models for homotopy types can be viewed as
generalizations of usual complexes, or of usual categories; I would
like to give a few others which go in the opposite direction -- they
may be viewed as particular cases of categories. One is the (pre)order
structure, which may be viewed as a category structure when the map
$\Fl{} \to \Ob{} \times \Ob{}$ defined by the source and target maps
is injective. Such category is equivalent (and hence homotopic) to the
category associated with the corresponding \emph{ordered} set (when
$x\ge y$ and $y\ge x$ imply $x=y$). Ordered sets are more familiar I
guess as model objects for describing combinatorially a topological
space, in terms of a ``cellular subdivision'' by compact subsets or
``cells'' (``strata'' would be a more\pspage{17} appropriate term),
which actually need not be topological cells in the strict sense, but
rather conical (and hence contractible) spaces, each being
homeomorphic to the cone over the union of all strictly smaller strata
(this union is compact). The ordered set associated to such a
(conically) stratified space $X$ is just the set of strata, with the
inclusion relation, and it can be shown that there is a perfect
dictionary between the topological objects (at least in the case of
finite or locally finite stratifications), and the corresponding
(finite or locally finite) ordered sets, via a ``topological
realization functor''
\[ X \mapsto \abs X\]
from ordered sets to (conically stratified) topological spaces. As a
matter of fact, when $K$ is finite, $K$ is endowed with a canonical
triangulation (the so-called barycentric subdivision), the
combinatorial model of which (of ``maquette'') is as follows: the
vertices are in one-to-one correspondence with the elements of $K$
(they correspond to the vertices of the corresponding cones), and the
combinatorial simplices are the ``flags'', or subsets of $K$ which are
totally ordered for the induced order. This is still OK when $K$ is
only locally finite (restricting of course to subsets of $K$ which are
finite, when describing simplices), but in any case we can define (via
infinite direct limits in the category \Spaces of all topological
spaces) the geometric realization of $K$, together with an
interpretation of $K$ as the ordered set of strata of $\abs K$. As a
matter of fact, we get a canonical isomorphism (nearly tautological)
\[ \abs K \simeq \abs{\text{Nerve}(K)} \]
where in the second member, we have written shortly $K$ for the
category defined by $K$. I did not reflect whether it was reasonable
to expect that the categories \Preord{} and \Ord{} of preordered and
ordered sets are model categories, or even closed model categories, in
Quillen's precise sense -- but it is clear though that using ordered
sets we'll get practically any homotopy type, in any case any homotopy
type which can be described in terms of locally finite
triangulations. However it should be noted that the inclusion functors
into \Cat{} or \Sssets
\[ \Preord \hookrightarrow \Ord \hookrightarrow
\Cat \hookrightarrow \Sssets,\]
while giving the correct results on homotopy types, do not satisfy
Quillen's general conditions on pairs of adjoint functors between
model categories -- namely the adjoint functor say
\[ \Cat \to \Preord, \]
associating to a category $A$ the set $\Ob A$ with the obvious
relation, does \emph{not} commute to formation of homotopy types, as
we see in the trivial case when $\Ob A$ is reduced to one point \ldots

There\pspage{18} is another amusing interpretation of the homotopy
type associated to any preordered set $K$, via the topological space
whose underlying set is $K$ itself, and where the closed sets are the
subsets $J$ of $K$ such that $x\in J$, $y\subseteq x$ implies $y\in
J$. This is a highly non-separated topology $\tau$ (except when the
preorder relation is the discrete one), where an arbitrary union of
closed subsets is again closed. I doubt its singular homotopy type to
make much sense, however its homotopy type as a topos does, and
(possibly under mild local finiteness restrictions) it should be the
same as the homotopy type of $K$ just envisioned. Thus, a sheaf on the
topological space $K$ can be interpreted via its fibers as being just
a \emph{covariant} functor
\[ K \to \Sets \]
(NB\enspace the open sets of $K$ are just the closed sets of $K\op$, namely
for the opposite order relation, and thus every $x\in K$ has a
smallest neighborhood, namely the set $K_{\ge x}$ of all $y\in K$ such
that $y\ge x$), or what amounts to the same, a sheaf on the topos
$(K\op)\uphat$ defined by the opposite order. Hence the derived
functors of the ``sections'' functor, when working with abelian
sheaves on $(K,\tau)$, i.e., the cohomology of the topological space
$(K,\tau)$, can be interpreted in terms of the topos associated to
$K\op$. This suggests that the definition I gave of the topology of
$K$ was awkward and maybe it is indeed (although it is the more
natural one in terms of incidence relations between open strata of
$\abs K$), and that we should have called ``open'' the sets I called
``closed'' and vice-versa, or equivalently, replace $K$ by $K\op$ in
the definition I gave of a topology on the set $K$. But as far as
homotopy types are concerned, it doesn't make a difference, namely the
homotopy types associated to $K$ and $K\op$ are canonically
isomorphic. This can be seen most simply on the topological
realizations, via a homeomorphism (not only a homotopism)
\[ \abs K \simeq \abs{K\op},\]
coming from the fact that the maquettes (by which I mean the
combinatorial model for a triangulation, which Cartan time ago called
``sch\'ema simplicial'' \ldots) of the two spaces are canonically
isomorphic, because the ``flags'' of $K$ are $K\op$ are the same. A
similar argument due to Quillen using the nerves shows that for any
category $K$ (not necessarily ordered), $K$ and $K\op$ are homotopic,
although $K\uphat$ and $(K\op)\uphat$ are definitely not
equivalent, i.e., not ``homeomorphic''.

To come back to the decreasing cascade of algebraic structure suitable
for describing homotopy types, we could go down one more step still,
to the category of ``maquettes'' (Maq), namely sets $S$ together with
a family $K$ of finite subsets (the simplices of the set of
vertices\pspage{19} $S$), such that one-point subsets are simplices
and a subset of a simplex is a simplex. This category, via the functor
$(S,K) \mapsto K$, is equivalent to the full subcategory of \Ord,
whose objects are those ordered sets $K$, such that for every $x\in
K$, the set $K_{\le x}$ be isomorphic to the ordered set of non-empty
subsets of some finite set (or ``simplex''). Here the question whether
this category is a model category in the technical sense doesn't
really arise, because this category doesn't even admit finite products
-- rien \`a faire!\footnote{\alsoondate{5.3.} indeed, the notion of a maquette is
  \emph{not} an algebraic structure species!}

% 23
\hangsection[Getting a basic functor $M\to\Hot$ from a site
structure \dots]{Getting a basic functor
  \texorpdfstring{$M\to\Hot$}{M->(Hot)} from a site structure
  \texorpdfstring{$M$}{M} \texorpdfstring{\textup(}{(}altering
  beginning of a systematic
  reflection\texorpdfstring{\textup)}{)}.}\label{sec:23}%
It may be about time to get back to stacks, still I can't help going
on pondering about algebraic structures as models for homotopy
types. If we have any algebraic structure species, giving rise to a
category $M$ of set-theoretic realizations, the basic question here
doesn't seem so much whether $M$ is a model category for a suitable
choice of the three sets of arrows (fibrations, cofibrations, weak
equivalences), but rather how to define a natural functor
\begin{equation}
  \label{eq:23.star}
  M \to \Hot, \tag{*}
\end{equation}
where \Hot{} is the category of usual homotopy types, and see whether
via this functor \Hot{} can be interpreted as a category of fractions
(or ``localization'') of $M$ -- namely, of course, by the operation of
making invertible those arrows in $M$ which are transformed into
isomorphisms in \Hot{}. In any case, if we have such a natural functor,
the natural thing to do is to call those arrows ``weak
equivalences''. If we want $M$ to be a category of models, various
examples suggest that the natural thing again is to take as
cofibrations the monomorphisms, and then (expecting that the model
categories we are going to meet will be closed model categories) to
define fibrations by the Serre-Quillen lifting property with respect
to cofibrations (=monomorphisms) which are weak equivalences. This
being done, it becomes meaningful to ask if indeed $M$ is a category
of models.

Now the reflections of the beginning of today's notes
(p.~\ref{p:14}--\ref{p:15})
suggest a rather natural way for describing a functor
\eqref{eq:23.star}, which makes sense in fact, in principle, for any
category $M$, namely: endow $M$ with its canonical topology (unless a
still more natural one appears at hand -- I am not sure there is any
better one in the present context),\footnote{\alsoondate{5.3.} the canonical
  topology is \emph{not} always suitable, see \S24 below.} and assume
that for every $X\in\Ob M$, the pro-homotopy type of the induced site
$M_{/X}$ is essentially constant, i.e., can be identified with an
object in \Hot{} itself. We then get the functor \eqref{eq:23.star} in
an obvious way. It then becomes meaningful to ask whether this functor
is a localization functor.

When $M$ is defined in terms of an algebraic structure species,
it\pspage{20} admits both types of limits, without finiteness
requirement -- and we certainly would expect indeed at least existence
of finite \emph{and} infinite direct sums in $M$, if objects of $M$
were to describe arbitrary homotopy types. However, in view of the
special exactness properties of \Hot{}, which are by no means autodual,
we will expect moreover direct sums in $M$ to be ``universal and
disjoint'', in Giraud's sense. This condition, which characterizes to
a certain extent categories which at least mildly resemble or parallel
categories such as \Sets, \Spaces{} and similar categories, whose
objects more or less express ``shapes'' -- this condition at once
rules out the majority of the most common algebraic structures, such
as rings, groups, modules over a ring or anything which yields for $M$
an abelian category, etc. If we describe an algebraic structure
species in terms of its universal realization in a category stable
under finite inverse limits, then such a structure species can be
viewed as being defined by such a category $\bC$, and its
realizations in any other such $C$ as the left-exact functors $\bC\to C$
(the universal realization of the structure within $\bC$
corresponding to the identity functor $\bC\to \bC$). In terms of
the dual category $\bB$, associating to every element in $\Ob{\bB}
= \Ob{\bC}$ the covariant functor $\bC\to\Sets$ it
represents, we get a fully-faithful embedding
\[\bB\hookrightarrow M,  \]
by which $\bB$ can be interpreted as the category of the
(set-theoretic) realizations of the given structure which are of
``finite presentation'' in a suitable sense (in terms of a given
family of generators of $\bB$ namely cogenerators of $\bC$,
considered as corresponding to the choice of ``base-sets'' for the
given structure species -- such choice however being considered as a
convenient way merely to describe the species in concrete terms\dots).
If I remember correctly, $M$ can be deduced from $\bB$, up to
equivalence, as being merely the category of Ind-objects of $\bB$,
i.e., the inclusion functor above yields an equivalence of categories
\[ \text{Ind}(\bB) \toequ M, \]
which implies, I guess, that the exactness properties of $M$ mainly
reflect those of $\bB$. Thus I would expect the condition we want
on direct sums in $M$ to correspond to the same condition for finite
direct sums in $\bB$, not more not less. Thus the algebraic
structure species satisfying this condition should correspond exactly
to small categories $\bB$, stable under finite direct limits, and
such that finite sums in $\bB$ are disjoint and universal. This
condition is presumably necessary, if we want the functor
\eqref{eq:23.star} from $M$ to \Hot{} to be defined and to be a
localization functor -- a condition which it would be nice to
understand directly in\pspage{21} terms of $\bB$, and
(presumably) in terms of the canonical topology of $\bB$, which
should give rise to a localization functor
\[ \bB \to \Hot_{\text{ft}}, \]
where the subscript ft means ``finite type'' -- granting that the
notion of homotopy types of finite type (presumably the same as
homotopy types of finite triangulations, or of finite CW space) is a
well-defined notion. As usual, it is in $\bB$, not in $\bC$,
that geometrical constructions take place which make sense for
topological intuition. More specifically, it seems that in the cases
met so far, there \emph{are} indeed privileged base-sets for the
structure species considered (such as the ``components'' or a
semisimplicial or cubical complex, or of a stack, etc.), indexed by
the natural integers or $n$-tuples of such integers, and which
``correspond'' to topological cells of various dimensions. Moreover,
some of the basic structural monomorphic maps between these objects of
$\bB$ define cellular decompositions of the topological spheres
building these topological cells. These objects and ``boundary maps''
between them define a (non-full, in general) subcategory of $\bB$,
say $\bB_\oo$, which looks like the core of the category $\bB$,
from which the topological significance of $\bB$ is springing. It
is in terms of $\bB_\oo$ that the ``correspondence'' (vaguely
referred to above) with topological cells and spheres takes a precise
meaning. Namely, associating to any object of $\bB_\oo$ the ordered
set of its subobjects (within $\bB_\oo$ of course, not $\bB$),
the (stratified) topological realization of this ordered set is a
cell, the family of subcells of smaller dimension (namely different
from the given one) is a cellular subdivision of the sphere bounding
this cell.

\bigbreak
\presectionfill\ondate{7.3.}\par

% 24
\hangsection{A bunch of topologies on \texorpdfstring{\Cat}{(Cat)}.}%
\label{sec:24}%
Yesterday there occurred to me a big ``\'etourderie'' again of the day
before, in connection with a reflection on a suitable ``natural'' site
structure on the category \Cat{} -- namely when asserting that for a
family of morphisms, i.e., functors in \Cat{}
\begin{equation}
  \label{eq:24.star}
  A_i \to A, \tag{*}
\end{equation}
in order for the corresponding family to be ``covering'' namely
epimorphic in the category (a topos, as a matter of fact) \Sssets,
namely for the corresponding families of mappings of sets
\begin{equation}
  \label{eq:24.starstar}
  \Fl_n(A_i) \to \Fl_n(A) \tag{**}
\end{equation}
to be epimorphic, it was sufficient that his condition be satisfied
for $n=2$ (which, according to Giraud, just means that the family
\eqref{eq:24.star} is covering for the canonical topology of
\Cat{}). This is obviously false -- morally, it would mean that an
$n$-simplex is ``covered'' in a reasonable sense by its
sub-$2$-simplices, which is pretty absurd. To\pspage{22} give
specific positive statements along these lines, let's for any
$N\in\bN$, denote by $T_N$ the topology on \Cat{} for which a family
\eqref{eq:24.star} is covering if{f} the families
\eqref{eq:24.starstar} are for $n\le N$ -- which means also that the
corresponding family of $N$-truncated nerves
\[ \text{Nerve}(A_i)[N] \to \text{Nerve}(A)[N] \]
is epimorphic. For $N=1$, this means that the family is ``universally
epimorphic'', for $N=2$, that it is ``universally effectively
epimorphic'', i.e., covering for the canonical topology of \Cat{}
(Giraud, loc.\ cit.\ page~28). It turns out that this decreasing
sequence of topologies on \Cat{} is strictly decreasing -- as a matter
of fact, denoting by $\Delta[N]$ the category of standard, ordered
simplices of dimension $\le N$, and using the inclusions
\[ \Delta[N] \hookrightarrow \Cat \hookrightarrow
\Delta[N]\uphat \]
(for $N\ge 2$ say), an immediate application of the ``comparison
lemma'' for sites shows that we have an equivalence
\[ (\Cat, T_N)^\sim \simeq {\Delta[N]}\uphat,\]
and hence, for any object $A$ in \Cat{}, namely a category $A$, we get
an equivalence
\[ (\Cat, T_N)_{/A}^\sim \simeq (\Delta[N]_{/A})\uphat,\]
and hence the homotopy type of the first hand member is \emph{not}
described by and equivalent to the homotopy type of the whole
Nerve$(A)$ object, but rather by its $N$-skeleton, which has the same
homotopy and cohomology invariant in dimension $\le N$, but by no
means for higher dimensions. This shows that among the topologies
$T_N$, none is suitable for recovering the homotopy type of objects of
\Cat{} in the way contemplated two days ago (page 15); the one topology
which \emph{is} suitable is the one which may be denoted by $T_\oo$,
and which is the one indeed which I first contemplated (page 14),
before the mistaken idea occurred to me that it was the same as the
canonical topology.

A very similar mistake occurred earlier, when I surmised that the left
adjoint functor $N'$ to the inclusion or Nerve functor $N$ from \Cat{}
to $\Sssets = \Simplexhat$ had the property that for any object
$K$ in \Sssets, $K$ and $N'(K)$ had the same homotopy type. Looking
up yesterday the description in Gabriel-Zisman of this functor, this
recalled to my mind that it factors (via the natural restriction
functor) through the category ${\Delta[2]}\uphat$, hence any
morphism $K \to K'$ inducing an isomorphism on the $2$-skeletons
(which by no means implies that it is a homotopy equivalence) induces
an isomorphism $N'(K) \tosim N'(K')$, and a fortiori a
homotopy equivalence. This shows that, even if it should be true that
\Cat{} is\pspage{23} a model category in Quillen's sense, the
situation with the inclusion functor
\[ N : \Cat \to \Sssets \]
and the left adjoint functor $N'$ is by no means the one of Quillen's
comparison theorem, where the two functors play mutually dual roles
and both induce equivalences on the corresponding localized
homotopical categories. Here only $N$, not $N'$, induces such
equivalence. This is analogous to the situation, already noted before,
of the inclusion functors
\[ \Preord \hookrightarrow \Cat
\quad\text{or}\quad
\Ord \hookrightarrow \Cat,\]
which by localization induce equivalences on the associate homotopy
categories, but the left adjoint functors do not share this property.

The topology $T_\oo$ just considered on \Cat{} as a ``suitable''
topology for describing homotopy types of objects of \Cat{}, was of
course directly inspired by the semi-simplicial approach to homotopy
types, via simplicial complexes, namely via the two associated
inclusion functors
\[ \Simplex \hookrightarrow \Cat \hookrightarrow
\Simplexhat,\]
the second functor associating to every category $A$ the ``complex''
of its ``simplicial diagrams'' $a_0 \to a_1 \to \dots \to a_n$. If we
had been working with cubical complexes rather than ss ones for
describing homotopy types, this would give rise similarly to two
functors
\[ \square \hookrightarrow \Cat \hookrightarrow
\square\uphat,\]
where $\square$ is the category of ``standard cubes'' and face and
degeneracy maps between them, and where the second inclusion
associates to every category $A$ the cubical complex of its ``cubical
diagrams'', namely commutative diagrams in $A$ modelled after the
diagram types $\square_n[1]$, the $1$-skeleton of the standard
$n$-cubes $\square_n$, with suitable orientations on its edges
(indicative of the direction of corresponding arrows in $A$). The
natural idea would be to endow \Cat{} with the topology, $T_\oo'$ say,
induced by $\square\uphat$, namely call a family \eqref{eq:24.star}
covering if{f} for every $n\in\bN$, the corresponding family of maps
of sets
\[ \text{Cub}_n(A_i) \to \text{Cub}_n(A)\]
is epimorphic. This topology appears to be coarser than $T_\oo$ (i.e.,
there are fewer covering families), and the comparison lemma gives now
the equivalence
\[ (\Cat,T_\oo')^\sim \simeq \square\uphat,\]
which shows that definitely the topology is \emph{strictly} coarser
than $T_\oo$, as it gives rise to a non-equivalent topos,
$\square\uphat$ instead of $\Simplexhat$.

These reflections convince me 1)\enspace there are indeed topologies
on \Cat{}, suitable for describing the natural homotopy types of
objects of \Cat{} namely of categories, and 2)\enspace that there is
definitely\pspage{24} no privileged choice for such a topology. We
just described two such, but using multicomplexes (cubical or
semisimplicial) rather than simple complexes should give us infinitely
many others just as suitable, and I suspect now that there must be a
big lot more of them still!

% 25
\hangsection{A tentative equivalence relation for topologists.}\label{sec:25}%
A fortiori, coming back to the intriguing question of characterizing
the algebraic structure species suitable for describing homotopy types
in the usual homotopy category \Hot{}, and for recovering \Hot{} as a
category of fractions of the category $M$ of all set-theoretic
realizations of this species, it becomes clear now that we cannot hope
reasonably for a ``natural'' topology on $M$, distinguished among all
others, giving rise to the wished-for functor
\begin{equation}
  \label{eq:25.star}
  M \to \Hot \tag{*}
\end{equation}
in the way contemplated earlier (page 19 ff.). Here it occurs to me
that anyhow, if concerned mainly with defining the functor
\eqref{eq:25.star}, we should consider that two topologies $T,T'$ on
$M$ such that $T \ge T'$ and hence giving rise to a morphism of topoi
\begin{equation}
  \label{eq:25.starstar}
  M_T^\sim \to M_{T'}^\sim \tag{**}
\end{equation}
(the direct image functor associated to this morphism being the
natural inclusion functor, when considering $T$-sheaves as particular
cases of $T'$-sheaves) are ``\emph{equivalent}'' -- maybe we should
rather say ``Hot-equivalent'' -- if for any object $A$ in $M$, the
induced morphism
\[ (M_T^\sim)_{/A} \to (M_{T'}^\sim)_{/A} \]
is a homotopy equivalence. This can be viewed as an intrinsic property
of the morphism of topoi \eqref{eq:25.starstar}, of a type rather
familiar I guess to people used to the dialectics of \'etale
cohomology, where a very similar notion was met and given the name of
a ``globally acyclic morphism''. The suitable name here would be
``globally aspheric morphism'' which is a reinforcement of the former,
in the sense of being expressible in terms of isomorphism relations in
cohomology with arbitrary coefficient sheaves on the base,
\emph{including non-commutative} coefficient sheaves.\footnote{For a
  proper map of paracompact spaces, this condition just means that the
  fibers are ``aspheric'', namely ``contractible'' (in \v Cech'
  sense)} The relation just introduced between two topologies $T,T'$
on a category $M$ makes sense for any $M$ (irrespective of the
particular way $M$ was introduced here), it is not yet an equivalence
relation though -- so why not introduce the equivalence relation it
generates, and call this ``Hot-equivalence'' -- unless we find a
coarser, and cleverer notion of equivalence, deserving this name. The
point which, one feels now, should be developed, is that this notion
of equivalence should be the coarsest we can find out, and which still
implies that to any Hot-equivalence class of topologies on $M$ there
should be canonically associated a functor \eqref{eq:25.star}, which
should essentially be ``the'' common value of all the similar
functors, associated to the topologies $T$ within\pspage{25} this
class. Maybe even it could be shown that this equivalence class can be
recovered in terms of the corresponding functor \eqref{eq:25.star}, in
the same way as (according to Giraud) a ``topology'' on $M$ can be
recovered from the associated subtopos of $M\uphat$, namely the
associated category of sheaves on $M$ (in such a way that the set of
``topologies'' on $M$ can be identified with the set of ``closed
subtopoi'' of $M\uphat$). The very best one could possibly hope for
along these lines would be a one-to-one correspondence between
isomorphism classes of functors \eqref{eq:25.star} (satisfying certain
properties?), and the so-called ``Hot-equivalence classes'' of
topologies on $M$.

Whether or not this tentative hope is excessive, when it comes to the
(still somewhat vague) question of understanding ``which algebraic
structure species are suitable for expressing homotopy types'', it
might not be excessive though to expect that in all such cases, $M$
should be equipped with just \emph{one} ``natural'' Hot-equivalence
class of topologies on $M$, which moreover (one hopes, or wonders)
should be expressible directly in terms of the intrinsic structure of
$M$, or, what amounts to the same, in terms of the full subcategory
$\bB$ of objects of ``finite presentation'', giving rise to $M$ via
the equivalence
\[ M \simeq \text{Ind}(\bB). \]
As we just saw in the case $M = \Cat$, the so-called
``canonical topology'' on $M$ need not be within the natural
Hot-equivalence class -- and I am at a loss for the moment to give a
plausible intrinsic characterization of the latter, in terms of the
category $M=\Cat$.

% 26
\hangsection{The dawn of test categories and test functors\dots}%
\label{sec:26}%
What comes to mind though is that the categories such as $\Simplex$,
$\Square$ and their analogons (corresponding to multicomplexes rather
than monocomplexes, for instance) can be viewed as (generally not
full) subcategories of $M$ (in fact, even of the smaller category
\Ord{} of all ordered sets). The topologies we found on $M$ were in
fact associated in an evident way to the choice of such
subcategories. As was already felt by the end of the reflection two
days ago (p.~\ref{p:21}),
these subcategories (denoted there by $\bB_\oo$)
have rather special features -- they are associated to simultaneous
cellular decompositions of spheres of all dimensions -- and it is this
feature, presumably, that makes the associated ``trivial'' algebraic
structure species, giving rise to the category of set-theoretic
realizations $\bB_\oo\uphat$, eligible for ``describing
homotopy types''. In the typical example $M = \Cat$ though,
contrarily to what was suggested on page~\ref{p:21}
(when thinking mainly of
the rather special although important case when $M$ is expressible as
a category $B\uphat$, for some category $B$ such as\pspage{26}
$\Simplex$, $\Square$ etc.), there is no really privileged choice of
such subcategory $\bB_\oo$ -- we found indeed a big bunch of such,
the ones just as good as the others. The point of course is that the
corresponding topologies on $M$, namely induced from the canonical
topology on $\bB_\oo\uphat$ by the canonical functor
\[ M \to \bB_\oo\uphat,\]
are Hot-equivalent for some reason or other, which should be
understood. The plausible fact that emerges here, is that the
``natural'' Hot-equivalence classes of topologies on $M$ is
associated, in the way just described, to a class (presumably an
equivalence class in a suitable set for suitable equivalence
relation\ldots) of subcategories $\bB_\oo$ in $M$. The question of
giving an intrinsic description of the former, is apparently reduced
to the (possibly more concrete) one, of giving an intrinsic
description of a bunch of subcategories $B=\bB_\oo$ of $M$. This
description, one feels, should both \namedlabel{step:26.a}{a)}\enspace
insist on intrinsic properties of $B$, independent of $M$, namely of
the structure species one is working with, and
\namedlabel{step:26.b}{b)}\enspace be concerned with the particular
way in which $B$ is embedded in $M$, which should by no means be an
arbitrary one.

The properties \ref{step:26.a} should be, I guess, no more no less
than those which express that the ``trivial'' algebraic structure
species defined by $B$, giving a category of set-theoretic
realizations $M_B=B\uphat$, should be ``suitable for describing
homotopy types''. The examples at hand so far suggest that in this
case, the canonical topology on the topos $M_B$ is within the natural
Hot-equivalence class, which gives a meaning to the functor
\[ M_B = B\uphat \to \Hot,\]
indeed it associates to any $a \in \Ob{B\uphat}$ the homotopy type
of the induced category $B_{/a}$ of all objects of $B$ ``over
$a$''.\footnote{because $B\uphat_{/a} \simeq (B_{/a})\uphat$.}
Thus a first condition on $B$ is that this functor should be a
``localization functor'', identifying \Hot{} with a category of
fractions of $M_B = B\uphat$. This does look indeed as an extremely
stringent condition on $B$, and I wonder if the features we noticed in
the special cases dealt with so far, connected with cellular
decompositions of spheres, have any more compulsive significance than
just giving some handy \emph{sufficient} conditions (which deserve to
be made explicit sooner or later!) for ``eligibility'' of $B$ for
recovering \Hot{}. Beyond this, one would of course like to have a
better understanding of what it really means, in terms of the internal
structure of $B$, that the functor above from $B\uphat$ to \Hot{} is
a localization functor.

Once\pspage{27} this internal condition on $B$ is understood, step
\ref{step:26.b} then would amount to describing, in terms of an arbitrary
``eligible'' algebraic structure species expressed by the category
$M$, of what we should mean by ``eligible functors''\footnote{we will
  rather say ``test functors'', see below\ldots}
\[ B \to M,\]
giving rise in the usual way to a functor
\[ M \to B\uphat.\]
In any case, the latter functor defines upon $M$ an induced topology,
$T$ say, and the comparison lemma tells us that if either $B \to M$ or
$M\to B\uphat$ is fully faithful, then the topos associated to $M$
is canonically equivalent to $B\uphat$ (using this comparison lemma
for the functor which happens to be fully faithful). From this follows
that the functor (*) $M^{\sim} \to \Hot$ defined by
$T$ is nothing but the compositum 
\[ M^\sim \tosimeq B\uphat \to \Hot,\]
and hence a localization functor. In other words, when $B$ is a
category satisfying the condition seen above,\footnote{a ``test
  category'', as we will say} then \emph{any} functor
$B \to M$ satisfying one of the two fullness conditions above yields a
corresponding description of \Hot{} as a localization of $M^\sim$. What
is still lacking though is a grasp on when two such functors $B \to
M$, $B' \to M$ define essentially ``the same'' functor $M \to
\Hot$, or (more or less equivalently) two \Hot{}-equivalent
topologies $T,T'$ on $M$; is it enough, for instance, that they give
rise to the same notion of ``weak equivalences'' (namely morphisms in
$M$ which are transformed into an isomorphism of \Hot{})? And moreover,
granting that this equivalence relation between certain full
subcategories (say) $B$ of $M$ is understood, how to define, in terms
of $M$, a ``natural'' equivalence class of such full subcategories,
giving rise to a canonical functor $M\to\Hot$?\footnote{Same
  question arises for Hot-equivalence on topologies on $M$\ldots}

Recalling that the algebraic structure considered can be described in
terms of an arbitrary small category $\bB$ where arbitrary finite
direct limits exist (namely $\bB$ is the full subcategory of $M$ of
objects of finite presentation), it seems reasonably to assume that
indeed
\[B \to \bB \quad\text{(a full embedding)},\]
and the question transforms into describing a natural equivalence
class of such full subcategories, in (more or less) any small category
$\bB$ where finite direct limits exist, and where moreover there
exist such full subcategories $B$. Also, we may have to throw in some
extra conditions on $\bB$, such as the condition that direct sums
be ``disjoint'' and ``universal'' already contemplated before.

Maybe\pspage{28} I was a little overenthusiastic, when observing
for any full embedding of a category $B$ in $M$ (let's call the
categories $B$ giving rise to a localization functor $B\uphat\to
\Hot$ \emph{homotopy-test} categories, or simply \emph{test}
categories) we get a localization functor
\[ M^\sim = (M, T_B)^\sim \to \Hot,\]
where $T_B$ is the topology on $M$ corresponding to the full
subcategory $B$. After all, there is a long way in between $M$ itself
and the category of sheaves $M^\sim$ -- and what we want is to get
\Hot{} as a localization of $M$ itself, not of $M^\sim$. It is not even
clear, without some extra assumptions, that the natural functor from
$M$ to $M^\sim$ is fully faithful, namely that $M$ can be identified
with a full subcategory of the category $M^\sim$ we've got to localize
to get \Hot{}. We definitely would like this to be true, or what
amounts to the same, that the functor $M\to B\uphat$ defined by
$B\to M$ should be fully faithful -- which means also that the full
subcategory $B$ of $M$ is ``generating by strict epimorphisms'' namely
that for every $K$ in $M$, there exists a strictly epimorphic family
of morphisms $b_i \to K$, with sources $b_i$ in $B$. This
interpretation of full faithfulness of $M \to B\uphat$ is OK when
$B \to M$ is fully faithful, a condition which I gradually put into
the fore without really compelling reason, except that in those
examples I have in mind and which are \emph{not} connected with the
theory of stacks of various kinds, this condition is satisfied
indeed. Apparently, with this endless digression on algebraic models
for homotopy types, stacks (which I am supposed to be after, after
all) are kind of fading into the background! Maybe we should after all
forget about the fully faithfulness condition on either $B\to M$ or
$M\to B\uphat$, and just insist that the compositum
\[ M \to M^\sim \to B\uphat \to \Hot\]
(which can be described directly in terms of the topology $T_B$ on $M$
associated to the functor $B\to M$) should be a localization
functor. I guess that for a given $M$ or $\bB$, the mere fact that
there should exist a test category $B$ and a functor $B\to M$, or
$B\to\bB$, having this property is already a very strong condition on
the structure species considered, namely on the category $\bB$ which
embodies this structure. It possibly means that the corresponding
functor
\[ \bB \to \Hot\]
factors through a functor
\[ \bB \to \Hot_{\mathrm{f.t.}} \quad \text{(f.t.\ means ``finite
  type'')}\]
which is itself a localization functor. It is not wholly impossible,
after all, that this condition on a functor $B \to \bB$ ($B$ a test
category) is so stringent, that all such functors (for variable $B$)
must be\pspage{29} already ``equivalent'', namely define
Hot-equivalent topologies on $\bB$ (or $M$, equivalently), and hence
define ``the same'' functor $M\to\Hot$ or
$\bB\to\Hot_{\mathrm{f.t.}}$.

% 27
\hangsection{Digression on ``geometric realization'' functors.}\label{sec:27}%
All this is pretty much ``thin air conjecturing'' for the time being
-- quite possibly the notion of a ``test category'' itself has to be
considerably adjusted, namely strengthened, as well as the notion of a
``test functor'' $B \to \bB$ or $B \to M$ -- some important features
may have entirely escaped my attention. The one idea though which may
prove perhaps a valid one, it that a suitable localization functor
\begin{equation}
  \label{eq:27.star}
  M \to \Hot \tag{*}
\end{equation}
may be defined, using either various topologies on $M$ (related by a
suitable ``Hot-equivalence'' relation), or various functors $B \to M$
or $B \to \bB$ of suitable ``test categories'' $B$, and how the two
are related. I do not wish to pursue much longer along these lines
though, but rather put now into the picture a third way still for
getting a functor \eqref{eq:27.star}, namely through some more or less
natural functor
\begin{equation}
  \label{eq:27.starstar}
  M \to \Spaces, \quad K \mapsto \abs K, \tag{**}
\end{equation}
called a ``geometric (or topological) realization functor''. There is
a pretty compelling choice for such a functor, in the case of
(semisimplicial or cubical) complexes or multicomplexes of various
kinds, and accordingly for the subcategories \Cat, \Preord, \Ord{} or
\Sssets, using geometric realization of semisimplicial complexes. In
the case of the considerably more sophisticated structure of Gr-stacks
though (or the relator\scrcommentinline{?} structure of stacks, which will be dealt
with in much the same way below), although there is a pretty natural
choice for geometric realization on the subcategory $\bB_\oo$ of $M$
embodying the ``primitive structure'' (namely the structure of an
\oo-graph, see below also); it has been seen that the extension of
this to a functor on the whole of $M$ (via its extension to the left
coherator defining the structure species, which we denoted by $\bB$ at
the beginning of these notes, but which is not quite the $\bB$
envisioned here) is by no means unique, that it depends on a pretty
big bunch of rather arbitrary choices. This indeterminacy now appears
as quite in keeping with the general aspect of a (still somewhat
hypothetical) theory of algebraic homotopy models, gradually emerging
from darkness. It parallels the corresponding indeterminacy in the
choice of an ``eligible'' topology on $M$ (call these topologies the
test topologies), or of a test functor $B \to M$. What I would like
now to do, before coming back to stacks, is to reflect a little still
about the relations between such choice of a ``geometric realization
functor'', and test topologies or test functors relative to $M$.

\bigbreak

\presectionfill\ondate{8.3.}\pspage{30}\par

% 28
\hangsection{The ``inspiring assumption''. Modelizers.}\label{sec:28}%
While writing down the notes yesterday, and this morning still while
pondering a little more, there has been the ever increasing feeling
that I ``was burning'', namely turning around something very close,
very simple-minded too surely, without quite getting hold of it
yet. In such a situation, it is next to impossible just to leave it at
that and come to the ``ordre du jour'' (namely stacks) -- and even the
``little reflection'' I was about to write down last night (but it was
really too late then to go on) will have to wait I guess, about the
``geometric realization functors'', as I feel it is getting me off
rather, maybe just a little, from where it is ``burning''!

There was one question flaring up yesterday
(p.~\ref{p:27}) which I nearly
dismissed as kind of silly, namely whether two localization functors
\begin{equation}
  \label{eq:28.star}
  M \to \Hot \tag{*}
\end{equation}
obtained in such and such a way were isomorphic (maybe even
canonically so??) provided they defined the same notion of ``weak
equivalence'', namely arrows transformed into isomorphisms by the
localization functors. Now this maybe isn't so silly after all, in
view of the following\par
\noindent \textbf{Assumption}: \scrcommentinline{unreadable because
  highlighted}

This means 1)\enspace any equivalence $\Hot \tosimeq \Hot$ is
isomorphic to the identity functor, and 2)\enspace any automorphism of
the identity functor (possibly even any endomorphism?) is the
identity.

Maybe these are facts well-known to the experts, maybe not -- it is
not my business here anyhow to set out to prove such kinds of
things. It looks pretty plausible, because if there was any
non-trivial autoequivalence of \Hot, or automorphism of its identity
functor, I guess I would have heard about it, or something of the sort
would flip to  my mind. It would not be so if we abelianized \Hot\
some way or other, as there would be the loop and suspension functors,
and homotheties by $-1$ of $\id_\Hot$.

This assumption now can be rephrased, by stating that a localization
functors \eqref{eq:28.star} from any category $M$ into \Hot\ is well
determined, up to a unique isomorphism, when the corresponding class
$W \subset \Fl(M)$ of weak equivalences is known, in positive response
to yesterday's silly question!

Such situation \eqref{eq:28.star} seems to me to merit a name. As the
work ``model category'' has already been used in a somewhat different
and more sophisticated sense by Quillen, in the context of homotopy, I
rather use another\pspage{31} one in the situation here. Let's call
a ``modelizing category'', or simply a ``\emph{modelizer}''
(``mod\'elisatrice'' in French), any category $M$, endowed with a set
$W \subset \Fl(M)$ (the weak equivalences), satisfying the obvious
condition:

\noindent\parbox[t]{0.1\textwidth}{(Mod)\par}
\parbox[t]{0.9\textwidth}{\vspace*{-11pt}%
  \begin{enumerate}[,label=\alph*)]
  \item\label{it:Mod.a}
    $W$ is the set of arrows made invertible by the localization
    functor $M \to W^{-1}M$, and
  \item\label{it:Mod.b}
    $W^{-1}M$ is equivalent to \Hot,
  \end{enumerate}}
or equivalently, there exists a localization functor
\eqref{eq:28.star} (necessarily unique up to unique isomorphism) such
that $W$ be the set of arrows made invertible by this functor.

Let $(M,W)$, $(M',W')$ be two modelizers, a functor $F: M\to M'$ is
called \emph{model-preserving}, or a \emph{morphism} between the
modelizers, if it satisfies either of the following equivalent
conditions:
\begin{enumerate}[label=(\roman*)]
\item $F(W) \subset W'$, hence a functor $F_{W,W'}: W^{-1}M \to
  W'^{-1}M'$, and the latter is an equivalence.
\item The diagram
\[  \begin{tikzcd}[column sep=tiny]
    M \ar[dr] \ar[rr, "F"] & & M'\ar[dl] \\ & \Hot &
  \end{tikzcd}\]
  is commutative up to isomorphism (where the vertical arrows are the
  ``type functors'' associated to $M,M'$ respectively.
\end{enumerate}
When dealing with a modelizer $(M,W)$, $W$ will be generally
understood so that we write simply $M$. When $M$ is defined in terms
of an algebraic structure species, the task arises to find out whether
(if any) there exists a \emph{unique} $W \subset \Fl(M)$ turning $M$
into a modelizer, and if not so, if we can however pinpoint one which
is a more natural one, and which we would call ``canonical''.

Here is a diagram including most of the modelizers and
model-preserving functors between these which we met so far (not
included however those connected with the theory of ``higher'' stacks
and Gr-stacks, which we will have to elaborate upon later on):
\[
\begin{tikzcd}[row sep=tiny,column sep=small]
  & & & \Simplexhat = \text{(ss~sets)}\ar[dr, "\xi"] & \\
  \Ord\ar[r, hook] & \Preord\ar[r, hook] & \Cat\ar[ur, hook, "\alpha"]
  \ar[dr, hook, swap, "\beta"] & & \Cat \\
  & & & \square\uphat = \text{(cub.\ sets)} \ar[ur, swap, "\eta"] &
\end{tikzcd}\]
where the two last functors, with values in \Cat, are the two obvious functors,
obtained from
\begin{equation}
  \label{eq:28.starstar}
  i_A : \Ahat \to \Cat, \quad F\mapsto A_{/F} \tag{**}
\end{equation}
by particularizing to $A=\Simplex$ or $\Square$. As we noticed before,
the four first among these six functors admit left adjoints, but
except for the first, these adjoints are \emph{not} model
preserving. The two last functors, and more\pspage{32} generally
the functor \eqref{eq:28.starstar}, admit right adjoints, namely the
functor
\[ j_A: \Cat\to\Ahat,\]
where $j_A(B) = (a \mapsto \Hom(A_{/a},B))$ ($B\in\Ob\Cat$). It should
be noted that the functors $j_\Simplex, j_\Square$ are \emph{not} the
two functors $\alpha,\beta$ which appear in the diagram above, the
latter are associated to the familiar functors
\begin{equation}
  \label{eq:28.starstarstar}
  \begin{tikzcd}[row sep=tiny]
    \Simplex\ar[dr] & \\ & \Cat & \\ \Square\ar[ur] &
  \end{tikzcd}\tag{***}
\end{equation}
(factoring in fact through \Ord), associating to every ordered
simplex, or to each multiordered cube, the corresponding $1$-skeleton
with suitable orientations on the edges, turning the vertices of this
graph into an ordered set; while the two former are associated to the
functors deduced from \eqref{eq:28.starstar} by restricting to $A
\subset \Ahat$, namely
\[ n \mapsto \Simplex_{/\Simplex_n}\]
in the case of $\Simplex$, and accordingly for $\Square$. The values of
these functors, contrarily to the two preceding ones, are
\emph{infinite} categories, and they cannot be described by (i.e.,
``are'' not) (pre)ordered sets. If however we had defined the
categories $\Simplex,\Square$ in terms of iterated boundary operations
only, excluding the degeneracy operations (which, I feel, are not
really needed for turning $\Simplexhat$ and $\Square\uphat$ into
modelators), we would get indeed \emph{finite} ordered sets, namely
the full combinatorial simplices or cubes, each one embodied by the
ordered set of all its facets of all possible dimensions.

Contrarily to what happens with the functors $\alpha,\beta$, I feel
that for the two functors $\xi,\eta$ in opposite direction, not only
are they model preserving, but the right adjoint functors
$j_\Simplex,j_\Square$ must be model preserving too, and we will have to
come back upon this in a more general context.

We could amplify and unify somewhat the previous diagram of
modelizers, by introducing multicomplexes, which after all can be as
well ``mixed'' namely partly semisimplicial, partly cubical. Namely,
we may introduce the would-be ``test categories''
\[ \Simplex^p \times \Square^q = T_{p,q}\quad (p,q\in\bN, p+q\ge1)\]
giving rise to the category ${T_{p,q}}\uphat$ of
$(p,q)$-multicomplexes ($p$ times simplicial, $q$ times cubical). We
have a natural functor (generalizing the functors
\eqref{eq:28.starstarstar})
\[ T_{p,q} \to \Ord {( \hookrightarrow \Cat)},\]
associating to a system of $p$ standard simplices and $q$ standard
cubes (of variable dimensions), the \emph{product} of the $p+q$
associated ordered sets. We get this way a functor
\[\alpha_{p,q} : \Cat \to {T_{p,q}}\uphat\]
which\pspage{33} presumably (as I readily felt yesterday, cf.\
first lines p.~\ref{p:24})
is not any less model-preserving than the functors
$\alpha,\beta$ it generalizes. Of course, taking $A=T_{p,q}$ above, we
equally get a natural functor
\[i_{p,q} : {T_{p,q}}\uphat \to \Cat \]
admitting a right adjoint $j_{p,q}$, and both functors I feel must be
model preserving.

% 29
\hangsection[The basic modelizer \Cat.  Provisional definition of test
\dots]{The basic modelizer \texorpdfstring{\Cat}{(Cat)}.  Provisional
  definition of test categories and elementary
  modelizers.}\label{sec:29}%
It is time now to elaborate a little upon the notion of a test
category, within the context of modelizers. Let $A$ be a small
category, and consider the functor \eqref{eq:29.starstar}
\begin{equation}
  \label{eq:29.starstar}
  i_A : \Ahat \to \Cat, \quad F \mapsto A_{/F}.\tag{**}
\end{equation}
Whenever we have a functor $i: M \to M'$, when $M'$ is equipped with a
$W'$ turning it into a modelizer, there is (if any) just one $W
\subset \Fl(M)$ turning $M$ into a modelizer and $i$ into a morphism
of such, namely $W = i_{\Fl{}}^{-1}(W')$. In any case, we may define
$W$ (``weak equivalences'') by this formula, and get a functor
\[ W^{-1}M \to W'^{-1}M',\]
which is an equivalence if{f} $(M,W)$ is indeed a modelizer and $i$
model preserving. We may say shortly that $i:M\to M'$ is model
preserving, even without any $W$ given beforehand. Now coming back to
the situation \eqref{eq:29.starstar}, the understanding yesterday was
to call $A$ a \emph{test category}, to express that the canonical
functor \eqref{eq:29.starstar} is model preserving. (In any case,
unless otherwise specified by the context, we will refer to arrows in
$\Ahat$ which are transformed into weak equivalences of \Cat\ as
``\emph{weak equivalences}''.) It may well turn out, by the way, that
we will have to restrict somewhat still the notion of a test category.

In any case, the basic modelizer, in this whole approach to homotopy
models, is by no means the category \Sssets\ (however handy) or the
category \Spaces\ (however appealing to topological intuition), but
the category \Cat\ of ``all'' (small) categories. In this setup, the
category \Hot\ is most suitably defined as the category of fractions
of \Cat\ with respect to ``weak equivalences''. These in turn are most
suitably defined in cohomological terms, via the corresponding notion
for topoi -- namely a morphism of topoi
\[ f : X \to X' \]
is a ``weak equivalence'' or homotopy equivalence, if{f} for every
locally constant sheaf $F'$ on $X'$, the maps
\[ \mathrm H^i(X',F') \to \mathrm H^i(X, f^{-1}(F') ) \]
are isomorphisms whenever defined -- namely for $i=0$, for $i=1$ if
moreover $F'$ is endowed with a group structure, and for any $i$ if
$F'$ is moreover commutative (criterion of Artin-Mazur). Accordingly,
a functor\pspage{34}
\[f : A \to A'\]
between small categories (or categories which are essentially small,
namely equivalent to small categories) is called a weak equivalence,
if{f} the corresponding morphism of topoi
\[ f\uphat : \Ahat \to {A'}\uphat \]
is a weak equivalence.

Coming back to test categories $A$, which allow us to construct the
corresponding modelizers $\Ahat$, our point of view here is
rather that the test categories are each just as good as the others,
and $\Simplex$ just as good as $\Square$ or any of the $T_{p,q}$ and not
any better! Maybe it's the one though which turns out the most
economical for computational use, the nerve functor $\Cat \to
\Simplexhat$ being still the neatest known of all model preserving
embeddings of \Cat\ into categories $\Ahat$ defined by
modelizers. Another point, still more important it seems to me, is
that the natural functor
\[\Topoi \to \Pro\Hot\]
defined by the \v Cech-Cartier-Verdier process, and which allows for
another description of weak equivalences of topoi, namely as those
made invertible by this functor, are directly defined via
semi-simplicial structures of simplicial structures (of the type
``nerve of a covering'').

Modelizers of the type $\Ahat$, with $A$ a test category, surely
deserve a name -- let's call them \emph{elementary} modelizers, as
they correspond to the case of an ``elementary'' or ``trivial''
algebraic structure species, whose set-theoretical realizations can be
expressed as just \emph{any} functors
\[ A\op \to \Sets, \]
without any exactness condition of any kind; in other words they can
be viewed as just diagrams of sets of a specified type, with specified
commutativity relations. A somewhat more ambitious question maybe is
whether on such a category $M=\Ahat$, namely an elementary
modelizer, there cannot exist any other modelizing structure. In any
case, the one we got is intrinsically determined in terms of $M$,
which is a topos, by the prescription that an arrow $f: a \to b$
within $M$ is a weak equivalence if and only if the corresponding
morphism for the induced topoi $M_{/a}$ and $M_{/b}$ is a weak
equivalence (in terms of the Artin-Mazur criterion above,
see p.~\ref{p:33}).

A more crucial question I feel is whether the right adjoint functor
$j_A$ to $i_A$ in \eqref{eq:29.starstar}
(cf.\ p.~\ref{p:33}) is equally
model preserving, whenever $A$ is a test category. This, as we have
seen, is \emph{not} automatic, whenever we have a model preserving
functor between modelizers, whenever this functor admits an adjoint
functor. In more general terms still, let
\[
\begin{tikzcd}
  M \ar[r, bend left, "i"] & M' \ar[l, bend left, "j"] 
\end{tikzcd}\]
be a pair of adjoint functors, with $M,M'$ endowed with a
``saturated''\pspage{35} set of arrows $W,W'$. Then the following
are equivalent:
\begin{enumerate}[label=(\alph*)]
\item\label{it:29.a}
  $i(W)\subset W'$, $j(W')\subset W$, and the two corresponding
  functors
  \[  W^{-1}M \rightleftarrows W'^{-1}M' \]
  are quasi-inverse of each other, the adjunction morphisms between
  them being deduced from the corresponding adjunction morphisms for
  the pair $(i,j)$.
\item\label{it:29.b}
  $W=i^{-1}(W')$, and for every $a'\in M'$, the adjunction morphism
  \[ ij(a') \to a'\]
  is in $W'$.
\item[(b')]\label{it:29.bprime}
  dual to \ref{it:29.b}, with roles of $M$ and $M'$ reversed.
\end{enumerate}
In the situation we are interested in here, $M = \Ahat$ and
$M'=\Cat$, we know already $W=i^{-1}(W')$ by definition, and hence all
have to see is whether for any category $B$, the functor
\begin{equation}
  \label{eq:29.T}
  i_Aj_A(B) \to B \tag{T}
\end{equation}
is a weak equivalence. This alone will imply that not only $i_{W,W'}$,
but equally $j_{W',W}$ is an equivalence, and that the two are
quasi-inverse of each other. (NB\enspace even without assuming beforehand that
$W,W'$ are saturated, \ref{it:29.b} (say) implies \ref{it:29.bprime}
and \ref{it:29.a}, provided we assume on $W'$ the very mild saturation
condition that for composable arrows $u',v'$, if two among
$u',v',v'u'$ are in $W'$, so is the third; if we suppose moreover that
$M'$ is actually saturated namely made up with all arrows made
invertible by $M' \to W'^{-1}M'$, then condition \ref{it:29.b} implies
that $M$ is saturated too -- which ensures that if $(M',W')$ is
modelizing, so is $(M,W)$.)

It is not clear to me whether for every test category $A$, the
stronger condition \eqref{eq:29.T} above is necessarily
satisfied. This condition essentially means that for any homotopy
type, defined in terms of an arbitrary element $B$ in \Cat\ namely a
category $B$, we get a description of this homotopy type by an object
of the elementary modelizer $\Ahat$, by merely taking
$j_A(B)$. This condition seems sufficiently appealing to me, for
reinforcing accordingly the notion of a test-category $A$, and of an
elementary modelizer $\Ahat$, in case it should turn out to be
actually stronger. Of course, any category equivalent to an elementary
modelizer $\Ahat$ will be equally called by the same name. It
should appear in due course whether this is indeed the better suited
notion. One point in its favor already is that it appears a lot more
concrete.

Another natural question, suggested by the use of simplicial or
cubical multicomplexes, is whether the product of two test categories
is again a test category -- which might furnish us with a way to
compare directly the description of \Hot\ by the associated elementary
modelizers, without\pspage{36} having to make a detour by the
``basic'' modelizer \Cat\ we started with. But here it becomes about
time to try and leave the thin air conjecturing, and find some simple
and concrete characterization of test categories, or possibly some
reinforcement still of that notion, which will imply stability under
the product operation.

Here it is tempting to use semi-simplicial techniques though, by lack
of independent foundations of homotopy theory in terms of the
modelizer \Cat. Thus we may want to take the map of semi-simplicial
sets corresponding to \eqref{eq:29.T} when passing to nerves, and
express that a)\enspace this is a fibration and b)\enspace the fibers
of this fibration are ``contractible'' (or ``aspheric''), which
together will imply that we have a weak equivalence in \Sssets. Or we
may follow the suggestion of Quillen's set-up, working heuristically
in \Cat\ as though we actually know it is a model category, and
expressing that the adjunction morphism in \eqref{eq:29.T}, which is a
functor between categories, is actually a ``fibration'', and that its
fibers are ``contractible'' namely weakly equivalent to a one-point
category. In any case, a minimum amount of technique seems needed
here, to give the necessary clues for pursuing.

\bigbreak
\presectionfill\ondate{14.3.}\par

% 30
\hangsection{Starting the ``asphericity game''.}\label{sec:30}%
Since last week when I stopped with my notes, I got involved a bit
with recalling to mind the ``Lego-Teichm\"uller construction game''
for describing in a concrete, ``visual'' way the Teichm\"uller groups
of all possible types and the main relationships between them, which I
had first met with last year. This and other non-mathematical
occupations left little time only for my reflections on homotopy
theory, which I took up mainly last night and today. The focus of
attention was the ``technical point'' of getting a handy
characterization of test categories. The situation I feel is beginning
to clarify somewhat. Last thing I did before reading last weeks' notes
and getting back to the typewriter, was to get rid of a delusion which
I was dragging along more or less from the beginning of these notes,
namely that our basic modelizer \Cat, which we were using as a giving
the most natural definition of \Hot\ in our setting, was a ``model
category'' in the sense of Quillen, more specifically a ``closed model
category'', where the ``weak equivalences'' are the homotopy
equivalences of course, and where cofibrations are just monomorphisms
(namely functors injective on objects and on arrows) -- fibrations
being defined in terms of these by the Serre-Quillen lifting
property. Without even being so demanding, it turns out still that
there is no reasonable structure of a model category on \Cat, having
the correct weak equivalences, and such that the standard ``Kan
inclusions'' of the following two ordered sets\pspage{37}
\[
\begin{tikzcd}[row sep=small,column sep=tiny]
  & b & \\ a\ar[ur]\ar[rr] & & c
\end{tikzcd} \quad\text{and}\quad
\begin{tikzcd}[row sep=small,column sep=tiny]
  & b\ar[dr] & \\ a\ar[rr] & & c
\end{tikzcd}\]
into
\[\Simplex_2 = a \to b \to c\]
be cofibering. Namely, for a category $bC$, to say that it is
``fibering'' (over the final category $\bullet$) with respect to one
or the other monomorphism, means respectively that every arrow in
$\bC$ has a left respectively a right inverse -- the two together mean
that $\bC$ is a groupoid. But groupoids are definitely \emph{not}
sufficient for describing arbitrary homotopy types, they give rise
only to sums of $K(\pi,1)$ spaces -- thus contradicting Quillen's
statement that homotopy types can be described by ``models'' which are
both fibering and cofibering!

The feeling however remains that any elementary modelizer, namely one
defined (up to equivalence) by a test category, should be a closed
model category in Quillen's sense -- I find it hard to believe that
this should be a special feature just of semi-simplicial complexes!

While trying to understand test categories, the notion of
\emph{asphericity} for a morphism of topoi
\[ f:X\to Y\]
came in naturally -- this is a natural variant of the notion of
$n$-acyclicity (concerned with commutative cohomology) which has been
developed in the context of \'etale topology of schemes in SGA~4. It
can be expressed by ``localizing upon $Y$'' the Artin-Mazur condition
that $f$ be a weak equivalence, by demanding that the same remain true
for the induced morphism of topoi
\[ X \times_Y Y' \to Y' \]
for any ``localization morphism'' $Y'\to Y$. In terms of the
categories of sheaves $E,F$ on $X,Y$, $Y'$ can be defined by an object
(equally called $Y'$) of $F$, the category of sheaves on $Y'$ being
$F_{/Y'}$, and the fiber-product $X'$ can be defined likewise by an
object of $E$, namely by $X'=f^*(Y')$, hence the corresponding
category of sheaves is $E_{/f^*(Y')}$. In case the functor $f^*$
associated to $f$ admits a left adjoint $f_!$ (namely if it commutes
to arbitrary inverse limits, not only to finite ones), the category
$E_{/f^*(Y')}$ can be interpreted conveniently as $E_{f_!/Y'}$ (or
simply $E_{/Y'}$ if $f_!$ is implicit), whose objects are pairs
\[ (U,\varphi), \quad U\in\Ob E, \quad \varphi : f_!(U) \to Y',\]
with obvious ``maps'' between such objects. For the time being I am
mainly interested in the case of a morphism of topoi defined by a
functor between categories, which I will denote by the same symbol
$f$:
\[ f: C' \to C \quad\text{defines $f$ or $f\uphat : {C'}\uphat
  \to C\uphat$.}\]
Using\pspage{38} the fact that in the general definition of
asphericity it is enough to take $Y'$ in a family of generators of the
topos $Y$, and using here the generating subcategory $C$ of $C\uphat$,
we get the following criterion for asphericity of $f\uphat$: it
is necessary and sufficient that for every $a \in \Ob C$, the induced
morphism of topoi
\[ {C'}\uphat_{/a} \equeq ({C'_{/a}})\uphat \to C\uphat_{/a}
\simeq (C_/a)\uphat\]
be a weak equivalence, i.e., that the natural functor $C'_{/a}\to
C_{/a}$ be a weak equivalence. But it is immediate that $C_{/a}$ is
``contractible'', i.e., ``aspheric'', namely the ``map'' from $C_{/a}$
to the final category is a weak equivalence (this is true for any
category having a final object). Therefore we get the following

\noindent{\bfseries Criterion of asphericity for a
  functor}\phantomlabel{lem:asphericitycriterion}{asphericity criterion}
$f: C'\to C$ between categories: namely it is n.\ and s.\ that for any
$a\in\Ob C$, $C'_{/a}$ be aspheric.

Let's come back to a category $A$, for which we want to express that
it is a test-category, namely for any category $B$, the natural
functor
\begin{equation}
  \label{eq:30.star}
  i_Aj_A(B) \to B \tag{*}
\end{equation}
is a weak equivalence. One immediately checks that for any $b\in\Ob
B$, the category $i_Aj_A(B)_{/b}$ over $B_{/b}$ is isomorphic to
$i_Aj_A(B_{/b})$. Hence we get the following

\noindent\textbf{Proposition.} For a category $A$, the following are
equivalent:
\begin{enumerate}[label=(\roman*)]
\item\label{it:30.i}
  $A$ is a test category, namely \eqref{eq:30.star} is a weak
  equivalence for any category $B$.
\item\label{it:30.ii}
  \eqref{eq:30.star} is aspheric for any category $B$.
\item\label{it:30.iii}
  $i_Aj_A(B)$ is aspheric for any category $B$ with final element.
\end{enumerate}

This latter condition, which is the most ``concrete'' one so far,
means also that the element
\[ F = j_A(B) = (a \mapsto \Hom(A_{/a}, B)) \in \Ahat\]
is an aspheric element of the topos $\Ahat$, namely the induced
topos $\Ahat_{/F}$ is aspheric (i.e., the category $A_{/F}$ is
aspheric), \emph{whenever} $B$ has a final element.

% 31
\hangsection[The end of the thin air conjecturing: a criterion for
  test \dots]{The end of the thin air conjecturing: a criterion for
  test categories.}\label{sec:31}%
For the notion of a test category to make at all sense, we should make
sure in the long last that $\Simplex$ itself, the category of standard
simplices, is indeed a test category. So I finally set out to prove at
least that much, using the few reflexes I have in semi-simplicial
homotopy theory. A proof finally peeled out it seems, giving clues for
handy conditions in the general case, which should be
\emph{sufficient} at least to ensure that $B$ is a test category, but
maybe not quite necessary. I'll try now to get it down explicitly.

Here\pspage{39} are the conditions I got:
\begin{enumerate}[label=(T~\arabic*)]
\item\label{it:31.T1}
  $A$ is aspheric.
\item\label{it:31.T2}
  For $a,b\in\Ob A$, $A_{/{a\times b}}$ is aspheric (NB\enspace $a\times
  b$ need not exist in $A$ but it is in any case well defined as an
  element of $\Ahat$).
\item\label{it:31.T3}
  There exists an \emph{aspheric} element $I$ of $\Ahat$, and
  two subobjects $e_0$ and $e_1$ of $I$ which are final elements of
  $\Ahat$, such that $e_0 \sand e_1 (\eqdef e_0 \times_I e_1) =
  \varnothing_{\Ahat}$, the initial or ``empty'' element of
  $\Ahat$.
\end{enumerate}

In case when $A=\Simplex$ (as well as in the cubical analogon
$\Square$), I took $I=\Simplex_1$ which is an element of $A$ itself, and
moreover $A$ has a final element (which is a final element of
$\Ahat$ therefore) $e$, thus $e_0$ and $e_1$ defined by well
defined arrows in $A$ itself, namely $\delta_0$ and $\delta_1$. But it
does not seem that these special features are really relevant. In any
case \emph{intuitively $I$ stands for the unit interval, with
  endpoints $e_0,e_1$}. If $F$ is any element in $\Ahat$, the
standard way for trying to prove it is aspheric would be to prove that
we can find a ``constant map'' $F \to F$, namely one which factors
into
\[ F \to e \xrightarrow c F\quad\text{($e$ the final element of $A$)}\]
for suitable $c : e \to F$ or ``section'' of $F$, which be
``homotopic'' to the identity map $F \to F$. When trying to make
explicit the notion of a ``homotopy'' $h$ between two such maps, more
generally between two maps $f_0,f_1 : F \rightrightarrows G$, we hit
of course upon the arrow $h$ in the following diagram, which should
make it commutative
\begin{equation}
  \label{eq:31.D}
  \begin{tikzcd}
    & F \times I \ar[dd, dashed, swap, "h"] & \\
    F \simeq F \times e_0 \ar[ur, hook] \ar[dr, swap, "f_0"] & &
    F \simeq F \times e_1 \ar[ul, hook] \ar[dl, "f_1"] \\
    & G &
  \end{tikzcd}.
  \tag{D}
\end{equation}
This notion of a homotopy is defined in any category $\mathscr A$
where we've got an element $I$ and two subobjects $e_0,e_1$ which are
final objects. Suppose we got such $h$, and we know moreover (for
given $F,G,f_0,f_1,h$) that the two inclusions of $F \times e_0$ and
$F\times e_1$ into $F \times I$ are weak equivalences, and that $f_0$
is a weak equivalence (for a given set of arrows called ``weak
equivalences'', for instance defined in terms of a ``topology'' on
$\mathscr A$, in the present case the canonical topology of the topos
$\Ahat$), then it follows (with the usual ``mild saturation
condition'' on the notion of weak equivalence) that $h$, and hence
$f_1$ are weak equivalences. Coming back to the case
$F=G, f_0=\id_F, f_1=$ ``constant map'' defined by a $c : e \to F$, we
get that this constant map $f_1$ is a weak equivalence. Does this
imply that $F \to e$ is equally a weak equivalence? This is not quite
formal for general $(\mathscr A,W)$, but it is true though in the case
$\mathscr A = \Ahat$ and with the usual meaning of
``weak\pspage{40} equivalence'', in this case it is true indeed
that if we have a situation of inclusion with retraction $E \to F$ and
$F \to E$ ($E$ need not be a final element of $\mathscr A$), such that
the compositum $p : F \to E \to F$ (a projector in $F$) is a weak
equivalence, then so are $E \to F$ and $F \to E$. To check this, we
are reduced to checking the corresponding statement in \Cat, in fact
we can check it in the more general situation with two topoi $E$ and
$F$, using the Artin-Mazur criterion. (We get first that $E\to F$ is a
weak equivalence, and hence by saturation that $F\to E$ is too.)

Thus the assumptions made on $F\in\Ob{\Ahat}$ imply that $F \to
e$ is a weak equivalence, i.e., $A_{/F} \to A$ is a weak equivalence,
and if we assume now that $A$ satisfies \ref{it:31.T1} namely $A$ is
aspheric, so is $F$.

We apply this to the case $F = j_A(B) = (a \mapsto \Hom(A_{/a},B))$,
where $B$ is a category with final element. We have to check (using
\ref{it:31.T1} to \ref{it:31.T3}):
\begin{enumerate}[label=(\alph*)]
\item\label{it:31.a}
  The inclusions of $F\times e_0, F\times e_1$ into $F\times I$
  are weak equivalences (this will be true in fact for any
  $F\in\Ob{\Ahat}$),
\item\label{it:31.b}
  there exists a ``homotopy'' $h$ making commutative the previous
  diagram \eqref{eq:31.D}, where $G=F$, $f_0=\id_F$, and where $f_1:
  F\to F$ is the ``constant map'' defined by the section $c : e\to F$
  of $F$, associating to every $a\in A$ the \emph{constant} functor
  $e_{a,B} : A_{/a}\to B$ with value $e_B$ (a fixed final element of
  $B$), thus $e_{a,B}\in F(a) = \Hom(A_{/a},B)$ (and it is clear that
  his is ``functorial in $a$'').
\end{enumerate}
Then \ref{it:31.a} and \ref{it:31.b} will imply that $F$ is aspheric
-- hence $A$ is a test-category by the criterion \ref{it:30.iii} of
the proposition above.

To check \ref{it:31.b} we do not make use of \ref{it:31.T1} nor
\ref{it:31.T2}, nor of the asphericity of $I$. We have to define a
``map''
\[ h : F \times I \to F, \]
i.e., for every $a \in \Ob A$, a map (functorial for variable $a$)
\[ h(a) : \Hom(A_{/a},B) \times \Hom(a,I) \to \Hom(A_{/a},B)\]
(two of the $\Hom$'s are in \Cat, the other is in $\Ahat$). Thus,
let
\[ f:A_{/a}\to B, \quad u: a\to I,\]
we must define
\[ h(a)(f,u) : A_{/a} \to B \]
a functor from $A_{/a}$ to $B$, depending on the choice of $f$ and
$u$. Now let, for any $u\in\Hom(a,I)$, $u:a\to I$, $a_u$ be defined in
$\Ahat$ as
\[a_u = u^{-1}(e_0) = (a,u)\times_I e_0,\]
viewed as a subobject of $a$, and hence $C_u=A_{/a_u}$ can be viewed as a
subcategory of $C=A_{/a}$, namely the full subcategory of those objects
$x$ over $a$, i.e., arrows $x\to a\in A$, which factor through $a_u$
namely such that the compositum $x\to a\to I$ factors through
$e_0$. This subcategory is clearly\pspage{41} a ``crible'' in
$A_{/a}$, namely for an arrow $y\to x$ in $A_{/a}$, if the target $x$
is in the subcategory, so is the source $y$. This being so, we define
the functor
\[ f' = h(f,u) : A_{/a}=C \to B\]
by the conditions that
\begin{align*}
  f' &| (A_{/a_u} = C_u) = f | C_u\\
  f' &| (C \setminus C_u) = \text{constant functor with value $e_B$.}
\end{align*}
(where $C\setminus C_u$ denotes the obvious \emph{full} subcategory of
$C$, complementary to $C_u$). This defines $f'$ uniquely, on the
objects first, and on the arrows too because the only arrows left in
$C$ where we got still to define $f'$ are arrows $x\to y$ with $x$ in
$C_u$ and $y\in C\setminus C_u$ (because $C_u$ is a crible), but then
$f'(y)=e_B$ and we have no choice for $f'(x)\to f'(y)$! It's trivial
checking that this way we get indeed a \emph{functor} $f':C \to B$,
thus the map $h(a)$ is defined -- and that this map is functorial with
respect to $a$, i.e., comes from a map $h : F\times I \to F$ as we
wanted. The commutativity of \eqref{eq:31.D} is easily checked: for
the left triangle, i.e., that the compositum $F\simeq F\times e_0 \to
F\times I \xrightarrow h F$ is the identity, it comes from the fact
that if $u$ factors through $e_0$, then $C_u=C$ hence $f'=f$; for the
right triangle, it comes from the fact that if $u$ factors through
$e_1$, then $C_u=\varnothing$ (here we use the assumption $e_0 \sand e_1
= \varnothing_{\Ahat}$), and hence $f'$ is the constant functor $C
\to B$ with value $e_B$. This settles \ref{it:31.b}.

We have still to check \ref{it:31.a}, namely that for any
$F\in\Ob{\Ahat}$, the inclusion of the objects $F\times e_i$ into
$F\times I$ are weak equivalences, or what amounts to the same, that
the projection
\[ F\times I \to F\]
is a weak equivalence -- this will be true in fact for any object $I$
of $\Ahat$ which is aspheric. Indeed, we will prove the stronger
result that $F \times I\to F$ is \emph{aspheric}, i.e., that the
functor
\[A_{/F\times I} \to A_{/F} \]
is aspheric, We use for this the criterion of
p.~\ref{p:38}, which here
translates into the condition that for any $a$ in $A$ (such that we
got an $a\to F$, i.e., such that $F(a)\ne\emptyset$, but never mind),
the category $A_{/a\times I}$ is aspheric, i.e., the lement $a\times
I$ of $\Ahat$ is aspheric. Again, \emph{as $I$ is aspheric}, we
are reduced to checking that $a\times I \to I$ is aspheric, which by
the same argument (with $F,I$ replaces by $I,a$) boils down to the
condition that $A_{x\times b}$ is aspheric for any $b$ in $A$. Now
this is just condition \ref{it:31.T2}, we are through.

\bigbreak

\presectionfill\ondate{15.3.}\pspage{42}\par

% 32
\hangsection{Provisional program of work.}\label{sec:32}%
I definitely have the feeling to be out of the thin air -- the
conditions \ref{it:31.T1} to \ref{it:31.T3} look to me so elegant and
convincing, that I have no doubts left they are ``the right ones''! A
lot of things come to mind what to do next, I'll have to look at them
one by one though. Let me make a quick provisional planning.
\begin{enumerate}[label=\arabic*)]
\item\label{it:32.1}
  Have a closer look at the conditions \ref{it:31.T1} to
  \ref{it:31.T3}, to see how far they are necessary for $A$ to be a
  test category in the (admittedly provisional) sense I gave to this
  notion last week and yesterday, and to pin down the feeling of these
  being just the right ones.
\item\label{it:32.2}
  Use these conditions for constructing lots of test categories,
  including all the simplicial and cubical types which have been used
  so far.
\item\label{it:32.3}
  Check that these conditions are stable under taking the product
  of two or more test categories, and possibly use this fact for
  comparing the homotopy theories defined by any two such categories.
\item\label{it:32.4}
  Look up (using \ref{it:31.T1} to \ref{it:31.T3}) if an
  elementary modelizer $\Ahat$ is indeed a ``closed model
  category'' in Quillen's sense, and maybe too get a better feeling of
  how far apart \Cat\ is from being a closed model category. Visibly
  there \emph{are} some natural constructions in homotopy theory which
  do make sense in \Cat.
\item\label{it:32.5}
  Using the understanding of test categories obtained, come back
  to the question of which categories $\Ahat$ associated to
  algebraic structure species can be viewed as modelizers, and to the
  question of unicity or canonicity of the modelizing structure $W
  \subset \Fl(M)$.
\end{enumerate}

That makes a lot of questions to look at, and the theory of stacks I
set out to sketch seems to be fading ever more into the background! It
is likely though that a better general understanding of the manifold
constructions of the category \Hot\ of homotopy types will not be
quite useless, when getting back to the initial program, namely
stacks. Quite possibly too, on my way I will have to remind myself of
and look up, in the present setting, the main structural properties of
\Hot, including exactness properties and generators and
cogenerators. This also reminds me of an intriguing foundational
question since the introduction of derived categories and their
non-commutative analogs, which I believe has never been settled yet,
namely the following:
\begin{enumerate}[label=\arabic*),resume]
\item\label{it:32.6}
  In an attempt to grasp the main natural structures associated to
  derived categories, and Quillen's non-commutative analogons
  including \Hot, try to develop a comprehensive notion of a
  ``triangulated category'', without the known drawbacks of Verdier's
  provisional notion.\pspage{43}
\end{enumerate}

% 33
\hangsection[Necessity of conditions T1 to T3, and transcription in
\dots]{Necessity of conditions T1 to T3, and transcription in terms of
  elementary modelizers.}\label{sec:33}%
For the time being, I'll use the word ``test category'' with the
meaning of last week, and refer to categories satisfying the
conditions \ref{it:31.T1} to \ref{it:31.T3} as \emph{strict} test
categories. (NB\enspace $A$ is supposed to be essentially small in any
case.)

First of all, the conditions \ref{it:31.T1} and \ref{it:31.T3} are
\emph{necessary} for $A$ to be a test category. For \ref{it:31.T1}
this is just the ``concrete'' criterion \ref{it:30.iii} of yesterday
(page 38), when $B$ is the final element of \Cat. For \ref{it:31.T3},
we get even a \emph{canonical choice} for $I,e_0,e_1$, namely starting
with the ``universal'' choice in \Cat:
\[ \cst I = \Simplex_1, \quad \parbox[t]{0.5\textwidth}%
{$\cst e_0$ and $\cst e_1$ the two subobjects of $\Simplex_1$ in \Cat\
  (or in \Ord) corresponding to the two unique sections of $\cst I$
  over the final element $\cst e$ of \Cat,}\]
i.e., in terms of $\cst I$ as an ordered set $\{0\} \to \{1\}$, $\cst
e_0$ and $\cst e_1$ are just the two subobjects defined by the two
vertices $\{0\}$ and $\{1\}$ (viewed as defining two one-point ordered
subsets of $\cst I=\Simplex_1$). We now apply $j_A$ to get
\[ I = j_A(\cst I) = j_A(\Simplex_1), \quad e_i = j_A(\cst e_i) \quad
\text{for $i=0,1$.} \]
As $\cst I\to \cst e$ is a weak equivalence so is $I\to e_{\Ahat}
= j_A(\cst e)$, and hence (as $e_{\Ahat}$ is aspheric by
\ref{it:31.T1}) $I$ is aspheric. As $\cst e_0 \sand \cst e_1 =
\varnothing_\Cat$ and $j_A$ commutes with inverse limits, and with sums
(and in particular transforms the initial element $\varnothing$ of \Cat\
in the initial element $\varnothing$ of $\Ahat$), it follows that
$e_0 \sand e_1 = \varnothing$.

It's worthwhile having a look at what this object just constructed is
like. For this end, let's note first that for any category $C$, we
have a canonical bijection, functorial in $C$
\[ \Hom(C , \Simplex_1) \tosim \Crib(C) \simeq \text{set of all
  subobjects of $e_{C\uphat}$,}\]
by associating to any functor $f : C \to \Simplex_1$ the full
subcategory $f^{-1}(\cst e_0)$ of $C$, which clearly is a ``crible''
in $C$. Thus we get, for $a\in\Ob A$
\[ I(a) \simeq \Crib(A_{/a}) \quad \text{(subobjects of $a$ in
  $\Ahat$).}\]
More generally, we deduce from this, for any $F\in\Ob{\Ahat}$:
\[ \Hom(F,I) = \Gamma(I_F = I\times F/F) = \text{set of all subobjects
  of $F$,} \]
in other words, the object $I$ is defined intrinsically in the topos
$\Ahat$ (up to unique isomorphism) as the ``\emph{Lawvere
  element}'' representing the functor $F \mapsto$ set of subobjects of
$F$, generalizing the functor $F \mapsto \mathfrak P(F)$ in the
category of sets (namely sheaves over the one-point topos),
represented by the two-point set $\{0,1\}$. The condition $e_0\sand e_1
=\varnothing_\scrA$ is automatic in any topos \scrA, provided \scrA\ is
not the ``empty topos'', namely corresponding to a category of sheaves
equivalent to a one-point category.

There\pspage{44} is a strong temptation now to diverge from test
categories, to expand on intrinsic versions of conditions
\ref{it:31.T1} to \ref{it:31.T3} for any topos, and extract a notion
of a (strictly) modelizing topos, generalizing the ``elementary
modelizers'' $\Ahat$ defined by strict test categories $A$. But
it seems more to the point for the time being to look more closely to
the one condition, namely \ref{it:31.T2}, which does not appear so far
as necessary for $A$ being a test category -- and I suspect it is
\emph{not} necessary indeed, as no idea occurred to me how to deduce
it. (I have no idea of how to make a counterexample though, as I don't
see any other way to check a category $A$ is a test category, except
precisely using yesterday's criterion via \ref{it:31.T1} to
\ref{it:31.T3}.) Still, I want to emphasize about the fact that
\ref{it:31.T2} is indeed a very natural condition. In this connection,
it is timely to remember that in the category \Hot, finite direct and
inverse limits exist (and even infinite ones, I guess, but I feel I'll
have to be a little careful about these\ldots). The existence of such
limits, in terms of the description of \Hot as a category of fractions
of \Cat, doesn't seem at all a trivial fact, for the time being I'll
admit it as ``well known'' (from the semisimplicial set-up, say), and
probably come back upon this with some care later. Now if $(M,W)$ is
any modelizer, hence endowed with a localization functor
\[ M \to \Hot,\]
it is surely not irrelevant to ask about which limits this functor
commutes with, and study the case with care. Thus in no practical
example I know of does this functor commute with binary amalgamated
sums or with fibered products without an extra condition on at least
one of the two arrows involved in $M$ -- a condition of the type that
one of these is a monomorphism or a cofibration (for co-products), or
a fibration (for products). However, in all cases known, it seems that
the functor commutes to (finite, say) sums and products. For sums, it
really seems hard to make a sense out of a localizing functor $M\to
\Hot$, namely playing a ``model'' game, without the functor commuting
at the very least to these! In this respect, it is reassuring to
notice that for any category $A$, the associated functors $i_A,j_A$
between $\Ahat$ and \Cat\ do indeed both commute with (arbitrary)
sums -- which of course is trivial anyhow for $i_A$ (commuting to
arbitrary direct limits), and easily checked for $j_A$ (which
apparently does not commute to any other type of direct limits, but of
course commutes with arbitrary inverse limits). Now for products too,
it is current use to look at products of ``models'' for homotopy
types, as models for the product type -- so much so that this fact is
surely tacitly used everywhere, without any feeling of a need to
comment. It seems too that any category $M$ which one has looked at
so far for possible use as a category of ``models'' in one sense or
other\pspage{45} for homotopy types, for instance set-theoretic
realizations of some specified algebraic structure species, or
topological spaces, and the like, do admit arbitrary direct and
inverse limits, and surely sums and products therefore, so that the
question of commutation of the localization functor to these arises
indeed and is felt to be important. Possibly so important even, that
the notion I introduced of a ``modelizer'' should take this into
account, and be strengthened to the effect that the canonical functor
$M\to\Hot$ should commute at least to finite sums and products, and
possibly even to infinite ones (whether the latter will have to be
looked up with care). I'll admit provisionally that in \Hot, finite
sums and products can be described in terms of the corresponding
operations in \Cat, namely that the canonical functor (going with our
very definition of \Hot\ as a localization of \Cat)
\[\Cat \to \Hot\]
commutes with finite (presumably even infinite) sums and
products. This is indeed reasonably, in view of the fact that the
Nerve functor
\[\Cat \to \Simplexhat = \Sssets\]
does commute to sums and products.

In the case of a test category $A$ and the corresponding elementary
modelizer $\Ahat$, the corresponding localizing functor is the
compositum
\[ \Ahat \xrightarrow{i_A} \Cat\to\Hot,\]
which therefore commutes with finite (presumably even infinite) sums
automatically, because $i_A$ does. Commutation with finite
\emph{products} though does not look automatic. The property of
commutation with final elements is OK and is nothing but condition
\ref{it:31.T1}, which we saw is necessary for $A$ to be a test
category. Thus remains the question of commutation with binary
products, which boils down to the following condition, for any two
elements $F$ and $G$ in $\Ahat$:
\[ i_A(F\times G) \to i_A(F) \times i_A(G) \quad\text{should be a weak
  equivalence,}\]
i.e.,
\begin{equation}
  \label{eq:33.star}
  A_{/F\times G} \to A_{/F} \times A_{/G} \quad\text{a weak equivalence.}
  \tag{*}
\end{equation}
This now implies condition \ref{it:31.T2}, as we see taking $F=a$,
$G=b$, in which case the condition \eqref{eq:33.star} just means
asphericity of $a\times b$ in $\Ahat$, namely \ref{it:31.T2}. To
be happy, we have still to show that conversely, \ref{it:31.T2}
implies \eqref{eq:33.star} for any $F,G$. As usual, it implies even
the stronger condition that the functor in question is aspheric, which
by the standard criterion (page 38) just means that the categories
$A_{/a\times b}$ (for $a,b$ in $A$, and such moreover that $F(a)$ and
$G(b)$ non-empty, but never mind) are aspheric.

Everything turns out just perfect -- it seems worthwhile to summarize
it in one theorem:\pspage{46}

\begin{theorem}
Let $A$ be an essentially small category,
and consider the composed functor
\[ m_A : \Ahat \xrightarrow{i_A} \Cat \to \Hot,\]
where $i_A(F)=A_{/F}$ for any $F\in\Ob{\Ahat}$, and where
$\Cat\to\Hot$ is the canonical functor from \Cat\ into the localized
category with respect to weak equivalences. The following conditions
are equivalent:
\begin{enumerate}[label=(\roman*),font=\normalfont]
\item\label{it:33.i}
  The functor $m_A$ commutes with finite products, and is a
  localization functor, i.e., induces an equivalence $W_A^{-1}\Ahat
  \toequ\Hot$, where $W_A$ is the set of weak equivalences in
  $\Ahat$, namely arrows transformed into weak equivalences by
  $i_A$.\footnote{\alsoondate{9.4.} Presumably, we'll have to add that $j_A=i_A^*$
    is model preserving, cf.\ remark~\ref{rem:65.D.3} on p.~\ref{p:174}.}
\item\label{it:33.ii}
  The functor $i_A$ and its right adjoint $j_A$ define functors
  between $W_A^{-1}\Ahat$ and $\Hot=W^{-1}\Cat$ \textup{(}thus $j_A$
  should transform weak equivalences of \Cat\ into weak equivalences
  of $\Ahat$\textup{)} which are \emph{quasi-inverse} to each other, the
  adjunction morphisms for this pair being deduced from the adjunction
  morphisms for the pair $i_A,j_A$. Moreover, the functor $m_A$
  commutes with finite products.
\item\label{it:33.iii}
  The category $A$ satisfies the conditions \textup{\ref{it:31.T1}} to
  \textup{\ref{it:31.T3}} \textup{(}page 39\textup{)}.
\end{enumerate}
\end{theorem}

I could go on with two or three more equivalent conditions, which
could be expressed intrinsically in terms of the topos $\Ahat$
and make sense (and are equivalent) for any topos, along the lines of
the reflections of p.~\ref{p:43} and of yesterday. But
I'll refrain for the time being!

In the proof of the theorem above, I did not make use of
semi-simplicial techniques nor of any known results about \Hot, with
the only exception of the assumption (a fact, I daresay, but not
proved for the time being in the present framework, without reference
to semi-simplicial theory say) that the canonical functor
$\Cat\to\Hot$ commutes with binary products. We could have avoided
this assumption, by slightly changing the statement of the theorem,
the condition that $m_A$ commute with finite products in \ref{it:33.i}
and \ref{it:33.ii} being replaced by the assumption that $i_A$ commute
with finite products ``up to weak equivalence'', as made explicit in
\eqref{eq:33.star} above for the case of binary products (which are
enough of course).

I suspect that the notion of a test category in the initial, wider
sense will be of no use any longer, and therefore I will reserve this
name\pspage{47} to the \emph{strict} case henceforth, namely to the
case of categories satisfying the equivalent conditions of the theorem
above. Accordingly, I'll call ``\emph{elementary modelizer}'' any
category \scrA\ equivalent to a category $\Ahat$, with $A$ a test
category. Such a category will be always considered as a modelizer, of
course, with the usual notion of weak equivalence $W \subset
\Fl(\scrA)$, namely of a ``map'' $F \to G$ in \scrA\ such that the
corresponding morphism for the induced topoi is a weak
equivalence. The category \scrA\ is an elementary modelizer,
therefore, if{f} it satisfies the following conditions:
\begin{enumerate}[label=\alph*)]
\item\label{it:33.a}
  \scrA\ is equivalent to a category $\Ahat$ with $A$ small,
  which amount to saying that \scrA\ is a topos, and has sufficiently
  many ``essential points'', namely ``points'' such that the
  corresponding fiber-functor $\scrA \to \Sets$ commutes with
  arbitrary products -- i.e., there exists a conservative family of
  functors $\scrA\to\Sets$ which commute to arbitrary direct and
  inverse limits. (Cf.\ SGA~4 IV~7.5.)
\item\label{it:33.b}
  The pair $(\scrA,W)$ is modelizing (where $W \subset \Fl(\scrA)$ is
  the set of weak equivalences), i.e., the category $W^{-1}\scrA$ is
  equivalent to \Hot.
\item\label{it:33.c}
  The canonical functor $\scrA\to\Hot$ (or, equivalently, $\scrA\to
  W^{-1}\scrA$) commutes with finite products (or, equivalently, with
  binary products -- that it commutes with final elements follows
  already from \ref{it:33.a}, \ref{it:33.b}).
\end{enumerate}
As I felt insistently since yesterday, there is a very pretty notion
of a ``modelizing topos'' generalizing the notion of an elementary
modelizer, where \scrA\ is a topos but not necessarily equivalent to
one of the type $\Ahat$ -- but where however the aspheric (=
``contractible'') objects form a generating family (which generalizes
the condition $\scrA\simeq\Ahat$, which means that the
$0$-connected projective elements of \scrA\ form a generating
family). I'll come back upon this notion later I guess -- it is not
the most urgent thing for the time being\ldots

% 34
\hangsection{Examples of test categories.}\label{sec:34}%
I will now exploit the handy criterion \ref{it:31.T1} to
\ref{it:31.T3} for test categories, for constructing lots of such. In
all cases I have in mind at present, the verification of
\ref{it:31.T1} and \ref{it:31.T3} is obvious, they are consequences
indeed of the following stronger conditions:
\begin{enumerate}[label=(T~\arabic*')]
\item\label{it:34.T1prime}
  $A$ has a final element $e_A$.
  \addtocounter{enumi}{1}
\item\label{it:34.T3prime}
  There exists an element $I = I_A$ in $A$, and two sections
  $e_0,e_1$ of $I$ over $e=e_A$
  \[ \delta_0,\delta_1 : e \to I,\]
  such that the corresponding subobjects $e_0,e_1$ of $I$ satisfy
  \[ e_0 \sand e_1 = \varnothing_{\Ahat},\]
  namely that for any $a\in\Ob A$, if $p_a : a \to e$ is the
  projection, we have
  \[ \delta_0 p_a \ne \delta_1p_a.\]
\end{enumerate}

Of\pspage{48} course, the element $I$ (playing the role of unit interval with
endpoints $e_0,e_1$) is by no means unique, for instance it can be
replaced by any cartesian power $I^n$ ($n\ge1$), provided it is in $A$
-- or in the case of $A = \Simplex$ we can take for $I$ any $\Simplex_n$
($n\ge1$) and for $\delta_0,\delta_1$ any two \emph{distinct} maps
from $e=\Simplex_0$ to $\Simplex_n$, instead of the usual choice
$\Simplex_1$ for $I$, and corresponding $\delta_0$ and $\delta_1$. This
high degree of arbitrariness in the choice of $I$ should be no
surprise, this was already one striking feature in Quillen's theory of
model categories.

There remains the asphericity condition \ref{it:31.T2} for the
categories $A_{/a\times b}$, with $a,b\in\Ob A$, which is somewhat
subtler. There is one very evident way though to ensure this, namely
assuming
\begin{enumerate}[label=(T~\arabic*')]
  \addtocounter{enumi}{1}
\item\label{it:34.T2prime}
  If $a,b$ are in $A$, so is $a\times b$, i.e., $A$ is stable under
  binary products.
\end{enumerate}

In other words, putting together \ref{it:34.T1prime} and
\ref{it:34.T2prime}, we may take categories (essentially small)
\emph{stable under finite products}. When such a category satisfies
the mild extra condition \ref{it:34.T3prime} above, it is a test
category! This is already an impressive bunch of test categories. For
instance, take any category $C$ with finite products and for which
there exist $I,e_0,e_1$ as in \ref{it:34.T3prime} -- never mind
whether $C$ is essentially small, for instance any ``non-empty'' topos
will do (taking for $I$ the Lawvere element for instance, if no
simpler choice comes to mind). Take any subcategory $A$ (full or not)
stable under finite products and containing $I$ and
$\Simplex_0,\Simplex_1$ and where the $C$-products are also $A$-products
-- for instance this is OK if $A$ is full. Then $A$ is a test
category.

The simplest choice here, the smallest in any case, it to take the
subcategory made up with all cartesian powers $I^n$ ($n\ge0$). We may
take the full subcategory made up with these elements, but instead we
may take, still more economically, only the arrows
\[ I^n \to I^m\]
where $m$ components $I^n\to I$ are each, either a projection $\pr_i$
($1\le i\le n$) or of the form $\delta_i(p_{I^n})$ ($i\in\{0,1\}$).
The category thus obtained, up to unique isomorphism, does not depend
on the choice of $(C,I,e_0,e_1)$, visibly -- it is, in a sense,
\emph{the} smallest test category satisfying the stronger conditions
\ref{it:34.T1prime} to \ref{it:34.T3prime}. Its elements can be
visualized as ``standard cubes'': One convenient way to do so is to
take $C = \Ord$ (category of all ordered sets), $I = \Simplex_1 =
(\{0\}\to\{1\})$, $e_0$ and $e_1$ as usual, thus $A$ can be
interpreted as a subcategory of \Ord, but this embedding is not full
(there are maps of $I^2$ to $I$ in \Ord\ which are not in $A$, i.e.,
do not ``respect the cubical structure''). We may also take
$C=\Spaces$, $I=$ unit interval $\subset\bR$, $e_0$ and $e_1$ defined
by the endpoints $0$ and $1$,\pspage{49} thus the elements of $A$
are interpreted as the standard cubes $I^n$ in $\bR^n$; the allowable
maps between them $I^n\to I^m$ are those whose components $I^n\to I$
are either constant with value $\in\{0,1\}$, or one of the projections
$\pr_i$ ($1\le i\le n$). We may denote this category of standard cubes
by $\square$, but recall that the cubes here have symmetry operations,
they give rise to the notion of ``(fully) cubical complexes''
$(K_n)_{n\ge0}$, in contrast with what might be called ``semicubical
complexes'' without symmetry operations on the $K_n$'s, in analogy
with the case of simplicial complexes, where there are likewise
variants ``full'' and ``semi''. To make the distinction, we better
call the corresponding test category (with symmetry operations, hence
with more arrows) $\widetilde\square$ rather than $\square$, and
accordingly for $\widetilde\Simplex$, $\Simplex$. On closer inspection, it
seems to me that even apart the symmetry operations, the maps between
standard cubes allowed here are rather plethoric -- thus we admit
diagonal maps such as $I\to I\times I$, which is I guess highly
unusual in the cubical game. It is forced upon us though if we insist
on a test category stable under finite products, for easier checking
of condition \ref{it:31.T2}.

A less plethoric looking choice really is
\[ A = \text{\emph{non-empty} finite sets, with arbitrary maps}\]
or, equivalently, the full subcategory $\widetilde\Simplex$ formed by
the standard finite sets $\bN\sand{[0,n]}$, which we may call
$\widetilde\Simplex_n$ in contrast to the $\Simplex_n$'s (viewed as being
endowed with their natural total order, and with correspondingly more
restricted maps). This gives rise to the category $\widetilde\Simplex$
of ``(fully) simplicial complexes (or sets)''. The elements of $A$, or
of $\widetilde\Simplex$, can be interpreted in the well-known way as
\emph{affine simplices}, and simplicial maps between these.

It is about time now to come to the categories $\Simplex$ and $\square$
and check they are test categories, although they definitely do not
satisfy \ref{it:34.T2prime}, thus we are left with checking the
somewhat delicate condition \ref{it:31.T2}. It is tempting to dismiss
the question by saying that it is ``well-known'' that the categories
\[\Simplex_{/\Simplex_n\times\Simplex_m}\quad(n,m\in\bN)\]
are aspheric -- but this I feel would be kind of cheating. Maybe the
very intuitive homotopy argument already used yesterday (pages 39--40)
will do. In general terms, under the assumptions \ref{it:31.T1},
\ref{it:31.T3}, we found a sufficient criterion of asphericity for an
element $F$ of $\Ahat$, which we may want to apply to the case
$F=a\times b$, with $a$ and $b$ in $A$.\pspage{50}

Now let's say that $e_{\Ahat}$ is a \emph{deformation retract of
  $F$} ($F$ any element in $\Ahat$) if the identity map of $F$ is
homotopic (with respect to $I$) to a ``constant'' map of $F$ into
itself. It is purely formal, using the diagonal map $I\to I\times I$
in $\Ahat$, that if (more generally) $F_0$ is a deformation
retract of $F$, and $G_0$ a deformation retract of $G$, then
$F_0\times G_0$ is a deformation retract of $F\times G$. Thus, a
\emph{sufficient} condition for \ref{it:31.T2} to hold is the
following ``homotopy-test axiom'':

\noindent\parbox[t]{0.3\textwidth}{\textbf{Condition}
  \namedlabel{cond:TH}{(T~H)}:\\[\baselineskip]%
  \hspace*{3em}(1)\phantomlabel{cond:TH1}{(T~H~1)}}%
\parbox[t]{0.7\textwidth}{For any $a\in A$, $e=e_{\Ahat}$ is a
  deformation retract of $a$ with respect to $(I,e_0,e_1)$, namely
  there exists a section $c_a : e\to a$ (hence a constant map $u_a =
  p_ac_a : a\to a$) and a homotopy $h_a:a\times I\to a$ from $\id_a$
  to $u_a$, i.e., a map $h_a$ such that
  \[h_a(\id_a \times \delta_0) = \id_a , \quad
  h_a(\id_a\times\delta_1) = u_a.\]}

We have to be cautious though, it occurs to me now, not to make a
``vicious circle'', as the homotopy argument used yesterday for
proving asphericity of $f$ makes use of the fact that $F\times I\to F$
is a weak equivalence. To check that $F\times I\to F$ is a weak
equivalence, and even aspheric, we have seen though that it amounts to
the same to prove $a\times I\to a$ is a weak equivalence for any $a$
in $A$, i.e., $a\times I$ is aspheric for any $a$ in $A$. This, then,
was seen to be a consequence of the assumption that all elements
$a\times b$ are aspheric (for $a,b$ in $A$) -- but this latter fact is
now what we want to prove! Thus it can't be helped, we have to
complement the condition \ref{cond:TH} above (which I will call
\ref{cond:TH1}, by the extra condition\phantomlabel{cond:TH2}{(T~H~2)}
\begin{equation}
  \label{eq:34.2}
  \text{For any $a$ in $A$, $a\times I$ is aspheric,}
  \tag{2}
\end{equation}
which, in case $I$ itself is in $A$, appears as just a particular case
of \ref{it:31.T2}, which now, it seems, we have to check directly some
way or other.

\bigbreak

\presectionfill\ondate{19.3.}\par

% 35
\hangsection[The notion of a modelizing topos. Need for revising the
\dots]{The notion of a modelizing topos. Need for revising the
  \v Cech-Verdier-Artin-Mazur construction.}\label{sec:35}%
I really feel I should not wait any longer with the digression on
modelizing topoi that keeps creeping into my mind, and which I keep
trying to dismiss as not to the point or not urgent or what not! At
least this will fix some terminology, and put the notion of an
elementary modelizer into perspective, in terms of a wider class of
topoi.

A topos $X$ is called \emph{aspheric} if the canonical ``map'' from
$X$ to the ``final topos'' (corresponding to a one-point space, and
whose category of sheaves is the category \Sets) is aspheric. It is
equivalent to say that $X$ is $0$-connected (namely non-empty, and not
decomposable non-trivially into a direct sum of two topoi), and that
for any constant sheaf of groups $G$ on\pspage{51} $X$, the
cohomology group $\mathrm H^i(X,G)$ ($i\ge1$) is trivial whenever defined,
namely for $i\ge1$ if $G$ commutative, and $i=1$ if $G$ is not
supposed commutative. The $0$-connectivity is readily translated into
a corresponding property of the final element in the category of
sheaves $\Sh(X)=\scrA$ on $X$. An element $U$ of \scrA\ is called
\emph{aspheric} if the induced topos $X/U$ or $\scrA_{/U}$ is
aspheric. An arrow $f : U' \to U$ in \scrA\ is called a \emph{weak
  equivalence} resp.\ \emph{aspheric}, if the corresponding ``map'' or
morphism of the induced topoi
\[ X/U'\to X/U \quad\text{or}\quad \scrA_{/U'}\to\scrA_{/U} \]
has the corresponding property. Of course, in the notations it is
often convenient simply to identify objects $U$ of \scrA\ with the
induced topos, and accordingly for arrows. Clearly, if $f$ is aspheric,
it is a weak equivalence, more specifically, $f$ is aspheric if{f} it
is ``universally a weak equivalence'', namely if{f} for any base
change $V\to U$ in \scrA, the corresponding map $g: V'=V\times_U U'\to
V$ is a weak equivalence. Moreover, it is sufficient to check this
property when $V$ is a member of a given generating family of \scrA.

An interesting special case is the one when \scrA\ admits a family of
aspheric generators, or equivalently a generating full subcategory $A$
whose objects are aspheric in \scrA; we will say in this case that the
topos $X$ (or \scrA) is ``\emph{locally aspheric}''. This condition is
satisfied for the most common topological spaces (it suffices that
each point admit a fundamental system of contractible neighborhoods),
as well as for the topoi $\Ahat$ associated to (essentially
small) categories $A$ (for such a topos, $A$ itself is such a
generating full category made up with aspheric elements of \scrA). In
the case of a locally aspheric topos endowed with a generating full
subcategory $A$ as above, a map $f:U'\to U$ is aspheric if{f} for any
$a\in\Ob A$, and map $a \to U$, the object $U' \times_U a$ is aspheric
(because for a map $g: V'\to V$ with $V$ aspheric, $g$ is a weak
equivalence if{f} $V'$ is aspheric, which is a particular case of the
corresponding statement valid for any map of topoi).

Using this criterion, we get readily the following

\begin{propositionnum}\label{prop:35.1}
  Let \scrA\ be a locally aspheric topos, $A$ a generating full
  subcategory made up with aspheric elements. The following conditions
  on \scrA\ are equivalent:
  \begin{description}[leftmargin=2em]
  \item[\namedlabel{it:35.a}{a)}]
    Any aspheric element of \scrA\ is aspheric over $e$ (the final
    object).
  \item[\namedlabel{it:35.b}{b)}]
    The product of any two aspheric elements of \scrA\ is again
    aspheric.
  \item[\namedlabel{it:35.aprime}{a')}, \namedlabel{it:35.bprime}{b')}]
    as {\normalfont\ref{it:35.a}}, {\normalfont\ref{it:35.b}} but with elements restricted to be
    in $\Ob A$.
  \end{description}
\end{propositionnum}

These conditions, in case $\scrA = \Ahat$ are nothing but
\ref{it:31.T2} for $A$. \emph{It turns out they imply already}
\ref{it:31.T1}. More generally, let's call a topos\pspage{52}
\emph{totally aspheric} if is locally aspheric, and satisfies more the
equivalent conditions above. It then turns out:
\begin{corollary}
  Any totally aspheric topos is aspheric.
\end{corollary}

This follows readily from the definition of asphericity, and the
\v{C}ech computation of cohomology of $X$ in terms of a family
$(a_i)_{i\in I}$ covering $e$, with the $a_i$ in $A$. As the mutual
products and multiple products of the $a_i$'s are all aspheric (a
fortiori acyclic for any constant coefficients), the \v Cech
calculation is valid (including of course in the case of
non-commutative $\mathrm H^1$) and yields the desired result.

The condition for a topos to be totally aspheric, in contrast to
\emph{local} asphericity, is an exceedingly strong one. If for
instance the topos is defined by a topological space, which I'll
denote by $X$, then it is seen immediately that any two non-empty open
subsets of $X$ have a non-empty intersection, in other words $X$ is an
\emph{irreducible space} (i.e., not empty and not the union of two
closed subsets distinct from $X$). It is well known on the other hand
that any irreducible space is aspheric, and clearly any non-empty open
subset of an irreducible space is again irreducible. Therefore:
\setcounter{corollarynum}{1}
\begin{corollarynum}\label{cor:35.1.2}
  A topological space is totally aspheric \textup{(}i.e., the
  corresponding topos is t.a.\textup{)} if{f} it is irreducible. In
  case $X$ is Hausdorff, means also that $X$ is a one-point space.
\end{corollarynum}

Let's now translate \ref{it:31.T3} in the context of general topoi. We
get:

\begin{propositionnum}\label{prop:35.2}
  Let $X$ be a topos. Then the following two conditions are
  equivalent:
  \begin{enumerate}[label=\alph*),font=\normalfont]
  \item\label{it:35.P2.a}
    The ``Lawvere element'' $L_X$ of $\scrA = \Sh(X)$ is aspheric
    over the final object $e_X=e$.
  \item\label{it:35.P2.b}
    There exists an object $I$ in \scrA\ which is aspheric over
    $e$, and two sections $\delta_0,\delta_1$ of $I$ \textup(over
    $e$\textup), such that $\Ker(\delta_0,\delta_1)=\varnothing$, i.e.,
    such that $e_0 \sand e_1 = \varnothing$, where $e_0,e_1$ are the
    subobjects of $I$ defined by $\delta_0,\delta_1$.
  \end{enumerate}
\end{propositionnum}

I recall that the Lawvere element is the one which represents the
functor
\[ F \mapsto \text{set of all subobjects of $F$}\]
on \scrA, it is endowed with two sections over $e$, corresponding to
the two ``trivial'' subobjects $\varnothing, e$ of $e$, and the kernel
of this pair of sections is clearly $\varnothing$, thus \ref{it:35.P2.a}
$\Rightarrow$ \ref{it:35.P2.b}. Conversely, by the homotopy argument
already used (p.~\ref{p:39}--\ref{p:40}),
we readily get that \ref{it:35.P2.b}
implies that\pspage{53} the projection $L\to e$ is a weak
equivalence. As the assumption \ref{it:35.P2.b} is stable by
localization, this shows that for any $U$ in \scrA, $L_U \to U$ is
equally a weak equivalence, and hence $L$ is aspheric over $e$.

In proposition~\ref{prop:35.2} we did not make any assumption of the
topos $X$ being locally aspheric, let alone totally aspheric. The
property of total asphericity, and the one of prop.~\ref{prop:35.2} about
the existence of a handy substitute for the unit interval, seem to be
independent of each other. The case of topological spaces is
instructive in this respect. Namely:
\setcounter{corollarynum}{0}
\begin{corollarynum}\label{cor:35.2.1}
  Assume $X$ is a topological space, and $X$ is $0$-connected. Then
  the Lawvere element $L_X$ in $\scrA=\Sh(X)$ is $0$-connected, except
  exactly when $X$ is irreducible, in which case $L_X$ decomposes into
  two connected components.
\end{corollarynum}

Indeed, $L$ is disconnected if{f} there is a direct summand $L'$ of
$L$ containing $e_0$ and $\ne L$, or equivalently, if for any open $U
\subset X$, and $U\to L$, namely an open subset $U_0$ of $U$, we can
associate a \emph{direct summand} $U'=L'(U,U_0)$ of $U$ containing
$U_0$, functorially for variable $U$, and such that
\[ L'(U,\emptyset) = \emptyset.\]
The functoriality in $U$ means that for $V$ open in $U$ and $V_0=U_0
\sand V$, we get
\begin{equation}
  \label{eq:35.star}
  L'(V,V_0) = V \sand L'(U,U_0).
  \tag{*} 
\end{equation}
This implies that $L'(U,U_0)$ is known when we know $L'(X,U_0)$,
namely
\[ L'(U,U_0) = U \sand L'(X,U_0).\]
As $L'(X,U_0)$ is a direct summand of $X$ containing $U_0$, and $X$ is
connected, we see that
\[ \text{$L'(X,U_0) = X$ if $U_0\ne\emptyset$,} \quad
L'(X,\emptyset)=\emptyset; \]
and hence
\begin{equation}
  \label{eq:35.starstar}
  \text{$L'(U,U_0) = U$ if $U_0\ne\emptyset$,} \quad
  \text{$L'(U,U_0) = \emptyset$ if $U_0=\emptyset$,}
  \tag{**}
\end{equation}
thus the association $(U,U_0) \mapsto L'(U,U_0)$ is uniquely
determined, and it remains to see whether the association
\eqref{eq:35.starstar} is indeed functorial, i.e., satisfies
\eqref{eq:35.star} for any open $V \subset U$, with $V_0=V\sand
U_0$. It is OK if $V_0\ne\emptyset$, or if $V_0=\emptyset$ and
$U_0=\emptyset$. If $V_0=\emptyset$ and $U_0\ne\emptyset$, it is OK
if{f} $V$ is empty, in other words, if both open subsets $U_0$ and $V$
of $U$ are non-empty, so must be their intersection $V_0$. But this
means that $X$ is irreducible.

\begin{corollarynum}\label{cor:35.2.2}
  A topological space $X$ cannot be totally aspheric (i.e.,
  irreducible) and satisfy the condition of prop.~\textup{\ref{prop:35.2}}.
\end{corollarynum}

Because for a connected topos, this latter condition implies that
$L_X$ is equally connected, which contradicts
corollary~\ref{cor:35.2.1}.

Thus, surely local asphericity for a topos \emph{does not} imply
the\pspage{54} condition of prop.~\ref{prop:35.2} -- the simplest
counterexample is the final topos, corresponding to one-point
spaces. But I don't expect either that condition of
prop.~\ref{prop:35.2}, even for a locally aspheric topos $X$, and even
granting $X$ is aspheric moreover, implies that $X$ is totally
aspheric. The positive result in cor.~\ref{cor:35.2.1} gives a hint
that the condition of prop.~\ref{prop:35.2} may be satisfied for locally
aspheric topological spaces which are sufficiently far away from those
awful non-separated spaces (including the irreducible ones) of the
algebraic geometry freaks -- possibly even provided only $X$ is
locally Hausdorff.

Now the conjunction of the conditions of prop.~\ref{prop:35.1} and
\ref{prop:35.2}, namely total asphericity plus existence of a
monomorphism of $e \amalg e$ into an aspheric element, seems an
extraordinarily strong assumption -- so strong indeed that no
topological space whatever can satisfy it! I feel like calling a topos
satisfying these conditions a ``\emph{modelizing topos}''. For a topos
of the type $\Ahat$, this just means it is an ``elementary
modelizer'', i.e., $A$ is a test category, which are notions which, I
feel, are about to be pretty well understood. I have not such feeling
yet for the more general situation though. For instance, there is
another kind of property of a topos, which from the very start of our
model story, one would have thought of pinpointing by the name of a
``modelizing topos''. Namely, when taking $W \subset\Fl{\scrA}$ to be
weak equivalences in the sense defined earlier, we demand that
$(\scrA,W)$ should be a modelizer in the general sense
(cf.\ p.~\ref{p:31}),
with some exactness reinforcement
(cf.\ p.~\ref{p:45}), namely that
$w^{-1}\scrA$ should be equivalent to \Hot, and $W$ saturated, namely
equal to the set of maps made invertible by the localization functor
-- and moreover I guess that this functor should commute with (at
least finite) sums and products. Of course, we would even expect a
little more, namely that the ``canonical functor''
\[\scrA \to \Hot\]
can be described, by associating to every $F\in\Ob{\scrA}$ the
(pro-)homotopy type associated to it by the Verdier-Artin-Mazur
process. One feels there is a pretty juicy bunch of intimately related
properties for a topos, all connected with the ``homotopy model
yoga'', and which one would like to know about.

This calls to my mind too the question of understanding the
Verdier-Artin-Mazur construction in the present setting, where
homotopy types are thought of in terms of categories as preferential
models, rather than semi-simplicial sets. This will be connected with
the question, which\pspage{55} has been intriguing me lately, of
understanding the ``structure'' of an arbitrary morphism of topoi
\[ X\to\Top(B),\]
where $\Top(B)$ is the topos associated to an arbitrary (essentially
small) category $B$ -- a situation, it seems, which generalizes the
situation of a ``fibered topos'' over the indexing category $B$, in
the sense of SGA~4 IV. The construction of Verdier in SGA~4 V
(appendix) corresponds apparently to the case of categories $B$ of the
type $\Simplex_{/F}$ with $F\in\Ob{\Simplexhat}$, as brought near by
the standard \v Cech procedure. When however the topos $X$ is already
pretty near a topos of the type $\Top(B)$, for instance if it is such
a topos, to describe its homotopy type in terms of a huge inverse
system of semi-simplicial objects of $B\uphat$, rather than just
take $B$ and keep it as it is, seems technically somewhat prohibitive,
at least in the present set-up (and with a distance of twenty years!).

% 36
\hangsection[Characterization of a particular type of test functors
  with \dots]{Characterization of a particular type of test functors
  with values in \texorpdfstring{\Cat}{(Cat)}.}\label{sec:36}%
Now back to test categories, and more specifically to her majesty
$\Simplex$ -- we still have to check it is a test category indeed! The
method suggested last Tuesday
(p.~\ref{p:50}) does work indeed, without any
reference to ``well-known'' facts from the semi-simplicial theory. To
check \ref{cond:TH1}, we take of course $I=\Simplex_1$ and $e_0,e_1$ as
usual, we then have for every $n\in\bN$ to construct a homotopy
\[ h_n : \Simplex_n \times \Simplex_1 \to \Simplex_n,\]
where the product and the arrow can be interpreted as one in \Cat\ or
even in \Ord\ of course (of which $\Simplex$ is a full subcategory), and
there is a unique such $h$ if we take for $u_a$ the ``constant''
endomorphism of $\Simplex_n$ whose value is the \emph{last} element of
$\Simplex_n$. There remains only \ref{cond:TH2} -- namely to prove that
the objects $\Simplex_n\times\Simplex_1$ in $\Simplex$ are aspheric
($n\ge0$). For this we cover $\Simplex_n'=\Simplex_n\times\Simplex_1$ by
maximal representable subobjects, namely maximal flags of the ordered
product set (these are I guess what are called the ``shuffles'' in
semi-simplicial algebra). It then turns out that $\Simplex_n'$ can be
obtained by successive gluing of flags, the intersection of each flag
we add with the sup or union of the preceding ones being just a
subflag, hence representable and aspheric. Thus we only have to make
use of the Mayer-Vietoris type easy lemma:
\begin{lemma}
  Let $U,V$ be two subobjects of the final object of a topos, assume
  $U,V$ and $U\sand V$ aspheric, then so is $e$, i.e., so is the topos.
\end{lemma}

This finishes the proof of $\Simplex$ being a test category. Of course,
one cannot help thinking of the asphericity of \emph{all} the products
$\Simplex_m\times\Simplex_n$ as just meaning asphericity of the product
ordered set (namely of the corresponding category), which follows for
instance from the fact that it\pspage{56} has a final object. But
we should beware the vicious circle, as we implicitly make the
assumption that for an object $C$ of \Cat\ (at least an object $C$
such as $\Simplex_m\times\Simplex_n$, if $C$ is aspheric, then so is the
corresponding element $\alpha(C)$ in $\Simplexhat$:
\[ \alpha(C) : \Simplex_n \mapsto \Hom(\Simplex_n,C),\]
namely the \emph{nerve} of $C$. We expect of course something stronger
to hold, namely that for any $C$, $\alpha(C)$ has the same homotopy
type as $C$ -- in fact that we have a canonical isomorphism between
the images of both in \Hot. This relationship has still to be
established, as well as the similar statement for $\beta : \Cat \to
\square\uphat$, and accordingly with $\widetilde\Simplex$ and
$\widetilde\square$. For all four categories (where the simplicial
cases are known to be test categories already), we have, together with
the category $A$ (the would-be test category), a functor
\[ A \to \Cat,\quad\text{say $a\mapsto\Simplex_a$}\]
(notation inspired by the case $A=\Simplex$), hence a functor
\[\Cat \xrightarrow\alpha \Ahat, \quad \alpha(C) = (a \mapsto
\Hom_{\Cat}(\Simplex_a, C)),\]
and the question is to compare the homotopy types of $C$ and
$\alpha(C)$, the latter being defined of course as the homotopy type
of $A_{/\alpha(C)} \simeq A_{/C}$, the category of all pairs $(a,p)$
with $a\in\Ob A$ and $p:\Simplex_a\to C$, arrows between pairs
corresponding to \emph{strictly} commutative diagrams in \Cat{}
\[\begin{tikzcd}[row sep=tiny,column sep=small]
  \Simplex_{a'} \ar[rr,"\Simplex_f"] \ar[dr, swap, "p'"] & & \Simplex_a \ar[dl,"p"] \\
  & C &
\end{tikzcd}.\]
\begin{proposition}
  Let the data be as just said, assume moreover that the categories
  $\Simplex_a$ \textup($a\in\Ob A$\textup) have final elements, that $A$
  is aspheric and that $\alpha(\Simplex_1)$ is aspheric over the final
  element of $\Ahat$ \textup(which is automatic if $A$ is a test
  category, $A\to\Cat$ is fully faithful and $\Simplex_1$ belongs to its
  essential image\textup). Then we can find for every $C$ in \Cat\ a
  "map" in \Cat\ \textup(i.e., a functor\textup)
  \begin{equation}
    \label{eq:36.star}
    \varphi : A_{/C} \to C,
    \tag{*}
  \end{equation}
  functorial in $C$ for variable $C$, and this map being aspheric, and
  a fortiori a weak equivalence \textup(hence induces an isomorphism between
  the homotopy types\textup).
\end{proposition}

This implies that the compositum $C \mapsto A_{/C}$:
\[\Cat \to \Ahat \xrightarrow{i_A} \Cat\]
transforms weak equivalences into weak equivalences, and that the
functor deduced from it by passage to the localized categories
$W^{-1}\Cat = \Hot$ is isomorphic to the identity functor of
\Hot.\pspage{57}

To define a functor \eqref{eq:36.star}, functorial in $C$ for variable
$C$, we only have to choose a final element $e_a$ in each
$\Simplex_a$. For an element $(a,p)$ of $A_{/C}$, we define
\[\varphi(a,p) = p(e_a),\]
with evident extension to arrows of $A_{/C}$ (NB\enspace Here it is important
that the $e_a$ be \emph{final} elements of the $\Simplex_a$, initial
elements for a change wouldn't do at all!), thus we get a functor
$\varphi$, and functoriality with respect to $C$ is clear. Using the
standard \hyperref[lem:asphericitycriterion]{criterion of asphericity
  of a functor} $\varphi$ we have to check that the categories
\[ (A_{/C})_{/x}\]
for $x\in\Ob C$ are aspheric, but one checks at once that the category
above is canonically isomorphic to $A_{/C'}$, where $C'=C_{/x}$. Thus
asphericity of $\varphi$ for arbitrary $C$ is equivalent to
asphericity of $A_{/C}$ when $C$ has a final element, or equivalently,
to asphericity of the element $\alpha(C)$ in $\Ahat$ for such
$C$. Now let $e_C$ be the final element of $C$, and consider the
unique homotopy
\[ h : \Simplex_1 \times C \to C\]
between $\id_C$ and the constant functor from $C$ to $C$ with value
$e_C$. Applying the functor $\alpha$, we get a corresponding homotopy
in $\Ahat$
\[ I \times \alpha(C) \to \alpha(C), \quad\text{where
  $I=\alpha(\Simplex_1)$,}\]
between the identity map of $F=\alpha(C)$ and a ``constant'' map $F\to
F$. As we assume that $I$ is aspherical over the final element of
$\Ahat$, hence $I\times F\to F$ is a weak equivalence, it follows
that $F\to e$ is a weak equivalence, hence $F$ is aspheric as $e$ is
supposed to be aspheric, which was to be proved.

If we apply this proposition to $A=\Simplex$ and $n\mapsto\Simplex_n$, we
are still reduced to check just \ref{cond:TH2} (which we did above
looking at shuffles), whereas \ref{cond:TH1} appears superfluous now
-- we get asphericity of the elements $\Simplex_m \times \Simplex_n$ in
$\Ahat$ as a consequence (but the homotopy argument used in the
prop.\ is essentially the same as the one used for proving
\ref{cond:TH1}). This proposition applies equally to the case when
$A=\widetilde\Simplex$ (category of non-ordered simplices
$\widetilde\Simplex_n$), here the natural functor
\[ i : \widetilde\Simplex \to \Cat, \quad a\mapsto\mathfrak P^*(a)\]
is obtained by associated to any finite set $a$ the combinatorial
simplex it defines, embodied by the ordered set $\mathfrak P^*(a)$ of
its facets of all dimensions, which can be identified with the ordered
set of all non-empty subsets of the finite set $a$. As we know already
$\widetilde\Simplex$ is a test category, all that remains to be done is
to check in this case that $I=\alpha(\Simplex_1)$ is aspheric (hence
aspheric over the final object of $\Ahat$). Now this follows
again from a homotopy argument involving the unit segment
substitute\pspage{58} in $\Ahat$. We could formalize it as
follows:
\begin{corollary}
  In the proposition above, the condition that $\alpha(\Simplex_1)$ be
  aspheric over $e_{\Ahat}$ is a consequence of the following
  assumptions:
  \begin{enumerate}[label=\alph*),font=\normalfont]
  \item\label{it:36.a}
    The condition \textup{\ref{it:31.T3}} is valid in $A$\kern1pt, with the
    stronger assumption that $I$ is aspheric over the final object $e$
    of $\Ahat$, and moreover:
  \item\label{it:36.b}
    for every $a\in\Ob A$ and $u:a\to I$, let $a_u=u^{-1}(e_0)$
    \textup(subobject of $a$ in $\Ahat$\textup), $\Simplex_{a_u} =
    \varinjlim_{\text{$b$ in $A_{/a_u}$}} \Simplex_b$, and $\Simplex_{a,u}
    = \Imm(\Simplex_{a_u} \to \Simplex_a)$. We assume that
    $\Ob{\Simplex_{a,u}}$ is a \emph{crible} in $\Simplex_a$, say $C_u$.
  \end{enumerate}
\end{corollary}

Indeed, interpreting $\alpha(\Simplex_1)$ as the functor $a \mapsto
\Crib{\Simplex_a}$, we define a homotopy
\[ h : I \times F \to F\quad\text{(where $F=\alpha(\Simplex_1)$)}\]
from the identity $\id_F$ to the constant map $c:F\to F$ associating
to every crible in some $\Simplex_a$ the empty crible in the same, by
taking for every $a$ in $A$ the map
\[ (u,C_0) \mapsto C_u \circ C_0.\]
It follows that $F\to e$ is a weak equivalence, and the same argument
in any induced category $A_{/a}$ shows that this is universally so,
i.e., $F\to e$ is aspheric, O.K.

In the case above, $A=\widetilde\Simplex$, we get indeed that for
$u:a\to I$, $a_u$ is empty in $A$ and $\Simplex_{a_u} \to \Simplex_a$ is a
full embedding, turning $\Simplex_{a_u}$ into a crible in $\Simplex_a$,
thus \ref{it:36.b} is satisfied (with the usual choice
$I=\widetilde\Simplex_1$ of course).

\bigbreak

\presectionfill\ondate{21.3.}\par

% 37
\hangsection{The ``asphericity story'' told anew -- the ``key
  result''.}\label{sec:37}%
By the end of the notes two days ago, there was the pretty strong
impression of repeating the same argument all over again, in very
similar situations. When I tried to pin down this feeling, the first thing
that occurred to me was that the homotopy equivalence I was after
$A_{/C}\to C$ (in the proposition of
p.~\ref{p:56}), when concerned with a
general functor $A\to\Cat$, and the criterion obtained, was applicable
to the situation I was concerned with at the very start when defined
test categories, namely when looking at the \emph{canonical} functor
$A\to\Cat$ given by $a\mapsto A_{/a}$; and that this gave a criterion
in this case for $A$ to be a test category in the wider sense, which I
had overlooked when peeling out the characterization of test
categories (cf.\ theorem of
p.~\ref{p:46}).Finally it becomes clear that it
is about time to recast from scratch the asphericity story, and (by
one more repetition) tell it anew in a way stripped from its
repetitive features!\pspage{59}

On a more technical level, I got aware too that the last corollary
stated was slightly incorrect, because it is by no means clear that
the crible $C_u$ constructed there is functorial in $u$, i.e.,
corresponds to a map $I\to\alpha(\Simplex_1)$, this in fact has still to
be assumed (and turns out to be satisfied in all cases which turned up
naturally so far and which I looked up). But let's now ``retell the story''!

One key notion visibly in the homotopy technique used, and which needs
a name in the long last, is the notion of a \emph{homotopy interval}
(``segment homotopique'' in French). To be really outspoken about the
very formal nature of this notion and the way it will be used, let's
develop it in any category \scrA\ endowed with a subset
$W\subset\Fl(\scrA)$ of the set of arrows of \scrA, and satisfying the
usual conditions:
\begin{enumerate}[label=\alph*)]
\item\label{it:37.a}
  $W$ contains all isomorphisms of \scrA,
\item\label{it:37.b}
  for two composable arrows $u,v$ in \scrA, if two among $u,v,vu$ are
  in $W$, so is the third, and
\item\label{it:37.c}
  if $i:F_0\to F$, and $r:F\to F_0$ is a left inverse (i.e., a
  retraction), and if $p=ir:F\to F$ is in $W$, so is $r$ (and hence
  $i$ too).
\end{enumerate}

The condition \ref{it:37.c} here is the ingredient slightly stronger
than what we used to call ``mild saturation property'' of $W$, meaning
\ref{it:37.a} and \ref{it:37.b}.

We'll call \emph{homotopy interval in \scrA} (with respect to the
notion of ``weak equivalence'' $W$, which will generally be implicit
and specified by context) a triple $(I,e_0,e_1)$, where $I$ is an
object of \scrA, $e_0$ and $e_1$ two subobjects, such that the
following conditions \ref{cond:HIa} to \ref{cond:HIc} hold:
\begin{enumerate}[label=\alph*)]
\item\label{cond:HIa}
  $e_0$ and $e_1$ are final objects of \scrA,
\end{enumerate}
which implies that \scrA\ has a final object, unique up to unique
isomorphism, which we'll denote by $e_\scrA$ or simply $e$, so that
the data $e_0,e_1$ in $I$ are equivalent to giving two sections
\[ \delta_0,\delta_1 : e \to I\]
of $I$ over $e$. Note that $e_0 \sand e_1 = \Ker(\delta_0,\delta_1)$.
\begin{enumerate}[label=\alph*),resume]
\item\label{cond:HIb}
  $e_0\sand e_1 = \varnothing_\scrA$ (a strict initial element of \scrA),
\item\label{cond:HIc}
  $I\to e$ is ``universally in $W$'' or, as we'll say, is
  \emph{$W$-aspheric} or simply aspheric,
\end{enumerate}
which just means here that $I\to e$ is ``squarable'' and that for any
$F$ in $\Ob\scrA$, $F\times I \to F$ is in $W$, i.e., is a ``weak
equivalence''.

It is clear that if $(I,e_0,e_1)$ is a homotopy interval in \scrA,
then for any $F\in\Ob\scrA$, the ``induced interval'' in $\scrA_{/F}$,
namely $(I\times F,e_0\times F,e_1\times F)$ is a homotopy interval in
$\scrA_{/F}$.\pspage{60}

We now get the (essentially trivial)
\begin{homotopylemma}\label{lem:homotopylemma}
  Let $h: F\times I\to F$ be a ``homotopy'' with respect to
  $(I,e_0,e_1)$ of $\id_F$ with a constant map $c = ir: F\to F$, where
  $r: F\to e$ and $i: e\to F$ is a section of $F$ over $e$. Then $F\to
  e$ is $W$-aspheric.
\end{homotopylemma}

In the proof of this lemma, we make use of
\ref{it:37.a}\ref{it:37.b}\ref{it:37.c} for $W$, but only
\ref{cond:HIa}\ref{cond:HIc} for $(I,e_0,e_1)$, namely we don't even
need $e_0\sand e_1=\varnothing_\scrA$. This is still the case for the
proof of the
\begin{comparisonlemmaforHI}\label{lem:comparisonlemmaforHI}
  Let $L$ be an object of \scrA, squarable over $e$ and endowed with a
  composition law $x \land y$, let $\delta_i^L:e \to L$ be two sections of
  $L$ \textup($i\in\{0,1\}$\textup), which are respectively a left
  unit and zero element for the multiplication, namely the
  corresponding elements $e_0^a,e_1^a$ in any $\Hom(a,L)$ satisfy
  \[ e_0^a \land x = x, \quad e_1^a\land x = e_1^a\]
  for any $x$ in $\Hom(a,L)$. Assume moreover we got a homotopy
  interval $(I,e_0,e_1)$ and a ``map of intervals''
  \[ \varphi: I \to L\]
  \textup(in the sense: compatible with endpoints\textup). Then
  $\id_L$ is homotopic \textup(with respect to $(I,e_0,e_1)$\textup)
  to the constant map $c_L : L\to L$ defined by $\delta_1^L$, and
  hence by the previous \hyperref[lem:homotopylemma]{homotopy lemma},
  $L\to e$ is $W$-aspheric and therefore $L$ endowed with
  $e_0^L,e_1^L$ defined by $\delta_0^L,\delta_1^L$ is itself a
  homotopy interval in \scrA\ \textup(provided, at least, we know that
  $e_0^L\sand e_1^L = \varnothing$\textup).
\end{comparisonlemmaforHI}

The homotopy $h$ is simply given (for $u:a\to I$, $x:a\to L$) by
\[ h(x,u) = \varphi(u) \land x;\]
when $u$ factors through $e_0$, then $\varphi(u) = e_0^a$ and hence
$h(x,u)=x$; if it factors through $e_1$, then $\varphi(u)=e_1^a$ and
we get $h(x,u)=e_1^a$, qed.

\begin{corollary}\label{cor:ofcomparisonlemmaforHI}
  Assume in \scrA\ \textup(endowed with $W$\textup) finite inverse limits
  exist \textup(i.e., final object and fibered products exist\textup),
  and moreover that the presheaf on \scrA
  \[F \mapsto \text{set of all subobjects of $F$}\]
  is representable by an element $L$ of \scrA\ \textup(the ``Lawvere
  element''\textup). Assume moreover \scrA\ has a strict initial element
  $\varnothing_\scrA$, i.e., an initial element and that any map
  $a\to\varnothing_\scrA$ is an isomorphism. Consider the two sections
  of $L$ over $e$, $\delta_0^L$ and $\delta_1^L$, corresponding to the
  full and to the empty subobject of $e$, so that we get visibly
  $e_0^L\sand e_0^L=\varnothing$. Then, for a homotopy interval to exist
  in \scrA, it is necessary and sufficient that $L$ be $W$-aspheric
  over $e$, i.e., that $(L,e_0^L,e_1^L)$ be a homotopy interval.
\end{corollary}

By the comparison lemma, it is enough to show that for any homotopy
interval $(I,e_0,e_1)$ in \scrA, there exists a morphism of
``intervals''\pspage{61} from $I$ into $L$, using the fact of
course that the intersection law on $L$ is a composition law admitting
$\delta_0^L,\delta_1^L$ respectively as unit and as zero element
(still using the fact that the initial object is strict). But the
subobject $e_0$ of $I$, by definition of $L$, can be viewed as the
inverse image of $e_0^L$ by a uniquely defined map $I\to L$. The
induced map $e_1\to L$ corresponds to the induced subobject $e_0 \sand
e_1$ of $e_1$ which by assumption is $\varnothing_\scrA$, and hence
$e_1\to L$ factors through $e_1^L$, qed.

Of course, the case for the time being which mainly interests us is
the one when \scrA\ is a topos (more specifically even, a topos
equivalent to a category $\Ahat$, with $A$ a small category), in
which case it is tacitly understood that $W$ is the set of weak
equivalences in the usual sense. We now come to the key result:
\begin{theorem}\label{thm:keyresult}
  Let $A$ be a small category, and
  \[ i : A \to \Cat\]
  a functor, hence a functor
  \[ i^* : \Cat\to\Ahat, \quad C\mapsto(a\mapsto \Hom(i(a),C)).\]
  Consider the canonical functor $i_A : \Ahat\to\Cat$, $F\mapsto
  A_{/F}$, and the compositum
  \[\Cat \xrightarrow{i^*} \Ahat\xrightarrow{i_A} \Cat, \quad
  C \mapsto A_{/i^*(C)} \eqdef A_{/C}.\]
  Assume that for any $a\in\Ob A$, $i(a)\in\Ob\Cat$ has a final
  element $e_a$ \textup(but we don't demand that for $u: a\to b$,
  $i(u): i(a)\to i(b)$ transforms $e_a$ into $e_b$ nor even into a
  final element of $i(b)$\textup). Consider the canonical functor
  \begin{equation}
    \label{eq:key.star}
    A_{/C}\to C, \quad (a, p:i(a)\to C) \mapsto p(e_a),\tag{*}
  \end{equation}
  which is functorial in $C$, and hence defines a map between functors
  from \Cat\ to \Cat:
  \begin{equation}
    \label{eq:key.starstar}
    i_Ai^* \to \id_\Cat.\tag{**}
  \end{equation}
  \begin{enumerate}[label=\alph*),font=\normalfont]
  \item\label{it:key.a}
    The following conditions are equivalent:
    \begin{enumerate}[label=(\roman*),font=\normalfont]
    \item\label{it:key.a.i}
      For any $C$ in \Cat, \textup{\eqref{eq:key.star}} is aspheric.
    \item\label{it:key.a.ii}
      For any $C$ in \Cat, \textup{\eqref{eq:key.star}} is a weak
      equivalence, i.e., $i_Ai^*$ transforms weak equivalences into
      weak equivalences, and \textup{\eqref{eq:key.starstar}} induces
      an isomorphism of the corresponding functor $\Hot\to\Hot$ with
      the identity functor.
    \item\label{it:key.a.iii}
      The functor $i_Ai^*$ transforms weak equivalences into weak
      equivalences, and the induced functor $\Hot\to\Hot$ transforms
      every object into an isomorphic one, i.e., for any $C$ in \Cat,
      $A_{/C}$ is isomorphic to $C$ in \Hot.
    \item\label{it:key.a.iv}
      For any $C$ with a final element, $A_{/C}$ is aspheric.
    \end{enumerate}
  \item\label{it:key.b}
    The following conditions are equivalent, \emph{and they imply the
      conditions in \textup{\ref{it:key.a}} provided $A$ is
      aspheric}:\pspage{62}
    \begin{enumerate}[label=(\roman*),font=\normalfont]
    \item\label{it:key.b.i}
      For any $C$ in $\Cat$ the functor
      \begin{equation}
        \label{eq:key.starstarstar}
        A_{/C} \to A\times C\quad\text{deduced from
          \textup{\eqref{eq:key.star}} and
          $A_{/C}\xrightarrow{\textup{can}} A$}
        \tag{***}
      \end{equation}
      is aspheric.
    \item\label{it:key.b.ii}
      For any $C$ in \Cat\ with final element and any $a$ in $A$\kern1pt,
      $a\times i^*(C)$ is an aspheric element in $\Ahat$, i.e.,
      for any such $C$, $i^*(C)$ is aspheric over the final element
      $e$ of $\Ahat$.
    \item\label{it:key.b.iii}
      The element $i^*(\Simplex_1)$ of $\Ahat$ is aspheric over the
      final object $e$.
    \end{enumerate}
  \end{enumerate}
\end{theorem}

\begin{remark}
  Of course the conditions in \ref{it:key.a} imply that $A$ is
  aspheric (take $C$ to be the final category), and hence by an easy
  lemma (of below, \S\ref{sec:40}) we get that $A\times C\to C$ is a
  weak equivalence (and even aspheric), and hence $A_{/C}\to A\times
  C$ is a weak equivalence (because its compositum with the weak
  equivalence $A\times C\to C$ is a weak equivalence by
  assumption). It is unlikely however that \ref{it:key.a} implies the
  conditions of \ref{it:key.b}, namely asphericity (not merely weak
  equivalence) of \eqref{eq:key.starstarstar} in
  b\ref{it:key.b.i}. But the opposite implication, namely
  b\ref{it:key.b.i} + asphericity of $A$ implies a\ref{it:key.a.i}, is
  trivial, because $A\times C\to C$ is aspheric.
\end{remark}

\begin{proof}[Proof of theorem] We stated \ref{it:key.a} and
  \ref{it:key.b} in a way to get a visibly decreasing cascade of
  conditions; and moreover that the weakest in \ref{it:key.a} implies
  the strongest, or that b\ref{it:key.b.ii} implies b\ref{it:key.b.i},
  is an immediate consequence of the standard \ref{lem:asphericitycriterion}
  for a functor between categories
  (p.~\ref{p:38}). The only point which is a
  little less formal is that b\ref{it:key.b.iii} implies
  b\ref{it:key.b.ii}. But using the final object in $C$, we get a
  (unique) homotopy in \Cat\ (relative to $\Simplex_1$)
  \[ \Simplex_1 \times C \to C,\]
  between the identity map of $C$ and the constant map with value
  $e_C$ -- hence by applying $i^*$, a homotopy relative to
  $i^*(\Simplex_1)$ (viewed as an ``interval'' by taking as
  ``endpoints'' the arrows deduced from
  $\delta_0,\delta_1 : e_\Cat=\Simplex_0\rightrightarrows \Simplex_1$ by
  applying $i^*$), between the identity map of $i^*(C)$ and a constant
  map of $i^*(C)$, which by the \hyperref[lem:homotopylemma]{homotopy
  lemma} implies that $i^*(C)$ itself is aspheric over
  $e_{\Ahat}$, qed.
\end{proof}
\addtocounter{remarknum}{1}
\begin{remarknum}
  The Condition b\ref{it:key.b.iii} can be stated by saying that
  $i^*(\Simplex_1)$ is a homotopy interval in $\Ahat$. All which
  needs to be checked for this, is that condition \ref{cond:HIb} for
  homotopy intervals (p.~\ref{p:59}) namely $e_0 \sand
  e_1 = \varnothing$ is satisfied in $\Ahat$, but this follows from
  the corresponding property for $\Simplex_1$ in \Cat\ (as the functor
  $i^*$ is left exact) and from the fact that $i^*$ transforms initial
  element $\varnothing_\Cat$ into initial element $\varnothing_{\Ahat}$.
  This last fact is equivalent to $i(a)\ne\varnothing$ for any
  $a$ in $A$, which is true as $i(a)$ has a final element.
\end{remarknum}

\bigbreak

\presectionfill\ondate{22.3.}\pspage{63}\par

% 38
\hangsection{Asphericity story retold
  \texorpdfstring{\textup(}{(}cont'd\texorpdfstring{\textup)}{)}:
  generalized nerve functors.}\label{sec:38}%
Let's get back to the ``asphericity story retold'' -- I had to stop
yesterday just in the middle, as it was getting prohibitively late.

I want to comment a little about the ``\hyperref[thm:keyresult]{key
  result}'' just stated and proved. The main point of this result,
forgetting the game of givings heaps of equivalent formulations of two
kinds of properties, is that the extremely simple condition
b\ref{it:key.b.iii}, namely that $i^*(\Simplex_1)$ is aspheric over the
unit element $e_\Ahat$ of \Ahat, plus asphericity of the latter,
ensure already the conditions in \ref{it:key.a}, which can be viewed
(among others) as just stating that the compositum
\[ i_Ai^* : \Cat\to\Ahat\to\Cat\]
from \Cat\ to \Cat\ induces an autoequivalence of the localized
category \Hot, or (what amounts still to the same) a functor
$\Hot\to\Hot$ isomorphic to the identity functor. (NB\enspace that two do
indeed amount to the same follows at once from the implication
\ref{it:key.a.iv} $\Rightarrow$ \ref{it:key.a.ii} in \ref{it:key.a}.)
It is interesting to note that both properties, the stronger one that
$i^*(\Simplex_1)$ is aspheric over $e_\Ahat$, and the weaker one in
terms of properties of the compositum $i_Ai^*$, make a sense without
any reference to the extra assumption that the categories $i(a)$ have
a final element each, nor to the corresponding map \eqref{eq:key.star}
$A_{/C}\to C$ (functorial in $C$).

This suggests that there should exist a more general statement than in
the theorem, without making the assumption about final objects in the
categories $i(a)$, and without the possibility of a \emph{direct}
comparison of $A_{/C}$ and $C$ through a functor between them. Indeed
I have an idea of a statement in this respect, however for the time
being it seems that the theorem as stated is sufficiently general for
handling the situations I have in mind.

Applying the theorem to the \emph{canonical} functor $i$
\[i_0 : a\mapsto A_{/a} : A \to \Cat,\]
whose canonical extension to \Ahat\ (as a functor $\Ahat\to\Cat$
commuting to direct limits) is the functor
\[i_A: F \mapsto A_{/F} : \Ahat\to\Cat,\]
giving rise to the right adjoint
\[i_0^* = j_A : \Cat\to\Ahat,\]
the condition \ref{it:key.a.i} in \ref{it:key.a} is nothing but the
familiar condition
\[ i_Aj_A(C)\to C\quad\text{a weak equivalence for any $C$ in \Cat,}\]
which we had used for our first (or rather, second already!)
definition of so-called ``test categories''. Later on we considerably
strengthened this condition -- we now call them ``test categories in
the wide sense''. On the other hand, as we already noticed before,
here $i_0^*(\Simplex_1) = j_A(\Simplex_1)$ \emph{is nothing but the
  Lawvere element $L_A$ of \Ahat}. Thus the main content of
the\pspage{64} theorem in the present special case can be
formulated thus:
\begin{corollarynum}\label{cor:38.1}
  Assume the Lawvere element $L_A$ in \Ahat\ is aspheric over the
  final object $e_\Ahat$ of \Ahat, and that moreover the latter be
  aspheric, i.e., $A$ aspheric. Then $A$ is a test category in the
  wide sense, namely for any $C$ in \Cat, the canonical functor
  $i_Aj_A(C)\to C$ is a weak equivalence \textup(and even
  aspheric\textup -- see prop.\ on p.~\ref{p:38}, and
  also p.~\ref{p:35}, for equivalent formulations).
\end{corollarynum}

Thus we did get after all a ``handy criterion'' sufficient to ensure
this basic test-property, which looks a lot less strong a priori than
the condition \ref{it:31.T2} of total asphericity of
\Ahat.\footnote{\alsoondate{25.3.} or rather, than the set of conditions
  \ref{it:31.T1} to \ref{it:31.T3}}

But let's now come back to the more general situation of the theorem,
with a functor
\[ i : A \to \Cat\]
subjected only to the mild condition that the categories $i(a)$ (for
$a$ in $A$) have final objects. Assume condition b\ref{it:key.b.iii}
to be satisfied, namely that $i^*(\Simplex_1)$ is aspheric over
$e_\Ahat$, and therefore it is a homotopy interval in \Ahat. Let again
$L_A$ be the Lawvere element in \Ahat, we define (independently of any
assumption on $i$) a morphism of ``intervals'' in \Ahat, compatible
even with the natural composition laws (by intersection) on both
members
\[\varphi : J=i^*(\Simplex_1) \to L_A.\]
For this we remember that for $a$ in $J$, we get
\[ J(a) \simeq \Crib{i(a)} \hookrightarrow \text{set of all subobjects
  of $i(a)$ in \Cat}\]
(this bijection and the inclusion being functorial in $a$), thus if
$C$ is in $J(a)$, i.e., a crible in $i(a)$, we associate to this
\[ \varphi(C) = \parbox[t]{0.6\textwidth}{subobject of $a$ in \Ahat,
  corresponding to the crible in $A_{/a}$ of all $b/a$ such that
  $i(b)\to i(a)$ factors through the crible $C\subset i(a)$;}\]
it is immediate that the map $\varphi_a : J(a)\to L(a)$ thus obtained
is functorial in $a$ for variable $a$, hence a map $\varphi : J\to L$,
and it is immediately checked too that this is ``compatible with
endpoints'' -- namely when $C$ is full, respectively empty, then so is
$\varphi(C)$ (for the ``empty'' case, this comes from the fact that
the categories $i(a)$ are non-empty). Applying now the comparison lemma
for homotopy intervals (p.~\ref{p:60}), we get the
following

\begin{corollarynum}\label{cor:38.2}
  Under the general conditions of the theorem, and assuming moreover
  that $i^*(\Simplex_1)$ is aspheric over $e_\Ahat$ \textup(condition
  \textup{b\ref{it:key.b.iii})}, it follows that the Lawvere element $L_A$ of
  \Ahat\ is aspheric over $e_\Ahat$. Assume moreover that $e_\Ahat$ is
  aspheric, i.e., $A$ aspheric. Then we get:\pspage{65}
  \begin{enumerate}[label=\alph*),font=\normalfont]
  \item\label{it:38.a}
    The category $A$ is a test category in the wide sense \textup(cf.\
    cor.\ \textup{\ref{cor:38.1})}.
  \item\label{it:38.b}
    Both functors $i^* : \Cat\to\Ahat$ and $i_A:\Ahat\to\Cat$ are
    ``modelizing'', namely the set of weak equivalences in the source
    category is the inverse image of the corresponding set of arrows
    in the target category, and the functor induced on the
    localizations with respect to weak equivalences is an equivalence
    of categories.
  \item\label{it:38.c}
    Let $W$ \textup(resp.\ $W_A$\textup) be the set of weak
    equivalences in \Cat\ \textup(resp.\ in \Ahat\textup). Then the
    functor
    \[\Hot \eqdef W^{-1}\Cat \to W_A^{-1}\Ahat\]
    induced by $i^*$ is canonically isomorphic to the quasi-inverse of
    the functor in opposite direction induced by $i_A$, or
    equivalently, can.\ isomorphic to the functor in the same
    direction induced by $i_0^*=j_A:\Cat\to\Ahat$ \textup(cf.\ again
    cor.\ \textup{\ref{cor:38.1}} above for the notations\textup).
  \end{enumerate}
\end{corollarynum}

Of course \ref{it:38.a} follows from cor.\ \ref{cor:38.1}, and implies
that $i_A$ has the properties stated in \ref{it:38.b}. That the
analogous properties hold for $i^*$ too, and the rest of the
statement, i.e., \ref{it:38.c}, follows formally, using the fact that
the compositum $i_Ai^*$ is canonically isomorphic to the identity
functor once we pass to the localized category \Hot\ (using the
theorem, b\ref{it:key.b.iii} $\Rightarrow$ \ref{it:key.a}).

This corollary shows that, up to canonical isomorphism, the functor
\[\Hot\to W_A^{-1}\Ahat\quad\text{induced by $i^*:\Cat\to\Ahat$}\]
\emph{does not depend on the choice of the functor} $i:A\to\Cat$,
provided only this functor satisfies the two conditions that it takes
its values in the full subcategory of \Cat\ of all categories with
final objects, and that moreover $i^*(\Simplex_1)$ be aspheric over the
final object of \Ahat\ (plus of course the condition of asphericity on
$A$ itself). There \emph{is} of course always the \emph{canonical}
choice of a functor $i:A\to\Cat$, namely $i_A: a\mapsto A_{/a}$ which
(because of its canonicity) looks as the best choice theoretically --
and it was the first one indeed we investigated into. But in practical
terms, the categories $A_{/a}$ are (in the concrete cases one might
think of) comparatively big (for instance, infinite) and the
corresponding functor $j_A=i_A^*$ gives comparatively clumsy
``models'' in \Ahat\ for describing the homotopy types of given
``models'' in \Cat, whereas we can get away with considerably neater
models in \Ahat, using a functor $i$ giving rise to categories $i(a)$
which are a lot easier to compute with (for instance, finite
categories of very specific type). The most commonly used is of course
the \emph{nerve functor} $i^*$, corresponding to the standard
embedding of $A=\Simplex$ into \Cat\ -- and in the general case of the
theorem above, complemented by cor.\ \ref{cor:38.2},
\emph{the functor $i^*$ should be viewed as a generalized nerve
  functor}.\pspage{66}

To be completely happy, we still need a down-to-earth sufficient
criterion to ensure that $i^*(\Simplex_1)$ is aspheric over $e_\Ahat$,
in the spirit of the somewhat awkward corollary on p.~%
\ref{p:58}. The following seems quite adequate for all
cases I have in mind at the present moment:

\begin{corollarynum}\label{cor:38.3}
  Under the general conditions of the theorem on $A$ and $i:A\to\Cat$,
  assume moreover we got a homotopy interval $(I,\delta_0,\delta_1)$
  in \Ahat, that $A$ has a final object $e$ \textup(which is therefore
  also a final object of \Ahat\ so we may view $\delta_0,\delta_1$ as
  maps $e\rightrightarrows I$, i.e., elements in $I(e)$\textup), and
  let $i_!:\Ahat\to\Cat$ be the canonical extension of $i$ to \Ahat\
  \textup(commuting with direct limits\textup). Assume moreover
  $i(e)=\Simplex_0$ \textup(the final object in \Cat\textup), and that
  we can find a map in \Cat
  \[i_!(I) \to \Simplex_1\]
  compatible with $\delta_0,\delta_1$, i.e., whose compositae with
  $i_!(\delta_n) : \Simplex_0\to\Simplex_1$ for $n\in\{0,1\}$ are the
  two standard maps $\boldsymbol\delta_0,\boldsymbol\delta_1$ from
  $\Simplex_0$ to $\Simplex_1$. Then the condition
  \textup{b\ref{it:key.b.iii}} of the theorem holds, namely
  $i^*(\Simplex_1)$ is aspheric over $e$.
\end{corollarynum}

Indeed, to give a map $i_!(I)\to\Simplex_1$ in \Cat\ amounts to the same
as giving a map $I\to i^*(\Simplex_1)$ in \Ahat\ (namely $i_!$ and $i^*$
are adjoint), moreover the extra condition involving
$\delta_0,\delta_1$ just means that his map respects endpoints. Using
the composition law of intersection on $i^*(\Simplex_1) = (a \mapsto
\Crib(i(a)))$, the comparison lemma for homotopy intervals (p.~%
\ref{p:60} implies that $i^*(\Simplex_1)$ is aspheric over
$e$, qed.

In the cases I have in mind, $I$ is even an element of $A$, hence
$i_!(I)=i(I)$, moreover $i(I)=\Simplex_1$ and the map
$i_!(I)\to\Simplex_1$ above is the \emph{identity}! The choice of the
functor $i$ is in every case ``the most natural one'' (discarding
however the clumsy $i_0=i_A$, and trying to get away with categories
$i(a)$ which give the simplest imaginable description of objects and
arrows of the would-be test category $A$), and the choice of $I$
itself is still more evident -- it is \emph{the} object of $A$ (or
\emph{one} among the objects, in cases such as $A=\Simplex^n$, giving
rise to simplicial \emph{multi}complexes\ldots) which suggests most
strongly the picture of an ``interval''. Thus the one key verification
we are left with (all the rest being ``formal'' in terms of what
precedes) is the asphericity of $I$ over $e$, i.e., that all the
products $I\times a$ are aspheric.

% 39
\hangsection[Returning upon terminology: strict test categories, and
\dots]{Returning upon terminology: strict test categories, and strict
  modelizers.}\label{sec:39}%
Maybe it is time now to come back to the property of total asphericity
of \Ahat, expressed by the condition \ref{it:31.T2} on $A$, namely
that the product in \Ahat\ of any two elements in $A$ is aspheric. As
we saw, this implies already asphericity of $A$, i.e., of the topos
\Ahat. In our present setting, total asphericity is of special
interest only when coupled with the property \ref{it:31.T3}, which
amounts to saying that the Lawvere element $L_A$ in \Ahat\
is\pspage{67} aspheric over $e_\Ahat$, or (what now amounts to the
same) that $L_A$ is aspheric. However, when faced with the question to
decide whether a given $A$ does indeed satisfy \ref{it:31.T2}, it will
be convenient to use a system $(I,e_0,e_1)$ in \Ahat\ of which we know
beforehand it is a homotopy interval (the delicate part of this notion
being the asphericity of all products $I\times a$ for $a$ in $A$). As
part of the ``story retold'', I recall now the most natural geometric
assumption which will ensure \ref{it:31.T2}, i.e., total asphericity
of \Ahat:

\begin{proposition}
  Let $A$ be a small category, $(I, e_0,e_1)$ a homotopy interval in
  \Ahat. Assume that for any $a$ in $A$\kern1pt, there exists a homotopy
  \[h_a : I\times a\to a\]
  between $\id_a$ and a constant map $c_a:a\to a$ \textup(defined by a
  section of $a$ over the final element of \Ahat, i.e., by a map $e\to
  a$\textup). Then \Ahat\ is totally aspheric, i.e., every $a\in\Ob A$
  is aspheric over $e$ \textup(or, equivalently, the product in \Ahat\ of any
  two elements $a,b$ in $A$ is aspheric\textup).
\end{proposition}

This is a particular case of the
``\hyperref[lem:homotopylemma]{homotopy lemma}'' (p.~%
\ref{p:60}). In fact we don't even use the condition $e_0
\sand e_1 = \varnothing_\Ahat$ on the ``interval'' $(I,e_0,e_1)$, but we
easily see that this condition follows from the existence of the
homotopy and the sections of any $a$ in $A$ over $e$.

Before returning to the investigation of specific test categories, I
want to come back on some terminology. The condition on \Ahat\ that
the Lawvere element $L_A$ should be aspheric over $e$ has taken lately
considerable geometric significance, and merits a name. I will say
from now on that $A$ is a \emph{test category}, and that \Ahat\ is an
\emph{elementary modelizer}, if this condition is satisfied, and if
moreover $A$ is aspheric. This condition (which amounts to our former
\ref{it:31.T1} + \ref{it:31.T3}) is weaker than what we had lately
called a test category, as we had overlooked so far the fact that
\ref{it:31.T1} and \ref{it:31.T3} alone already imply the basic
requirement about $i_Aj_A(C)\to C$ being a weak equivalence for any
$C$, and hence $W_A^{-1}\Ahat$ being canonically equivalent to
\Hot. Thus it seemed for a while that the only handy conditions we
could get for ensuring this requirement were
\ref{it:31.T1}\ref{it:31.T2}\ref{it:31.T3}, all together (in fact, it
turned out later that \ref{it:31.T2} already implies
\ref{it:31.T1}). When all three conditions \ref{it:31.T1} to
\ref{it:31.T3} are satisfied, I'll now say that\pspage{68} $A$ is a
\emph{strict test category}, and that \Ahat\ is an \emph{elementary
  strict modelizer}. Here the notion of a ``strict modelizer'' (not
necessary an elementary one) makes sense independently, it means a
category $M$ endowed with a set $W\subset\Fl(M)$ satisfying conditions
\ref{it:37.a}\ref{it:37.b}\ref{it:37.c}
of p.~\ref{p:59}, such that $W^{-1}M$ is equivalent to
\Hot, and such moreover that the localization functor $M\to W^{-1}M$
\emph{commutes with finite products} (perhaps we should also insist on
commutation with finite sums, and possibly include also infinite sums
and products in the condition -- whether this is adequate is not quite
clear yet). The mere condition that $i_Aj_A(C)\to C$ should be a weak
equivalence for any $C$ in \Cat, or equivalently that $i_Aj_A(C)$
should be aspheric when $C$ has a final element, will be referred to
by saying that $A$ is a \emph{test category in the wide sense}. It
means little more than the fact that $(\Ahat,W_A)$ is a
modelizer. Finally, if we merely assume that \Ahat\ admits a homotopy
interval, or equivalently, that $L_A$ is aspheric over $e_\Ahat$, we
will say that $A$ is a \emph{local test category} (because it just
means that the induced categories $A_{/a}$ for $a$ in $A$ are test
categories), and accordingly \Ahat\ will be called a \emph{local
  elementary modelizer}. More generally, we call a topos \scrA\ such
that the Lawvere element $L_\scrA$ be aspheric over the final element
a \emph{locally modelizing topos}, and we call it a \emph{modelizing
  topos} if it is moreover aspheric. When the topos is locally
aspheric, i.e., admits a generating family made up with aspheric
objects of \scrA, then \scrA\ is indeed a locally modelizing topos
if{f} the final object can be covered by elements $U_i$ such that the
induced topoi $\scrA_{/U_i}$ be modelizing topoi. This terminology for
topoi more general than of the type \Ahat\ is possibly somewhat hasty,
as the relation with actual homotopy models, namely with the question
whether $(\scrA,W_\scrA)$ is a modelizer, has not been investigated
yet. Still I have the feeling the relation should be a satisfactory
one, much along the same lines as we got in the case of topoi of the
type \Ahat. We will not dwell upon this now any longer.

For completing conceptual clarification, we should still make sure
that a test category $A$ need not be a strict test category, i.e.,
need not be totally aspheric. As a candidate for a counterexample, one
would think about a category $A$ endowed with a final element and an
element $I$, together with $\delta_0,\delta_1:e\to I$, satisfying
$\Ker(\delta_0,\delta_1)=\varnothing_\Ahat$ in \Ahat, $I$ being
squarable in $A$, namely the products $I\times a$ ($a \in \Ob A$) are
in $A$ -- this alone will imply that $I$ is aspheric over $e$ in
\Ahat, hence $A$ is a test category, but it is unlikely that this
alone will imply equally total asphericity of \Ahat. We would think,
as the most ``economical'' example, one where $A$ is made up with
elements of the type $I^n$ ($n\in\bN$) namely cartesian\pspage{69}
powers of $I$, plus products $a_0\times I^n$ ($n\in\bN$), where $a_0$
is an extra element, and those maps between these all those (and not
more) which can be deduced from $\delta_0,\delta_1$ and the assumption
that the elements $I^n$ and $a\times I^n$ are cartesian products
indeed. This is much in the spirit of the construction of a variant of
the category $\square$ of ``standard cubes'', which naturally came to
mind a while ago (cf.\ p.~%
\ref{p:48}--\ref{p:49}). I very much doubt $A$
satisfies \ref{it:31.T2}, and would rather bet that $a_0 \times a_0$
in \Ahat\ is \emph{not} aspheric.

Another type of example comes to my mind, starting with a perfectly
good (namely \emph{strict}) test category $A$, and taking an induced
category $A_{/a_0}$, with $a_0$ in $A$. This is of course a test
category (it would have been enough that $A$ be just a local test
category), however it is unlikely that it will satisfy \ref{it:31.T2},
namely the induced topos be totally aspheric. This would imply for
instance that for any two non-empty subobjects of $a_0$ in \Ahat\
(namely subobjects of the final element in $\Ahat_{/a_0} \simeq
(A_{/a_0})\uphat$) have a non-empty intersection. Now this is an
exceedingly strong property of $a_0$, which is practically never
satisfied, except for the final object of $A$. Now here is a kind of
``universal'' counterexample. Take $A$ any test category and $I$ a
homotopy interval in \Ahat, thus $I$ is aspheric and hence $A_{/I}$ is
again a test category (namely locally test and aspheric), but it is
\emph{never} a strict test category, because the two standard
subobjects $e_0,e_1$ given with the structure of $I$ are non-empty,
and however their intersection is empty!

These reflections bring very near how much stronger the strictness
requirement \ref{it:31.T2} is for test categories, than merely the
conditions \ref{it:31.T1}, \ref{it:31.T3} without \ref{it:31.T2}.

% 40
\hangsection[Digression on cartesian products of weak equivalences in
\dots]{Digression on cartesian products of weak equivalences in
  \texorpdfstring{\Cat}{(Cat)}; 4 weak equivalences \emph{relative} to
  a given base object.}\label{sec:40}%
Yesterday I incidentally made use of the fact that if $A$ is an
aspheric element in \Cat, then for any other object $C$,
$C\times A\to C$ is aspheric (and a fortiori a weak equivalence). The
usual \ref{lem:asphericitycriterion} for a functor shows that it is
enough to prove that for any $C$ with final element, $C\times A$ is
aspheric. For this again, it is enough to prove that the projection
$C\times A\to A$ is aspheric, which by localization upon $A$ means
that the product categories $C\times (A_{/a})$ (with $a$ in $A$) are
aspheric. Finally we reduced to checking that the product of two
categories with final element is aspheric, which is trivial because
such a category has itself a final element.

The result just proved can be viewed as a particular case of the
following
\begin{proposition}
  In \Cat, the cartesian product of two weak equivalences is a weak equivalence.
\end{proposition}

(We\pspage{70} get the previous result by taking the weak
equivalences $A\to e_\Cat$ and $\id_C:C\to C$.) To prove the
proposition, we are immediately reduced to the case when one of the
two functors is an identity functor, i.e., proving the
\begin{corollarynum}\label{cor:40.1}
  If $f:A'\to A$ is a weak equivalence in \Cat, then for any $C$ in
  \Cat, $f\times \id_C: A'\times C\to A\times C$ is a weak equivalence.
\end{corollarynum}

For proving this, we view the functor $f\times\id_C$ as a morphism of
categories ``over $C$'', which corresponds to a situation of a
morphism of topoi over a third one
\[\begin{tikzcd}[baseline=(Y.base),column sep=tiny,row sep=small]
  X' \ar[dr]\ar[rr,"f"] & & X\ar[dl] \\ & |[alias=Y]| Y &
\end{tikzcd},\]
we will say that $f$ is a \emph{weak equivalence relative to $Y$}, if
not only this is a weak equivalence, but remains so by any base change
by a localization functor
\[ Y_{/U} \to U \]
giving rise to
\[ f_{/U} : X'_{/U} \to X_{/U}.\]
As usual, standard arguments prove that it is enough to take $U$ in a
set of generators of the topos $Y$. In case $Y=\Chat$, we may
take $U$ in $C$. In case moreover $X,X'$ are defined by small
categories $P,P'$ and a functor of categories over $C$
\[\begin{tikzcd}[baseline=(C.base),column sep=tiny,row sep=small]
  P' \ar[dr]\ar[rr,"F"] & & P\ar[dl] \\ & |[alias=C]| C &
\end{tikzcd},\]
this condition amounts to demanding that for any $c$ in $C$, the
induced functor
\[F_{/c} : P'_{/c} \to P_{/c}\]
be a weak equivalence. In our case $P=A\times C$, $P'=A'\times C$,
$F=f\times\id_C$, the induced functor can be identified with
$f\times\id_{C_{/c}}$. This reduces us, for proving the corollary, to
the case when $C$ has a final element. But consider now the
commutative diagram
\[\begin{tikzcd}[baseline=(A.base),row sep=small]
  A'\times C\ar[d]\ar[r,"f\times\id_C"] & A\times C\ar[d]\\
  A' \ar[r,"f"] & |[alias=A]| A
\end{tikzcd},\]
where the vertical arrows are the projections and hence, by what was
proved before, weak equivalences. As $f$ is a weak equivalence, it
follows that $f\times\id_C$ is a weak equivalence too, qed.

The proposition above goes somewhat in the direction of looking at
``homotopy properties of \Cat'' and ``how far \Cat\ is from being a
closed model category in Quillen's sense''. It is very suggestive
for\pspage{71} having a closer look at functors $f: B \to A$ in
\Cat\ which are ``universally weak equivalences'', i.e.,
$W_\Cat$-aspheric, namely such that for \emph{any} map in \Cat\ $A'\to
A$ (not only a localization $A_{/a}\to A$), the induced functor
$B'=B\times_A A' \to A'$ is a weak equivalence. This property is a lot
stronger than just asphericity, and reminds of the ``trivial
fibrations'' in Quillen's theory. The usual criterion of asphericity
for $B'\to A'$ shows that $f$ is $W$-aspheric if{f} for any $A'$ with
final element, and any functor $A'\to A$, the fiber product
$B'=B\times_A A'$ is aspheric. The feeling here is (suggested partly
by Quillen's terminology) that this property is tied up some way with
the property for $f$ to be a fibering (or cofibering?) functor, in the
sense of fibered and cofibered categories, with aspheric fibers
moreover. Presumably, fibered or cofibered categories, with ``base
change functors'' which are weak equivalences, will play the part of
Serre-Quillen's fibrations -- and it is still to be guessed what kind
of properties of a functor will play the part of
\emph{co}fibrations. Apparently they will have to be a lot more
stringent than just monomorphisms in \Cat, cf.\ p.~\ref{p:37}.

However, I feel it is not time yet to dive into the homotopy theory
properties of the all-encompassing basic modelizer \Cat, but rather
come back to the study of general (and less general) test categories.

% 41
\hangsection[Role of the ``inspiring assumption'', and of saturation
\dots]{Role of the ``inspiring assumption'', and of saturation
  conditions on ``weak equivalences''.}\label{sec:41}%
One comment still, upon the role played in the theory I am developing
of the assumption (p.~\ref{p:30}) that the category of
autoequivalences of \Hot\ is equivalent to the final category. This
assumption has been a crucial guide for putting the emphasis where it
really belongs, namely upon the set $W\subset\Fl(M)$ of weak
equivalences within a category $M$ which one would like to take in
some sense as a category of models for homotopy types -- the functors
$M\to\Hot$ following along automatically. However, in no statement
whatever I proved so far, was this assumption ever used. On the other
hand, the notion of a \emph{modelizer} introduced in the wake of the
``assumption'' (cf.\ p.~\ref{p:31}) was tacitly changed during the
reflection, by dropping altogether the condition \ref{it:Mod.a} of
(strong) saturation of $W$, namely that $W$ is just the set of arrows
made invertible by the localization functor $M\to W^{-1}M$. Instead of
this, it turned out that the saturation condition we really had at
hands and which was adequate for working, was the conditions
\ref{it:37.a} to \ref{it:37.c} I finally wrote down explicitly
yesterday (p.~\ref{p:59}). As for the strong saturation condition,
for the time being (using nothing but what has actually been proved so
far, without reference to ``well-known facts'' from homotopy theory),
it is not even clear that the basic modelizer \Cat\ is one in the
initial sense, namely that the set of weak equivalences\pspage{72} in
\Cat\ is strongly saturated. However, from the known relation between
\Cat\ and an elementary modelizer \Ahat, it follows that $W_\Cat$ is
strongly saturated if{f} $W_A\subset\Fl(\Ahat)$ is. This implies that
it is enough to prove strong saturation in \emph{one} elementary
modelizer \Ahat, to deduce it in all others, as well as in
\Cat. However, in the case at least when $A=\Simplex$, it is indeed
``well-known'' that the weak equivalences in \Ahat\ satisfy the strong
saturation condition. In terms of Quillen's set-up, it follows from
the fact that $\Simplexhat$ is a ``closed model category'', and prop.\
1 in I~5.5 of Quillen's expos\'e. The only thing which is not quite
understood, not by me at any rate, is why $\Simplexhat$ \emph{is}
indeed a closed model category -- Quillen's proof it seems relies
strongly on typical simplicial techniques. I'll have to look if the
present set-up will suggest a more conceptual proof, valid possibly
for any test category (or at least, any strict test category). I'll
have to come back upon this later. For the time being, I feel a
greater urge still to understand about the relationship between
different test categories -- also, I did not really finish with my
review of what may be viewed as the ``standard'' test categories such
as $\Simplex$, $\square$ and their variants.

\bigbreak

\presectionfill\ondate{25.3.}\par

% 42
\hangsection[Terminology revised (model preserving
functors). \dots]{Terminology revised
  \texorpdfstring{\textup(}{(}model preserving
  functors\texorpdfstring{\textup)}{)}. Submodelizers of the basic
  modelizer \texorpdfstring{\Cat}{(Cat)}.}\label{sec:42}%
In the notes last time I made clear what finally has turned out for me
to correspond to the appelation of a ``modelizer'', as prompted by the
internal logics of the situations I was looking at, in terms of the
information available to me. I should by then have added what a
\emph{model-preserving} (or \emph{modelizing}) \emph{functor} between
modelizers $(M,W)$ and $(M',W')$ has turned out to mean, which has
undergone a corresponding change with respect to what I first
contemplated calling by that name (p.~\ref{p:31}). Namely, here it
turned out that I should be more stringent for this notion, replacing
the condition $f(W)\subset W'$ by the stronger one
\[ W = f^{-1}(W'),\]
and moreover, of course, still demanding that the corresponding
functor
\[W^{-1}M \to W'^{-1}M'\]
should be an equivalence. It is by now established, with the exception
of just the two first that, all the functors occurring in the diagram
on p.~\ref{p:31} are indeed model preserving, namely it
is so for $\alpha$, $\beta$, $\xi$, $\eta$ -- and also the right
adjoints to $\xi$, $\eta$, as expected by then. It should be more or
less trivial that the first of the functors in this diagram, namely
the canonical inclusion $\Ord\to\Preord$, as well as the left adjoint
from \Preord\ to \Ord, are model preserving -- except for the fact
that it has not been yet established that these two categories are
indeed modelizers (for the natural notion of weak equivalences, induced
from \Cat)),\pspage{73} namely that the localized categories with
respect to weak equivalences are indeed equivalent to \Hot. The only
natural way one might think of this to be proved, is by proving that
the inclusion functor from either into \Cat\ induces an equivalence
between the localizations, which would imply at the same time that
this inclusion functor is indeed model-preserving, and hence that all
the functors in the diagram of p.~\ref{p:31} are model
preserving functors between modelizers. I still do believe this should
be so, and want to give below a reflection which might lead to a proof
of this.

Before, there is still one noteworthy circumstance I want to
emphasize. Namely, it occurred in a number of instances that we got in
a rather natural way several modelizing functors between two
modelizers $(M,W)$ and $(M',W')$, in one direction or the other -- for
instance from \Cat\ to \Ahat\ using different functors $i:A\to\Cat$,
or from \Ahat\ to \Cat\ using $a\mapsto A_{/a}$. It turned out that
the corresponding functors between $W^{-1}M$ and $W'^{-1}M'$ were
always canonically isomorphic when in the same direction, and
quasi-inverse of each other when in opposite directions. This is
indeed very much in the spirit of the ``inspiring assumption'' of p.~%
\ref{p:30}, that the category of autoequivalences of \Hot\
is equivalent to the unit category, which implies indeed that for two
categories $H,H'$ equivalent to \Hot, for two equivalences from $H$ to
$H'$, there is a unique isomorphism between them. Quite similarly,
there have been a number of situations (more or less summed up in the
end in the ``\hyperref[thm:keyresult]{key theorem}'' of p.~%
\ref{p:61}) when by localization we got a natural functor
$f$ from some $W^{-1}M$ to another $W'^{-1}M'$, and another $F$ which
is known already to be an equivalence, and it turns out that $f$ is
isomorphic to $F$ (and in fact, then, canonically so) if{f} for any
object $x$ in the source category, $f(x)$ and $F(x)$ are
isomorphic. This suggests that \emph{presumably, every functor
  from \Hot\ into itself, transforming every object into an isomorphic
  one, is in fact isomorphic to the identity functor}.

I want now to make a comment, implying that there are many full
subcategories $M$ of \Cat, such that for the induced notion of weak
equivalence, $M$ becomes a modelizer, and the inclusion functor a
modelizing functor -- or, what amounts to the same, that the canonical
functor
\[W_M^{-1}M \to W_\Cat^{-1}\Cat = \Hot\]
is an equivalence. To see this, let more generally $(M,W)$ be any
category endowed with a subset $W\subset\Fl(M)$, and let $h:M\to M$ be
a functor such that $h(W)\subset W$, and such that the induced functor
$W^{-1}\to W^{-1}M$ is an equivalence. Let now $M'$ be any full
subcategory of $M$ such that $h$ factors through $M'$, let $h' : M\to
M'$ be the corresponding induced functor, and $W'=W\sand
\Fl(M')$. Then it is formal that the inclusion $g:M'\to
M$\pspage{74} and $h':M\to M'$ induce functors between $W'^{-1}M'$
and $W^{-1}M$, which are quasi-inverse to each other -- hence these
two categories are equivalent. In case $(M,W)$ is a modelizer, this
implies that $(M',W')$ is a modelizer too and the inclusion functor
$g$, as well as $h'$, are model preserving.

We can apply this remark to the modelizer \Cat, and to the functor
\[ i_Ai^*:\Cat\to\Cat\]
defined by any functor $i:A\to\Cat$ satisfying the conditions of the
``\hyperref[thm:keyresult]{key theorem}'' (p.~%
\ref{p:61}), for instance the functor $i_A\restrto A: a
\mapsto A_{/a}$ (where $A$ is any test category). We get the following

\begin{proposition}
  Let $M$ be any full subcategory of \Cat, assume there exists a
  test-category $A$ such that for any $F$ in \Ahat, the category
  $A_{/F}$ belongs to $M$ \textup(i.e., the functor $i_A:\Ahat\to\Cat$,
  $F\mapsto A_{/F}$, factors through $M$\textup). Let $W_M$ be the set
  of weak equivalences in $M$. Then $(M,W_M)$ is a modelizer and the
  inclusion functor $M\to\Cat$ is model preserving.
\end{proposition}

The same of course will be true for any full subcategory of \Cat\
containing $M$ -- which makes an impressive bunch of modelizers
indeed! When the test category $A$ is given, one natural choice for
$M$ is to take all categories $C$ which are ``locally isomorphic to
$A$'', namely such that for any $x$ in $C$, the induced category
$C_{/x}$ be isomorphic to a category of the type $A_{/a}$, with $a$ in
$A$.

It would be tempting to apply this result to the full subcategory
\Ord\ of \Cat\ -- but for this to be feasible, would mean exactly that
there exists a test-category $A$ defined by an ordered set (or at
least ``locally ordered''). To see whether there exists indeed such an
ordered set looks like a rather interesting question -- maybe it would
give rise to algebraic models for homotopy types, simpler than those
used so far, namely simplicial and cubical complexes and
multicomplexes. It is interesting to note that if such a test category
should exist, it will \emph{not} be in any case a strict test
category. Indeed, the topos \Ahat\ associated to an ordered set $A$
can be viewed also, as we saw before (p.~\ref{p:18}), as
associated to a suitable topological space (namely $A$ endowed with a
suitable topology, the open sets being just the ``cribles'' in
$A$). But we have seen that the topos associated to a topological
space \emph{cannot} be \emph{strictly} modelizing (cor.\
\ref{cor:35.2.2} on page \ref{p:53}).

This remark confirms the feeling that it was worth while emphasizing
the notion of a test category (just \ref{it:31.T1} to \ref{it:31.T3})
by a simple and striking name as I finally did, rather than bury it
behind the notion I now call a strict test category, which is
considerably more stringent and, moreover, more ``rigid''. For
instance, it is not stable under localization $A_{/a}$, whereas the
notion of a test category is -- indeed, for any aspheric $I$ in \Ahat,
$A_{/I}$ is still a test category.\pspage{75}

% 43
\hangsection[The category \protect\smashSimplexf{} of simplices without
degeneracies as a \dots]{The category
  \texorpdfstring{\protect\Simplexf}{Delta-f}{} of simplices without
  degeneracies as a weak test category -- or ``face complexes'' as
  models for homotopy types.}\label{sec:43}%
Now let's come back for a little while again to the so-called
``standard test categories'', and check how nicely the ``story
retold'' applies to them.

Not speaking about multicomplexes, there are essentially two variants
for ``categories of simplices'' as test categories. The smaller, more
commonly used one, is the category $\Simplex$ of ``ordered simplices'',
most conveniently described as the full subcategory of \Cat\ defined
by the family of simplices $\Simplex_n$ ($n\in\bN$). Here the most
natural choice for $i:\Simplex\to\Cat$ is of course the inclusion
functor. As $i$ is fully faithful and $\Simplex_1$ is in the image, it
follows that $i^*(\Simplex_1)=\Simplex_1$, and we have only to check (for
$A$ to be a test-category with ``test-functor'' $i$) that $\Simplex_1$
is aspheric over $e=\Simplex_0$, namely that all products
$\Simplex_1\times\Simplex_n$ are aspheric -- which we did. The extra
condition of the ``total asphericity criterion'' (proposition on p.~%
\ref{p:67}), namely existence of a homotopy in
$\Simplexhat$ from the identity map to a constant map, for any
$\Simplex_n$, is indeed satisfied: it is enough to define such a homotopy
in \Cat, which is trivial using the final element of $\Simplex_n$. Thus
$\Simplex$ is in fact a \emph{strict} test category.

As for $\widetilde\Simplex$, the most elegant choice theoretically is to
take the category of all non-empty finite sets, but his leads to
set-theoretic difficulties, as this category is not small - thus we
take again the standard non-ordered simplices
\[\widetilde\Simplex_n = \bN \sand {[0,n]} \quad (n\in\bN),\]
so as to get a ``reduced'' category with a countable set of
objects. This time, as $\widetilde\Simplex$ is stable under finite
products, and contains the ``interval''
$(\widetilde\Simplex_1,\delta_0,\delta_1)$ (which is necessarily then a
homotopy interval, as all elements of $\widetilde\Simplex$ are aspheric
over $e=\widetilde\Simplex_0$), the fact that $\widetilde\Simplex$ is a
strict test category is trivial. As for a test functor, the neatest
choice is the one we said before, namely associating to every finite
set the ordered set of all non-empty subsets. We thus get
\[ \widetilde i : \widetilde\Simplex\uphat \to \Cat
\quad \text(\parbox[t]{.5\textwidth}{factoring in fact through \Ord, as
  does $i:\Simplex\to\Cat$ above)}\]
To prove it is indeed a test functor, the corollary
\ref{cor:38.3} to the ``key theorem'' (p.~%
\ref{p:66}) applies, taking of course
$I=\widetilde\Simplex_1$, hence
\[\widetilde i_!(I) = i(\widetilde\Simplex_1) =
\begin{tikzcd}[baseline=(A.base),row sep=-7pt,column sep=small]
  \{1\} \ar[dr] & \\ & |[alias=A]| \{0,1\} \\ \{0\}\ar[ur] &
\end{tikzcd}.\]
We\pspage{76} map this into the object $\Simplex_1$ of \Cat, by
taking $\{0\}$ into $0$, $\{1\}$ and $\{0,1\}$ into $1$, we do get
indeed a morphism compatible with endpoints, which implies that
$\widetilde i$ is a test functor.

If we denote still by $i,\widetilde i$ the functors from
$\Simplex,\widetilde\Simplex$ to the category \Ord\ of ordered sets
factoring the previous two test functors, we get a commutative diagram
of functors
\[\begin{tikzcd}[baseline=(O.base)]
  \Simplex\ar[r]\ar[d,swap,"i"] & \widetilde\Simplex\ar[d,"\widetilde i"] \\
  \Ord\ar[r,"\text{bar}"] & |[alias=O]| \Ord
\end{tikzcd},\]
where the first horizontal arrow is the inclusion functor (bijective
on objects, and injective but not bijective on arrows), and the second
is the ``barycentric subdivision'' functor, or ``flag''-functor,
associating to every ordered set the set of all ``flags'', namely
non-empty subsets which are totally ordered for the induced order
(here, all subsets, as the simplices are totally ordered).

For a while I thought there was an interesting third variant, namely
ordered simplices with \emph{strictly} increasing maps between them --
which means ruling out degeneracy operators. This feeling was prompted
of course by the fact that the face operators in a complex are enough
for computing homology and cohomology groups, which are felt to be
among the most important invariants of a complex. Equally, the
fundamental groupoid of a semi-simplicial set can be described, using
only the face operators. As a consequence, for a map between
semisimplicial complexes $K_*\to K_*'$, to check whether this is a
weak equivalence, in terms of the Artin-Mazur cohomological criterion,
depends only on the underlying map between ``simplicial face
complexes'' (namely, forgetting degeneracies). These \emph{are} indeed
striking facts, which will induce us to put greater emphasis on the
face operators than on degeneracies. It seems, though that the
degeneracies play a stronger role than I suspected, even though it is
a somewhat hidden one. In any case, as soon as we try to check (``par
acquit de conscience'') that the category \Simplexf\ of simplices with
strictly increasing maps is a test category, it turns out that it is
very far from it! Thus, as there is no map from any $\Simplex_n$ with
$n>1$ into $\Simplex_1$, it follows that in
$(\Simplexf)\uphat$, we get
\[\Simplexf_{/\Simplex_0\times\Simplex_n} = \text{discrete category with
  $n+1$ elements,}\]
thus these products are by no means aspheric, poor them! Even throwing
out $\Simplex_0$ (a barbarous thing to do anyhow!) doesn't rule out the
trouble, and restricting moreover to products $\Simplex_1\times\Simplex_n$
(to have at least a test category, if not a strict one). In any case,
$\Simplex_1$ wouldn't be of much use, because it has got no ``section''
anymore (nor does any other element of \Simplexf) -- because this would
imply that \emph{any} element of \Simplexf{} maps into it -- but for
given $\Simplex_n$,\pspage{77} only the $\Simplex_m$'s with $m\le n$
map into it.

Maybe I am only being imprisoned still by the preconception of finding
a homotopy interval in \Simplexf{} itself, rather than in
$(\Simplexf)\uphat$. After all, just applying the definition of a
test-category $A$ with test-functor $i:A\to\Cat$, all we have to care
about is whether a)\enspace $A$ is aspheric and b)\enspace
$i^*(\Simplex_1)$ is an aspheric element of \Ahat. We just got to
apply this to the case of the functor
\[ i^{\mathrm f} : \Simplexf \to \Cat\]
induced by $i:\Simplex\to\Cat$ above, taking into account of course the
extra trouble that $i^{\mathrm f}$ is no longer fully faithful.

I just stopped to look, with a big expectation that \Simplexf{}
\emph{is} a test category after all -- but it turns out it definitely
isn't! Indeed, $A_{/\Simplex_0\times I}$ (where
$I=(i^{\mathrm f})^*(\Simplex_1)$) is again a discrete two-point
category, not aspheric. Taking the canonical functor $a\mapsto A_{/a}$
from $A=\Simplexf$ to \Cat, which is the ultimate choice for checking
whether or not $A$ is a test category, finally gives the answer: it is
not. Because with $I$ now the Lawvere element, we still have that
$A_{/\Simplex_0\times I}$ is a two-point discrete category. Thus the
topos $\Ahat=(\Simplexf)\uphat$ isn't locally modelizing, i.e., it
hasn't got any homotopy interval, which at any rate is a very big
drawback I would think. The only hope which still remains, to account
for the positive features of face-complexes recalled above, is that
\Simplexf{} is at least a test category \emph{in the wide sense}, namely
that for any category $C$ with final element, the category
$i_Aj_A(C) = A_{/j_A(C)}$ ($=$ category of all pairs $(n,u)$, with $u$
a map of the (ordered) category $A_{/\widetilde\Simplex_n}$) is
aspheric. The first case to check is for
$C=\text{final category $\Simplex_0$}$, i.e., asphericity of $A$, next
step would be $C=\Simplex_1$, i.e., asphericity of the Lawvere element
$L_A$ (but of course \emph{not} asphericity over the final element
$e_\Ahat$!).

The question certainly deserves to be settled. If the answer is
affirmative, i.e., \Simplexf{} \emph{is} a test category in the wide
sense, then the proposition stated earlier (p.~%
\ref{p:74}), which clearly applies equally when $A$ is a
test category in the wide sense, implies that if $M$ is any full
subcategory of \Cat{} containing all those $C$ which are ``locally
isomorphic'' to \Simplexf, (i.e., such that for every $x\in C$, $C_{/x}$
is isomorphic to the ordered category of all subsimplices of some
simplex), then for the induced notion $W_M$ of weak equivalences, $M$
is a modelizer and the inclusion functor from $M$ into \Cat{} is
modelizing. This does not yet apply to \Ord, however, it reopens the
question whether the full subcategory of \Ord{} made up by all ordered
sets $J$ which are locally isomorphic to \Simplexf{} in the sense above
(namely for any $x\in J$, the ordered subset $J_{\le x}$ is
isomorphic\pspage{78} to the ordered set of subsimplices of some
simplex) is a modelizer. To ensure this, it would be enough to find an
ordered set $J$ satisfying the previous condition (for instance on
stemming from a ``simplicial maquette''), such that the corresponding
category is a test category, or at least a test category in the wide
sense. The first candidate that comes to my mind, is to take any
\emph{infinite} set $S$ of vertices, and take $J$ to be the ordered
set of all \emph{finite non-empty} subsets (called the simplices --
thus the elements of $S$ can be interpreted in terms of $J$ as the
minimal simplices). By the way, the category associated to $J$, in
case $S=\bN$, can be interpreted in terms of $A=\Simplexf$ as the
category $A_{/\Simplex_\oo}$, where $\Simplex_\oo$ is defined as the
filtering direct limit in $\Ahat$ of the $\Simplex_n$'s, arranged into a
direct system in the obvious way:
\[\Simplex_\oo = \varinjlim_n \Simplex_n\quad \text{in
  $(\Simplexf)\uphat$.}\]
Asphericity of $J$ looks intuitively evident, and should be easy by a
direct limit argument, as a matter of fact any filtering category (the
next best to having a final element) should be aspheric, at least if
it has a countable cofinal family of objects. The Lawvere element
$L_A$ in \Ahat{} is not aspheric though over the final object, because
when inducing over a zero simplex, we get the same contradiction as
before. As a matter of fact, I am getting aware I have been very silly
and prejudiced not to see one trivial common reason, applicable to
\Simplexf{} as to $J$, showing that they are not test categories nor
even local test categories: namely the induced categories $A_{/a}$
should be test categories too, but among these there are one-point
categories (take $a=\Simplex_0$, or a zero-simplex), and such a category
is \emph{not} a test-category!

Still, $J$ may be a test category in the wide sense -- the basic test
here, as we know already asphericity of $J$ itself, would be
(absolute) asphericity of $L_J$, or equivalently, of the category
$J_{/L_J}$, an ordered set in fact (as is the case for any category
$A_{/F}$ for $A$ defined by an ordered set and $F$ in \Ahat). This is
now the ordered set of all pairs $(K\subset T)$ of finite subsets of
$S$, with $T$ in $J$, namely $T$ non-empty, with the rule
\[ (K',T') \le (K,T) \quad\text{if{f}}\quad \text{$T'\subset T$ and
  $K'=K\sand T'$.}\]

\bigbreak

\presectionfill\ondate{26.3.}\pspage{79}\par

% 44
\hangsection{Overall review of the basic notions.}\label{sec:44}%
\renewcommand*{\thesubsection}{\alph{subsection})}%
I finally convinced myself that \Simplexf, the category of standard
ordered simplices with face operations (and no degeneracies) \emph{is}
a ``test category in the wide sense'' after all -- although definitely
not a test category, as was seen yesterday (turning out to be a
practically trivial observation). This now does rehabilitate the
notion of a test category in the wide sense, which I expected to be of
little or no interest -- much the way as previously, the notion I now
call by the name ``test category'' was rehabilitated or rather,
discovered, after I expected that the only one proper notion for
getting modelizers of the form \Ahat{} was in terms of conditions
\ref{it:31.T1} to \ref{it:31.T3} on $A$ (including the very strong
condition \ref{it:31.T2} of total asphericity). This notion I finally
called by the name ``\emph{strict} test category'', and I was
fortunately cautious enough to reserve a name too for the notion which
appeared then as rather weak and unmanageable, of test categories in
the wide sense, or, as I will say now more shortly, \emph{weak test
  categories} (by which of course I do not mean to exclude the
possibility that it be even a test category), thus getting the trilogy
of notions with strict implications
\[\text{weak test categories} \Leftarrow \text{test categories}
\Leftarrow \text{strict test categories}.\]
In order not to get confused, I will recall what exactly each of these
notions means.

\subsection{Weak test categories.}
\label{subsec:44.a}
For a given small category $A$, we
look at the functor
\begin{equation}
  \label{eq:44.1}
  i_A : \Ahat\to\Cat, \quad F \mapsto A_{/F}, \tag{1}
\end{equation}
which commutes with direct limits, and at the right adjoint functor
\begin{equation}
  \label{eq:44.2}
  j_A = i_A^*:\Cat\to\Ahat, \quad C\mapsto
  j_A(C)=(a\mapsto\Hom(A_{/a},C)). \tag{2}
\end{equation}
We get an adjunction morphism
\begin{equation}
  \label{eq:44.3}
  i_Aj_A(C) \to C\quad\text{in \Cat,}\tag{3}
\end{equation}
and another
\begin{equation}
  \label{eq:44.4}
  F \to j_Ai_AF\quad\text{in \Ahat,}\tag{4}
\end{equation}
functorially in $C$ resp.\ in $F$. Presumably, the following are
equivalent (I'll see in a minute how much I can prove about these
equivalences):
\begin{enumerate}[label=(\roman*)]
\item\label{it:44.a.i}
  The functors \eqref{eq:44.1} and \eqref{eq:44.2} are compatible
  with weak equivalences, and the two induced functors between the
  localized categories
  \begin{equation}
    \label{eq:44.5}
    W_A^{-1}\Ahat \rightleftarrows W^{-1}\Cat \eqdef \Hot\tag{5}
  \end{equation}
  are equivalences.
\item\label{it:44.a.ii}
  As in \ref{it:44.a.ii}, and moreover the two equivalences are
  quasi-inverse of each other, with adjunction morphism in
  $W^{-1}\Cat=\Hot$ deduced from \eqref{eq:44.3} by localization.
\item\label{it:44.a.iii}
  As\pspage{80} in \ref{it:44.a.ii}, but moreover the adjunction
  morphism in $W_A^{-1}\Ahat$ for the pair of quasi-inverse
  equivalences in \eqref{eq:44.5} being likewise deduced from
  \eqref{eq:44.4}.
\item\label{it:44.a.iv}
  For any $C$ in \Cat, \eqref{eq:44.3} is a weak equivalences.
\item\label{it:44.a.v}
  Same as \ref{it:44.a.iv}, with $C$ restricted to having a final
  element, i.e., for any such $C$, $j_Ai_A(C)=A_{/j_A(C)}$ is
  aspheric.
\item\label{it:44.a.vi}
  For any $F$ in \Ahat, \eqref{eq:44.4} is a weak equivalence,
  moreover
  \[W_\Cat = j_A^{-1}(W_A),\]
  i.e., a map $f:C'\to C$ in \Cat{} is a weak equivalence if{f}
  $j_A(f)$ is a weak equivalence in \Ahat{} -- which means, by
  definition (more or less) that $i_Aj_A(f)$ is a weak equivalence.
\item\label{it:44.a.vii}
  The functor
  \begin{equation}
    \label{eq:44.6}
    i_Aj_A : \Cat\to\Cat,\quad C\mapsto A_{/j_A(C)},\tag{6}
  \end{equation}
  transforms weak equivalences into weak equivalences, i.e., gives
  rise to a functor $\Hot\to\Hot$, and moreover the latter respects
  final object (i.e., $A$ is aspheric).
\item\label{it:44.a.viii}
  Same as \ref{it:44.a.vii}, but restricting to weak equivalences of
  the type $C\to e$, where $C$ has a final element and $e$ is the
  final object in \Cat{} (the one-point category), plus asphericity
  of $A$.
\end{enumerate}

The trivial implications between all these conditions can be
summarized in the diagram
\[\begin{tikzcd}[baseline=(O.base),math mode=false,%
  row sep=small,arrows=Rightarrow]
  \ref{it:44.a.iii} \ar[r,Leftrightarrow]\ar[d] & \ref{it:44.a.vi} \ar[r] &
  \ref{it:44.a.vii} \ar[r] & \ref{it:44.a.viii} \ar[dl] \\
  \ref{it:44.a.ii} \ar[r]\ar[d] & \ref{it:44.a.iv} \ar[r] &
  \ref{it:44.a.v} & \\
  |[alias=O]| \ref{it:44.a.i} & , & &
\end{tikzcd}\]
the only slightly less obvious implication here is \ref{it:44.a.viii}
$\Rightarrow$ \ref{it:44.a.v}, which is seen by looking at the
commutative square deduced from $C\to e$ ($C$ in \Cat{} is in
\ref{it:44.a.v} namely with final element) by applying
\eqref{eq:44.3}
\[\begin{tikzcd}[baseline=(O.base)]
  i_Aj_A(C)\ar[d]\ar[r] & C\ar[d] \\
  A=i_Aj_A(e)\ar[r] & |[alias=O]| e
\end{tikzcd},\]
by assumption the vertical arrows are weak equivalences, and so is
$A\to e$ (because $A$ is supposed to be aspheric), therefore the same
holds for the fourth arrow left. On the other hand, an easy
asphericity argument showed us that \ref{it:44.a.v} $\Rightarrow$
\ref{it:44.a.iii}, hence all conditions \ref{it:44.a.ii} to
\ref{it:44.a.viii} are equivalent, and they are equivalent to the
stronger form of \ref{it:44.a.iv}, say
\namedlabel{it:44.a.ivprime}{(iv')}, saying that \eqref{eq:44.3} is
\emph{aspheric} for any $C$ in \Cat. The only equivalence which is not
quite clear yet is that \ref{it:44.a.i} implies the other
conditions. But it is so\pspage{81} if we grant the ``inspiring
assumption'', implying that any autoequivalence of \Hot{} is
isomorphic to the identity functor -- in this case it is clear that
even the weaker form \namedlabel{it:44.a.iprime}{(i')} of \ref{it:44.a.i},
demanding only that $i_Aj_A$ induce an autoequivalence of \Hot{} and
nothing on either factor $i_A$, $j_A$ in \eqref{eq:44.5}, implies
\ref{it:44.a.v}. Also, when we assume moreover $A$ aspheric, it is
clear that \ref{it:44.a.i} (and even \ref{it:44.a.iprime}) implies
\ref{it:44.a.vii}, i.e., all other conditions. Thus, instead of the
assumption on \Hot, it would be enough to know that the particular
autoequivalence of \Hot{} induced by $i_Aj_A$ transforms the final
element of \Hot{} (represented by the element $e$ of \Cat) into an
isomorphic one,\footnote{\alsoondate{27.3.} But this is true for \emph{any}
  equivalence of categories -- I'm really being very dull!} which
looks like a very slight strengthening of \ref{it:44.a.i} indeed.

In any case, the conditions \ref{it:44.a.ii} to \ref{it:44.a.viii} are
equivalent, and equivalent to \ref{it:44.a.i} (or
\ref{it:44.a.iprime}) \emph{plus}\footnote{The ``plus'' is unnecessary
  visibly, see note above.} asphericity of $A$. The formally strongest
form is \ref{it:44.a.iii}, the formally weakest one (with the
exception of \ref{it:44.a.i} or \ref{it:44.a.iprime}) is
\ref{it:44.a.v}, which is also the one which looks the most concrete,
namely amenable to practical verification. This was indeed what at the
very beginning was attractive in the condition, in comparison to the
first one that came to my mind when introducing the notion of a
modelizer and of model preserving functors -- namely merely that $i_A$
should induce an equivalence between the localized categories, or
equivalently, that \Ahat{} should be a ``modelizer'' and $i_A$ should
be model-preserving. That we should be more demanding and ask for
$j_A$ to be equally model preserving crept in first rather timidly --
and I still don't know (and didn't really stop to think) if the first
one implies the other.\footnote{If fact, it is the condition that
  $j_A=i_A^*$, not $i_A$, should be model preserving, which is ``the
  right'' condition on $A$, equivalent to $A$ being a weak test
  category, if $A$ is assumed aspheric.}

In any case, as far as checking a property goes, I would consider
\ref{it:44.a.v} to be \emph{the} handy definition of a weak test
category, wheres \ref{it:44.a.iii} is the best, when it goes to making
use of the fact that $A$ is indeed a weak test category. As for
\ref{it:44.a.i}, it corresponds to the main intuitive content of the
notion, which means that homotopy types can be ``modelled'' by
elements of \Ahat, using $i_A$ for describing which homotopy type is
described by an element $F$ of \Ahat, and using $j_A$ for getting a
model in \Ahat{} for a given homotopy type, described by an object $C$
in \Cat.

% b
\subsection{Test categories and local test categories.}
\label{subsec:44.b}
Let $A$ still be a small category. Then the following conditions are
equivalent:
\begin{enumerate}[label=(\roman*)]
\item\label{it:44.b.i}
  All\pspage{82} induced categories $A_{/a}$ (with $a$ in $A$)
  are weak test categories.
\item\label{it:44.b.ii}
  For any aspheric $F$ in \Ahat, $A_{/F}$ is a weak test category.
\item\label{it:44.b.iii}
  There exists a ``homotopy interval'' in \Ahat, namely an element $I$
  in \Ahat, \emph{aspheric over the final element} $e_\Ahat$ (i.e.,
  such that all products $I\times a$ (for $a$ in $A$) are aspheric),
  endowed with two sections $\delta_0$, $\delta_1$ ever $e=e_\Ahat$,
  i.e., with two subobjects $e_0,e_1$ isomorphic to $e$, such that
  $\Ker(\delta_0,\delta_1)=\varnothing_\Ahat$, i.e., $e_0\sand
  e_1=\varnothing_\Ahat$.
\item\label{it:44.b.iv}
  The Lawvere element $L_A$ or $L_\Ahat$ in \Ahat{} (i.e., the
  presheaf $a\mapsto$ subobjects of $a$ in $\Ahat\simeq$ cribles of
  $A_{/a}$) is aspheric over the final element $e$ (in other words, it
  is a homotopy interval, when endowed with the two sections
  $\delta_0,\delta_1$ corresponding to the full and the empty crible
  in $A$).
\item\label{it:44.b.v}
  For any $C$ in \Cat, the canonical map in \Cat
  \begin{equation}
    \label{eq:44.7}
    i_Aj_A(C) = A_{/j_A(C)} \to A \times C \tag{7}
  \end{equation}
  (with second component \eqref{eq:44.3}) is (not only a weak
  equivalence, which definitely isn't enough, but even) \emph{aspheric}.
\end{enumerate}

By the usual \hyperref[lem:asphericitycriterion]{criterion of
  asphericity} for a map in \Cat, condition \ref{it:44.b.v} is
equivalent with the condition \namedlabel{it:44.b.vprime}{(v')}: For
any $a$ in $A$ and any $C$ in \Cat{} with final element, $A_{/a\times
  j_A(C)}$ is aspheric, i.e., $a\times j_A(C)$ is an aspheric element
in \Ahat; now this is clearly equivalent to \ref{it:44.b.i} (by the
checking-criterion \ref{it:44.a.v} above for weak test categories,
applied to the categories $A_{/a}$). Thus we get the purely formal
implications
\[\begin{tikzcd}[baseline=(O.base),math mode=false,arrows=Rightarrow]
  \ref{it:44.b.ii} \ar[r] & \ref{it:44.b.i}\ar[d,Leftrightarrow]
  \ar[r] & \ref{it:44.b.iv} \ar[r] & \ref{it:44.b.iii} \\
  & |[alias=O]| \ref{it:44.b.v} & &
\end{tikzcd},\]
where \ref{it:44.b.i} $\Rightarrow$ \ref{it:44.b.iv} is obtained by
applying the criterion \ref{it:44.b.vprime} just recalled to the case
of $C=\Simplex_1$. As the condition \ref{it:44.b.iii} is clearly stable
by localizing to a category $A_{/F}$ (using the equivalence
$(A_{/F})\uphat\simeq \Ahat_{/F}$), we see that \ref{it:44.b.iii}
implies \ref{it:44.b.ii}, we are reduced (replacing $A$ by $A_{/F}$)
to the case when $F$ is the final element in \Ahat, and thus to
proving that if $A$ has a final element, then \ref{it:44.b.iii}
implies that $A$ is a weak test category, i.e., that the categories
$A_{/j_A(C)}$, with $C$ having a final element, are aspheric. This was
done by a simple homotopy argument in two steps. One step (``the
comparison lemma for homotopy intervals'' on p.~%
\ref{p:60}) shows that \ref{it:44.b.iii} $\Rightarrow$
\ref{it:44.b.iv}, i.e., existence of a homotopy interval implies that
the Lawvere interval is a homotopy interval, the other step
(presented in a more general set-up in the
``\hyperref[thm:keyresult]{key result}'' on page
\ref{p:61}) proving that \ref{it:44.b.iv} implies
asphericity of the elements $j_A(C)$ ($C$ with\pspage{83} final
object) over $e_A=e_\Ahat$.

We express the conditions \ref{it:44.b.i} to \ref{it:44.b.v} by saying
that $A$ is a \emph{local test category}, or that \Ahat{} is a
\emph{locally modelizing topos}, or (if we want to recall that this
topos is of the type \Ahat) an \emph{elementary local modelizer}. If
$A$ is moreover aspheric (i.e., $e_\Ahat$ aspheric), or what amounts
to the same, if $A$ is moreover a weak test category, we say that $A$
is a \emph{test category}, or that \Ahat{} is a \emph{modelizing
  topos}, or (to recall it is an \Ahat{} and not just any topos) an
\emph{elementary modelizer}. These are intrinsic properties on the
topos \Ahat, the first one of a local nature, the second not (as
asphericity of \Ahat{} is a global notion).

Here the question arises whether the condition for $A$ to be a weak
test category can be likewise expressed intrinsically as a property of
the topos \Ahat{} -- which we then would call a \emph{weakly
  modelizing topos}, or a \emph{weak elementary modelizer}. This
doesn't look so clear, as all conditions stated in \ref{subsec:44.a}
make use of at least one among the two functors $i_A, j_A$, which do
not seem to make much sense in the more general case, except possibly
when using a specified small generating subcategory $A$ of the given
topos -- and possibly checking that the condition obtained (if
something reasonable comes out, as I do expect) does not depend on the
choice of the generating site. This should be part of a systematic
reflection on modelizing topoi, to make sure for instance they are
modelizing indeed with respect to weak equivalences -- but I'll not
enter into such reflection for the time being.

% c
\subsection{Strict test categories.}
\label{subsec:44.c}
The following conditions on the small category $A$ are equivalent:
\begin{enumerate}[label=(\roman*)]
\item\label{it:44.c.i}
  $A$ is a weak test category (cf.\ \ref{subsec:44.a}
  \ref{it:44.a.iii} above), and thus induces a localization functor
  \[ \Ahat \to \Hot=W^{-1}\Cat,\]
  and this functor moreover \emph{commutes with binary products}; or
  equivalently, the canonical functor $\Ahat\to W_A^{-1}\Ahat$
  commutes with binary products.
\item\label{it:44.c.ii}
  $A$ is a test category (a local test category would be enough,
  even), and moreover the topos \Ahat{} is totally aspheric, namely
  (apart having a generating family of aspheric generators, which is
  clear anyhow) the product of any two aspheric elements in \Ahat{} is
  aspheric, i.e., any aspheric element of \Ahat{} is aspheric over the
  final object.
\item\label{it:44.c.iii}
  $A$\pspage{84} satisfies the two conditions:
  \begin{enumerate}[label=T~\arabic*),start=2]
  \item\label{it:44.T2}
    The product in \Ahat{} of any two elements in $A$ is aspheric.
  \item\label{it:44.T3}
    \Ahat{} admits a homotopy interval (which is equivalent to saying
    that $A$ is a local test category).
  \end{enumerate}
\end{enumerate}

Of course, \ref{it:44.c.ii} implies \ref{it:44.c.iii}. The condition
\ref{it:44.c.iii} can be expressed in terms of the topos $\scrA=\Ahat$
by saying that this is a totally aspheric and locally modelizing
topos, which implies already (as we saw by the \v Cech computation of
cohomology) that the topos is aspheric, and therefore modelizing,
i.e., $A$ a test category, which is just \ref{it:44.c.ii}. Thus
\ref{it:44.c.ii} $\Leftrightarrow$ \ref{it:44.c.iii}. Total
asphericity of \scrA, and the property that \scrA{} be a local
modelizer, are expressed respectively by the two neat conditions
\ref{it:44.T2} and \ref{it:44.T3}, neither of which implies the other
even for an \Ahat, with $A$ a category with final object.

For expressing in more explicit terms condition \ref{it:44.c.i}, and
check equivalence with the two equivalent conditions \ref{it:44.c.ii},
\ref{it:44.c.iii}, we need to admit that the canonical functor from
\Cat{} to its localization \Hot{} commutes with binary products. This
being so, the condition of commutation of $\Ahat\to\Hot$ with binary
products can be expressed by the more concrete condition that for
$F,G$ in \Ahat{}, the canonical map in \Cat
\begin{equation}
  \label{eq:44.8}
  i_A(F\times G) \to i_A(F) \times i_A(G), \quad\text{i.e.,
    $A_{/F\times G}\to A_{/F}\times A_{/G}$}\tag{8}
\end{equation}
is a weak equivalence. In fact, the apparently stronger condition that
the maps \eqref{eq:44.8} are aspheric will follow, because the usual
criterion of asphericity for a map in \Cat{} shows that it is enough
for this to check that \eqref{eq:44.8} is a weak equivalence when $F$
and $G$ are representable by elements $a$ and $b$ in $A$, which also
means that $A_{/a\times b}$ is aspheric, namely $a\times b$ in \Ahat{}
aspheric -- which is nothing but condition \ref{it:44.T2} in
\ref{it:44.c.iii}, i.e., total asphericity of \Ahat. On the other
hand, the condition that $A$ be a weak test category means that the
elements $j_A(C)$ in \Ahat, for $C$ in \Cat{} admitting a final
object, are aspheric, or (what amount to the same when \Ahat{} is
totally aspheric) that they are aspheric over the final object
$e_\Ahat$, i.e., that the products $a\times j_A(C)$ are aspheric,
which also means $A$ is a local test category. Thus \ref{it:44.c.i} is
equivalent to \ref{it:44.c.ii}.

The equivalent conditions \ref{it:44.c.i} to \ref{it:44.c.iii} are
expressed by saying that $A$ is a \emph{strict test category}, or that
$\scrA=\Ahat$ is a \emph{strictly modelizing topos}, or also that it
is an \emph{elementary strict modelizer} (when emphasizing the topos
\scrA{} should be of the type \Ahat{} indeed). This presumably will
turn out to be the more important among the three notions of a ``test
category'' and the weak and strict variant. The two less stringent
notions, however, seem interesting in their own right. The notion of a
weak test category mainly (at present) because it turns out that the
category\pspage{85} \Simplexf{} of standard ordered simplices with
only ``face-like'' maps between them, namely strictly increasing ones
(that is, ruling out degeneracies) is a weak test category, and not a
test category. On the other hand, for any test category $A$, we can
construct lots of test categories which are not strict, namely all
categories $A_{/F}$ where $F$ is any aspheric object in \Ahat{} which
admits two ``non-empty'' subobjects whose intersection is ``empty'' --
take for instance for $F$ any homotopy interval. In the case of
$A=\Simplex$, we may take for $F$ any objects $\Simplex_n$ ($n\ge1$) in
$A$, except just the final object $\Simplex_0$.

% d
\subsection[Weak test functors and test functors (with values in
\texorpdfstring{\Cat}{(Cat)}).]{Test functors.}

Let $A$ be a weak test category, and let
\begin{equation}
  \label{eq:44.9}
  i : A \to \Cat\tag{9}
\end{equation}
be a functor, giving rise to a functor
\begin{equation}
  \label{eq:44.10}
  i^*: \Cat\to\Ahat,\quad C\mapsto i^*(C)=(a\mapsto\Hom(i(a),C)).\tag{10}
\end{equation}
We'll say that $i$ is a \emph{weak test functor} if $i^*$ is a
morphism of modelizers, i.e., model-preserving, namely
\[
\begin{cases}
  W_\Cat=(i^*)^{-1}(W_\Ahat) \quad\text{and} \\
  W_\Cat^{-1}\Cat = \Hot \to W_\Ahat^{-1}\Ahat\text{ an equivalence.}
\end{cases}\]
As by assumption we know already that \eqref{eq:44.1}
\[ i_A : \Ahat\to\Cat, F\mapsto A_{/F}\]
is model-preserving, this implies that $i$ is a weak test functor
if{f} the compositum
\begin{equation}
  \label{eq:44.11}
  i_Ai^*:\Cat\to\Cat,\quad C\mapsto A_{i^*(C)} \eqdef A_{/C}\tag{11}
\end{equation}
is model-preserving, i.e., (essentially) induces an
\emph{autoequivalence} of \Hot.

The basic example is to take $i=i_A$, hence $i^*=j_A$ \eqref{eq:44.2},
the fact that $A$ is a weak test category can be just translated (p.~%
\ref{p:80}, \ref{it:44.a.vii}) by saying that this functor
\eqref{eq:44.11} $=$ \eqref{eq:44.6} transforms weak equivalences into
weak equivalences, and the localized functor $\Hot\to\Hot$ is an
equivalent (or, which is enough, transforms final object into final
object of \Hot). But as we say before, this weak test functor gives
rise to categories $i(A)=A_{/a}$ which are prohibitively large, and
one generally prefers working with weak test functors more appropriate
for computations -- including\pspage{86} the customary ones for
categories of simplices or cubes, such as \Simplexf, $\Simplex$,
$\widetilde\Simplex$ and the cubical analogons. If we admit the
``inspiring assumption'' that any autoequivalence of \Hot{} is
uniquely isomorphic to the identity functor, it will follow that the
autoequivalence $\Hot\to\Hot$ induced by \eqref{eq:44.11} (where again
$i$ is any weak test functor) is canonically isomorphic to the
identity. This we checked directly (as part of the
``\hyperref[thm:keyresult]{key result}''
p.~\ref{p:61} and following) when we make on $i$ the
extra assumption that each category $i(a)$ has a final object. In
practical terms, \emph{the role of a weak test functor $i$ is to
  furnish us with a quasi-inverse of}
\[W_A^{-1}\Ahat \to W_\Cat^{-1}\Cat = \Hot\]
\emph{induced by $i_A$, more handy than the one deduced from $j_A$ by
  localization, namely taking $i^*$ instead of $j_A=i_A^*$}. In other
terms, for every homotopy type, described by an object $C$ in \Cat, we
get a ready model in \Ahat, just taking
$i^*(C)=(a\mapsto\Hom(i(a),C))$.

In the theorem on p.~\ref{p:61} just referred to, one
point was that we did not assume beforehand that $A$ was a weak test
category, but we were examining two sets \hyperref[it:key.a]{(a)} and
\hyperref[it:key.b]{(b)} of mutually equivalent conditions, where
\hyperref[it:key.a]{(a)} just boils down to $i_Ai^*$ \eqref{eq:44.11}
inducing an autoequivalence of \Hot, while \hyperref[it:key.b]{(b)}
would be expressed (in the present terminology) by stating that
moreover $A$ is a test category (if we assume beforehand in
\hyperref[it:key.b]{(b)} that $A$ is aspheric, plus a little more
still on $i$). It was not quite clear by then, as it is now, that this
is actually strictly stronger. The somewhat bulky statement
essentially reduces, with the present background, to the following
\begin{proposition}
  Let $A$ be a small category, $i:A\to\Cat$ a functor, such that for
  any $a$ in $A$\kern1pt, $i(a)$ has a final object. Consider the following
  conditions:
  \begin{enumerate}[label=(\roman*),font=\normalfont]
  \item\label{it:44.prop.i}
    $A$ is a test category \textup(NB\enspace not only a weak one\textup), and
    $i$ is a weak test functor.
  \item\label{it:44.prop.ii}
    The element $i^*(\Simplex_1)$ in \Ahat{} \textup(namely $a \mapsto
    \Crib(i(a))$\textup) is aspheric over $e_\Ahat$, i.e., all
    products $i^*(\Simplex_1)\times a$ \textup($a\in\Ob A$\textup) are
    aspheric; moreover $e_\Ahat$ is aspheric, i.e., $A$ itself is
    aspheric.
  \item\label{it:44.prop.iii}
    There exists a homotopy interval $(I,\delta_0,\delta_1)$ in \Ahat,
    and a map in \Cat{} compatible with $\delta_0,\delta_1$
    \begin{equation}
      \label{eq:44.12}
      i_!(I) \to \Simplex_1,\tag{12}
    \end{equation}
    where $i_!:\Ahat\to\Cat$ is the canonical extension of
    $i:A\to\Cat$; moreover, $A$ is aspheric.
  \end{enumerate}
  The conditions \textup{\ref{it:44.prop.ii}} and \textup{\ref{it:44.prop.iii}} are
  equivalent and imply \textup{\ref{it:44.prop.i}}.
\end{proposition}

That \ref{it:44.prop.ii} implies \ref{it:44.prop.iii} is trivial by
taking $I=i^*(\Simplex_1)$; the converse, as we saw, is a corollary of
the comparison lemma for homotopy intervals (p.~%
\ref{p:60}), applied to $I\to i^*(\Simplex_1)$ corresponding
to \eqref{eq:44.12}, which is\pspage{87} a morphism of intervals,
namely compatible with endpoints. (However, in practical terms
\eqref{eq:44.12} is the more ready-to-use criterion, because in most
cases $I$ will be in $A$ and therefore $i_!(I)=i(I)$, and
\eqref{eq:44.12} will be more or less trivial, for instance because
$I$ will be chosen so that $i(I)$ is either $\Simplex_1=\{0\}\to\{1\}$
or its barycentric subdivision \begin{tikzcd}[baseline=(A.base),row sep=-12pt,column sep=small]
  \{0\} \ar[dr] & \\ & |[alias=A]| \{0,1\} \\ \{1\}\ar[ur] &
\end{tikzcd}.) The same argument shows that \ref{it:44.prop.iii}
implies that $A$ is a test category, namely the Lawvere element $L_A$
in \Ahat{} is a homotopy interval in \Ahat, namely is aspheric over
$e_\Ahat$. This being so, we only got to prove (under the assumption
\ref{it:44.prop.iii}) that $i_Ai^*$ \eqref{eq:44.11} induces an
autoequivalence of \Hot, and more specifically a functor isomorphic to
the identity functor. This we achieved by comparing directly $A_{/C}$
to $C$ by the functor $(a,p)\mapsto p(e_a)$
\begin{equation}
  \label{eq:44.13}
  A_{/C} \to C,\tag{13}
\end{equation}
where $e_a$ is the final element in $i(a)$. It is enough to check this
is a weak equivalence for any $C$, a fortiori that it is aspheric, and
the usual asphericity criterion reduces us to showing it is a weak
equivalence when $C$ has a final object, namely that in this case
$A_{/C}=A_{/i^*(C)}$ is aspheric, i.e., $i^*(C)$ is aspheric in
\Ahat. This was achieved by an extremely simple homotopy argument,
using a homotopy in \Cat
\[\Simplex_1\times C\to C\]
between the identity functor in $C$ and the constant functor with
value $e_C$ (cf.\ p.~\ref{p:62}).

I feel like reserving the appellation of a \emph{test functor}
$A\to\Cat$ (in contract to a \emph{weak} test functor) to functors
satisfying the equivalent conditions \ref{it:44.prop.ii} and
\ref{it:44.prop.iii}, implying that $A$ is a test category (not merely
a weak one!), and which seem stronger a priori than assuming moreover
that $i$ is a weak test functor (condition \ref{it:44.prop.i}). I
confess I did not make up my mind if for a test category $A$, there
may be weak test functors, with all $i(a)$ having final objects, which
are not test functors in the present sense, namely whether it may be
true that for any $C$ in \Cat{} with final object, $i^*(C)$ is
aspheric, and therefore $i^*(\Simplex_1)$ aspheric, without the latter
being aspheric over $e_\Ahat$, i.e., all products $i^*(\Simplex_1)\times
a$ ($a\in\Ob A$) being aspheric (which would be automatic though if
$A$ is even a \emph{strict} test category). The difficulty here is
that it looks hard to check the condition for arbitrary $C$ with final
object, except through the homotopy argument using relative
asphericity of $i^*(\Simplex_1)$. Of course, the distinction between
weak test functors and test functors is meaningful only as long as it
is expected that the two are actually distinct, when applied to test
categories. Besides this, the restriction to categories $i(a)$ with
final objects looks theoretically a little awkward, and it shouldn't
be too hard I believe to get rid of it, if need be. But for the time
being the only would-be test functors or\pspage{88} weak test
functors which have turned up, do satisfy this condition, and
therefore it doesn't seem urgent to clean up the notion in this
respect.

\needspace{6\baselineskip}
\subsection*{Remarks.}

\paragraph{1.} I regret I was slightly floppy when translating the
condition that $\varphi=i_Ai^*$ \eqref{eq:44.11} be model-preserving,
by the (a priori less precise) condition that it induce an
autoequivalence of the localized category \Hot{} (which is of course
meant to imply that it transforms weak equivalences into weak
equivalences, and thus does induce a functor from \Hot{} into
itself). It isn't clear that if this functor is an equivalence,
$W=\varphi^{-1}(W)$, except if we admit that $W$ is strongly
saturated, i.e., an arrow in \Cat{} which becomes an isomorphism in
\Hot, is indeed a quasi-equivalence. This, as we saw, follows from the
fact that there are weak test categories (such as $\Simplex$, which is
even a strict one), such that \Ahat{} be a ``closed model category''
in Quillen's sense -- but I didn't check yet in the present set-up
that \Simplexhat{} does indeed satisfy Quillen's condition. \emph{Which}
elementary modelizers are ``closed model categories'' remains one of
the intriguing questions in this homotopy model story, which I'll have
to look into pretty soon now.

\paragraph{2.} It should be noted that if $i:A\to\Cat$ is a test
functor, with a strict test category $A$ (such as $\Simplex$, with the
standard embedding $i$ of $\Simplex$ into \Cat{} say), whereas
$i^*:\Cat\to\Ahat$ is model preserving by definition, it is by no
means always true that the left adjoint functor
\[i_! : \Ahat\to\Cat\]
(which can be equally defined as the canonical extension of $i$ to
\Ahat, commuting with direct limits) is equally model preserving. This
seems to be in fact an extremely special property of just the
``canonical'' test functor $i_A\restrto A$.

\paragraph{3.} On the other hand, I do not know if for any small
category $A$, such that $(A, W_A=W_\Ahat)$ is a modelizer and
$i_A:\Ahat\to\Cat$ is model preserving, is a weak test
category. Assume that $j_A=i_A^*:\Cat\to\Ahat$ is model preserving,
and moreover $A$ aspheric, then $A$ is a weak test category, because
$j_A$ transforms weak equivalences into weak equivalences and we apply
criterion \ref{it:44.a.vii} (p.~\ref{p:80}). More
generally, if we have a functor $i:A\to\Cat$ such that
$i^*:\Cat\to\Ahat$ is model preserving (assuming already $(\Ahat,W_A)$
to be a ``modelizer''), I wonder whether this implies that $A$ is
already a weak test category, as it does when $i$ is the canonical
functor $i_A$ and hence $i^*=j_A$. The answer isn't clear to me even
when $i_A$, or equivalently $\varphi=i_Ai^*$ model-preserving too.

%%% Local Variables:
%%% mode: latex
%%% TeX-master: "main.tex"
%%% End:
