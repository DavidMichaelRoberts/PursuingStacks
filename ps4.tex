%% Anti-Copyright 2015 - the scrivener

\chapter{Asphericity structures and canonical modelizers}
\label{ch:IV}

\presectionfill\alsoondate{11.4.}\pspage{188}\par

% 67
\hangsection[Setting out for the asphericity game again: variance of
\dots]{Setting out for the asphericity game again: variance of the
  category \texorpdfstring{\HotA}{(Hot-A)}, for arbitrary small
  category $A$ and aspheric functors.}\label{sec:67}%
A little more pondering and scribbling finally seems to show that the
real key for an understanding of modelizers isn't really the notion of
contractibility, but rather the notion of aspheric objects (besides,
of course, the notion of weak equivalence). At the same time it
appears that the notion of an \emph{aspheric map} in \Cat, more
specifically of a \scrW-aspheric ``map'' (i.e., a functor between
small categories) is a lot more important than being just a highly
expedient technical convenience, as it has been so far -- it is indeed
one of the basic notions of the theory of modelizers we got into. As a
matter of fact, I should have known this for a number of weeks
already, ever since I did some scribbling about the plausible notion
of ``morphism'' between test-categories (as well as their weak and
strong variants), and readily convinced myself that the natural
``morphisms'' here were nothing else but the aspheric functors between
those categories. I kind of forgot about this, as it didn't seem too
urgent to start moving around the category I was working with. If I
had been a little more systematic in grinding through the usual
functorialities, as soon as a significant notion (such as the various
test notions) appears, I presumably would have hit upon the crucial
point about modelizers and so-called ``asphericity structures'' a lot
sooner, without going through the long-winded detour of homotopy
structures, and the still extremely special types of test functors
suggested by the contractibility assumptions. However, I believe that
most of the work I went through, although irrelevant for the
``asphericity story'' itself, will still be useful, especially when it
comes to pinpointing the so-called ``canonical'' modelizers, whose
modelizing structure is intrinsically determined by the category
structure.

First thing now which we've to do is to have a closer look at the
meaning of asphericity for a functor between small categories. There
is no reason whatever to put any restrictions on these categories
besides smallness (namely the cardinals of the sets of objects and
arrows being in the ``universe'' we are implicitly working in, or even
more stringently still, $A$ and $B$ to be in \Cat{} namely to be
objects of that universe). Thus, we will not assume $A$ and $B$ to be
test categories or the like. We will be led to consider, for
\emph{any} small category $A$, the localization of \Ahat{} with
respect to \scrW-equivalences, which I'll denote by \HotAW{} or simply
\HotA:
\begin{equation}
  \label{eq:67.1}
  \HotAW=\HotA=\scrW^{-1}\Ahat.\tag{1}
\end{equation}
These\pspage{189} categories, I suspect, are quite interesting in
themselves, and they merit to be understood. Thus, one of Quillen's
results asserts that (at least for $\scrW=W_\Cat=$ ordinary weak
equivalences, but presumably his arguments will carry over to an
arbitrary \scrW) in case $A$ is a product category $\Simplex\times A_0$,
where $A_0$ is any small category and $\Simplex$ the category of
standard, ordered simplices, then \Ahat{} is a closed model category
admitting as weak equivalences the set \scrWA\footnote{\alsoondate{3.5.}
  This, I quickly became aware, is a misrepresentation of Quillen's
  result -- that ``weak equivalences'' he introduced are a lot
  stronger than \scrWA. I'll have to come back upon this soon
  enough!}; and hence \HotA, the corresponding ``homotopy category'',
admits familiar homotopy constructions, including the two types of
Dold-Puppe exact sequences, tied up with loop- and suspension
functors. It is very hard to believe that this should be a special
feature of the category $\Simplex$ as the multiplying factor -- surely
any test category or strict test category instead should do as
well. As we'll check below, the product of a local test category with
any category $A_0$ is again a local test category, hence a
test-category if both factors are (\scrW-)aspheric. Thus suggests that
maybe for any local test category $A$, the corresponding \Ahat{} is a
closed model category -- but it isn't even clear yet if the same
doesn't hold for \emph{any} small category $A$ whatever! It's surely
something worth looking at.

As we'll see presently, it is a tautology more or less that a functor
\begin{equation}
  \label{eq:67.1again}
  i:A\to B,\tag{1}
\end{equation}
giving rise to a functor
\begin{equation}
  \label{eq:67.2}
  i^*:\Bhat\to\Ahat\tag{2}
\end{equation}
(commuting to all types of direct and inverse limits), induces a
functor on the localizations
\[\overline{i^*}:\HotB\to\HotA,\]
provided $i$ is \scrW-aspheric. Thus,
\[A\mapsto\HotA\]
can be viewed as a functor with respect to $A$, provided we take as
``morphisms'' between ``objects'' $A$ the \emph{aspheric} functors
only -- i.e., it is a functor on the subcategory
$\Cat_{\textup{\scrW-asph}}$ of \Cat, having the same objects as \Cat,
but with maps restricted to be \scrW-aspheric ones.

We'll denote by
\[\HotOf(\scrW) = \scrW^{-1}\Cat\]
the homotopy category defined in terms of the basic localizer
\scrW. For any small category $A$, we get a commutative
diagram\pspage{190}
\begin{equation}
  \label{eq:67.3}
  \begin{tabular}{@{}c@{}}
    \begin{tikzcd}[baseline=(O.base)]
      \Ahat\ar[r,"i_A"]\ar[d,swap,"\gamma_A"] &
      \Cat\ar[d,"\text{$\gamma_\scrW$ or $\gamma$}"] \\
      \HotA \ar[r,"\overline{i_A}"] &
      |[alias=O]| (\HotOf(\scrW))
    \end{tikzcd},
  \end{tabular}
  \tag{3}
\end{equation}
we denote by
\[\varphi_A=\gamma i_A = \overline{i_A}\gamma_A :
\Ahat\to(\HotOf(\scrW))\]
the corresponding composition.

Coming back to the case of a functor $i$ and corresponding $i^*$
(\eqref{eq:67.1again} and \eqref{eq:67.2}), the functors $i^*$, $i_A$,
$i_B$ do not give rise to a commutative triangle, but to a triangle
\emph{with commutation morphism} $\lambda_i$:
\[\begin{tikzcd}[baseline=(O.base),column sep=small]
  \Bhat\ar[rr,"i^*"]\ar[dr,swap,"i_B",""{name=iB,right}] & &
  \Ahat\ar[dl,"i_A",""{name=iA,left}] \\
  & |[alias=O]| \Cat \arrow[bend right=10, from=iA, to=iB, swap, "\lambda_i"] &
\end{tikzcd},\]
i.e., for any $F$ in \Bhat, we get a map
\begin{equation}
  \label{eq:67.4}
  \lambda_i(F): i_Ai^*(F)=A_{/i^*(F)} \to i_B(F)=B_{/F} ,\tag{4}
\end{equation}
the first hand side of \eqref{eq:67.4}, also written simply $A_{/F}$
when there is no ambiguity for $i$, can be interpreted as the category
of pairs
\[(a,p), \quad a\in\Ob A,\quad p:i(a)\to F,\]
where $p$ is a map in \Bhat, $B$ identified as usual to a full
subcategory of \Bhat{} (hence $i(a)$ identified with an object of
\Bhat). The map $\lambda_i(F)$ for fixed $F$ is of course the functor
\[ (a,p) \mapsto (i(a),p).\]
The topological significance of course is clear: interpreting $i$ as
defining a ``map'' or morphisms of the corresponding topoi \Ahat{} and
\Bhat, having $i^*$ as inverse image functor, an object $F$ of \Bhat{}
gives rise to an \emph{induced topos} $\Bhat_{/F} \tosimeq
(B_{/F})\uphat$, and the restriction of the ``topos above'' \Ahat{} to
the induced $\Bhat_{/F}$, or equivalently the result of base change
$\Bhat_{/F}\to\Bhat$, gives rise to the induced morphism of topoi
$\Ahat_{/i^*(F)}\to\Bhat_{/F}$, represented precisely by the map
$\lambda_i(F)$ in \Cat.

The condition of \scrW-asphericity on $i$ may be expressed in manifold
ways, as properties of either one of the three aspects
\[\lambda_i, \quad i^*, \quad i_Ai^*\]
of the situation created by $i$, with respect to the localizing sets
\scrW, \scrWA, $\scrW_B$, or to the notion of aspheric object. As
\scrWA{} is defined in terms of \scrW{} as just the inverse image of
the latter by $i_A$, and the same for aspheric objects, it turns out
that each of the conditions we\pspage{191} are led to express on
$i^*$, can be formulated equivalently in terms of the composition
$i_Ai^*$. I'll restrict to formulate these in terms of $i^*$ only,
which will be the form most adapted to the use we are going to make
later of the notion of a \scrW-aspheric map, when introducing the
so-called ``asphericity structures'' and corresponding ``testing
functors''.

\bigbreak
\presectionfill\ondate{27.5.}\par

% 68
\hangsection{Digression on a ``new continent''.}\label{sec:68}%
It has been over six weeks now that I didn't write down any notes. The
reason for this is that I felt the story of asphericity structures and
canonical modelizers was going to come now without any problem, almost
as a matter of routine to write it down with some care -- therefore, I
started doing some scratchwork on a few questions which had been
around but kept in the background since the beginning, and which were
a lot less clear in my mind. Some reflection was needed anyhow, before
it would make much sense to start writing down anything on
these. Finally, it took longer than expected, as usual -- partly
because (as usual too!), a few surprises would turn up on my
way. Also, I finally allowed myself to become distracted by some
reflection on the ``Lego-Teichmüller construction game'', and pretty
much so during last week. The occasion was a series of informal talks
Y.\scrcomment{Y.\ = Yves Ladegaillerie} has been giving in Molino's seminar,
on Thurston's hyperbolic geometry game and his compactification of
Teichm\"uller space. Y.\ was getting interested again in mathematics
after a five year's interruption. He must have heard about my seminar
last year on ``anabelian algebraic geometry'' and the ``Teichmüller
tower'', and suggested I might drop in to get an idea about Thurston's
work. This work indeed appears as closely related in various respects
to my sporadic reflections of the last two years, just with a
diametrically opposed emphasis -- mine being on the algebro-geometric
and arithmetic aspects of ``moduli'' of algebraic curves, his on
hyperbolic riemannian geometry and the simply connected transcendental
Teichmüller spaces (rather than the algebraic modular
varieties). The main intersection appears to be interest in surface
surgery and the relation of this to the Teichmüller modular group. I
took the occasion to try and recollect about the Lego-Teichmüller
game, which I had thought of last year as a plausible, very concrete
way for modelizing and visualizing the whole tower of Teichmüller
groupoids $T_{g,\nu}$ and the main operations among these, especially
the ``cutting'' and ``gluing'' operations. The very informal talk I
gave was mainly intended for Y.\ as a matter of fact, and
it\pspage{192} was an agreeable surprise to notice that the message
this time was getting through. For the fiver or six years since my
attention became attracted by the fascinating melting-pot of key
structures in geometry, topology, arithmetic, discrete and algebraic
groups, intertwining tightly in a kind of very basic
Galois-Teichmüller theory, Y.\ has been the very first person I met
so far to have a feeling for (a not yet dulled instinct I might say,
for sensing) the extraordinary riches opening up here for
investigation. The series of talks I had given in a tentative seminar
last year had turned short, by lack of any active interest and
participation of anyone among the handful of mere listeners. And the
two or three occasions I had the years before to tell about the matter
of two-dimensional maps (``cartes'') and their amazing
algebro-arithmetic implications, to a few highbrow colleagues with
incomparably wider background and know-how than anyone around here, I
met with polite interest, or polite indifference which is the same. As
there was nobody around anyhow to take any interest the these juicy
greenlands, nobody would care to see, because there was no text-book
nor any official seminar notes to prove they existed, after a few
years I finally set off myself for a preliminary voyage.

I thought it was going to take me a week or two to tour it and kind of
recense\scrcomment{``recense'' could be translated as ``survey''}
resources. It took me five months instead of intensive work, and two
impressive heaps of notes (baptized ``La Longue Marche à travers la
théorie de Galois''), to get a first, approximative grasp of some of
the main structures and relationships involved. The main emphasis was
(still is) on an understanding of the action of profinite
Galois-groups (foremost among which \GalQQ{} and the subgroups of
finite index) on non-commutative profinite fundamental groups, and
primarily on fundamental groups of algebraic curves -- increasingly
too on those of modular varieties (more accurately, modular
\emph{multiplicities}) for such curves -- the profinite completions of
the Teichmüller group. The voyage was the most rewarding and exciting
I had in mathematics so far -- and still it became very clear that it
was just like a first glimpse upon a wholly new landscape -- one
landscape surely among countless others of a continent unknown, eager
to be discovered.

This was in the first half of the year 1981 -- just two years ago, it
turns out, but it look almost infinitely remote, because such a lot of
things took place since. Looking back, it turns out there have been
since roughly four main alternating periods of reflection, one period
of reflection on personal matters alternating with one on
mathematics. The next mathematical reflection started with a long
digression on tame topology and the ``déployment''
(``unfolding'')\pspage{193} of stratified structures, as a leading
thread towards a heuristic understanding of the natural stratification
of the Mumford-Deligne compactifications of modular multiplicities
$M_{g,\nu}$ (for curves of genus $g$ endowed with $\nu$ points). This
then led to the ``anabelian seminar'' which turned short, last
Spring. Then a month or two sicknees, intensive meditation for three
or four months, a few more months for settling some important personal
matter; and now, since February, another mathematical reflection
started.

I am unable to tell the meaning of this alternation of periods of
meditation on personal matters and periods of mathematical reflection,
which has been going through my life for the last seven years, more
and more, very much like the unceasing up and down of waves, or like a
steady breathing going through my life, without any attempt any longer
of controlling it one way or the other. One common moving force surely
is the inborn curiosity -- a thirst for getting acquainted with the
juicy things of the inexhaustible world, whether they be the breathing
body of the beloved, or the evasive substance of one's own life, or
the much less evasive substance of mathematical structure and their
delicate interplay. This thirst in itself is of a nature quite
different of the ego's -- it is the thirst of life to know about
itself, a primal creative force which, one suspects, has been around
forever, long before a human ego -- a bundle mainly of fears, of
inhibitions and self-deceptions -- came into being. Still, I am aware
that the ego is strongly involved in the particular way in which the
creative force expresses itself, in my own life or anyone else's (when
this force is allowed to come into play at all\ldots). The motivations
behind any strong energy investment, and more particularly so when it
is an activity attached with any kind of social status or prestige,
are a lot more complex and ego-driven than one generally cares to
admit. True, ambition by itself is powerless for discovering or
understanding or perceiving anything substantial whatever, neither a
mathematical relationship nor the perfume of a flower. In moments of
work and of discovery, in any creative moments in life, as a matter of
fact, ambition is absent; the Artisan is a keen interest, which is
just one of the manifold aspect of love. What however pushes us so
relentlessly to work again, and so often causes our life and passion
gradually to dry off and become insensitive, even to the kindred
passion of a follow-being, or to the unsuspected beauties and
mysteries of the very field we are supposed to be plowing -- this
force is neither love nor keen interest for the things and beings in
this world. It is interesting enough though, and surely deserves a
close look!

I\pspage{194} thought I was starting a retrospective of six weeks of
scratchwork on homotopical algebra -- and it turns out to be a (very)
short retrospective rather of the up-and-down movement of my
mathematical interests and investments during the last
years. Doubtless, the very strongest attraction, the greatest
fascination goes with the ``new world'' of anabelian algebraic
geometry. It may seem strange that instead, I am indulging in this
lengthy digression on homotopical algebra, which is almost wholly
irrelevant I feel for the Galois-Teichmüller story. The reason is
surely an inner reluctance, an unreadiness to embark upon a long-term
voyage, well knowing that it is so enticing that I may well be caught
in this game for a number of years -- not doing anything else day and
night than making love with mathematics, and maybe sleeping and eating
now and then. I have gone through this a number of times, and at times
I thought I was through. Finally, I came to admit and to accept, two
years ago, I was not through yet -- this was during the months of
meditation after the ``long march through Galois theory'' -- which had
been, too, a wholly unexpected fit of mathematical passion, not to say
frenzy. And during the last weeks, just reflecting a little here and
there upon the Teichmüller-Lego game and its arithmetical
implications, I let myself be caught again by this fascination -- it
is becoming kind of clear now that I am going to finish writing up
those notes on algebra, almost like some homework that has got to be
done (anyhow I like to finish when I started something) -- and as soon
as I'm through with the notes, back to geometry in the long last!
Also, the idea is in the air for the last few months -- since I
decided to publish these informal notes on stacks or whatever it'll
turn out to be -- that I may well go on the same way, writing up and
publishing informal notes on other topics, including tame topology and
anabelian algebraic geometry. In contrast to the present notes, I got
heaps of scratchwork done on these in the years before -- in this
respect time is even riper for me to ramble ``publicly'' than on
stacks and homotopy theory!

From Y.\ who looked through a lot of literature on the subject, it
strikes me (agreeably of course) that nobody yet hit upon ``the''
natural presentation of the Teichmüller groupoids, which kind of
imposes itself quite forcibly in the set-up I let myself be guided
by. Technically speaking (and this will rejoice Ronnie Brown I'm
sure!), I suspect one main reason why this is so, is that people are
accustomed to working with fundamental groups and generators and
relations for these and stick to it, even in contexts when this is
wholly inadequate, namely when you get a clear\pspage{195} description
by generators and relations only when working simultaneously with a
whole bunch of base-points chosen with case -- or equivalently,
working in the algebraic context of \emph{groupoids}, rather than
groups. Choosing paths for connecting the base-points natural to the
situation to just one among them, and reducing the groupoid to a
single group, will then hopelessly destroy the structure and inner
symmetries of the situation, and result in a mess of generators and
relations no-one dares to write down, because everyone feels they
won't be of any use whatever, and just confuse the picture rather than
clarify it. I have known such perplexity myself a long time ago,
namely in van Kampen-type situations, whose only understandable
formulation is in terms of (amalgamated sums of) groupoids. Still,
standing habits of thought are very strong, and during the long march
through Galois theory, two years ago, it took me weeks and months
trying to formulate everything in terms of groups or ``exterior
groups'' (i.e., groups ``up to inner automorphism''), and finally
learning the lesson and letting myself be convinced progressively, not
to say reluctantly, that groupoids only would fit nicely. Another
``technical point'' of course is the basic fact (and the wealth of
intuitions accompanying it) that the Teichmüller groups are
fundamental groups indeed -- a fact ignored it seems by most
geometers, because the natural ``spaces'' they are fundamental groups
of are not topological spaces, but the modular ``multiplicities''
$M_{g,\nu}$ -- namely topoi! The ``points'' of these ``spaces'' are
just the structures being investigated (namely algebraic curves of
type $(g,\nu)$), and the (finite) automorphism groups of these
``points'' enter into the picture in a very crucial way. They can be
adequately chosen as part of the system of basic generators for the
Teichmüller groupoid $T_{g,\nu}$. The latter of course is essentially
(up to suitable restriction of base-points) just the fundamental
groupoid of $M_{g,\nu}$. It is through this interpretation of the
Teichmüller groups or groupoids that it becomes clear that the
profinite Galois group \GalQQ{} operates on the profinite completion
of these and of their various variants, and this (it turns out) in a
way respecting the manifold structures and relationships tying them
tightly together.

\bigbreak

\presectionfill\ondate{29.5.}\pspage{196}\par

% 69
\hangsection[Digression on six weeks' scratchwork: \emph{derivators},
and \dots]{Digression on six weeks' scratchwork:
  \texorpdfstring{\emph{derivators}}{derivators}, and integration of
  homotopy types.}\label{sec:69}%
Before resuming more technical work again, I would like to have a
short retrospective of the last six weeks' scratchwork, now lying on
my desk as a thickly bunch of scratchnotes, nobody but I could
possibly make any sense of.

The first thing I had on my mind has been there now for nearly twenty
years -- ever since it had become clear, in the SGA~5 seminar on
$L$-functions and apropos the formalism of traces in terms of derived
categories, that Verdier's set-up of derived categories was
insufficient for formulating adequately some rather evident situations
and relationships, such as the addition formula for traces, or the
multiplicative formula for determinants. It then became apparent that
the derived category of an abelian category (say) was too coarse an
object various respects, that it had to be complemented by similar
``triangulated categories'' (such as the derived category of a
suitable category of ``triangles'' of complexes, or the whole bunch of
derived categories of categories of filtered complexes of order $n$
with variable $n$), closely connected to it. Deligne and Illusie had
both set out, independently, to work out some set-up meeting the most
urgent requirements (Illusie's treatment in terms of filtered
complexes was written down and published in his thesis six years later
(Springer Lecture Notes N\textsuperscript{\b o}~239)).%
\scrcomment{\cite{Illusie1971,Illusie1972}}
While adequate for the main tasks then at hand, neither treatment was
really wholly satisfactory to my taste. One main feature I believe
making me feel uncomfortable, was that the extra categories which had
to be introduced, to round up somewhat a stripped-and-naked
triangulated category, were triangulated categories in their own
right, in Verdier's sense, but remaining nearly as stripped by
themselves as the initial triangulated category they were intended to
provide clothing for. In other words, there was a lack of inner
stability in the formalism, making it appear as very much provisional
still. Also, while interested in associating to an abelian category a
handy sequence of ``filtered derived categories'', Illusie made no
attempt to pin down what exactly the inner structure of the object he
had arrived at was -- unlike Verdier, who had introduced, alongside
with the notion of a derived category of an abelian category, a
general notion of triangulated categories, into which these derived
categories would fit. The obvious idea which was in my head by then
for avoiding such shortcomings, was that an abelian category \scrA{}
gave rise, not only to the single usual derived category $D(\scrA)$ of
Verdier, but also, for every type of diagrams,\pspage{197} to the
derived category of the abelian category of all \scrA-valued diagrams
of this type. In precise terms, for any small category $I$, we get the
category $D(\bHom(I,\scrA))$, depending functorially in a
contravariant way on $I$. Rewriting this category $D_\scrA(I)$ say,
the idea was to consider
\begin{equation}
  \label{eq:69.star}
  I\mapsto D_\scrA(I),\tag{*}
\end{equation}
possibly with $I$ suitably restricted (for instance to finite
categories, or to finite ordered sets, corresponding to finite
\emph{commutative} diagrams), as embodying the ``full'' triangulated
structure defined by \scrA. This of course at once raises a number of
questions, such as recovering the usual triangulated structure of
$D(\scrA)=D_\scrA(e)$ ($e$ the final object of \Cat) in terms of
\eqref{eq:69.star}, and pinning down too the relevant formal
properties (and possibly even extra structure) one had to assume on
\eqref{eq:69.star}. I had never so far taken the time to sit down and
play around some and see how this goes through, expecting that surely
someone else would do it some day and I would be informed -- but
apparently in the last eighteen years nobody ever was
interested. Also, it had been rather clear from the start that
Verdier's constructions could be adapted and did make sense for
non-commutative homotopy set-ups, which was also apparent in between
the lines in Gabriel-Zisman's\scrcomment{\cite{GabrielZisman1967}}
book on the foundations of homotopy theory, and a lot more explicitly
in Quillen's axiomatization of homotopical algebra. This
axiomatization I found very appealing indeed -- and right now still
his little book\scrcomment{\cite{Quillen1967}} is my most congenial
and main source of information on foundational matters of homotopical
algebra. I remember though my being a little disappointed at Quillen's
not caring either to pursue the matter of what exactly a
``non-commutative triangulated category structure'' (of the type he
was getting from his model categories) was, just contenting himself to
mumble a few words about existence of ``higher structure'' (then just
the Dold-Puppe sequences), which (he implies) need to be understood. I
felt of course that presumably the variance formalism
\eqref{eq:69.star} should furnish any kind of ``higher'' structure one
was looking for, but it wasn't really my business to check.

It still isn't, however I did some homework on \eqref{eq:69.star} --
it was the first thing indeed I looked at in these six weeks, and some
main features came out very readily indeed. It turns out that the main
formal variance property to demand on \eqref{eq:69.star}, presumably
even the only one, is that for a given map $f:I\to J$ on the indexing
categories of diagram-types $I$ and $J$, the corresponding functor
\[f^*: D(J) \to D(I)\]
\emph{should have both a left and a right adjoint}, say $f_!$ and
$f_*$. In case $J=e$, the two\pspage{198} functors we get from $D(I)$
to $D(e)=\scrD$ (the ``stripped'' triangulated category) can be viewed
as a \emph{substitute for taking, respectively, direct and inverse
  limits in \scrD} (for a system of objects indexed by $I$),
\emph{which in the usual sense don't generally exist in \scrD} (except
just finite sums and products). These operations admit as important
special cases, when $I$ is either one of the two mutually dual
categories
\begin{equation}
  \label{eq:69.starstar}
  \begin{tikzcd}[baseline=(O.base),column sep=small,row sep=tiny]
    & b \\ |[alias=O]| a\ar[ur]\ar[dr] & \\ & c
  \end{tikzcd}
  \quad\text{or}\quad
  \begin{tikzcd}[baseline=(O.base),column sep=small,row sep=tiny]
    & b\ar[dl] \\ |[alias=O]| a & \\ & c\ar[ul]
  \end{tikzcd}\quad,\tag{**}
\end{equation}
the operation of (binary) amalgamated sums or fibered products, and
hence also of taking ``cofibers'' and ``fibers'' of maps, in the sense
introduced by Cartan-Serre in homotopy theory about thirty years
ago. I also checked that the two mutually dual Dold-Puppe sequences
follow quite formally from the set-up. One just has to fit in a
suitable extra axiom to ensure the usual exactness properties for
these sequences.

Except in the commutative case when starting with an abelian category
as above, I did not check however that there is indeed such ``higher
variance structure'' in the usual cases, when a typical ``triangulated
category'' in some sense or other turns up, for instance from a model
category in Quillen's sense. What I did check though in this last
case, under a mild additional assumption which seems verified in all
practical cases is the existence of the operation $f_!=\int_I$
(``integration'') and $f_*=\prod_I$ (cointegration) for the special
case $f:I\to e$, when $I$ is either of the two categories
\eqref{eq:69.starstar} above. I expect that working some more, one
should get under the same assumptions at least the existence of $f_!$
and $f_*$ for any map $f:I\to J$ between finite ordered sets.

My main interest of course at present is in the category \Hot{}
itself, more generally in $\HotOf(\scrW)=\scrW^{-1}\Cat$, where
\scrW{} is a ``basic localizer''. More generally, if $(M,W)$ is any
modelizer say, the natural thing to do, paraphrasing
\eqref{eq:69.star}, is for any indexing category $I$ to endow
$\bHom(I,M)$ with the set of arrows $W_I$ defined by componentwise
belonging to $W$, and to define
\[ D_{(M,W)}(I) = D(I) \eqdef W_I^{-1}\bHom(I,M),\]
with the obvious contravariant dependence on $I$, denoted by $f^*$ for
$f:I\to J$. The question then arises as to the existence of left and
right adjoints, $f_!$ and $f_*$. In case we take $M=\Cat$, the
existence of $f_!$ goes through with amazing smoothness: interpreting
a ``model'' object of $\bHom(I,\Cat)$, namely a functor
\[I\to \Cat\]
in\pspage{199} terms of a cofibered category $X$ over $I$
\[p: X\to I,\]
and assuming for simplicity $f$ cofibering too, $f_!(X)$ is just $X$
itself, the total category of the cofibering, viewed as a (cofibered)
category over $J$ by using the functor $g=f\circ p$! This applies for
instance when $J$ is the final category, and yields the operation of
``integration of homotopy types'' $\int_I$, in terms of the total
category of a cofibered category over $I$. If we want to rid ourselves
from any extra assumption on $f$, we can describe $D(I)$ (up to
equivalence) in terms of the category $\Cat_{/I}$ of categories $X$
over $I$ (not necessarily cofibered over $I$), $W_I$ being replaced by
the corresponding notion of ``\emph{\scrW-equivalences relative to
  $I$}'' for maps $u:X\to Y$ of objects of \Cat{} over $I$, by which
we mean a map $u$ such that the localized maps
\[u_{/i}: X_{/i} \to Y_{/i}\]
are in \scrW, for any $i$ in $I$. Regarding now any category $X$ over
$I$ as a category over $J$ by means of $f\circ p$, this is clearly
compatible with the relative weak equivalences $\scrW_I$ and
$\scrW_J$, and yields by localization the looked-for functor $f_!$.

This amazingly simple construction and interpretation of the basic
$f_!$ and $\int_I$ operations is one main reward, it appears, for
working with the ``basic localizer'' \Cat, which in this occurrence,
as in the whole test- and asphericity story, quite evidently deserves
its name. It has turned out since that in some other respects -- for
instance, paradoxically when it comes to the question of the
relationship between this lofty integration operation, and true honest
amalgamated sums -- the modelizers \Ahat{} associated to test
categories $A$ (namely the so-called ``elementary modelizers'') are
more convenient tools than \Cat. Thus, it appears very doubtful still
that \Cat{} is a ``model category'' in Quillen's sense, in any
reasonable way (with \scrW{} of course as the set of ``weak
equivalences''). I finally got the feeling that a good mastery of the
basic aspects of homotopy types and of basic relationships among
these, will require mainly great ``aisance''\scrcomment{``aisance''
  here could be ``ease'' or ``fluency''} in playing around with a
number of available descriptions of homotopy types by models, no one
among which (not even by models in \Cat, and surely still less by
semisimplicial structures) being adequate for replacing all others.

As for the $f_*$ and cointegration $\prod_I$ operations among the
categories $D(I)$, except in the very special case noted above
(corresponding to fibered products), I did not hit upon any
ready-to-use candidate for it, and I doubt there is any. I do believe
the operations exist indeed, and I even have\pspage{200} in mind a
rather general condition on a pair $(M,W)$ with $W\subset\Fl(M)$, for
both basic operations $f_!$ and $f_*$ to exist between the
corresponding categories $D_{(M,W)}(I)$ -- but to establish this
expectation may require a good amount of work. I'll come back upon
these matters in due course.

There arises of course the question of giving a suitable name to the
structure $I\to D(I)$ I arrived at, which seems to embody at least
some main features of a satisfactory notion of a ``triangulated
category'' (not necessarily commutative), gradually emerging from
darkness. I have thought of calling such a structure a
``\emph{derivator}'', with the implication that its main function is
to furnish us with a somehow ``complete'' bunch (in terms of a
rounded-up self-contained formalism) of categories $D(I)$, which are
being looked at as ``\emph{derived categories}'' in some sense or
other. The only way I know of for constructing such a derivator, is as
above in terms of a pair $(M,W)$, submitted to suitable conditions for
ensuring existence of $f_!$ and $f_*$, at least when $f$ is any map
between finite ordered sets. We may look upon $D(I)$ as a refinement
and substitute for the notion of family of objects of $D(e)=D_0$
indexed by $I$, and the integration and cointegration operations from
$D(I)$ to $D_0$ as substitutes (in terms of these finer objects) of
direct and inverse limits in $D_0$. When tempted to think of these
latter operations (with values in $D_0$) as the basic structures
involved, one cannot help though looking for the same kind of
structure on any one of these subsidiary categories $D(I)$, as these
are being thought of as derived categories in their own right. It then
appears at once that the ``more refined substitutes'' for $J$-indexed
systems of objects of $D(I)$ are just the objects of $D(I\times J)$,
and the corresponding integration and cointegration operations
\[ D(I\times J)\to D(I)\]
are nothing but $p_!$ and $p_*$, where $p:I\times J\to I$ is the
projection. Thus, one is inevitably conducted to look at operations
$f_!$ and $f_*$ instead of merely integration and cointegration --
thus providing for the ``inner stability'' of the structure described,
as I had been looking for from the very start.

The notion of integration of homotopy types appears here as a natural
by-product of an attempt to grasp the ``full structure'' of a
triangulated category. However, I had been feeling the need for such a
notion of integration of homotopy types for about one or two years
already (without any clear idea yet that this operation should be one
out of two\pspage{201} main ingredients of a (by then still very
misty) notion of a triangulated category of sorts). This feeling arose
from my ponderings on stratified structures and the ``screwing
together'' of such structures in terms of simple building blocks
(essentially, various types of ``tubes'' associated to such
structures, related to each other by various proper maps which are
either inclusion or -- in the equisingular case at any rate -- fiber
maps). This ``screwing together operation'' could be expressed as
being a direct limit of a certain finite system of spaces. In the
cases I was most interested in (namely the Mumford-Deligne
compactifications $M\uphat_{g,\nu}$ of the modular multiplicities
$M_{g,\nu}$), these spaces or ``tubes'' have exceedingly simple
homotopy types -- they are just $K(\pi,1)$-spaces, where each $\pi$ is
a Teichmüller-type discrete group (practically, a product of usual
Teichmüller groups). It then occurred to me that the whole homotopy
type of $M\uphat_{g,\nu}$, or of any locally closed union of strata,
or (more generally still) of ``the'' tubular neighborhood of such a
union in any larger one, etc.\ -- that all these homotopy types should
be expressible in terms of the given system of spaces, and more
accurately still, just in terms of the corresponding system of
fundamental groupoids (embodying their homotopy types). In this
situation, what I was mainly out for, was precisely an accurate and
workable description of this direct system of groupoids (which could
be viewed as just one section of the whole ``Teichmüller tower'' of
Teichmüller groupoids\ldots). Thus, it was a rewarding extra feature
of the situation (by then just an expectation, as a matter of fact),
that such a description should at the same time yield a ``purely
algebraic'' description of the homotopy types of all the spaces
(rather, multiplicities, to be wholly accurate) which I could think of
in terms of the natural stratification of $M\uphat_{g,\nu}$. There was
an awareness that this operation on homotopy types could not be
described simply in terms of a functor $I\to\Hot$, where $I$ is the
indexing category, that a functor $i\mapsto X_i: I\to M$ (where $M$ is
some model category such as \Spaces{} or \Cat) should be available in
order to define an ``integrated'' homotopy type $\int X_i$. This
justified feeling got somewhat blurred lately, for a little while, by
the definitely unreasonable expectation that finite limits should
exist in \Hot{} after all, why not! It's enough to have a look though
(which probably I did years ago and then forgot in the meanwhile) to
make sure they don't\ldots

Whether or not this notion of ``integration of homotopy types'' is
more or less well known already under some name or other, isn't quite
clear to me. It isn't familiar to Ronnie Brown visibly, but it seems
he heard about such a kind of thing, without his being specific about
it. It\pspage{202} was the episodic correspondence with him which
finally pushed me last January to sit down for an afternoon and try to
figure out what there actually was, in a lengthy and somewhat rambling
letter to Illusie (who doesn't seem to have heard at all about such
operations). This preliminary reflection proved quite useful lately,
I'll have to come back anyhow upon some of the specific features of
integration of homotopy types later, and there is not much point
dwelling on it any longer at present.

% 70
\hangsection[Digression on scratchwork (2): cohomological
\dots]{Digression on scratchwork
  \texorpdfstring{{\normalfont(2)}}{(2)}: cohomological properties of
  maps in \texorpdfstring{\Cat}{(Cat)} and in
  \texorpdfstring{\Ahat}{Ahat}. Does any topos admit a ``dual'' topos?
  Kan fibrations rehabilitated.}\label{sec:70}%
This was a (finally somewhat length!) review of ponderings which
didn't take me more than just a few days, because it was about things
some of which were on my mind for a long time indeed. It took a lot
more work to try to carry through the standard homotopy constructions,
giving rise to the Dold-Puppe sequences, within the basic modelizer
\Cat. Most of the work arose, it now seems to me, out of a block I got
(I couldn't tell why) against the Kan-type condition on complexes, so
I tried hard to get along without anything of the sort. I kind of
fooled myself into believing that I was forced to do so, because I was
working in an axiomatic set-up dependent upon the ``basic localizer''
\scrW, so the Kan condition wouldn't be relevant anyhow. The main
point was to get, for any map $f:X\to Y$ in \Cat, a factorization
\[ X \xrightarrow i Z \xrightarrow p Y,\]
where $p$ has the property that base-change by $p$ transforms weak
equivalences into weak equivalences (visibly a Serre-fibration type
condition), and $i$ satisfies the dual condition with respect to
co-base change; and moreover where either $p$, or $i$ can be assumed
to be in \scrW, i.e., to be a weak equivalence. (This, by the way, is
the extra condition on a pair $(M,W)$ I have been referring to above
(page \ref{p:200}, for (hopefully) getting $f_!$ and $f_*$
operations.) At present, I do not yet know whether such factorization
always exists for a map in \Cat, without even demanding that either
$i$ or $p$ should be in \scrW.

I first devoted a lot of attention to Serre-type conditions on maps in
\Cat, which turned out quite rewarding -- with the impression of
arriving at a coherent and nicely auto-dual picture of cohomology
properties of functors, i.e., maps in \Cat, as far as these were
concerned with \emph{base change behavior} (and \emph{not} co-base
change). Here I was guided by work done long ago to get the étale
cohomology theory off the ground, and where the two main theorems
achieving this aim were precisely two theorems of commutation of
``higher direct images'' $\mathrm R^ig_*$ with respect to base-change
by a map $h$ -- namely, it is OK when either $g$ is \emph{proper}, or
$h$ is \emph{smooth}. It was rather natural then to introduce the
notion of\pspage{203} (cohomological) \emph{smoothness} and (coh.)
\emph{properness} of a map in \Cat, by the obvious base-change
properties. It turned out that these could be readily characterized by
suitable asphericity conditions, which are formally quite similar to
the well-known valuative criterion for a map of finite type between
noetherian schemes to be ``universally open'' (which can be viewed as
a ``purely topological'' variant of the notion of smoothness), resp.\
``universally closed'', or rather, more stringently, proper. These
conditions, moreover, are trivially satisfied when $f: X\to Y$ turns
$X$ into a category over $Y$ which is fibered over $Y$ (i.e.,
definable in terms of a contravariant pseudo-functor $Y\op\to\Cat$),
resp.\ cofibered over $Y$ (namely, definable in terms of a
pseudo-functor $Y\to\Cat$). If we call such maps in \Cat{}
``fibrations'' and ``cofibrations'' (very much in conflict, alas, with
Quillen's neat set-up of fibrations-cofibrations!), it turns out that
\emph{fibrations are smooth}, \emph{cofibrations are proper}. This is
all the more remarkable, as the notions of fibered and cofibered
categories were introduced with a view upon ``large categories'', in
order to pin down some standard properties met in all situations of
``base change'' (and later the dual situation of co-base change) --
the main motivation for this being the need to formulate with a
minimum of precision a set-up for ``descent techniques'' in algebraic
geometry. (These techniques, as well as the cohomological base change
theorems, make visibly a sense too in the context of analytic spaces
say, or of topological spaces, but they don't seem to have been
assimilated yet by geometers outside of algebraic geometry.) That
these typically ``general nonsense'' notions should have such precise
topological implications came as a complete surprise! As a
consequence, a bifibration (namely a map which is both a fibration and
a cofibration) is smooth and proper and hence a (cohomological)
``Serre fibration'', for instance the sheaves $\mathrm R^if_*(F)$,
when $F$ is a constant abelian sheaf on $X$ (i.e., a constant abelian
group object in $X\uphat$), namely $y\mapsto\mathrm H^i(X_{/y},F)$,
are \emph{local systems} on $Y$, i.e., factor through the fundamental
groupoid of $Y$.

A greater surprise still was the \emph{duality between the notions of
  properness and smoothness:} just as a map $f:X\to Y$ in \Cat{} is a
cofibration if{f} the ``dual'' map $f\op:X\op\to Y\op$ is a fibration,
it turns out that \emph{$f$ is proper if{f} $f\op$ is smooth}. This
was a really startling fact, and it caused me to wonder, in the
context of more general topoi than just those of the type $X\uphat$,
whether there wasn't \emph{a notion of duality generalizing the
  relationship between two topoi $X\uphat$ and $X\upvee \eqdef
  (X\op)\uphat$}. Indeed, these two categories of sheaves can be
described intrinsically, one in terms of the other, up to equivalence,
by a natural pairing\pspage{204}
\[X\uphat \times X\upvee \to \Sets\]
commuting componentwise with (small) direct limits, and inducing an
equivalence between either factor with the category of ``co-sheaves''
on the other, namely covariant functors to \Sets{} commuting with
direct limits. But it isn't at all clear, starting with an arbitrary
topos \scrA{} say, whether the category $\scrA'$ of all cosheaves on
\scrA{} is again a topos, and still less whether \scrA{} can be
recovered (up to equivalence) in terms of $\scrA'$ as the category of
all cosheaves on $\scrA'$.

To come back though to the factorization problem raised above (p.\
\ref{p:202}), the main trouble here is that, except the case of an
isomorphism $i: X\tosim Z$, I was unable to pin down a single case of
a map $i$ in \Cat{} such that co-base change by $i$ transforms weak
equivalences (in the usual sense say) into weak equivalences. One
candidate I had in mind, the so-called ``\emph{open immersions}'',
namely functors $i:X\to Z$ inducing an isomorphism between $X$ and a
``sieve'' (or ``crible'') in $Z$ (corresponding to an open sub-topos
of $Z\uphat$), and dually the ``closed immersions'', finally have
turned out delusive -- a disappointment maybe, but still more a big
relief to find out at last how the score was! Almost immediately in
the wake of this negative result in \Cat, and in close connection with
the fairly well understood $\int$ substitute for amalgamated sums in
\Cat, came the big compensation valid in any category $\Ahat$, namely
the fact that for a cocartesian square
\[\begin{tikzcd}
  Y\ar[r,"i"] \ar[d,swap,"g"] & X\ar[d,"f"] \\
  Y'\ar[r,"i'"] & X'
\end{tikzcd}\]
in \Ahat, where $i$ is a monomorphism, if $i$ (resp.\ $g$) is in
\scrWA, so is $i'$ (resp.\ $f$). This implies that \emph{co-base
  change by a monomorphism in \Ahat{} transform weak equivalences into
  weak equivalences}. The common main fact behind these statements is
that for a diagram as above (without assuming $i$ nor $g$ to be in
\scrWA), $X'$ can be interpreted up to weak equivalence as the
``integral'' of the diagram
\[\begin{tikzcd}[baseline=(O.base),sep=small]
  Y\ar[r] \ar[d] & X \\
  |[alias=O]| Y' &
\end{tikzcd},\]
more accurately, the natural map in \Cat
\[\int \begin{tikzcd}[cramped,sep=small]
  A_{/Y}\ar[r]\ar[d] & A_{/X} \\ A_{/Y'} &
\end{tikzcd} \longrightarrow\; A_{/X'}\]
is in \scrW.

The corresponding statement in \Cat{} itself, even if $i$ is
supposed\pspage{205} to be an open immersion in \Cat{} say (or a
closed immersion, which amounts to the same by duality), is definitely
false, in other words: while the modelizer \Cat{} allows for a
remarkably simple description of integration of homotopy types, as
seen in the previous section, in the basic case however of an
integration\scrcomment{this integral is set inline in the typescript}
\[\int \begin{tikzcd}[cramped,sep=small]
  A_{/Y}\ar[r,"i"]\ar[d,"g"] & A_{/X} \\ A_{/Y'} &
\end{tikzcd}\]
(corresponding  to ``amalgamated sums''), and even when $i:Y\to X$ is
an open or closed immersion, this operation does definitely \emph{not}
correspond to the operation of taking the amalgamated sum $X'$ in
\Cat. It does though when $X\to X'$ is smooth resp.\ proper, for
instance if it is a fibration resp.\ a cofibration, and this in fact
implies the positive result in \Ahat{} noted above. This condition
moreover is satisfied if $g: Y\to Y'$ is equally an open resp.\ closed
immersion, in which case the situation is just the one of an ambient
$X'$, and two open resp.\ closed subobjects $X$ and $Y'$, with
intersection $Y$. This is a useful result, but wholly insufficient for
the factorization problem we were after in \Cat, with a view of
performing the standard homotopy constructions in \Cat{} itself. It
may be true still that if $i:Y\to X$ is not only an open or closed
immersion, but a weak equivalence as well, that then $i':Y'\to X'$ is
equally a weak equivalence, or what amounts to the same, that $X'$ can
be identified up to weak equivalence with the homotopy integral (which
indeed, up to weak equivalence, just reduces to $Y'$); but I have been
unable so far to clear up this matter. If true, this would be quite a
useful result, but still insufficient it seems in order to carry out
the standard homotopy constructions in \Cat{} itself.

To sum up, the main drawback of \Cat{} as a modelizer is that, except
in very special cases which are just too restricted, amalgamated sums
in \Cat{} don't have a reasonable meaning in terms of homotopy
operations -- whereas in a category \Ahat, a topos indeed where
therefore amalgamated sums have as good exactness properties as if
working in \Sets, these amalgamated sums (for a two-arrow diagram with
one arrow a monomorphism) \emph{do} have a homotopy-theoretic
meaning. This finally seems to force us, in order to develop some of
the basic structure in $\HotOf(\scrW)$, to leave the haven of the
basic modelizer \Cat, and work in an elementary modelizer \Ahat{}
instead, where $A$ is some \scrW-test category. This then brought me
back finally to the question whether these modelizers are \emph{closed
  model categories} in Quillen's sense, when we take of course for
``weak equivalences'' \scrWA, and moreover as ``cofibrations'' (in the
sense of Quillen's set-up) just the monomorphisms. Relying heavily
upon the result on monomorphisms in \Ahat{} stated above, it seems to
come out that we do get a closed model category indeed -- and even a
simplicial model category, if we are out for this. There is still a
cardinality question to be\pspage{206} settles to get the Quillen
factorizations in the general case, but this should not be too serious
a difficulty I feel. What however makes me still feel a little unhappy
in all this, is rather that I did not get a direct proof for an
elementary modelizer being a closed model category -- I finally have
to make a reduction to the known case of semisimplicial complexes,
settles by Quillen in his notes. This detour looks rather artificial
-- it is the first instance, and presumably the last one, where the
theory I am digging out seems to depend on semi-simplicial techniques,
techniques which moreover I don't really know and am not really eager
to swallow. It's just a prejudice maybe, a block maybe against the
semi-simplicial approach which I never really liked nor assimilated --
but I do have the feeling that the more refined and specific
semi-simplicial techniques and notions (such as minimal fibrations,
used in Quillen's proof, alas!) are irrelevant for an understanding of
the main structures featuring homotopy theory and homotopical
algebra. As for the notion of a Kan complex or a Kan fibration --
namely just a ``fibration'' in Quillen's axiomatic set-up, which I was
finally glad to find, ``ready for use'' -- I came to convince myself
at last that it was a basic notion indeed, and it was no use trying to
bypass it at all price. Thus, I took to the opposite, and tried to pin
down a Quillen-type factorization theorem, and his characteristic
seesaw game between right and left lifting properties, in as great
generality as I could manage.

\bigbreak
\presectionfill\ondate{30.5.}\par

% 71
\hangsection[Working program and rambling questions (group objects
\dots]{Working program and rambling questions \texorpdfstring{\textup
    (}{(}group objects as models, Dold-Puppe
  theorem\ldots\texorpdfstring{\textup )}{)}.}\label{sec:71}%
The scratchwork done since last month has of course considerably
cleared up the prospects of my present pre-stacks reflection on
homotopy models, on which I unsuspectingly embarked three months
ago. I would like to sketch a provisional working program for the
notes still ahead.

\namedlabel{it:71.A}{A)} Write down at last the story of asphericity
structures and canonical modelizers, as I was about to when I
interrupted the notes to do my scratchwork.

\namedlabel{it:71.B}{B)} Study the basic modelizer \Cat, and the
common properties of elementary modelizers \Ahat, with a main emphasis
upon base change and co-base change properties, and upon Quillen-type
factorization questions. Here it will be useful to dwell somewhat on
the ``homotopy integral'' variant of taking amalgamated sums in \Cat,
on the analogous constructions for topoi, and how these compare to the
usual amalgamated sums, including the interesting case of topological
spaces. It turns out that the homotopy\pspage{207} integral variant
for amalgamated sums is essentially characterized by a Mayer-Vietoris
type long exact sequence for cohomology, and the cases when the
homotopy construction turns out to be equivalent to usual amalgamated
sums, are just those when such a Mayer-Vietoris sequence exists for
the latter. An interesting and typical case is for topological spaces,
taking the amalgamated sum for a diagram
\[\begin{tikzcd}[baseline=(O.base),sep=small]
  Y\ar[r,hook,"i"] \ar[d,swap,"g"] & X \\
  |[alias=O]| Y' &
\end{tikzcd},\]
when $i$ is a closed immersion and $g$ is proper (which is also the
basic type of amalgamations which occur in the ``unfolding'' of
stratified structures).

In the course of the last weeks' reflections, there has taken place
also a substantial clarification concerning the relevant properties of
a basic localizer \scrW{} and how most of these, including strong
saturation of \scrW, follow from just the first three (a question which
kept turning up like a nuisance throughout the notes!). This should be
among the very first things to write down in this part of the
reflections, as \scrW{} after all is \emph{the} one axiomatic data
upon which the whole set-up depends.

\namedlabel{it:71.C}{C)}\enspace A reflection on the main common features of
the various contexts met with so far having a ``homotopy theory''
flavor, with a hope to work out at least some of the main features of
an all-encompassing new structure, along the lines of Verdier's
(commutative) theory of derived categories and triangulated
categories. The basic idea here, for the time being, seems to be the
notion of a \emph{derivator}, which should account for all the kind of
structure dealt with in Verdier's set-up, as well as in Deligne's and
Illusie's later elaborations. There seems to be however some important
extra features which are not accounted for by the mere derivator, such
as external $\Hom$'s with values in \Hot{} or some closely related
category, and the formalism of basic invariants (such as $\pi_i$,
$\mathrm H_i$ or $\mathrm H^i$), with values in suitable categories
(often abelian ones), which among others allow to check weak
equivalence. Such features seem to be invariably around in all cases I
know of, and they need to be understood I feel.

Coming back to \Hot{} itself and to modelizers $(M,W)$ giving rise to
it, there is the puzzling question about when exactly can we assert
that taking group objects of $M$ and weak equivalences between these
(namely group-object homomorphisms which are also in $W$), we get by
localization a category equivalent to the category of pointed
$0$-connected homotopy types. This is a well-known basic fact when we
take as models semisimplicial complexes or (I guess) topological
spaces -- a fact closely connected to the game of
associating\pspage{208} to any topological group its
``\emph{classifying space}'', defined ``up to homotopy''. I suspect
the same should hold in any elementary modelizer \Ahat, $A$ a test
category, at least in the ``strict'' case, namely when \Ahat{} is
totally aspheric, i.e., the canonical functor $\Ahat\to\Hot$ commutes
to finite products. The corresponding statement for the basic
localizer \Cat{} itself is definitely false. Group objects in \Cat{}
are indeed very interesting and well-known beings (introduced, I
understand from Ronnie Brown, by Henry Whitehead long time ago, under
the somewhat misleading name of ``crossed modules''), yet they embody
not arbitrary pointed $0$-connected homotopy types $X$, but merely those
for which $\pi_i(X)=0$ for $i>2$. Thus, we get only $2$-truncated
homotopy types -- and presumably, starting with $n\mathrm{-Cat}$
instead of \Cat{} as a modelizer, we then should get $(n+1)$-truncated
homotopy types. This ties in with the observation that taking group
objects either in \Cat, or in the full subcategory
$(\mathrm{Groupoids})$ of the latter, amounts to the same -- and
similarly surely for $(n\mathrm{-Cat})$; on the other hand it has been
kind of clear from the very beginning of this reflection that at any
\emph{finite} level, groupoids and $n$-groupoids ($n$ finite) will
only yield \emph{truncated} homotopy types.

A related intriguing question is when exactly does a modelizer $(M,W)$
give rise to a \emph{Dold-Puppe theorem} -- namely when do we get an
actual \emph{equivalence} between the category of \emph{abelian} group
objects of $M$, and the category of chain complexes of abelian groups?
The original statement was in case $M=\Simplexhat=$ semisimplicial
complexes, and doubtlessly it was one main impetus for the sudden
invasion of homotopy and cohomology by semisimplicial calculus -- so
much so it seems that for many people, ``homotopy'' has become
synonymous to ``semisimplicial algebra''. The impression that
semi-simplicial complexes is the God-given ground for doing homotopy
and even cohomology, comes out rather strong also in Quillen's
foundational notes, and in Illusie's thesis. Still, there are too some
cubical theory chaps I heard, who surely must have noticed long ago
that the Dold-Puppe theorem is valid equally for cubical complexes (I
could hardly imagine that it possibly couldn't). Now it turns out that
semisimplicial and cubical complexes are part of a trilogy, together
with so-called ``\emph{hemispherical complexes}'',\scrcomment{I guess
  today we call these ``globular sets''} which look at lot simpler
still, with just two boundary operators and one degeneracy in each
dimension. They can be viewed as embodying the ``primitive structure''
of an \oo-groupoid, the boundary operators being the ``source'' and
``target'' maps, and the degeneracy the map associating to any
$i$-object the corresponding ``identity''. I hit upon this structure
among the first examples of test categories and elementary modelizers,
and have been told since by Ronnie Brown that he has already known for
a while these models, under\pspage{209} the similar name of ``globular
complexes''). Roughly speaking, it can be said that the three types of
complexes correspond to three ``series'' of \emph{regular cellular
  subdivisions} of all spheres $S_n$ where the two-dimensional pieces
are respectively (by increasing order of ``intricacy'') bigons,
triangles, and squares. It shouldn't be hard to show these are the
only series of regular cellular subdivisions of all spheres (one in
each dimension) such that for any cell of such subdivision, the
induced subdivision of the bounding sphere should still be in the
series (up to isomorphism). The existence moreover of suitable
``degeneracy'' maps, which merit a careful general definition in this
context of cellular subdivisions of spheres, is an important common
extra feature of the three basic contexts, whose exact significance I
have not quite understood still. To come back to Dold-Puppe, sure
enough it is still valid in the hemispherical context. Writing down
the equivalence of categories in explicit terms comes out with
baffling simplicity. I wrote it down without even looking for it, in a
letter to Ronnie Brown, while explaining in a PS the ``yoga'' of
associating to a chain complex of abelian groups an \oo-groupoid with
additive structure (a so-called ``Picard category'' but within the
context of \oo-categories or \oo-groupoids, rather than usual
categories).

On the other hand, taking multicomplexes instead of simple ones, which
still can be interpreted as working in a category \Ahat{} for a
suitable test category $A$ (namely, a product category of categories
of the types $\Square$, $\Simplex$, and $\Globe$),
it is clear that Dold-Puppe's theorem as originally stated is no
longer true in these: in such case the category of abelian group
objects of \Ahat{} is equivalent to a category of \emph{multiple}
chain complexes. This shows that definitely, among all possible
elementary modelizers, the three in our trilogy are distinguished
indeed, as giving rise to a Dold-Puppe theorem. A thorough
understanding of this theorem would imply, I feel, an understanding of
which exactly are the modelizers, or elementary modelizers at any rate
giving rise to such a theorem. I wouldn't be too surprised if it
turned out that the three we got are the only ones, up to equivalence.

Another common feature of these modelizers, is that they allow for a
sweeping computational description of cohomology (or homology)
invariants in terms of the so-called ``boundary operations''. This is
visibly connected to the (strongly intuitive) tie between these kinds
of models, and cellular subdivisions of spheres. However, at this
level (unlike the Dold-Puppe story) the regularity feature of the
cellular subdivisions we got, and the fact that we allow for just one
(up to isomorphism) in each dimension,\pspage{210} seems to be
irrelevant. It might be worth while to write down with care what
exactly is needed, in order to define, in terms of a bunch of cellular
structures of spheres, a corresponding test-category (hopefully even a
strict one), and a way of computing in terms of boundary operators the
cohomology of the corresponding models. The more delicate point here
may be that if we really want to get an actual test category, not just
a weak one, we should have ``enough'' degeneracy maps between our
cellular structures, which might well prove an extremely stringent
requirement. Again, it will be interesting if we can meet it otherwise
than just sticking to our trilogy.

\namedlabel{it:71.D}{D)} ``Back to topoi''. They have been my main
intuitive leading thread in the reflections so far, but have remained
somewhat implicit most times. When working with categories $X$ as
models for homotopy types, we have been thinking in reality of the
associated topos $X\uphat$. In the same way, when relativizing over an
object $I$ of \Cat{} the construction of \Hot{} as a derived category,
namely working with categories over $I$ as models for a derived
category $D(I)$, the leading intuition again has been to look at $X$
over $I$ as the topos $X\uphat$ over $I\uphat$. This gives strong
suggestions as to defining $D(I)$ not only in terms of a given
category $I$, but also in terms of an arbitrary topos $T$ as well,
standing for $I\uphat$, and look to what extent the $f^*, f_!, f^*$
formalism of derivators extends to the case when $f$ is a ``map''
between topoi. The definition of some $D(T)$ for a topos $T$ can be
given in a number of ways it seems. As far as I know, the only one
which has been written and used so far consists in taking a suitable
derived category of the category of semisimplicial objects of
$T$. this is done in Illusie's thesis, where there is no mention
though of $f_!$ and $f_*$ functors -- which, maybe, should rather be
written $\mathrm Lf_!$ and $\mathrm Rf_*$, suggesting they are
respectively left and right derived functors of the more familiar
$f_!$ and $f_*$ functors for sheaves (the left and right adjoints of
the inverse image functor $f^*$ for sheaves of sets). One would
expect, at best, $\mathrm Lf_1$ to exist when $f_!$ itself does,
namely when $f^*$ commutes not only with finite inverse limits, but
with infinite products as well. As for $\mathrm Rf_*$, whereas there
is no problem for the existence of $f_*$ itself, already in the case
of a morphism of topoi coming from a map in \Cat, a map say between
finite ordered sets, the existence of $\mathrm Rf_*$ has still to be
established. Thus, presumably I'll content myself with writing down
and comparing a few tentative definitions of $D(T)$ and make some
reasonable guesses as to its variances. Intuitively, the objects of
$D(T)$ may be viewed as ``sheaves of homotopy types over $T$'', or
``relative homotopy types over $T$'', or ``non-commutative chain
complexes over $T$ up to quasi-isomorphism''. As in the case when $T$
is\pspage{211} the final (one point) topos, $D(T)$ is just the
homotopy category \Hot, there must be of course a vast variety of ways
of defining $D(T)$ in terms of model categories, and I would like to
review some which seem significant. In fact, one cannot help but
looking at two mutually dual groups of models categories, giving rise
to (at least) two definitely non-equivalent derived categories $D(T)$
and $D'(T)$ say, which, in case $T=I\uphat$, would correspond to
$D(I)$ and $D(I\op)$. A typical model category for the former is made
up with Illusie's semisimplicial sheaves; a typical model category for
$D'(T)$, on the other hand, should be made up with $1$-stacks on $T$
(in Giraud's sense), for a suitable notion of weak equivalence
between these.

Maybe the most natural models of all, in this context, for ``relative
homotopy types over the topos $T$'', should be \emph{topoi} $X$ over
$T$ (generalizing the categories over an ``indexing category''
$I$). The only trouble with this point of view though is that the best
we can hope for, in term's of Illusie's $D(T)$ say, is that a topos
$X$ over $T$ gives rise to a pro-object of $D(T)$, which needs not
come from an object of $D(T)$ itself, i.e., is not necessarily
``essentially constant''. This brings us back to the simpler and still
more basic question of associating a pro-homotopy type, namely a
pro-object of \Hot, to any topos $X$ -- namely back to the \v
Cech-Verdier-Artin-Mazur construction. This has been handled so far
using the semisimplicial models for \Hot, I suspect though that using
\Cat{} as a modelizer will give a more elegant treatment, as already
suggested earlier. Whichever way we choose to get the basic functor
\[\mathrm{Topoi} \to \Pro\Hot,\]
this functor will allow us, given a basic localizer \scrW, to define
\scrW-equivalences between topoi as maps which become isomorphisms
under the composition of the basic functor above, and the canonical
functor
\[\Pro\Hot \to \Pro(\HotOf(\scrW))\]
deduced from the localization functor
\[\Hot\to(\HotOf(\scrW)).\]
(It turns out, using Quillen's theory, that usual weak equivalences is
indeed the finest of all possible basic localizers \scrW, hence
$\HotOf(\scrW)$ is indeed a localization of \Hot.)

Maybe it is not too unreasonable to expect that all, or most homotopy
constructions, involving a topos $T$, can be expressed replacing
$T$ by its image in $\Pro\Hot$, or in $\Pro(\HotOf(\scrW))$ if the
construction are relative to a given basic localizer \scrW. The very
first example one would like\pspage{212} to look up in this respect,
is surely $D(T)$, defined say à la Illusie.

In the present context, the main point of the property of \emph{local
  asphericity} for a topos (cf.\ section \ref{sec:35}) is that the
corresponding prohomotopy type is essentially constant, i.e., the
topos defines an actual homotopy type. Thus, locally aspheric topoi
and weak equivalences between these should be eligible models for
homotopy types (more accurately, make up a modelizer), just as the
basic modelizer \Cat{} contained in it. The corresponding statements,
when introducing a basic localizer \scrW, should be equally valid. One
might expect, too, a relative variant for the notion of local
asphericity or \scrW-asphericity, in case of a topos $X$ over a given
base topos $T$ (which we may have to suppose already locally
aspheric), with the implication that the corresponding object of
$\Pro(D(T))$ should be again essentially constant.

A last question I would like to mention here is about the meaning of
the notion of a so-called ``modelizing topos'', introduced in a
somewhat formal way in section \ref{sec:35}, as a locally aspheric and
aspheric topos $T$ such that the Lawvere element $L_T$ of $T$ is
aspheric over the final object. (We assumed at first, moreover, that
$T$ be even totally aspheric, but soon after the point of view and
terminology shifted a little and the totally aspheric case was
referred to as a \emph{strictly} modelizing topos, cf.\ page
\ref{p:68}. As made clear there, the expectation suggested by the
terminology is of course that such a topos should indeed be a
modelizer, when endowed with the usual notion of weak
equivalence. This sill makes sense and seem plausible enough, when
usual weak equivalence is replaced by the weaker notion defined in
terms of an arbitrary basic localizer \scrW. A related question is to
get a feeling for how restrictive the conditions put on a modelizing
topos are. It is clear that nearly all topoi met with in practice,
including those associated to the more common topological spaces (such
as locally contractible ones) are locally aspheric -- but what about
the condition on the Lawvere element? For instance, taking a
topological space admitting a finite triangulation and which is
aspheric, i.e., contractible, is the corresponding topos modelizing?

\bigbreak

\presectionfill\ondate{10.6.}\pspage{213}\par

% 72
\hangsection{Back to asphericity: criteria for a map in
  \texorpdfstring{\Cat}{(Cat)}.}\label{sec:72}%
In the long last, we'll come back now to the ``asphericity game''!
Let's take up the exposition at the point where we stopped two months
ago (section \ref{sec:67}, p.\ \ref{p:188}). We were then about to
reformulate in various ways the property of asphericity, more
specifically \scrW-asphericity, for a given map
\[ i :A\to B\]
in \Cat. To this end, we introduced the corresponding diagram of maps
in \Cat{} with ``commutation morphism'' $\lambda_i$:
\[\begin{tikzcd}[baseline=(O.base),column sep=small]
  \Bhat\ar[rr,"i^*"]\ar[dr,swap,"i_B",""{name=iB,right}] & &
  \Ahat\ar[dl,"i_A",""{name=iA,left}] \\
  & |[alias=O]| \Cat \arrow[bend right=10, from=iA, to=iB, swap, "\lambda_i"] &
\end{tikzcd}.\]
As already stated, the asphericity condition on $i$ can be expressed
in a variety of ways, as a condition on either of the three
``aspects''
\[\lambda_i,\quad i^*,\quad i_Ai^*\]
of the situation created by $i$, with respect to the localizing sets
\scrW, \scrWA, $\scrW_B$ in the three categories under consideration,
or with respect to the notion of \scrW-aspheric objects in these. As
\scrWA, and aspheric objects in \Ahat, are defined respectively in
terms of \scrW{} and aspheric objects in \Cat{} via the functor $i_A$,
it turns out that the formulations in terms of properties of $i_Ai^*$
reduce trivially to the corresponding formulations in terms of $i^*$,
which will be the most directly useful for our purpose for later work
-- hence well's omit them, and focus attention instead on $\lambda_i$
and $i^*$. Notations are those of loc.\ sit., in particular, we are
working with the categories
\[ \mathrm{Hot}_A^\scrW \quad\text{or} \quad
\mathrm{Hot}_A \eqdef \scrWA^{-1}\Ahat\]
and
\[\HotOf(\scrW) = \scrW^{-1}\Cat,\]
where \scrW{} is a given ``basic localizer'', namely a set of arrows
in \Cat{} satisfying certain conditions.

It seems worthwhile here to be careful to state which exactly are the
properties of \scrW{} we are going to use in the ``asphericity game''
-- therefore I'm going to list them again here, using a labelling
which, hopefully, will not have to be changed again:
\begin{description}
\item[\namedlabel{loc:1}{Loc~1)}]
  ``Mild saturation'' (cf.\ page \ref{p:59}),
\item[\namedlabel{loc:2}{Loc~2)}]
  ``Homotopy condition'': $\Simplex_1\times X\to X$ is in \scrW{} for
  any $X$ in \Cat,
\item[\namedlabel{loc:3}{Loc~3)}]
  ``Localization\pspage{214} condition'': If $X,Y$ are objects in \Cat{} over an
  object $A$,\scrcomment{AG seems to have replaced $A$ with $\mathscr
    S$ here, but it's not clear\ldots} and $u: X\to Y$ an $A$-morphism
  such that for any $a$ in $A$, the induced $u_{/a}:X_{/a}\to Y_{/a}$
  is in \scrW, then so is $u$.
\end{description}

It should be noted that the ``mild saturation'' condition is slightly
weaker than the saturation condition introduced later (p.\
\ref{p:101}, conditions
\ref{it:48.aprime}\ref{it:48.bprime}\ref{it:48.cprime}), namely in
condition \ref{it:48.cprime} (if $f:X\to Y$ and $g:Y\to X$ are such
that $gf,fg\in\scrW$, then $f,g\in\scrW$) we restrict to the case when
$gf$ is the identity, namely $f$ an inclusion and $g$ a retraction
upon the corresponding subobject. On the other hand, in what follows
we are going to use \ref{loc:3} only in case $Y=A$ and $Y\to A$ is the
identity. I stated the condition in greater generality than needed for
the time being, in view of later convenience -- as later it will have
to be used in full strength. It is not clear whether the weaker form
of \ref{loc:3} (plus of course \ref{loc:1} and \ref{loc:2}) implies
already the stronger. We'll see later a number of nice further
properties of \scrW{} implied by these we are going to work with for
the time being -- including \emph{strong} saturation of \scrW, namely
that \scrW{} is the set of arrows made invertible by the localization
functor
\[\Cat\to\scrW^{-1}\Cat=\HotOf(\scrW).\]
\begin{propositionnum}\label{prop:72.1}
  Let as above $i:A\to B$ be a map in \Cat. Consider the following
  conditions on $i$:
  \begin{description}
  \item[\namedlabel{it:72.1.i}{(i)}]
    For any $F$ in \Bhat, $\lambda_i(F): A_{/F} \to B_{/F}$ is in \scrW.
  \item[\namedlabel{it:72.1.iprime}{(i')}]
    For any $F$ as above, $\lambda_i(F)$ is \scrW-aspheric
    \textup(i.e., satisfies the assumption on $u$ in
    \textup{\ref{loc:3}} above, when $Y\to A$ is an identity\textup).
  \item[\namedlabel{it:72.1.idblprime}{(i'')}]
    Same as \textup{\ref{it:72.1.i}}, but restricting to $F=b$ in $B$.
  \item[\namedlabel{it:72.1.ii}{(ii)}]
    For any $F$ in \Bhat, $F$ \scrW-aspheric $\Rightarrow i^*(F)$
    \scrW-aspheric.
  \item[\namedlabel{it:72.1.iiprime}{(ii')}]
    For any $F$ in \Bhat, $F$ \scrW-aspheric $\Leftrightarrow i^*(F)$
    \scrW-aspheric.
  \item[\namedlabel{it:72.1.iii}{(iii)}]
    For any $b$ in $B$, $A_{/b} (\eqdef A_{/i^*(b)})$ is
    \scrW-aspheric, i.e., $i$ is \scrW-aspheric.
  \item[\namedlabel{it:72.1.iv}{(iv)}]
    For any map $f$ in \Bhat, $f\in\scrW_B \Rightarrow i^*(f)\in\scrWA$.
  \item[\namedlabel{it:72.1.ivprime}{(iv')}]
    For any map $f$ as above, $f\in\scrW_B \Leftrightarrow i^*(f)\in\scrWA$.
  \item[\namedlabel{it:72.1.v}{(v)}]
    Condition \textup{\ref{it:72.1.iv}} holds, i.e., $i^*$ induces a
    functor
    \[\overline{i^*} : \mathrm{Hot}_B \to \mathrm{Hot}_A,\]
    and moreover the latter is an \emph{equivalence}.
  \end{description}
  The conditions \textup{\ref{it:72.1.i}} up to
  \textup{\ref{it:72.1.iii}} are all equivalent, call this set of
  conditions \textup{\namedlabel{cond:72.As}{(As)}}
  \textup(\scrW-asphericity\textup). We moreover have the following
  implications between \textup{\ref{cond:72.As}} and the remaining
  conditions \textup{\ref{it:72.1.iv}} to \textup{\ref{it:72.1.v}}:
  \begin{equation}
    \label{eq:72.1.star}
    \begin{tabular}{@{}c@{}}
      \begin{tikzcd}[baseline=(O.base),math mode=false,%
        arrows=Rightarrow,column sep=-3pt]
        \textup{\ref{cond:72.As}}
        \ar[rrrr,dashed,"{if $A,B$ ps.test}"{inner sep=3pt}]\ar[dr] & & & &
        \textup{\ref{it:72.1.v}} \ar[ddll] \\
        & \textup{\ref{it:72.1.ivprime}} \ar[dr] & & \phantom{hello} & \\
        & & |[alias=O]| \textup{\ref{it:72.1.iv}}
        \ar[uull,dashed,bend left,%
        "{\begin{tabular}{@{}l@{}}
            if $A,B$ \\ \scrW-asph.
          \end{tabular}}"{inner sep=-3pt,near end}] & &
      \end{tikzcd},
    \end{tabular}\tag{*}
  \end{equation}
  where\pspage{215} the implication \textup{\ref{it:72.1.iv} $\Rightarrow$
    \ref{cond:72.As}} is subject to $A,B$ begin \scrW-aspheric, and
  \textup{\ref{cond:72.As} $\Rightarrow$ \ref{it:72.1.v}} to $A,B$
  being ``pseudo-test categories'' \textup(for \scrW\textup), namely the canonical
  functors
  \[\overline{i_A}: \mathrm{Hot}_A \to \HotOf(\scrW), \quad
  \overline{i_B} : \mathrm{Hot}_B \to \HotOf(\scrW)\]
  being equivalence.
\end{propositionnum}
\begin{corollary}
  Assume that $A$ and $B$ are pseudo-test categories, and are
  \scrW-aspheric. Then all conditions \textup{\ref{it:72.1.i}} to
  \textup{\ref{it:72.1.v}} of the proposition above are equivalent.
\end{corollary}
\begin{remarknum}
  If we admit strong saturation of \scrW{} (which will be proved
  later), it follows at once that a pseudo-test category is
  necessarily \scrW-aspheric -- hence the conclusion of the corollary
  holds assuming only $A$ and $B$ are pseudo-test categories. Of
  course, it holds a fortiori if $A$ and $B$ are weak test categories,
  or even test categories. Also, strong saturation implies that in
  \eqref{eq:72.1.star} above, we have even the implication:
  \ref{it:72.1.v} $\Rightarrow$ \ref{it:72.1.ivprime}, stronger than
  \ref{it:72.1.v} $\Rightarrow$ \ref{it:72.1.iv}.
\end{remarknum}

\begin{proof}[Proof of proposition]
  It is purely formal -- for the first part, it follows from the
  diagram of tautological implications
  \[\begin{tikzcd}[baseline=(O.base),math mode=false,%
    arrows=Rightarrow,column sep=0pt]
    & \ref{it:72.1.iprime} \ar[d] & & $\phantom{\text{(i')}}$ & \\
    & \ref{it:72.1.i} \ar[dl] \ar[dr] & & & \\
    \ref{it:72.1.iiprime}\ar[d] & &
    \ref{it:72.1.idblprime}\ar[d,Leftrightarrow] & & \\
    \ref{it:72.1.ii}\ar[rr] & & \ref{it:72.1.iii} \ar[rr] & &
    |[alias=O]| \ref{it:72.1.iprime}
  \end{tikzcd},\]
  where the implication \ref{it:72.1.iprime} $\Rightarrow$
  \ref{it:72.1.i} is contained in the assumption
  \hyperref[loc:3]{Loc~3} on \scrW. The implications of the diagram
  \eqref{eq:72.1.star} are about as formal -- there is no point I
  guess writing it out here.
\end{proof}
\begin{remarknum}
  Using the canonical functors
  \[\varphi_A: \Ahat\to\HotOf(\scrW), \quad
  \varphi_B:\Bhat\to\HotOf(\scrW),\]
  there are two other amusing versions still of \ref{it:72.1.i}, which
  look a lot weaker still, and are equivalent however to
  \ref{it:72.1.i}, i.e., to \scrW-asphericity of $i$, namely:
  \begin{description}
  \item[\namedlabel{it:72.1.vi}{(vi)}]
    For any $F$ in \Bhat, $\varphi_B(F)=\gamma(B_{/F}$ and
    $\varphi_A(i^*(F))=\gamma(A_{/F})$ are isomorphic objects of
    $\HotOf(\scrW)$.
  \item[\namedlabel{it:72.1.viprime}{(vi')}]
    Same\pspage{216} as \ref{it:72.1.vi}, with $F$ restricted to be
    object $b$ in $B$.
  \end{description}
  Indeed, we have of course \ref{it:72.1.i} $\Rightarrow$
  \ref{it:72.1.vi} $\Rightarrow$ \ref{it:72.1.viprime}, but also
  \ref{it:72.1.viprime} $\Rightarrow$ \ref{it:72.1.iii} if we admit
  strong saturation of \scrW, which implies that an object $X$ of
  \Cat{} is \scrW-aspheric if{f} its image in $\HotOf(\scrW)$ is a
  final object.
\end{remarknum}
\begin{remarknum}
  If we don't assume $A$ and $B$ to be \scrW-aspheric, the
  implications
  \[ \text{\ref{cond:72.As}} \Rightarrow \text{\ref{it:72.1.ivprime}}
    \Rightarrow \text{\ref{it:72.1.iv}}\]
  are both strict. As an illustration of this point, take for $B$ a
  discrete category (defined in terms of a set of indices $I=\Ob B$),
  thus a category $A$ over $B$ is essentially the same as a family
  $(A_b)_{b\in I}$ of objects of \Cat{} indexed by $I$. In terms of
  this family, we see at once that the three conditions above on
  $i:A\to B$ mean respectively (a)\enspace that all categories $A_b$
  are \scrW-aspheric, (b)\enspace that all categories $A_b$ are
  non-empty, and (c)\enspace condition vacuous. This example brings to
  mind that \emph{condition \textup{\ref{it:72.1.iv}} is a
    Serre-fibration type condition}, we'll come back upon this
  condition when studying homotopy properties of maps in \Cat, with
  special emphasis on base change questions. Likewise, condition
  \ref{it:72.1.ivprime} appears as a strengthening of such Serre-type
  condition, to the effect that the restriction of $A$ over any
  connected component of $B$ should be moreover non-empty. As an
  example, we may take the projection $B\times C\to C$, where $C$ is
  any non-empty object in \Cat.
\end{remarknum}

I would like now to dwell still a little on the case, of special
interest of course for the modelizing story, when $A$ and $B$ are test
categories or something of the kind. The formulation \ref{it:72.1.i}
of the asphericity condition for the functor $i$ can be expressed by
stating that the functor $\overline{i^*}$ between the localizations
$\mathrm{Hot}_B$ and $\mathrm{Hot}_A$ exists, and gives rise (via
$\lambda_i$) to a commutation morphism which is an \emph{isomorphism}
\[\begin{tikzcd}[baseline=(O.base),column sep=small]
  \mathrm{Hot}_B\ar[rr,"\overline{i^*}"]\ar[dr,swap,"\overline{i_B}",""{name=iB,right}] & &
  \mathrm{Hot}_A\ar[dl,"\overline{i_A}",""{name=iA,left}] \\
  & |[alias=O]| \Cat \arrow[bend right=10, from=iA, to=iB, swap,
  "\overline{\lambda_i}"{inner sep=4pt}, "\sim"{inner sep=0pt}] &
\end{tikzcd}.\]
This shows, when $A$ and $B$ are pseudo-test categories for \scrW,
i.e., the functors $\overline{i_A}$ and $\overline{i_B}$ are
equivalences, that the functor $\overline{i^*}$ deduced from the
\scrW-aspheric map $i:A\to B$, \emph{does not depend} (up to canonical
isomorphism) on the choice of $i$, and can be described as the
composition of $\overline{i_B}$ followed by a quasi-inverse for
$\overline{i_A}$. This of course is very much in keeping with the
``inspiring assumption'' (section \ref{sec:28}), which just means that
up to unique isomorphism, there is indeed but \emph{one}\pspage{217}
equivalence from $\mathrm{Hot}_B$ to $\mathrm{Hot}_A$ (both categories
being equivalent to $\HotOf(\scrW)$). Here we are thinking of course
of the extension of the ``assumption'' contemplated earlier, when
usual weak equivalence is replaced by a basic localizer \scrW{} as
above. It seems plausible that the assumption holds true in all cases
-- anyhow we didn't have at any moment to make explicit use of it
(besides at a moment drawing inspiration from it\ldots).
\begin{propositionnum}\label{prop:72.2}
  Let $i:A\to B$ be a map in \Cat.
  \begin{enumerate}[label=\alph*),font=\normalfont]
  \item\label{it:72.2.a}
    If $i$ is \scrW-aspheric, and \Ahat{} is totally \scrW-aspheric,
    then \Bhat{} is totally \scrW-aspheric too.
  \item\label{it:72.2.b}
    Assume $A$ \scrW-aspheric, and that $B$ is a \scrW-test category,
    admitting the separating \scrWB-homotopy interval
    $\bI=(I,\delta_0,\delta_1)$, satisfying the homotopy condition
    \textup{\ref{cond:TH1}} of page \textup{\ref{p:50}}. Consider the following
    conditions:
    \begin{enumerate}[label=\arabic*),font=\normalfont]
    \item\label{it:72.2.b.1}
      $i^*(I)$ is \scrW-aspheric over $e_\Ahat$,
    \item\label{it:72.2.b.2}
      $i$ is \scrW-aspheric,
    \item\label{it:72.2.b.3}
      $i^*(I)$ is \scrW-aspheric.
    \end{enumerate}
    We have the implications
    \[ \textup{\ref{it:72.2.b.1}} \Rightarrow
    \textup{\ref{it:72.2.b.2}} \Rightarrow
    \textup{\ref{it:72.2.b.3}},\]
    hence, if $A$ is totally \scrW-aspheric \textup(hence
    \textup{\ref{it:72.2.b.3}} $\Rightarrow$
    \textup{\ref{it:72.2.b.1})}, all three conditions are equivalent.
  \end{enumerate}
\end{propositionnum}
\begin{proof}
  \ref{it:72.2.a}\enspace We have to prove that if $b,b'$ are in $B$, then
  their product in \Bhat{} is \scrW-aspheric, but by assumption on $i$
  we know that the images of $b,b'$ by $i^*$ are \scrW-aspheric, hence
  (as \Ahat{} is totally aspheric) their product
  \[ i^*(b) \times i^*(b') = i^*(b\times b')\]
  is \scrW-aspheric too, hence so is $b\times b'$ by criterion
  \ref{it:72.1.iiprime} of prop.\ above.

  \ref{it:72.2.b}\enspace The homotopy condition \ref{cond:TH1} referred to
  means that all objects of $B$ are \bI-contractible. As $i^*$
  commutes with finite products, it follows that the objects $i^*(b)$
  of \Ahat, for $b$ in $B$, are $i^*(\bI)$-contractible. When $i^*(I)$
  is \scrW-aspheric over $e_\Ahat$, this implies that so are the
  objects $i^*(b)$, a fortiori they are \scrW-aspheric (as by
  assumption $e_\Ahat$ is \scrW-aspheric). Thus \ref{it:72.2.b.1}
  $\Rightarrow$ \ref{it:72.2.b.2}, and \ref{it:72.2.b.2} $\Rightarrow$
  \ref{it:72.2.b.3} is trivial.
\end{proof}
\begin{corollary}
  Let $i:A\to B$ be a \scrW-aspheric map in \Cat, assume that $A$ is
  totally \scrW-aspheric, and $B$ is a local \scrW-test category. Then
  both $A$ and $B$ are strict \scrW-test categories.
\end{corollary}

Indeed, by \ref{it:72.2.a} above we see that $B$ is totally
\scrW-aspheric, hence $B$ is a strict \scrW-test category. In order to
prove that so is $A$, we only have to show that \Ahat{} admits a
separating homotopy interval for \scrWA. By assumption\pspage{218} on
$B$, there is a separating \scrWB-homotopy interval
$\bI=(I,\delta_0,\delta_1)$ in \Bhat. The exactness properties of $i^*$
imply that $i^*(\bI)$ is a separating interval in \Ahat, the
asphericity condition on $i$ implies that moreover $i^*(I)$ is
\scrW-aspheric, hence \scrW-aspheric over $e_\Ahat$ as \Ahat{} is
totally \scrW-aspheric, qed.

To finish with the more formal properties of the notion  of
\scrW-aspheric maps in \Cat, let's give a list of the standard
stability conditions for this notion, with respect notably to
composition, base change, and cartesian products:

\begin{propositionnum}\label{prop:72.3}
  \textup{\namedlabel{it:72.3.a}{a)}}\enspace
  Consider two maps
  \[ A \xrightarrow i B \xrightarrow j C\]
  in \Cat. Then if $i$ and $j$ are \scrW-aspheric, so is $ji$. If $ji$
  and $i$ are \scrW-aspheric, so is $j$. Any isomorphism in \Cat{} is
  \scrW-aspheric.

  \textup{\namedlabel{it:72.3.b}{b)}}\enspace
  Let
  \[\begin{tikzcd}
    A \ar[d,swap,"i"] & A' \ar[l,swap,"f"]\ar[d,"i'"] \\
    B & B' \ar[l,swap,"g"]
  \end{tikzcd}\]
  be a cartesian square in \Cat, assume $i$ is \scrW-aspheric and $g$
  is fibering \textup(for instance an induction functor $B_{/F}\to B$,
  with $F$ in \Bhat\textup). Then $i'$ is \scrW-aspheric \textup(and,
  of course, $f$ is equally fibering\textup).
  
  \textup{\namedlabel{it:72.3.c}{c)}}\enspace
  Let
  \[i:A\to B, \quad i':A'\to B'\]
  be two \scrW-aspheric maps in \Cat, then
  \[ i\times i' : A\times A' \to B\times B'\]
  is \scrW-aspheric.
\end{propositionnum}
\begin{proof}
  Property \ref{it:72.3.a} is formal, in terms of criterion
  \ref{it:72.1.iiprime} of prop.\ \ref{prop:72.1}. Property
  \ref{it:72.3.c} follows formally from the criterion
  \ref{it:72.1.iii}, and the canonical isomorphism
  \[ (A\times A')_{/b\times b'} \simeq (A_{/b}) \times (A'_{/b'}),\]
  and the fact that a product of two \scrW-aspheric objects of \Cat{}
  is again \scrW-aspheric (cf.\ prop.\ of page \ref{p:167}, making use
  of the localization condition \ref{loc:3} on \scrW{} in its full
  generality). Another proof goes via \ref{it:72.3.a} and
  \ref{it:72.3.b}, by reducing first (using \ref{it:72.3.a}) to the
  case when either $i$ or $i'$ are identities, and using the fact that
  a projection map $C \times B \to B$ in \Cat{} is a fibration.

  We are left with proving property \ref{it:72.3.b}. For this, we note
  that the asphericity condition \ref{it:72.1.iii} on a map $i:A\to B$
  just means that for\pspage{219} any base change of the type
  \[ B_{/b} \to B,\]
  where $b$ is in $B$, the corresponding map
  \[i_{/b}: A\times_B B_{/b} \simeq A_{/b} \to B_{/b}\]
  is in \scrW. Applying this to the case of $i':A'\to B'$, and an
  object $b'$ in $B'$, and denoting by $b$ its image in $B$, using
  transitivity of base change, we get a cartesian square
  \[\begin{tikzcd}[baseline=(O.base)]
    A_{/b} \ar[d,"i_{/b}"] & A'_{/b'}\ar[l]\ar[d,"i'_{/b'}"] \\
    B_{/b} & |[alias=O]| B'_{/b'} \ar[l]
  \end{tikzcd},\]
  and we got to prove $i'_{/b'}$ is in \scrW, i.e., $A'_{/b'}$ is
  \scrW-aspheric, using the fact that we know the same holds for
  $i_{/b}$, i.e., $A_{/b}$ is \scrW-aspheric. Thus, all we have to
  prove is that the first horizontal arrow is in \scrW. But we check
  at once that the condition that $B'\to B$ is fibering implies that
  the induced functor
  \[ B'_{/b'} \to B_{/b}\]
  is fibering too, and moreover has fibers which have final
  objects. Hence by base change, the same properties hold for
  \[ A'_{/b'} \to A_{/b}.\]
  This reminds us of the ``fibration condition''
  \hyperref[it:64.L5]{L~5} (page \ref{p:164}), which should ensure
  that a fibration with \scrW-aspheric fibers is in \scrW. We did not
  include this axiom among the assumptions (recalled above) we want to
  make on \scrW. However, it turns out that the assumptions we do make
  here imply already the fibration condition, as well as the dual
  condition on cofibrations. We'll give a proof later -- in order not
  to diverge at present from our main purpose. There will not be any
  vicious circle, as all we're going to use of prop.\ \ref{prop:72.3}
  for the formalism of asphericity structures and canonical modelizers
  is the first part \ref{it:72.3.a}. I included \ref{it:72.3.b} and
  \ref{it:72.3.b} for the sake of completeness, and because
  \ref{it:72.3.c} is useful for dealing with products of two
  categories, notably of two test categories -- a theme which has been
  long pending, and on which I would like to digress next, before
  getting involved with asphericity structures.
\end{proof}
\begin{remarknum}
  In part \ref{it:72.3.a} of the proposition, if $j$ and $ji$ are
  \scrW-aspheric, we cannot conclude that $i$ is. If $C$ is the final
  object of \Cat, this means that a map between \scrW-aspheric objects
  in \Cat{} need not be \scrW-aspheric.
\end{remarknum}

\bigbreak

\presectionfill\ondate{11.6.}\pspage{220}\par

% 73
\hangsection{Asphericity criteria
  \texorpdfstring{\textup(}(continued\texorpdfstring{\textup)}).}%
\label{sec:73}%
I am not quite through yet with generalities on asphericity criteria
for a map in \Cat, it turns out -- it was just getting prohibitively
late last night to go on!

From now on, I'll drop the qualifying \scrW{} when speaking of
asphericity, test categories, modelizers and the like, as by now it is
well understood, I guess, there is a given \scrW{} around in all we
are doing. It'll be enough to be specific in those (presumably rare)
instances when working with more than one basic localizer.

Coming back to the last remark in yesterday's notes, a good
illustration is the case of a functor $i:A\to B$ of aspheric objects
of \Cat, when $A$ is a final object in \Cat, i.e., a one-point
discrete category. Then $i$ \emph{is aspheric if{f} $i(a)$ is an
  initial object of $B$} -- an extremely stringent extra condition
indeed!

In part \ref{it:72.3.a} of proposition \ref{prop:72.3} above, stating
that isomorphisms in \Cat{} are aspheric, it would have been timely to
be more generous -- indeed, \emph{any equivalence of categories in
  \Cat{} is aspheric}. This associates immediately with a map of topoi
which is an equivalence being (trivially so) aspheric (in the case,
say, when \scrW{} is the usual notion of weak equivalence, the only
one for the time being when the notion is extended from maps in \Cat{}
to maps between topoi). In the context of \Cat, the basic modelizer,
we can give a still more general case of aspheric maps, both
instructive and useful:
\addtocounter{propositionnum}{3}
\begin{propositionnum}\label{prop:73.4}
  Let
  \[\begin{tikzcd}
    A \ar[r,shift left,"f"] & B \ar[l,shift left,"g"] B
  \end{tikzcd}\]
  be a pair of adjoint functors between the objects $A,B$ in \Cat,
  with $f$ left and $g$ right adjoint. Then $f$ is aspheric.
\end{propositionnum}

Indeed, by the adjunction formula we immediately get
\[A_{/b} \simeq A_{/g(b)}\]
(as a matter of fact, $f^*(b)=g(b)$), hence this category has a final
object and hence is aspheric, qed.

\addtocounter{remarknum}{4}
\begin{remarknum}
  The conclusion of prop.\ \ref{prop:73.4} is mute about $g$, which
  will rightly strike as unfair. Dualizing, we could say that
  $(g\op,f\op)$ is a pair of adjoint functors between $A\op$ and
  $B\op$, and therefore $g\op$ is aspheric. We will express this fact
  (by lack of a more suggestive name) by saying that $g$ is a
  \emph{coaspheric} map. For a functor $i:A\to B$, in terms of the
  usual criterion for $i\op$, it just\pspage{221} means that for any
  $b$ in $B$, the category
  \[\preslice Ab \eqdef A \times_B(\preslice Bb)
  = \parbox[t]{0.5\textwidth}{category of pairs $(x,p)$, with $x$ in
    $A$ and $p:b\to i(x)$}\]
  is aspheric. (NB\enspace In the case of the functor $g$ from $B$ to
  $A$, the corresponding categories $\preslice Ba$ even have
  \emph{initial objects}.) This conditions comes in here rather
  formally, we'll see later though that it has a quite remarkable
  interpretation, in terms of a very strong property of cohomological
  ``cofinality'' of the functor $i$, implying the usual notion of $i$
  being a ``cofinal'' functor, or $A$ being ``cofinal'' in $B$ (namely
  $i$ giving rise to isomorphisms $\varinjlim_A{} \to \varinjlim_B{}$
  for any direct system $B\to M$ with values in a category $M$
  admitting direct limits), as the ``dimension zero'' shadow of this
  ``all dimensions'' property. To make an analogy which will acquire
  more precise meaning later, the asphericity property for a map $i$
  in \Cat{} can be viewed as a (slightly weakened) version of a
  \emph{proper map with aspheric fibers}, whereas the coasphericity
  property appears as the corresponding version of a \emph{smooth map
    with aspheric fibers}. The qualification ``slightly weakened''
  reflects notably in the fact that the notions of asphericity and
  coasphericity are \emph{not} stable under arbitrary base change --
  but rather, asphericity for a map is stable under base change by
  \emph{fibration functors} (more generally, by smooth maps in \Cat),
  whereas coasphericity is stable under base change by
  \emph{cofibration functors} (more generally, by proper maps in
  \Cat). Thus, whereas a functor $i:A\to B$ which is an equivalence of
  categories is clearly both aspheric and coaspheric, this property is
  not preserved by arbitrary base change, e.g., passage to fibers:
  e.g., some fibers may be empty, and therefore are neither aspheric
  nor coaspheric!
\end{remarknum}

The most comprehensive property for a map in \Cat, implying both
asphericity and coasphericity, is to be in \UW, i.e., a
``\emph{universal weak equivalence}'' -- namely it is in \scrW{} and
remains so after any base change. These maps deserve the name of
``\emph{trivial Serre fibrations}''. They include all maps with
aspheric fibers which are either ``proper'' or ``smooth'', for
instance those which are either cofibering or fibering functors (in
the usual sense of category theory). We'll come back upon these
notions in a systematic way in the next part of the reflections.

% 74
\hangsection{Application to products of test categories.}\label{sec:74}%
As announced yesterday, I would like still to make an overdue
digression on products of test categories, before embarking on the
notion of asphericity structure. It will be useful to begin with
some\pspage{222} generalities on presheaves on a product category
$A\times B$, where for the time begin $A,B$ are any two objects of
\Cat. The following notation is often useful, for a pair of presheaves
\[ F\in\Ob\Ahat, \quad G\in\Ob\Bhat,\]
introducing an ``\emph{external product}''
\[ F \boxtimes G\in \Ob(A\times B)\uphat\]
by the formula
\[\mathop{F \boxtimes G}(a,b) = F(a) \times G(b).\]
Introducing the two projections
\[ p_1 : A\times B\to A, \quad p_2:A\times B\to B,\]
and the corresponding inverse image functors $p_i^*$, we have
\[ F \boxtimes G = p_1^*(F) \times p_2^*(G).\]
Of course, $F\boxtimes G$ depends bifunctorially on the pair $(F,G)$,
and it is easily checked, by the way, that the corresponding functor
\[ \Ahat \times \Bhat \to (A\times B)\uphat, \quad
(F,G)\mapsto F\boxtimes G,\]
is fully faithful.

We have a tautological canonical isomorphism
\[ (A\times B)_{/{F \boxtimes G}} \tosim A_{/F} \times B_{/G},\]
and hence the
\begin{propositionnum}\label{prop:74.1}
  \textup{\namedlabel{it:74.1.a}{a)}}\enspace
  If $F$ and $G$ are aspheric objects in \Ahat{} and \Bhat{}
  respectively, then $F\boxtimes G$ is an aspheric object of $(A\times
  B)\uphat$.

  \textup{\namedlabel{it:74.1.b}{b)}}\enspace
  If $u: F\to F'$ and $v: G\to G'$ are aspheric maps in \Ahat{} and
  \Bhat{} respectively, then
  \[ {u\boxtimes v} : F \boxtimes G \to F' \boxtimes G'\]
  is an aspheric map in $(A\times B)\uphat$.
\end{propositionnum}

Indeed, this follows respectively from the fact that a product of two
aspheric objects (resp.\ aspheric maps) in \Cat{} is again aspheric.
\begin{corollarynum}\label{cor:74.1.1}
  Let $F$ in \Ahat{} be aspheric over the final object $e_\Ahat$, then
  $F \boxtimes e_\Bhat = p_1^*(F)$ is aspheric over the final object
  of $(A\times B)\uphat$.
\end{corollarynum}
\begin{corollarynum}\label{cor:74.1.2}
  Assume $A$ and $B$ are totally aspheric, then so is $A\times B$.
\end{corollarynum}

We have to check that the product elements
\[ (a,b) \times (a',b') = (a\times a') \boxtimes (b\times b')\]
are aspheric, which follows from the assumption (namely $a\times a'$
and $b\times b'$ aspheric) and part \ref{it:74.1.a} of the
proposition.

Assume\pspage{223} now $A$ is a local test category, i.e., \Ahat{}
admits a separating interval
\[ \bI=(I,\delta_0,\delta_1)\]
such that $I$ is aspheric over the final object of \Ahat. Then
$p_1^*(\bI)$ is of course a separating interval in $(A\times
B)\uphat$, which by cor.\ \ref{cor:74.1.1} is aspheric over the final
object. Hence
\begin{propositionnum}\label{prop:74.2}
  If $A$ is a local test category, so is $A\times B$ for any $B$ in
  \Cat. If $A$ is a test category, then so is $A\times B$ for any
  aspheric $B$.
\end{propositionnum}

The second assertion follows, remembering that a test category is just
an \emph{aspheric} local test category. Using cor.\ \ref{cor:74.1.2},
we get:
\begin{corollary}
  If $A$ is a strict test category, and $B$ totally aspheric, then
  $A\times B$ is a strict test category. In particular, if $A$ and $B$
  are strict test categories, so is their product.
\end{corollary}

We need only remember that strict test categories are just test
categories that are totally aspheric.

As an illustration, we get the fact that the categories of
multicomplexes of various kinds, which we can even take mixed
(semisimplicial in some variables, cubical in others, and
hemispherical say in others still) are strict modelizers (as generally
granted), which corresponds to the fact that a finite product of
standard semisimplicial, cubical and hemispherical (strict) test
categories $\Simplex$, $\Square$ and $\Globe$, is again a strict test
category.

I would like now to dwell a little upon the comparison of ``homotopy
models'', using respectively two test categories $A$ and $B$, namely
working in \Ahat{} and \Bhat{} respectively. More specifically, we
have two description of $\HotOf$ (short for $\HotOf(\scrW)$ here),
namely as
\[ \HotOf_A = \scrWA^{-1}\Ahat \quad\text{and}\quad \HotOf_B =
\scrWA^{-1}\Bhat,\]
and we want to describe conveniently the tautological equivalence
between these two categories (this equivalence being defined up to
unique isomorphism). The most ``tautological'' way indeed is to use
the basic modelizer \Cat{} and its localization \Hot{} as the
intermediary, i.e., using the diagram of equivalences of categories
\begin{equation}
  \label{eq:74.1}
  \begin{tabular}{@{}c@{}}
    \begin{tikzcd}[baseline=(O.base),column sep=small]
      \HotOf_A\ar[dr,"\equ","\overline{i_A}"'] & &
      \HotOf_B\ar[dl,"\overline{i_B}","\equ"'] \\
      & |[alias=O]| \Hot &
    \end{tikzcd}.
  \end{tabular}
  \tag{1}
\end{equation}
Remember we have a handy quasi-inverse $\overline{j_B}$\pspage{224} to
$\overline{i_B}$, using the canonical functor\scrcomment{In this display, AG
  originally put the definition of $i_B$ instead $j_B$, cf., e.g.,
  \eqref{eq:65.1} and \eqref{eq:65.2} in \S\ref{sec:65} to recall the definitions.}
\[ j_B = i_B^*: \Cat\to\Bhat, \quad X\mapsto(b\mapsto\Hom(B_{/b},X)).\]
Thus, we get a description of an equivalence
\begin{equation}
  \label{eq:74.1prime}
  \overline{j_B}\, \overline{i_A} : \HotOf_A \toequ \HotOf_B,\tag{1'}
\end{equation}
whose quasi-inverse of course is just the similar $\overline{j_A}
\,\overline{i_B}$. We can vary a little this description, admittedly
cumbersome in practice, by replacing $\overline{j_B}$ by the
isomorphic functor $\overline{i^*}$, where $i:B\to\Cat$ is any test
functor from $B$ to \Cat{} (while we have to keep however
$\overline{i_A}$ as it is, without the possibility of replacing $i_A$
by a more amenable test functor).

Another way for comparing $\HotOf_A$ and $\HotOf_B$ arises, as we saw
yesterday, whenever we have an \emph{aspheric} functor
\begin{equation}
  \label{eq:74.2}
  i: A\to B,\tag{2}
\end{equation}
by just taking
\begin{equation}
  \label{eq:74.2prime}
  \overline{i^*}: \HotOf_B \toequ \HotOf_A.\tag{2'}
\end{equation}
This of course is about the simplest way imaginable, all the more as
the functor $i^*:\Bhat\to\Ahat$ commutes to arbitrary direct and
inverse limits, just perfect for comparing constructions in \Bhat{}
and constructions in \Ahat{} -- whereas the functor
$j_Bi_A:\Ahat\to\Bhat$, giving rise to \eqref{eq:74.1} above commutes
just to sums and fibered products, not to amalgamated sums nor to
products, sadly enough! We could add here that if we got \emph{two}
aspheric functors from $A$ to $B$, namely $i$ plus
\[ i':A\to B,\]
then (as immediately checked) any map between these functors
\[u: i\to i'\]
gives rise to an isomorphism between the corresponding equivalences
\[\begin{tikzcd}[baseline=(O.base),column sep=large]
  |[alias=O]| \HotOf_B
  \ar[r,bend left,"\overline{i^*}",""{name=i,below}]
  \ar[r,bend right,"\overline{(i')^*}"',""{name=iprime,above}] &
  \HotOf_A \arrow[from=iprime,to=i,"\overline u"']
\end{tikzcd},\]
which is nothing but the canonical isomorphism referred to yesterday
(both functors being canonically isomorphic to \eqref{eq:74.1prime}),
and yields the most evident way for ``computing'' the latter. Thus, it
turns out that the isomorphism $\overline u$ does not depend upon the
choice of $u$. If we want to ignore this fact and look at the
situation sternly, in a wholly computational spirit, we could present
things by stating that we get a contravariant functor, from aspheric
maps $i:A\to B$ to equivalences $\HotOf_B\to\HotOf_A$:\pspage{225}
\[ \bAsph(A,B)\op \to \bHom(\HotOf_B,\HotOf_A), \quad i\mapsto
\overline{i^*},\]
where $\bAsph$ denotes the full subcategory of the functor category
$\bHom$ made up with aspheric functors. This functor transforms
arbitrary arrows from the left hand side into isomorphisms on the
right, and therefore, it factors through the fundamental groupoid
(i.e., localization of the left hand category with respect to the set
of \emph{all} its arrows):
\[ \bigl(\Pi_1(\bAsph(A,B))\bigr)\op\to\bHom(\HotOf_B,\HotOf_A).\]
If we now remember that we had assumed $A$ and $B$ to be test
categories (otherwise the functors just written would be still
defined, but their values would not necessarily be \emph{equivalences}
between $\HotOf_B$ and $\HotOf_A$, but merely functors between these),
we may hope that this might imply that the category $\bAsph(A,B)$ of
aspheric functors from $A$ to $B$, whenever non-empty, to be
$1$-connected. Whenever this is so, in any case, we get ``a priori''
(namely without any reference to \Hot{} itself) a transitive system of
isomorphisms between the functors $\overline{i^*}$, for $i$ in
$\bAsph(A,B)$, hence a \emph{canonical functor} $\HotOf_B\to\HotOf_A$,
\emph{defined up to unique isomorphism} (and which, in case $A$ and
$B$ are test categories, or more generally pseudo-test categories, is
an equivalence, and the one precisely stemming from the diagram
\eqref{eq:74.1}).

\begin{remark}
  Here the reflection slipped, almost against will, into a related
  one, about comparison of $\HotOf_A$ and $\HotOf_B$ for arbitrary $A$
  and $B$ (not necessarily test categories nor \emph{even} pseudo test
  categories), using aspheric functors $i:A\to B$ to get
  $\overline{i^*}:\HotOf_B\to\HotOf_A$ (not necessarily an
  equivalence). As seen above, this functor depends rather loosely on
  the choice of $i$, and we could develop comprehensive conditions on
  $A$ and $B$, not at all of a test-condition nature, implying that
  just using the isomorphisms $\overline u$ between these functors, we
  get a canonical \emph{transitive} system of isomorphisms between
  them, hence a \emph{canonical} functor $\HotOf_B\to\HotOf_A$, not
  depending on the particular choice of any aspheric functor $i:A\to
  B$. As we are mainly interested in the modelizing case though, I
  don't think I should dwell on this much longer here. Anyhow, in case
  both $A$ and $B$ are pseudo-test categories, and provided only
  $\bAsph(A,B)$ is $0$-connected (not necessarily $1$-connected), it
  follows from comparison with the diagram \eqref{eq:74.1} above that
  the isomorphisms $\overline u$ do give rise indeed to a
  \emph{transitive} system of isomorphisms between the equivalences $\overline{i^*}$.
\end{remark}

In most cases though, such as $A=\Simplex$ and $B=\Square$ say (the
test\pspage{226} categories of standard simplices and standard cubes
respectively), we do not have any aspheric functor $A\to B$ at hand,
and presumably we may well have that $\bAsph(A,B)$ is empty, poor it!
We now assume again that $A$ and $B$ are test categories, hence
$A\times B$ is a test category, and the natural idea for comparison of
$\HotOf_A$ and $\HotOf_B$ is to use the diagram
\[\begin{tikzcd}[baseline=(O.base),column sep=tiny]
  & A\times B\ar[dl,"p_1"']\ar[dr,"p_2"] & \\
  A & & |[alias=O]| B \end{tikzcd},\]
where now $p_1$ and $p_2$ are aspheric (because $B$ and $A$ are
aspheric). Thus, we get a corresponding diagram on the corresponding
modelizers\scrcomment{in the typescript this equation is tagged (2)}
\begin{equation}
  \label{eq:74.3}
  \begin{tabular}{@{}c@{}}
    \begin{tikzcd}[baseline=(O.base),column sep=tiny]
      & \HotOf_{A\times B} & \\
      \HotOf_A\ar[ur,"\overline{p_1^*}","\equ"'] & & |[alias=O]|
      \HotOf_B\ar[ul,"\overline{p_2^*}"',"\equ"]
    \end{tikzcd}.
  \end{tabular}\tag{3}
\end{equation}
These equivalences are compatible with the canonical equivalences with
$\HotOf$ itself, and hence they give rise to an equivalence $\HotOf_A
\toequ \HotOf_B$ which (up to canonical isomorphism) is indeed the
canonical one. This way for comparing $\HotOf_A$ and $\HotOf_B$ looks
a lot more convenient than the first one, as the functors $p_1^*$ and
$p_2^*$ which serve as intermediaries have all desirable exactness
properties, and their very description is the simplest imaginable.

We are very close here to an Eilenberg-Zilber situation, which will
arise more specifically when $A$ and $B$ are both \emph{strict} test
categories, i.e., in the corresponding modelizers $\Ahat,\Bhat$
products of models do correspond to products of the corresponding
homotopy types. As seen above, this implies the same for $A\times
B$. Thus, if $F$ is in \Ahat, $G$ in \Bhat, the objects $F\boxtimes G$
in $(A\times B)\uphat$ is just a $(A\times B)$-model for the homotopy
type described respectively by $F$ (in terms of $A$) and $G$ (in terms
of $B$). As a matter of fact, the relation
\[(A\times B)_{/{F \boxtimes G}} \simeq A_{/F} \times B_{/G},\]
already noticed before, implies that this interpretation holds,
independently of strictness. In the classical statement of
Eilenberg-Zilber (as I recall it), we got $A=B$ (both categories being
just $\Simplex$);\pspage{227} on the one hand the $A\times B$-model is
used in order to get readily the Künneth type relations for homology
and cohomology, whereas we are interested really in the $A$-model
$F\times G$. Using the diagonal map
\[ \delta: A\to A\times A,\]
we have of course
\[F \times G\simeq \delta^*(F\boxtimes G),\]
more generally it is expected that the functor
\[ \delta^*: (A\times A)\uphat \to \Ahat,\]
is modelizing, i.e., gives a means of passing from $(A\times
A)$-models to $A$-models. This essentially translates into $\delta$
being an aspheric map -- something which will not be true for
arbitrary $A$ in \Cat. We get in this respect:
\begin{propositionnum}\label{prop:74.3}
  Let $A$ be an object in \Cat. Then the diagonal map $\delta:A\to
  A\times A$ is aspheric if{f} $A$ is totally aspheric.
\end{propositionnum}

This is just a tautology -- one among many which tell us that the
conceptual set-up is OK indeed! A related tautology:
\begin{corollary}
  Let $i:A\to B$, $i':A\to B'$ be two aspheric functors with same
  source $A$\kern1pt, then the corresponding functor
  \[(i,i'):A \to B\times B'\]
  is aspheric, provided $A$ is totally aspheric.
\end{corollary}

This can be seen either directly, or as a corollary of the
proposition, by viewing $(i,i')$ as a composition
\[A \xrightarrow \delta A\times A \xrightarrow{i\times i'} B\times
B'.\]

To sum up: when we got a bunch of strict test categories, and a
(possibly empty) bunch of aspheric functors between some of these,
using finite products we get a (substantially larger!) bunch of strict
test categories, giving rise to corresponding strict modelizers for
homotopy types; and using projections, diagonal maps, and the given
aspheric functors, we get an impressive lot of aspheric maps between
all these, namely as many ways to ``commute'' from one type of
``homotopy models'' to others. The simplest example: start with just
one strict test category $A$, taking products
\[ A^I\]
where $I$ is any finite set, and the ``simplicial'' maps between
these, expressing contravariance of $A^I$ with respect to $I$. The
case most commonly used is $A=\Simplex$, giving rise to the formalism
of semisimplicial multicomplexes.

To\pspage{228} finish these generalities on aspheric functors, I would
like still to make some comments on $\bAsph(A,B)$, in case $B$ admits
binary products, with $A$ and $B$ otherwise arbitrary in \Cat. We are
interested, for two functors
\[ i,i' : A \rightrightarrows B,\]
in the functor
\[ i\times i': a\mapsto i(a)\times i'(a) : A \to B,\]
which can be viewed indeed as a product object in $\bHom(A,B)$ of $i$
and $i'$. Our interest here is mainly to give conditions ensuring that
$i\times i'$ is aspheric. It turns out that there is no point to this
end assume \emph{both} $i$ and $i'$ to be aspheric, what counts is
that one, say $i$, should be aspheric. The point is made very clearly
in the following
\begin{propositionnum}\label{prop:74.4}
  Let $i:A\to B$ be an aspheric map in \Cat, we assume for simplicity
  that in $B$ binary products exist.
  \begin{enumerate}[label=\alph*),font=\normalfont]
  \item\label{it:74.4.a}
    Let $b_0\in\Ob B$, $i_{b_0}:A\to B$ be the constant functor with
    value $b_0$, consider the product functor
    \[ i\times i_{b_0} : a\mapsto i(a)\times b_0 : A\to B.\]
    In order for this functor to be aspheric, it is necessary and
    sufficient that for any object $y$ in $B$, the object
    $\bHom(b_0,y)$ in \Bhat{} be aspheric \textup(a condition which
    depends only on $b_0$, not upon $i$ nor even upon $A$\textup).
  \item\label{it:74.4.b}
    Assume this condition satisfied for any $b_0$ in $B$, assume
    moreover $A$ totally aspheric. Then for \emph{any} functor
    $i':A\to B$, $i\times i'$ is aspheric.
  \end{enumerate}
\end{propositionnum}
\begin{proof}
  \ref{it:74.4.a}\enspace Asphericity of $i\times i_{b_0}$ means that
  for any object $y$ in $B$, the corresponding presheaf on $A$
  \[ (i\times i_{b_0})^*(y) = (a \mapsto \Hom(i(a)\times b_0, y))\]
  is aspheric. Now the formula defining the presheaf $\bHom(b_0,y)$ on
  $B$ yields
  \[ \Hom(i(a)\times b_0,y) \simeq \Hom(i(a),\bHom(b_0,y)),\]
  hence the presheaf we get on $A$ is nothing but
  \[i^*(\bHom(b_0,y)).\]
  As $i$ is aspheric, the criterion \ref{it:72.1.iiprime} of prop.\
  \ref{prop:72.1} (page \ref{p:214}) implies that this presheaf is
  aspheric if{f} $\bHom(b_0,y)$ is, qed.

  \ref{it:74.4.b}\enspace We may view $i\times i'$ as a composition
  \[ A \xrightarrow\delta A\times A \xrightarrow{i\boxtimes i'} B,\]
  where the ``external product'' $i\boxtimes i': A\times A\to B$ is
  defined by\pspage{229}
  \[{i\boxtimes i'}(a,a') \eqdef i(a)\times i'(a').\]
  By prop.\ \ref{prop:74.3} we know that the diagonal map for $A$ is
  aspheric, thus we are left with proving that $i\boxtimes i'$ is
  aspheric. More generally, we get:
\end{proof}
\setcounter{corollarynum}{0}
\begin{corollarynum}\label{cor:74.4.1}
  Let $B$ in \Cat{} satisfy the conditions of \textup{\ref{it:74.4.a}}
  and \textup{\ref{it:74.4.b}} above, and let $A,A'$ be two objects in
  \Cat, and
  \[i : A\to B, \quad i':A' \to B\]
  two maps with target $B$, hence a map
  \[ i\boxtimes i' : (a,a')\mapsto i(a)\times i'(a') : A\times A'\to
  B.\]
  If $i$ is aspheric, then $i\boxtimes i'$ is aspheric if{f} $A'$ is aspheric.
\end{corollarynum}

We have to express that for any object $b$ in $B$, the presheaf
\[ (a,a') \mapsto \Hom(i(a)\times i'(a'),b) \simeq \Hom(i(a),
\bHom(i'(a'),b))\]
on $A\times A'$ is aspheric. For $a'$ fixed in $A'$, the corresponding
presheaf on $A$ is aspheric, as we saw in \ref{it:74.4.a}. The
conclusion now follows from the useful
\begin{lemma}
  Let $F$ be a presheaf on a product category $A\times A'$, with
  $A,A'$ in \Cat. Assume that for any $a'$ in $A'$, the corresponding
  presheaf $a\mapsto F(a,a')$ on $A$ be aspheric. Then the composition
  \[(A\times A')_{/F} \to A\times A' \xrightarrow{\mathrm{pr}_2} A'\]
  is in \scrW, and hence $F$ is aspheric if{f} $A'$ is aspheric.
\end{lemma}
\begin{proof}[Proof of lemma]
  The composition is fibering, as both factors are. This reminds us of
  the ``fibration condition'' \hyperref[it:64.L5]{L~5} on \scrW{}
  (page \ref{p:164}), as yesterday (page \ref{p:219}), where we stated
  that this condition follows from the conditions \ref{loc:1} to
  \ref{loc:3} reviewed yesterday. This condition asserts that a
  fibration with aspheric fibers is in \scrW{} -- hence the lemma.
\end{proof}
\begin{remarks}
  The results stated in prop.\ \ref{prop:74.4} and its corollary give
  a lot of elbow freedom for getting new aspheric functors in terms of
  old ones, with target category $B$ -- provided $B$ satisfies the two
  assumptions: stability under binary products (a property frequently
  met with, although the standard test categories $\Simplex$,
  $\Square$ and $\Globe$ lack it\ldots), and the asphericity of the
  presheaves $\bHom(b,y)$, for any two objects $b,y$ in $B$. A
  slightly stronger condition (indeed an equivalent one, for fixed
  $b$, when $b$ admits already a section over $e_\Bhat$ and if \Bhat{}
  is totally aspheric, hence an aspheric object is even aspheric over
  $e_\Bhat$\ldots) is \emph{contractibility} of the objects $b$ of
  $B$, for the homotopy interval structure on \Bhat{} defined by
  \scrWB{} (i.e., in terms of homotopy intervals aspheric over
  $e_\Bhat$).\pspage{230} (Compare with propositions on pages
  \ref{p:121} and \ref{p:143}.) Whereas this latter assumption
  admittedly is quite a stringent one, it is however of a type which
  has become familiar to us in relation with test categories, where it
  seems a rather common lot.

  If $B$ satisfies these conditions (as in prop.\ \ref{prop:74.4} and
  if $A$ is totally aspheric, we see from prop.\ \ref{prop:74.4} that
  the category $\bAsph(A,B)$ of aspheric functors from $A$ to $B$ is
  stable under binary products. Now, a non-empty object $C$ of \Cat{}
  stable under binary products gives rise to a category $\Chat$
  which is clearly totally aspheric for any basic localizer \scrW, and
  in particular for the usual one \scrWz{}\scrcomment{later, we'll
    write \scrWoo{} for this instead\ldots} corresponding to usual
  weak equivalence. A fortiori, such a category $C$ is $1$-connected
  (which is easily checked too by down-to-earth direct
  arguments). Thus, the reflections of page \ref{p:225} apply, and
  imply that if $\bAsph(A,B)$ is non-empty, i.e., if there is at least
  \emph{one} aspheric functor $i:A\to B$, then there is a canonical
  transitive system of isomorphisms between all functors
  \begin{equation}
    \label{eq:74.star}
    \HotOf_B \to \HotOf_A\tag{*}
  \end{equation}
  of the type $\overline{i^*}$, and hence there is a canonical functor
  \eqref{eq:74.star}, defined up to unique isomorphism, as announced
  in the remark on page \ref{p:225}.
\end{remarks}

\bigbreak
\presectionfill\ondate{13.6.}\par

% 75
\hangsection{Asphericity structures: a bunch of useful tautologies.}%
\label{sec:75}%
The generalities on aspheric maps of the last three sections should be
more than what is needed to develop now the notion of
\emph{asphericity structure} -- which, together with the closely
related notion of \emph{contractibility structure}, tentatively dealt
with before, and the various ``test-notions'' (e.g., \emph{test
  categories} and \emph{test functors}) seems to me the main pay-off
so far of our effort to come to a grasp of a general formalism of
``homotopy models''.

In the case of asphericity structures, just as for the kindred notion
of a contractibility structure, in all instances I could think of a
present, the underlying category $M$ of an asphericity structure is
\emph{not} an object in \Cat{} nor even a ``small category'', but is
``large'' -- namely the cardinality of $\Ob M$ and $\Fl(M)$ are not in
the ``universe'' we are working in, still less is $M$ an object of
\scrU{} -- all we need instead, as usual, is that $M$ be a
\scrU-category, namely that for any two objects of $M$, $\Hom(x,y)$ be
an element of \scrU. Till now, the universe \scrU{} has been present
in our reflections in a very much implicit way, in keeping with the
informal nature of the reflections, which however by and by have
become more formal (as I finally let myself become involved
in\pspage{231} a minimum of technical work, needed for keeping out of
the uneasiness of ``thin air conjecturing''). An attentive reader will
have felt occasionally this implicit presence of \scrU, for instance
in the definition of the basic modelizer \Cat{} (which is, as all
modelizers, a ``large'' category), and in our occasional reference to
various categories as being ``small'' or ``large''. He\scrcomment{or
  she} will have noticed that whereas modelizers are by necessity
large categories (just as \Hot{} itself, whose set of isomorphism
classes of objects is large), test categories are supposed to be small
(and often even to be in \Cat) -- with the effect that \Ahat, the
category of presheaves on $A$, is automatically a \scrU-category
(which would not be the case if $A$ was merely assumed to be a
\scrU-category).

An \emph{asphericity structure} (with respect to the basic localizer
\scrW) on the \scrU-category $M$ consists of a subset
\begin{equation}
  \label{eq:75.1}
  M\subas\subset\Ob M,\tag{1}
\end{equation}
whose elements will be called the \emph{aspheric objects} of $M$ (more
specifically, the \scrW-aspheric objects, if confusion may arise),
this subset being submitted to the following condition:
\begin{description}
\item[\namedlabel{cond:Asstr}{(Asstr)}]
  There exists an object $A$ in \Cat, and a functor $i:A\to M$,
  satisfying the following two conditions:
  \begin{enumerate}[label=(\roman*)]
  \item\label{cond:Asstr.i}
    For any $a$ in $A$, $i(a)\in M\subas$, i.e., $i$ factors through
    the full subcategory (also denoted by $M\subas$) of $M$ defined by
    $M\subas$.
  \item\label{cond:Asstr.ii}
    Let
    \[i^*:M\to\Ahat, \quad x\mapsto i^*(x)=(a\mapsto\Hom(i(a),x)),\]
    then we have
    \begin{equation}
      \label{eq:75.2}
      M\subas = (i^*)^{-1}(\Ahatas),\tag{2}
    \end{equation}
    where $\Ahatas$ is the subset of $\Ob\Ahat$ of all
    (\scrW-)aspheric objects of \Ahat, i.e., the presheaves $F$ on $A$
    such that the object $A_{/F}$ of \Cat{} is \scrW-aspheric.
  \end{enumerate}
\end{description}

In other words, for $x$ in $M$, we have the equivalence
\begin{equation}
  \label{eq:75.2bis}
  x\in M\subas \Leftrightarrow i^*(x)\in\Ahatas,
  \quad\text{i.e.,}\quad
  A_{/x} (\eqdef A_{/i^*(x)}) \in \Cat\subas,\tag{2 bis}
\end{equation}
where $\Cat\subas$ is the subset of $\Ob\Cat$ made up with all
\scrW-aspheric objects of \Cat{} -- i.e., objects $X$ such that
$X\to\Simplex_0$ ($=$ final object) be in \scrW.

Thus, an asphericity structure on $M$ can always be defined by a
functor
\[i:A\to M,\]
with $A$ in \Cat, by the formula \eqref{eq:75.2}; and conversely, any
such functor\pspage{232} defined an asphericity structure $M\subas$ on
$M$, admitting $i$ as a ``\emph{testing functor}'' (i.e., a functor
satisfying \ref{cond:Asstr.i} and \ref{cond:Asstr.ii} above), provided
only we assume that
\begin{equation}
  \label{eq:75.3}
  \text{For any $a$ in $A$, $i^*(i(a))$ is an aspheric object of
    \Ahat.}\tag{3}
\end{equation}
This latter condition is automatically satisfied if $i$ is fully
faithful, for instance, if it is the inclusion functor of a full
subcategory $A$ of $M$. Thus, \emph{any small full subcategory $A$ of
  $M$ defines an asphericity structure on $M$}, and we'll see in a
minute that any asphericity structure on $M$ can be defined this way.

\begin{propositionnum}\label{prop:75.1}
  Let $M\subas$ be an asphericity structure on $M$, and $A,B$ objects
  in \Cat, and
  \[ A \xrightarrow f B \xrightarrow j M\]
  be functors, with $j$ factoring through $M\subas$.
  \begin{enumerate}[label=\alph*),font=\normalfont]
  \item\label{it:75.1.a}
    If $f$ is aspheric, then $j$ is a testing functor if{f} $i=jf$ is.
  \item\label{it:75.1.b}
    If $j$ is fully faithful, and if $i=jf$ is a testing functor, then
    $f$ is aspheric, and $j$ is a testing functor too.
  \end{enumerate}
\end{propositionnum}
\begin{proof}
  \ref{it:75.1.a} follows trivially from
  \[\Bhatas = (f^*)^{-1}(\Ahatas),\]
  which is one of the ways of expressing that $f$ is aspheric
  (criterion \ref{it:72.1.iiprime} of prop.\ \ref{prop:72.1},
  \ref{p:214}). And the first assertion in \ref{it:75.1.b} is a
  trivial consequence of the definition of testing functors, and of
  asphericity of $f$ (by criterion \ref{it:72.1.iii} on p.\
  \ref{p:214}) -- and the second assertion of \ref{it:75.1.b} now
  follows from \ref{it:75.1.a}.
\end{proof}
\begin{corollary}
  Let $i:A\to M$ be a testing functor for $(M,M\subas)$, and let $B$
  any \emph{small} full subcategory of $M\subas$ containing
  $i(A)$. Then the induced functor $A\to B$ is aspheric, and the
  inclusion functor $B\hookrightarrow M$ is a testing functor.
\end{corollary}
\begin{remarks}
  This shows, as announced above, that any asphericity structure on
  $M$ can be defined by a small full subcategory of $M$ (for instance,
  the smallest full subcategory of $M$ containing $i(A)$). We have
  been slightly floppy though, while we defined testing functors by
  insisting that the source should be in Cat{} (which at times will be
  convenient), whereas the $B$ we got here is merely small, namely
  \emph{isomorphic} to an object of \Cat, but not necessarily in
  \Cat{} itself. This visibly is an ``inessential floppiness'', which
  could be straightened out trivially, either by enlarging accordingly
  the notion of testing functor, or by submitted $M$ to the (somewhat
  artificial, admittedly) restriction that all\pspage{233} its small
  subcategories should be in \Cat. This presumably will be satisfied
  by most large categories we are going to consider, and it shouldn't
  be hard moreover to show that any \scrU-category is isomorphic to a
  category $M'$ satisfying the above extra condition.
\end{remarks}


A little more serious maybe is the use we are making here of the name
of a ``testing functor'', which seems to be conflicting with an
earlier use (def.\ \ref{def:65.5}, p.\ \ref{p:175} and def.\
\ref{def:65.6}, p.\ \ref{p:176}), where we insisted for instance that
the source $A$ should be a test category. That's why I have been using
here the name ``testing functor'' rather than ``test functor'', to be
on the safe side formally speaking -- but this is still playing on
words, namely cheating a little. Maybe the name ``\emph{aspherical
  functor}'' instead of ``testing functor'' would be less misleading,
thinking of the case when $M$ is small itself, and endowing $M$ with
the canonical asphericity structure, for which
\[M\subas = M,\]
(admitting the identity functor as a testing functor) -- in which case
the ``testing functors'' $A\to M$ are indeed just the aspherical
functors. The drawback is that when $M$ is small, it may well be
endowed with an asphericity structure $M\subas$ different from the
previous one, in which case the proposed extension of the name
``aspheric functor'' again leads to an ambiguity, unless specified by
``\emph{$M\subas$-aspheric}'' (where, after all, $M\subas$ could be
\emph{any} full subcategory of $M$) -- but then the notion reduces to
the one of an aspheric functor $A\to M\subas$. But the same after all
holds even for large $M$ -- the notion of a ``testing functor'' $A\to
M$ (with respect to an asphericity structure $M\subas$ on $M$) does
not really depend on the \emph{pair} $(M,M\subas)$, but rather on the
(possibly large) category $M\subas$ itself -- namely it is no more no
less than a functor
\[A\to M\subas\]
which satisfies the usual asphericity condition \ref{it:72.1.iii} (of
prop.\ \ref{prop:72.1}, p.\ \ref{p:214}), with the only difference
that $M\subas$ may not be small, and therefore $M\subas\uphat$ may not
be a \scrU-category (and we are therefore reluctant to work with this
latter category at all, unless we first pass to the next larger
universe $\scrU'$\ldots).

This short reflection rather convinces us that the designation of
``testing functors'' as introduced on the page before, by the
alternative name of \emph{$M\subas$-aspheric functors}, or just
\emph{aspheric functors} when no confusion is likely to arise as to
existence and choice of $M\subas$, is satisfactory indeed. I'll use it
tentatively, as a synonym to ``testing functor'', and it will appear
soon enough if this terminology is a good one\pspage{234} or not --
namely if it is suggestive, and not too conducive to
misunderstandings.
\begin{propositionnum}\label{prop:75.2}
  Let $(M,M\subas)$ be an asphericity structure, and let
  \[i:A\to M\]
  be an $M\subas$-aspheric functor, hence $i^*:M\to\Ahat$. Let
  \begin{equation}
    \label{eq:75.4}
    W = (i^*)^{-1}(\scrWA) \subset \Fl(M).\tag{4}
  \end{equation}
  Then $W$ is a mildly saturated subset of $\Fl(M)$, independent of
  the choice of $(A,i)$.
\end{propositionnum}

Mild saturation of $W$ follows from mild saturation of \scrWA, the set
of \scrW-weak equivalences in \Ahat{} (which follows from mild
saturation of \scrW{} and the definition
\[ \text{$\scrWA=i_A^{-1}(\scrW)$ ).}\]
To prove that $W_i$ for $(A,i)$ is the same as $W_{i'}$ for $(A',i')$,
choose a small full subcategory $B$ of $M\subas$ such that $i$ and
$i'$ factor through $B$, let $j:B\to M$ be the inclusion, we only have
to check $W_i=W_j$ (and similarly for $W_{i'}$), which follows from
$i^*=f^*j^*$ (where $f:A\to B$ is the induced functor) and the
relation
\[\scrWA = (f^*)^{-1}(\scrWA),\]
which follows from $f$ being aspheric by prop.\ \ref{prop:75.1}
\ref{it:75.1.b}, and the known property \ref{it:72.1.iv} (prop.\
\ref{prop:72.1}, p. \ref{p:214}) of asphericity, qed.
\begin{remark}
  Once we prove that \scrW{} is even strongly saturated, it will
  follow of course that the sets \scrWA, and hence $W$ above, are
  strongly saturated too.
\end{remark}

We'll call $W$ the set of \emph{weak equivalences} in $M$, for the
given asphericity structure $M\subas$ in $M$. We are interested now in
giving a condition on $M\subas$ ensuring that conversely, $M\subas$ is
known when the corresponding set $W$ of weak equivalences is. We'll
assume for this that $M$ has a final object $e_M$, and we'll call the
asphericity structure $(M,M\subas)$ ``\emph{aspheric}'' if $e_M$ is
aspheric, i.e., $e_M\in M\subas$ (which does not depend of course on
the choice of $e_M$). Now we get the tautology:
\begin{propositionnum}\label{prop:75.3}
  Let $(M,M\subas)$ be an asphericity structure, and $u:x\to y$ a map
  in $M$. If $y$ is aspheric, then $x$ is aspheric if{f} $u$ is
  aspheric.
\end{propositionnum}
\begin{corollary}
  Assume $M$ admits a final object $e_M$, and that $e_M$ is aspheric
  \textup(i.e., $(M,M\subas)$ is aspheric\textup). Then an object $x$
  in $M$ is aspheric if{f} the map $x\to e_M$ is aspheric.
\end{corollary}

Next question then is to state the conditions on a pair $(M,W)$, with
$W\subset\Fl(M)$, for the existence of an aspheric asphericity
structure on $M$,\pspage{235} admitting $W$ as its set of weak
equivalences. We assume beforehand $M$ admits a final object $e_M$. A
n.s.\ condition is the existence of a small category $A$ and a functor
\[i:A\to M\]
satisfying the following two conditions:
\begin{enumerate}[label=(\roman*)]
\item\label{it:75.i}
  for $x$ in $M$ of the form $i(a)$ ($a\in\Ob A$) \emph{or} $e_M$,
  $i^*(x)\in\Ahatas$,
\item\label{it:75.ii}
  $W=(i^*)^{-1}(\scrWA)$.
\end{enumerate}
Another n.s.\ condition, which looks more pleasant I guess, is that
there exist a small full subcategory $B$ of $M$, containing $e_M$,
with inclusion functor $j:B\to M$, such that
\begin{equation}
  \label{eq:75.5}
  W=(j^*)^{-1}(\scrWB).\tag{5}
\end{equation}
In any case, if we want a n.s.\ condition on $W$ for it to be the set
of weak equivalences for some asphericity structure on $M$ (not
necessarily an aspheric one, and therefore maybe not unique), we get:
there should exist a small full subcategory $B$ of $M$, with inclusion
functor $j$, such that \eqref{eq:75.5} above holds. (Here we do not
assume $e_M$ to exist, still less $B$ to contain it.) Similarly, the
corollary of prop.\ \ref{prop:75.1} implies that a subset $M\subas$ of
$\Ob M$ is an asphericity structure on $M$, if{f} there exists a small
full subcategory $B$ in $M$, such that
\begin{equation}
  \label{eq:75.6}
  M\subas = (j^*)^{-1}(\Bhatas);\tag{6}
\end{equation}
moreover, if $M$ admits a final object $e_M$, the asphericity
structure $M\subas$ is aspheric if{f} $B$ can be chosen to contain
$e_M$.
\addtocounter{remarksnum}{1}
\begin{remarksnum}
  The relationship between aspheric asphericity structures $M\subas$
  on $M$, and sets $W \subset \Fl(M)$ of ``weak equivalences'' in $M$
  satisfying the condition above (and which we may call ``weak
  equivalence structures'' on $M$), in case $M$ admits a final object
  $e_M$, is reminiscent of the relationship between ``contractibility
  structures'' $M\subc\subset\Ob M$ on $M$, and those ``homotopism
  structures'' $h_M\subset\Fl(M)$ on $M$ which can be described in
  terms of such a contractibility structure (cf.\ sections
  \ref{sec:51} and \ref{sec:52}). Both pairs can be viewed as giving
  two equivalent ways of expressing one and the same kind of structure
  -- the structure concerned by the first pair being centered on
  asphericity notions, whereas the second is concerned with typical
  homotopy notions rather. We'll see later that any contractibility
  structure defines in an evident way an asphericity structure, and in
  the most interesting cases (e.g., canonical modelizers), it is
  uniquely determined by the latter.
\end{remarksnum}

To\pspage{236} sum up some of the main relationships between the three
asphericity notions just introduced ($M\subas$, $W$,
$M\subas$-aspheric functors), let's state one more tautological
proposition, which is very much a paraphrase of the display given
earlier (prop.\ on p.\ \ref{p:214}) of the manifold aspects of the
notion of an aspheric map between small categories:
\begin{propositionnum}\label{prop:75.4}
  Let $(M,M\subas)$ be an asphericity structure, $A$ a small category,
  $i:A\to M$ a functor, factoring through $M\subas$. Consider the
  following conditions on $i$:
  \begin{description}
  \item[\namedlabel{it:75.4.i}{(i)}]
    $i$ is $M\subas$-aspheric \textup(or ``testing functor'' for
    $M\subas$\textup), i.e.,
    \[ M\subas = (i^*)^{-1}(\Ahatas).\]
  \item[\namedlabel{it:75.4.iprime}{(i')}]
    For $x$ in $M\subas$, $i^*(x)$ is aspheric, i.e.,
    \[ M\subas \subset (i^*)^{-1}(\Ahatas).\]
  \item[\namedlabel{it:75.4.idblprime}{(i'')}]
    \textup(Here, $B$ is a given small full subcategory of $M\subas$
    containing $i(A)$ and which ``\emph{generates}'' the asphericity
    structure $M\subas$, namely such that the inclusion functor
    $j:B\hookrightarrow M$ is $M\subas$-aspheric, i.e.,
    $M\subas=(j^*)^{-1}(\Bhatas)$.\textup)
    For any $x$ in $B$,
    $i^*(x)$ is aspheric, i.e.,
    \[ B \subset (i^*)^{-1}(\Ahatas).\]
  \item[\namedlabel{it:75.4.ii}{(ii)}]
    For any map $u$ in $M$, $u$ is a weak equivalence if{f} $i^*(u)$
    is, i.e.,
    \[ W = (i^*)^{-1}(\scrWA).\]
  \item[\namedlabel{it:75.4.iiprime}{(ii')}]
    If the map $u$ in $M$ is a weak equivalence, so is $i^*(u)$, i.e.,
    \[ W \subset (i^*)^{-1}(\scrWA).\]
  \item[\namedlabel{it:75.4.iidblprime}{(ii'')}]
    \textup(Here, $B$ is given as in \textup{\ref{it:75.4.idblprime}}
    above\textup)
    For any map $u$ in $M$, $i^*(u)$ is a weak equivalence, i.e.,
    \[ \Fl(B) \subset (i^*)^{-1}(\scrWA).\]
  \end{description}
  The conditions
  \textup{\ref{it:75.4.i}\ref{it:75.4.iprime}\ref{it:75.4.idblprime}}
  are equivalent and imply all others, and we have the tautological
  implications \textup{\ref{it:75.4.ii}} $\Rightarrow$
  \textup{\ref{it:75.4.iiprime}} $\Rightarrow$
  \textup{\ref{it:75.4.iidblprime}}. If $M$ admits a final object
  $e_M$ and if $A$ and the asphericity structure $M\subas$ are
  aspheric, then all six conditions \textup(except the last\textup)
  are equivalent; and all six are equivalent if moreover $e_M\in\Ob B$.
\end{propositionnum}
\begin{proof}
  The implications \ref{it:75.4.i} $\Rightarrow$ \ref{it:75.4.iprime}
  $\Rightarrow$ \ref{it:75.4.idblprime} are tautological, on the other
  hand \ref{it:75.4.idblprime} just means that the induced functor $f:
  A\to B$ is aspheric, which by prop.\ \ref{prop:75.1} \ref{it:75.1.a}
  implies that $i$ is aspheric. On the other hand \ref{it:75.4.i}
  $\Rightarrow$ \ref{it:75.4.ii} by the definition of $W$ (cf.\ prop.\
  \ref{prop:75.2}). If $e_M$ exists and $A$ and the asphericity
  structure $M\subas$ are aspheric, and if $B$ contains $e_M$, then
  \ref{it:75.4.iidblprime} implies that the maps $x\to e_M$ for $x$ in
  $B$ are transformed by $i^*$ into a weak equivalence, and as
  $e_\Ahat$ is aspheric, this implies $i^*(x)$ is aspheric,
  i.e., \ref{it:75.4.idblprime}, which proves the last
  statement\pspage{237} of the proposition -- all six conditions are
  equivalent in this case. If no $B$ is given, but still assuming $A$
  and $(M,M\subas)$ aspheric, we can choose a generating subcategory
  for $(M,M\subas)$ large enough in order to contain $i(A)$ and $e_M$,
  and we get that conditions \ref{it:75.4.i} to \ref{it:75.4.iiprime}
  are equivalent, qed.
\end{proof}

% 76
\hangsection{Examples. Totally aspheric asphericity structures.}%
\label{sec:76}%
It's time to give some examples.

\namedlabel{ex:76.1}{1)}\enspace
Take $M=\Cat$, $M\subas=$ set of \scrW-aspheric objects in \Cat. We
then got an asphericity structure, as we see by taking any weak test
category $A$ in \Cat{} (def.\ \ref{def:65.2} on page \ref{p:172}), and
the functor
\[i_A: a\mapsto A_{/a} : A \to\Cat,\]
which satisfies indeed \ref{cond:Asstr.i} and \ref{cond:Asstr.ii} of
p.\ \ref{p:231} above.

We may call
\begin{equation}
  \label{eq:76.7}
  (\Cat, \Cat_{\scrW\mathrm{-as}} \eqdef \Cat\subas)\tag{7}
\end{equation}
the ``basic asphericity structure'', giving rise (by taking the
corresponding set $W$ of ``weak equivalences'') to the ``basic
modelizer'' $(\Cat,\scrW)$. Turning attention towards the former
corresponds to a shift in emphasis; whereas previously, our main
emphasis has been dwelling consistently with the notion of ``weak
equivalence'', namely with giving on a category $M$ a bunch of
\emph{arrows} $W$, here we are working rather with notion of
``aspheric \emph{objects}'' as the basic notion. One hint in this
direction comes from prop.\ \ref{prop:72.1} on p.\ \ref{p:214}, when
we saw that for a functor $f:A\to B$ between small categories, giving
rise to $f^*:\Bhat\to\Ahat$ (a map between asphericity structures, as
a matter of fact), asphericity of $f$ can always be expressed in
manifold ways as a property of $f$ relative to the notion of aspheric
\emph{objects} in \Ahat{} and \Bhat, but not as a property relative to
the notion of weak equivalences, unless $A$ and $B$ are assumed to be
aspheric.

Coming back to the case of the ``basic asphericity structures''
\eqref{eq:76.7}, we get more general types of ``aspheric'' functors
$A\to\Cat$ than functors $i_A$, by taking any weak test category $A$
and any weak test functor $i:A\to\Cat$ (cf.\ def.\ \ref{def:65.5}, p.\
\ref{p:175}), provided however the objects $i(a)$ are aspheric. It
would seem though that, for a given $A$, even assuming $A$ to be a
\emph{strict} test category say (and even a ``contractor'' moreover),
and restricting to functors $i:A\to\Cat$ which factor through
$\Cat\subas$ (to make them eligible for being ``aspheric functors''),
the condition for $i$ to be a weak test functor, namely for $i^*$ to
be ``model preserving'', is substantially stronger than mere
``asphericity'': indeed, the latter just means that $i^*$ transforms
weak\pspage{238} equivalences into weak equivalences, i.e., gives rise
to a functor
\[\HotOf{} = \scrW^{-1}\Cat \to \HotOf_A=\scrWA^{-1}\Ahat,\]
whereas the latter insists that this functor moreover should be an
equivalence of categories. Theorem~\ref{thm:65.1} (p.\ \ref{p:176})
gives a hint though that the two conditions may well be equivalent --
this being so at any rate provided the objects $i(a)$ in \Cat{} are,
not only aspheric, but even \emph{contractible}. This reminds us at
once of the ``silly question'' of section \ref{sec:46} (p.\
\ref{p:95}), which was the starting point for the subsequent
reflections leading up to the theorem~\ref{thm:65.1} recalled above;
and, beyond this still somewhat technical result, the ultimate
motivation for the present reflections on asphericity structures. The
main purpose for these, I feel, is to lead up to a comprehensive
answer to the ``silly question''. We'll have to come back to this very
soon!

\namedlabel{ex:76.2}{2)}\enspace
Take $M=\Spaces$, the category of topological spaces, and $M\subas$
the spaces which are weakly equivalent to a point. We get an
asphericity structure, indeed an \emph{aspheric} asphericity structure
(I forgot to make this evident specification in the case \ref{ex:76.1}
above), as we see by taking $A=\Simplex$ for instance, and
\[ i:\Simplex\to\Spaces\]
the ``geometric realization functor'' for simplices, which satisfies
conditions \ref{cond:Asstr.i} and \ref{cond:Asstr.ii} of page
\ref{p:231}, as is well known (cf.\ the book of
Gabriel-Zisman);\scrcomment{\cite{GabrielZisman1967}}
as a matter of fact, $i$ is even a test functor for the modelizer
$(M,W)$ (which is one way of stating the main content of GZ's book).

\namedlabel{ex:76.3}{3)}\enspace
The two examples above are \emph{aspheric} asphericity structures, and
such moreover that $(M,W)$ is (\scrW-)modelizing. These extra features
however are not always present in the next example
\begin{equation}
  \label{eq:76.8}
  M=\Ahat, \quad M\subas=\Ahatas,\quad
  \text{$A$ a small category,}
  \tag{8}
\end{equation}
which is indeed tautologically an asphericity structure, by taking the
canonical inclusion functor (which is fully faithful)
\[ A \to \Ahat,\]
satisfying the conditions \ref{cond:Asstr.i}\ref{cond:Asstr.ii} above
(p.\ \ref{p:231}). This shows moreover that the corresponding notion
of weak equivalence is the usual one, (I forgot to state the similar
fact in example \ref{ex:76.2}, sorry):
\[ W = \scrWA.\]
The asphericity structure is aspheric if{f} $A$ is aspheric. A functor
\[ A'\to A\]
where\pspage{239} $A'$ is another small category, is
$M\subas$-aspheric as a functor from $A'$ to \Ahat, if{f} it is
aspheric.

This last example suggests to call an asphericity structure
$(M,M\subas)$ ``\emph{totally aspheric}'' if $M$ is stable under
finite products, and if the final object of $M$, as well as the
product of any two aspheric objects of $M$, is again aspheric; in
other words, if any finite product in $M$ whose factors are aspheric
is aspheric. We have, in this respect:
\addtocounter{propositionnum}{4}
\begin{propositionnum}\label{prop:76.5}
  Let $(M,M\subas)$ be an asphericity structure, where $M$ is stable
  under finite products. The following conditions are equivalent:
  \begin{enumerate}[label=(\roman*),font=\normalfont]
  \item\label{it:76.5.i}
    $M$ is totally aspheric, i.e., $e_M$ and the product of any two
    aspheric objects of $M$ are aspheric.
  \item\label{it:76.5.ii}
    There exists a small subcategory $B$ of $M$, stable under finite
    products \textup(i.e., containing a final object $e_M$ of $M$ and,
    with any two objects $x$ and $y$, a product $x\times y$ in
    $M$\textup), and which generates the asphericity structure
    \textup(i.e., $j:B\hookrightarrow M$ is
    $M\subas$-aspheric\textup).
  \item\label{it:76.5.iii}
    There exists a small category $A$ such that \Ahat{} is totally
    aspheric \textup(def.\ \ref{def:65.1}, p.\ \ref{p:170}\textup),
    and a $M\subas$-aspheric functor $A\to M$.
  \end{enumerate}
\end{propositionnum}

The proof is immediate. Of course, the asphericity structure in
example \ref{ex:76.3} is totally aspheric if{f} \Ahat{} is totally
aspheric in the usual sense referred to above.

% 77
\hangsection{The canonical functor
  \texorpdfstring{$\HotM\to\Hot$}{Hot-M -> (Hot)}.}\label{sec:77}%
Let
\begin{equation}
  \label{eq:77.9}
  \scrM = (M,M\subas)
  \tag{9}
\end{equation}
be an asphericity structure, which will be referred to also as merely
$M$, when no ambiguity concerning $M\subas$ is feared. We'll write
\begin{equation}
  \label{eq:77.10}
  \HotOf_\scrM \text{ (or simply $\HotOf_M$) } = W^{-1}M,
  \tag{10}
\end{equation}
where $W$ of course is the set of weak equivalences in $M$. We are now
going to define a canonical functor
\begin{equation}
  \label{eq:77.11}
  \HotOf_\scrM \to \Hot_\scrW \quad
  (\text{or simply $\Hot\eqdef\scrW^{-1}\Cat$}),
  \tag{11}
\end{equation}
defined at any rate up to canonical isomorphism. For this, take any
aspheric (namely, $M\subas$-aspheric) functor
\[ i : A\to M,\]
and consider the composition
\begin{equation}
  \label{eq:77.12}
  M \xrightarrow{i^*} \Ahat \xrightarrow{i_A} \Cat.
  \tag{12}
\end{equation}
The\pspage{240} three categories in \eqref{eq:77.12} are endowed
respectively with sets of arrows $W$, \scrWA, \scrW, and the two
functors satisfy the conditions
\begin{equation}
  \label{eq:77.13}
  W = (i^*)^{-1}(\scrWA)\quad
  \text{and}\quad
  \scrWA=(i_A)^{-1}(\scrW),\quad
  \tag{13}
\end{equation}
hence $W=(i_{M,i})^{-1}(\scrW)$, where
\[i_{M,i} : M\to \Cat\]
is the composition $i_Ai^*$. Therefore, we get a functor
\begin{equation}
  \label{eq:77.14}
  \overline{i_{M,i}} : \HotOf_M \to \Hot,\tag{14}
\end{equation}
which a priori depends upon the choice of $(A,i)$. If we admit strong
saturation of \scrW, it follows that this functor is ``conservative'',
namely an arrow in $\HotOf_M$ is an isomorphism, provided its image in
\Hot{} is.

To define \eqref{eq:77.11} in terms of \eqref{eq:77.14}, we have to
describe merely a transitive system of isomorphisms between the
functors \eqref{eq:77.14}, for varying pair $(A,i)$. Therefore,
consider two such $(A,i)$ and $(A',i')$, choose a small full
subcategory $B$ of $M\subas$ containing both $i(A)$ and $i'(A')$
(therefore, $B$ is a generating subcategory for the asphericity
structure $M\subas$), and consider the inclusion functor
$j:B\hookrightarrow M$. From prop.\ \ref{prop:75.1} \ref{it:75.1.b} it
follows that the functors
\[f: A\to B, \quad f':A'\to B\]
induced by $i,i'$ are aspheric. Using this, and the criterion
\ref{it:72.1.i} of asphericity (prop.\ \ref{prop:72.1} on p.\
\ref{p:214}) we immediately get two isomorphisms
\[\begin{tikzcd}[baseline=(O.base),column sep=tiny,row sep=small,arrows=phantom]
  & \overline{i_{M,j}} \ar[dl,swap,sloped,"\simeq"{description}]
    \ar[dr,sloped,"\simeq"{description}] & \\
  \overline{i_{M,i}} & & |[alias=O]| \overline{i_{M,i'}}
\end{tikzcd},\]
and hence an isomorphism
\[ \overline{i_{M,i}} \tosim \overline{i_{M,i'}},\]
depending only on the choice of $B$. As a matter of fact, it doesn't
depend on this choice. To see this, we are reduced to comparing the
two isomorphisms arising from a $B$ and a $B'$ such that $B\subset
B'$, in which case the inclusion functor $g:B\to B'$ is aspheric
(prop.\ \ref{prop:75.1}). We leave the details of the verification
(consisting mainly of some diagram chasing and some compatibilities
between the $\lambda_i(F)$-isomorphisms of prop.\ \ref{prop:75.1} on
page \ref{p:214}) to the skeptical reader. As for transitivity for a
triple $(A,i)$, $(A',i')$, $(A'',i'')$, it now follows at once, by
using a $B$ suitable simultaneously\pspage{241} for all three.

Having thus well in hand the basic functor \eqref{eq:77.11} (which in
case of example \ref{ex:76.3} above with $M=\Ahat$, reduces to the
all-important functor $\HotOf_A\to\Hot$), we cannot but define an
asphericity structure to be \emph{modelizing}, as meaning that this
canonical functor is an equivalence of categories. In the case of
$M=\Ahat$ above, this means that $A$ is a pseudo-test category -- a
relatively weak test notion still. It appears as just as small bit
stronger than merely assuming the pair $(M,W)$ to be modelizing, i.e.,
to be a ``\emph{modelizer}'', namely assuming $\HotOf_M$ to be
equivalent (in some way or other\ldots) to \Hot. Maybe we should be a
little more cautious with the use of the word ``modelizing'' though,
and devise a terminology which should reflect very closely the
hierarchy of progressively stronger test notions
\begin{multline*}
  \text{(pseudo-test cat.)}
  \supset
  \text{(weak test cat.)}
  \supset\\
  \text{(test cat.)}
  \supset
  \text{(strict test cat.)}
  \supset
  \text{(contractors)}
\end{multline*}
which gradually has peeled out of our earlier reflections, by
pinpointing corresponding qualifications for an asphericity structure,
as being ``pseudo-modelizing'', ``weakly modelizing'', ``modelizing'',
``strictly modelizing'', and ???. It now appears that a little extra
reflection is needed here -- for today it's getting a little late
though!

\bigbreak
\presectionfill\ondate{18.6} and \ondate{19.6.}\par

% 78
\hangsection[Test functors and modelizing asphericity structures: the
\dots]{Test functors and modelizing asphericity structures: the
  outcome \texorpdfstring{\textup(}{(}at
  last!\texorpdfstring{\textup)}{)} of an easy
  ``observation''.}\label{sec:78}%
It is about time now to get a comprehensive treatment, in the context
of asphericity structures, of the relationships suggested time ago in
the ``observation'' and the ``silly question'' of section \ref{sec:46}
(pages \ref{p:94} and \ref{p:95}). The former is concerned with
test-functors from test categories to modelizers, the second more
generally with model-preserving maps between modelizers, having
properties similar to the map $M\to\Ahat$ stemming from a
test-functor. For the time being, as I found but little time for the
``extra reflection'' which seems needed, only the first situation is
by now reasonably clear in my mind.

As no evident series of ``asphericity structure''-notions has
appeared, paralleling the series of test-notions recalled by the end
of last Monday's reflections (see above), I'm going to keep
(provisionally only, maybe) the name of a \emph{modelizing asphericity
  structure} $(M,M\subas)$ as one for which the canonical functor
\begin{equation}
  \label{eq:78.1}
  \HotOf_M = W^{-1}M \to \Hot
  \tag{1}
\end{equation}
(cf.\ \eqref{eq:77.14}, p.\ \ref{p:240}) is an equivalence. This notion
at any rate is\pspage{242} satisfactory for formulating the following
statement, which comes out here rather tautologically, and which
however appears to me as exactly what I had been looking for in the
``observation'' recalled above:
\begin{theoremnum}\label{thm:78.1}
  Let $(M,M\subas)$ be a \emph{modelizing} asphericity structure, $A$
  a pseudo-test category, and
  \[i: A\to M\]
  a functor, factoring through $M\subas$ \textup(i.e., $i(a)$ is
  aspheric for any $a$ in $A$\textup). We assume moreover $M$ has a
  final object $e_M$. Then the six conditions \textup{\ref{it:75.4.i}}
  to \textup{\ref{it:75.4.iidblprime}} of prop.\ \ref{prop:75.4}
  \textup(p.\ \ref{p:236}\textup) \textup(the first of which
  expresses that $i$ is $M\subas$-aspheric\textup) are equivalent, and
  they are equivalent to the following condition:
  \begin{description}
  \item[\namedlabel{it:78.1.iii}{(iii)}]
    The functor $i^*:M\to\Ahat$ gives rise to a functor
    \[ \HotOf_M = W^{-1}M \to \HotOf_A = \scrWA^{-1}\Ahat\]
    \textup(i.e., \textup{\ref{it:75.4.iiprime}} of prop.\
    \ref{prop:75.4} holds, namely $i^*(W)\subset\scrWA$\textup),
    \emph{and} this functor moreover is an equivalence of categories.
  \end{description}
\end{theoremnum}
\begin{proof}
  For the first statement (equivalence of conditions \ref{it:75.4.i}
  to \ref{it:75.4.iidblprime}), by prop.\ \ref{prop:75.4} we need
  only show that the asphericity structure and $A$ as aspheric. But
  this follows from the assumptions and from the
\end{proof}
\begin{lemma}
  Let $(M,M\subas)$ be any modelizing asphericity structure, then a
  final object of $M$ is aspheric \textup(i.e., the structure is
  ``aspheric''\textup). In particular, if $A$ is a pseudo-test
  category \textup(i.e., $(\Ahat,\Ahatas)$ is a modelizing asphericity
  structure\textup), then $A$ is aspheric.
\end{lemma}

We only have to prove the first statement. It is immediate that $e_M$
is a final object of $W^{-1}M = \HotOf_M$ (this is valid whenever we
got a localization $W^{-1}M$ of a category $M$ with final object
$e_M$), hence its image in $\Hot=\scrW^{-1}\Cat$ is a final
object. Thus, we are reduced to proving the following
\begin{corollary}
  Let $(M,M\subas)$ be any asphericity structure, consider the
  composition
  \[ \varphi_M:M\to\HotOf_M=W^{-1} \xrightarrow{\mathrm{can.}} \Hot,\]
  then we get
  \[ M\subas = \set[\Big]{x\in\Ob M}{\text{$\varphi_M(x)$ is a final
      object of \Hot}}.\]
\end{corollary}

Indeed, using the construction of $\HotOf_M\to\Hot$ in terms of a
given $M\subas$-aspheric functor $B\to M$, we are reduced to the same
statement, with $(M,M\subas)$ replaced by $(\Bhat,\Bhatas)$. The
statement then reduced to: the object $B$ in \Cat{} is aspheric if{f}
its image in \Hot{} is a final object (this can be viewed also as the
particular case of the\pspage{243} corollary, when $M=\Cat$,
$M\subas=\Cat\subas$). Now, this follows at once from strong
saturation of \scrW, which (as we announced earlier) followed from
\ref{loc:1} to \ref{loc:3} (and will be proved in part \ref{ch:V} of
the notes). The reader who fears a vicious circle may till then
restrict use of the theorem to the case when we assume beforehand that
$e_M$ and $e_\Ahat$ are aspheric.

It is now clear that the six conditions of prop.\ \ref{prop:75.4} are
equivalent, and they are of course implied by
\ref{it:78.1.iii}. Conversely, they imply \ref{it:78.1.iii}, as
follows from the fact that in the canonical diagram (commutative up to
canonical isomorphism)
\[\begin{tikzcd}[baseline=(O.base),column sep=tiny]
  \HotOf_M\ar[rr]\ar[dr] & & \HotOf_A\ar[dl] \\
  & |[alias=O]| \Hot &
\end{tikzcd},\]
the two downwards arrows are equivalences, qed.

The theorem above seems to me to be exactly \emph{the} ``something
very simple-minded surely'' which I was feeling to get burningly
close, by the end of March, nearly three months ago (p.\ \ref{p:89});
at least, to be ``just it'' as far as the case of \emph{test-functors}
is concerned. We may equally view this theorem as giving the precise
relationship between the notion of a weak test functor or a \emph{test
  functor} (the latest version of which (in the context of
\scrW-notions) appears in section \ref{sec:65} (def.\ \ref{def:65.5}
and \ref{def:65.6}, pp.\ \ref{p:175} and \ref{p:176})), and the notion
of \emph{aspheric functors}, more precisely of $M\subas$-aspheric
functors, introduced lately (p.\ \ref{p:233}). This now is the moment
surely to check if the terminology of test functors and weak ones
introduced earlier, before the relevant notion of asphericity
structures was at hand, is really satisfactory indeed, and if needed,
adjust it slightly.

So let again
\[i:A\to M\]
be a functor, with $A$ small, and $(M,M\subas)$ an asphericity
structure. We don't assume beforehand, neither that $A$ is a
test-category or the like, nor that $(M,M\subas)$ be modelizing. We
now paraphrase def.\ \ref{def:65.5} (p.\ \ref{p:175}) of weak test
functors as follows:\scrcomment{it looks like there are only two
  conditions, but see \ref{it:78.1.1}--\ref{it:78.1.3} below\ldots}
\begin{definitionnum}\label{def:78.1}
  The functor $i$ above is called a \emph{pseudo-test functor}, if it
  satisfies the following three conditions:
  \begin{enumerate}[label=\alph*)]
  \item\label{it:78.1.a}
    The corresponding functor $i^*:M\to\Ahat$ is ``model-preserving'',
    by\pspage{244} which is meant here that
    \[ W = (i^*)^{-1}(\scrWA),\]
    \emph{and} the induced functor
    \[\HotOf_M=W^{-1}M \to \HotOf_A=\scrWA^{-1}\Ahat\]
    is an equivalence.
  \item\label{it:78.1.b}
    The functor $i_A:\Ahat\to\Cat$ is model preserving (for
    $\scrWA,\scrW$), which reduces to the canonical functor
    \[\HotOf_A=\scrWA^{-1}\Ahat\to\Hot=\scrW^{-1}\Cat\]
    being an equivalence (as we know already that
    $\scrWA=(i_A)^{-1}(\scrW)$, by definition of \scrWA).
  \end{enumerate}
\end{definitionnum}

Condition \ref{it:78.1.b} just means that $A$ is a pseudo-test
category, i.e., that the asphericity structure $(\Ahat,\Ahatas)$ it
defines is modelizing. By the corollary of lemma above, it implies
that $A$ is aspheric (which corresponds to condition \ref{it:65.E.c}
of def.\ \ref{def:65.5} in loc.\ cit.). Condition \ref{it:78.1.a}
comes in two parts, the first just meaning that $i$ is
$M\subas$-aspheric -- this translation being valid, at any rate, in
case we assume already $A$ aspheric, and the given asphericity
structure is aspheric, i.e., $M$ admits a final object $e_M$, and
$e_M$ is aspheric. We certainly do want a pseudo-test functor to be
(at the very least) $M\subas$-aspheric, so we should either strengthen
condition \ref{it:78.1.a} to this effect (which however doesn't look
as nice), or assume beforehand $(M,M\subas)$ aspheric. At any rate, if
we use the first variant of the definition, condition \ref{it:78.1.a}
in full then is equivalent (granting \ref{it:78.1.b}) to:
\begin{description}
\item[\namedlabel{it:78.1.aprime}{a')}]
  The functor $i$ is $M\subas$-aspheric, \emph{and} $(M,M\subas)$ is modelizing,
\end{description}
which in turn implies that $(M,M\subas)$ is aspheric. So we may as
well assume asphericity of $(M,M\subas)$ beforehand! At any rate, we
see that the notion we are after can be decomposed into three
conditions, namely:
\begin{enumerate}[label=\arabic*)]
\item\label{it:78.1.1} $A$ is a pseudo-test category, i.e.,
  $(\Ahat,\Ahatas)$ is modelizing.
\item\label{it:78.1.2} $(M,M\subas)$ is modelizing.
\item\label{it:78.1.3} The functor $i$ is $M\subas$-aspheric.
\end{enumerate}

The two first conditions are just conditions on $A$ and on
$(M,M\subas)$ respectively, the third is just the familiar asphericity
condition on $i$.

To get the notion of a \emph{weak test-functor} (def.\ \ref{def:65.5},
p.\ \ref{p:175}) we have to be just one step more specific in
\ref{it:78.1.1}, by demanding that $A$ be even a\pspage{245} weak test
category, namely that the functor
\[i_A^*=j_A:\Cat\to\Ahat\]
be model-preserving (which implies that $i_A$ is too).

Following def.\ \ref{def:65.6} (p.\ \ref{p:176}), we'll say that $i$
is a \emph{test functor} if $i$ \emph{and} the induced functors
$i_{/a}:A_{/a}\to A\to M$ are weak test functors. Using the definition
of a test category (def.\ \ref{def:65.3}, p.\ \ref{p:173}), we see
that this just means that the following conditions hold:
\begin{enumerate}[label=\arabic*')]
\item\label{it:78.1.1prime} $A$ is a test category.
\item\label{it:78.1.2prime} $(M,M\subas)$ is modelizing (i.e., same as
  \ref{it:78.1.2} above).
\item\label{it:78.1.3prime} The functor $i$ and the induced functors
  $i_{/a}:A_{/a}\to M$ are $M\subas$-aspheric.
\end{enumerate}

This last condition merits a name by itself, independently of other
assumptions:

\begin{definitionnum}\label{def:78.2}
  Let $(M,M\subas)$ be any asphericity structure, $A$ a small
  category, and $i:A\to M$ a functor. We'll say that $i$ is
  \emph{totally $M\subas$-aspheric} (or simply \emph{totally
    aspheric}, if no confusion is feared), if $i$ and the induced
  functors $i_{/a}:A_{/a}\to M$ are $M\subas$-aspheric (for any $a$ in
  $A$). We'll say that $i$ is \emph{locally $M\subas$-aspheric} (or
  simply \emph{locally aspheric}) if for any $a$ in $A$, the induced
  functor $i_{/a}:A_{/a}\to M$ is $M\subas$-aspheric.
\end{definitionnum}

Thus, $i$ is totally aspheric if{f} it is aspheric and locally
aspheric. On the other hand, $i$ is a test functor if{f} $A$ is a test
category, $(M,M\subas)$ is modelizing, and $i$ is totally aspheric.

In order for $i$ to be locally aspheric, it is n.s.\ that $i$ factor
through $M\subas$ and for any $x$ in $M\subas$, $i^*(x)$ be
\emph{aspheric over $e_\Ahat$}; if $B$ is any subcategory of $M$
containing $i(A)$ and generating the asphericity structure, it is
enough in this latter condition to take $x$ in $B$.

\begin{remarks}
  Thus, we see that the three gradations for the test-functor notion,
  as suggested by the definitions \ref{def:65.5} and \ref{def:65.6} of
  section \ref{sec:65} and now by the present context of asphericity
  structures, just amount to gradations for the test conditions on the
  category $A$ itself (namely, to be a pseudo-test, a weak test or
  just a plain test -category), and a two-step gradation on the
  asphericity condition for $i$ (namely, that $i$ be either just
  aspheric in the two first cases, or totally aspheric in the third),
  while these two asphericity conditions on $i$ are of significance,
  independently of any specific assumption which we may make on either
  $A$ or $(M,M\subas)$. This\pspage{246} seems to diminish somewhat
  the emphasis I had put formerly upon the notion of a test functor
  and its weak variant, and enhance accordingly the notion of an
  aspheric functor (with respect to a given asphericity structure) and
  the two related asphericity notions for $i$ which spring from it
  (namely, the notions of locally and of totally aspheric functors),
  which now seem to come out as the more relevant and the more general
  ones.
\end{remarks}

To be wholly happy, we still need the relevant reformulation, in terms
of asphericity structures, of the main result of section \ref{sec:65},
namely theorem \ref{thm:65.1} (p.\ \ref{p:177}) characterizing test
functors with values in \Cat, under the assumption that the objects
$i(a)$ be contractible. This theorem, I recall, has been the main
outcome of the ``grinding'' reflections taking their start with the
``observation'' on p.\ \ref{p:94} about ten days earlier. The appreach
then followed, as well as the contractibility assumption for the
objects $i(a)$ made in the theorem, retrospectively look awkward -- it
is clear that the relevant notions of asphericity structures, and of
aspheric functors into these, had been lacking. The theorem stated
last (p.\ \ref{p:242}) looks indeed a lot more satisfactory than the
former, except however in one respect -- namely that the asphericity
condition on $i$, in theorem \ref{thm:65.1} of p.\ \ref{p:177}, can be
expressed by asphericity over $e_\Ahat$ \emph{of just
  $i^*(\Simplex_1)$ alone}, rather than having to take $i^*(x)$ for
all $x$ in some fixed generating subcategory of $(M,M\subas)$
containing $i(A)$. To recover such kind of minimal criterion in a more
general case than \Cat, we'll have to relate the notion of an
asphericity structure with the earlier one of a \emph{contractibility
  structure}; the latter in our present reflections has faded somewhat
into the background, while at an earlier stage the homotopy and
contractibility notions had invaded the picture to an extent as to
overshadow and nearly bring to oblivion the magic of the ``asphericity
game''.

\scrcommentinline{The rest of this page is unreadable in my scan of
  the typescript.}

\bigbreak
\presectionfill\ondate{20.6.}\pspage{247}\par

% 79
\hangsection[Asphericity structure generated by a contractibility
\dots]{Asphericity structure generated by a contractibility structure:
  the final shape of the ``awkward main result'' on test
  functors.}\label{sec:79}%
I would like now to write out the relationship between asphericity
structures, and contractibility structures as defined in section
\ref{sec:51} \ref{subsec:51.D} (pages \ref{p:117}--\ref{p:119}). First
we'll need to rid ourselves of the smallness assumption for a
generating category of an asphericity structure:
\begin{propositionnum}\label{prop:79.1}
  Let $M$ be a \scrU-category \textup(\scrU{} being our basic
  universe\textup), and $N$ any full subcategory. The following
  conditions are equivalent:
  \begin{enumerate}[label=(\roman*),font=\normalfont]
  \item\label{it:79.1.i}
    There exists an asphericity structure $M\subas$ in $M$ such that
    \textup{\namedlabel{it:79.1.a}{(a)}}\enspace $M\subas\supset N$
    and \textup{\namedlabel{it:79.1.b}{(b)}}\enspace $N$ admits a
    \emph{small} full subcategory $N_0$ generating the asphericity
    structure, i.e., such that the inclusion functor $i_0$ from $N_0$
    to $M$ be $M\subas$-aspheric.
  \item\label{it:79.1.ii}
    There exists a small full subcategory $N_0$ of $N$, such that for
    any $x$ in $N$, $i_0^*(x)$ in $N_0\uphat$ be aspheric, where
    $i_0:N_0\to M$ is the inclusion functor.
  \item\label{it:79.1.iii}
    The couple $(N,N)$ is an asphericity structure.
  \end{enumerate}
  Moreover, when these conditions are satisfied, the asphericity
  structure $M\subas$ in \textup{\ref{it:79.1.i}} is unique.
\end{propositionnum}

Of course, we'll say it is the asphericity structure on $M$
\emph{generated} by the full subcategory $N$ of $M$, and the latter
will be called a \emph{generating subcategory} for the given
asphericity structure.

The proposition is a tautology, in view of the definitions and of
prop.\ \ref{prop:75.1} (p.\ \ref{p:232}) and its corollary. The form
\ref{it:79.1.ii} or \ref{it:79.1.iii} of the condition shows that it
depends only upon the category $N$, not upon the way in which $N$ is
embedded in a larger category $M$. We may call a category $N$
satisfying condition \ref{it:79.1.iii} above an \emph{aspherator}, by
which we would like to express that this category represents a
standard way of generating asphericity structures, through any full
embedding of $N$ into a category $M$. This condition is automatically
satisfied if $N$ is small, more generally it holds if $N$ is
equivalent to a small category. It should be kept in mind that the
condition depends both on the choice of the basic universe \scrU{} and
on the choice of the basic localizer \scrW{} in the corresponding
large category \Cat. It looks pretty sure that the condition is not
always satisfied (I doubt it is for $N=\Sets$ say), but I confess I
didn't sit down to make an explicit example.

Let\pspage{248} now $(M,M\subc)$ be a contractibility structure, such
that there exists a \emph{small} full subcategory $C$ of $M$ which
generates the contractibility structure, i.e., such that (a)\enspace
the objects of $C$ are ``contractible'', i.e., are in $M\subc$ and
(b)\enspace any object of $M\subc$ is $C$-contractible (i.e.,
contractible for the homotopy interval structure admitting the
intervals made up with objects of $C$ as a generating
family). Independently of any smallness assumption upon $C$, we gave
in section \ref{sec:51} \ref{subsec:51.D} (p.\ \ref{p:118}), under the
name of ``basic assumption'' \ref{cond:51.Bas.4}, the n.s.\ condition
on a full subcategory $C$ of a given category $M$, in order that $C$
can be viewed as a generating set of contractible objects, for a
suitable contractibility structure
\[ M\subc\subset \Ob M\]
in $M$ (which is uniquely defined by $C$). It turns out that in case
$C$ contains a final object of $M$, and is stable under binary
products in $M$, the condition \ref{cond:51.Bas.4} depends only upon
the category structure of $C$, and not upon the particular way this
category is embedded in another one $M$.

In loc.\ sit.\ we did not impose, when defining a contractibility
structure, a condition that there should exist a small set of
generators for the structure. From now on, we'll assume that the
(possibly large) categories $M$ we are working with are
\scrU-categories, and that ``contractibility structure'' means
``\scrU-contractibility structure'', namely one such that $M$ admit a
small full subcategory $C$, generating the structure.

We recall too that in the definition of a contractibility structure
$(M,M\subc)$, it has always been understood that $M$ is stable under
finite products.

\begin{propositionnum}\label{prop:79.2}
  With the conventions above, let $(M,M\subc)$ be any contractibility
  structure. Then:
  \begin{enumerate}[label=\alph*),font=\normalfont]
  \item\label{it:79.2.a}
    The full subcategory $M\subc$ of $M$ generates an asphericity
    structure $M\subas$ in $M$ \textup(cf.\ prop.\
    \ref{prop:79.1}\textup).
  \item\label{it:79.2.b}
    Any small full subcategory $C$ of $M$ which generates the
    contractibility structure and such that $\Chat$ be totally
    aspheric, \emph{generates} the asphericity structure $M\subas$,
    i.e., the inclusion functor $i:C\to M$ is $M\subas$-aspheric.
  \item\label{it:79.2.c}
    The asphericity structure $M\subas$ is totally aspheric \textup(cf.\
    prop.\ \ref{prop:76.5}, p.\ \ref{p:239}\textup).
  \end{enumerate}
\end{propositionnum}

The\pspage{249} first statement \ref{it:79.2.a} can be rephrased, by
saying that the category $M\subc$ of contractible objects of $M$ is an
\emph{aspherator}. To prove this, we use the fact that $M\subc$ is
stable under finite products, which implies that we can find a small
full subcategory $C$ of $M\subc$, stable under such products, and
which generates the contractibility structure. The stability condition
upon $C$ implies that \Chat{} is totally aspheric. Therefore,
\ref{it:79.2.a} and \ref{it:79.2.b} will be proved, if we prove that
any small full subcategory $C$ of $M\subc$, such that \Chat{} is
totally aspheric, satisfies the conditions of prof.\ \ref{prop:79.1}
(with $N=M\subc$), namely that for any $x$ in $M\subc$, $i^*(x)$ is an
aspheric object of \Chat, where $i:C\to M$ (or $C\to M\subc$,
equivalently) is the inclusion functor. But from the fact that $i^*$
commutes with finite products, it follows that $i^*(x)$ is
contractible in \Chat, for the homotopy interval structure admitting
as a generating family of homotopy intervals the intervals in \Chat{}
made up with objects of $C$. The assumption of total asphericity upon
\Chat{} implies that these homotopy intervals are aspheric over
$e_\Chat$, and from this follows (as already used a number of times
earlier) that $i^*(x)$ too is aspheric over $e_\Chat$, and hence
aspheric as $e_\Chat$ is aspheric (because of the assumption of total
asphericity). This proves \ref{it:79.2.a} and \ref{it:79.2.b}, and
\ref{it:79.2.c} follows via prop.\ \ref{prop:76.5} (p.\ \ref{p:239}).

We'll call of course the asphericity structure described in prop.\
\ref{prop:79.2} the \emph{asphericity structure generated by the given
  contractibility structure}.

\begin{remarknum}
  Clearly, not every asphericity structure can be generated by a
  contractibility structure, as a necessary condition (presumably not
  a sufficient one) is total asphericity. We don't expect either, in
  case it can, that the generating contractibility structure is
  uniquely defined; however, we do expect in this case that there
  should exist a canonical (largest) choice -- we'll have to come back
  upon this in due course. For the time being, let's only remark that
  all modelizing asphericity structures met with so far, it seems, do
  come from contractibility structures.
\end{remarknum}

It occurs to me that the last statement was a little hasty -- after
all we have met with test categories which are not strict ones, hence
$(\Ahat,\Ahatas)$ is a modelizing asphericity structure (it would be
enough even that $A$ be a pseudo-test category), which isn't totally
aspheric, and a fortiori does not come from a contractibility
structure. Thus,\pspage{250} I better correct the statement, to the
effect that, it seems, all modelizing \emph{totally aspheric}
asphericity structures met with so far are generated by suitable
contractibility structures.

Let $M$ be a category endowed with a contractibility structure
$M\subc$, hence also with an asphericity structure $M\subas$, and let
\[i: A\to M\]
be a functor from a small category $A$ to $M$. We want to give n.s.\
conditions for $i$ to be a \emph{test functor}, in terms of homotopy
notions in $M$ and in \Ahat. To this end, it seems necessary to
refresh memory somewhat and recall some relevant notions which were
developed in part \ref{ch:III} of our notes (sections \ref{sec:54} and
\ref{sec:55}).

It will be convenient to call an object $F$ of \Ahat{} \emph{locally
  aspheric} (resp.\ \emph{totally aspheric}), if its product in
\Ahat{} by any object of $A$, and hence also its product by any
aspheric object of \Ahat, is aspheric (resp.\ and if moreover $F$
itself is aspheric). With this terminology, \Ahat{} is totally
aspheric if{f} every aspheric object of \Ahat{} is totally aspheric,
and if moreover $A$ is aspheric, i.e., $e_\Ahat$ is aspheric. Note
that $F$ is locally aspheric if{f} the map $F\to e_\Ahat$ is aspheric,
or what amounts to the same, if this map is universally in \scrWA, or
equivalently, if the corresponding functor $A_{/F}\to A$ is
aspheric. If $A$ itself is aspheric, and in this case only, this
condition implies already that $F$ is aspheric, i.e., that $F$ is
totally aspheric. We'll denote by
\[\Ahattotas\quad(\text{resp.}\quad\Ahatlocas)\]
the full subcategory of \Ahat{} made up with the totally aspheric
(resp.\ locally aspheric) objects of \Ahat. Thus we get
\[\begin{tikzcd}[baseline=(O.base),column sep=tiny]
  & \Ahattotas = \Ahatlocas \sand \Ahatas
  \ar[dl,hook]\ar[dr,hook] & \\
  \Ahatlocas & & |[alias=O]| \Ahatas
\end{tikzcd}.\]

Now recall that (section \ref{sec:54}) in terms of the set \scrWA{} of
weak equivalences in \Ahat, we constructed a homotopy structure on
\Ahat, more specifically a homotopy interval structure, admitting as a
generating family of homotopy intervals
\[\bI = (I, \delta_0,\delta_1)\]
the set of intervals such that $I$ be a locally aspheric object of
\Ahat, i.e.,
\[ I\in\Ob \Ahatlocas .\]
Let\pspage{251}
\[ h = h_\scrWA\]
be this homotopy structure, hence a corresponding notion of
\emph{$h$-equivalence} or \emph{$h$-homotopy} $\hsim$ for arrows in
\Ahat, a corresponding notion of $h$-homotopisms, i.e., a set of
arrows
\[ W^h\subset\Fl\Ahat, \quad\text{such that}\quad W^h\subset\scrWA,\]
a notion of $h$-homotopy interval (namely an interval such that
$\delta_0$ and $\delta_1$ be $h$-homotopic, for which it is
sufficient, but not necessary, that $I$ be locally aspheric\ldots),
and last not least, a notion of \emph{contractible objects}, making up
a full subcategory
\[ \Ahatc \subset \Ahat, \quad\text{such that}\quad
\Ahatc \subset \Ahatlocas.\]
The latter inclusion, in case $A$ is aspheric, can be equally written
\[\Ahatc \subset \Ahattotas \quad(\text{if $A$ aspheric}).\]

Coming back now to the contractibility structure $(M,M\subc)$, and a
functor $i:A\to M$, we are interested in the corresponding functor
\[u=i^*: M \to M'=\Ahat,\]
where both members will be viewed as being endowed with their
respective homotopy structures -- the one of $M$ being of the most
restrictive type envisioned in section \ref{sec:51}, namely it is
defined in terms of a contractibility structure, whereas the one of
$M'$ is a priori defined in terms of a homotopy interval structure,
but not necessarily in terms of a contractibility structure. Now this
situation has been described in section \ref{sec:53}, as far as
compatibility conditions with homotopy structures are concerned,
independently by the way of any special assumption on $M'$ (such as
being a category of presheaves on some small category $A$), or on the
functor $u$, except for commuting to finite products. Compatibility of
$u$ with the homotopy structures on $M,M'$ can be expressed by either
one of the following six conditions, which are all equivalent:
\begin{description}
\item[\namedlabel{cond:79.H1}{H~1)}]
  $u$ transforms homotopic arrows of $M$ into homotopic arrows in $M'$.
\item[\namedlabel{cond:79.H2}{H~2)}]
  $u$ transforms homotopisms in $M$ into homotopisms in $M'$.
\item[\namedlabel{cond:79.H3}{H~3)}]
  $u$ transforms any homotopy interval $\bI=(I,\delta_0,\delta_1)$ in
  $M$ into a homotopy interval in $M'$ (i.e., if two sections of an
  object $I$ over $e_M$ are homotopic, so are $u(\delta_0)$ and $u(\delta_1)$).
\item[\namedlabel{cond:79.H3prime}{H~3')}]
  Same as \ref{cond:79.H3}, but \bI{} being restricted to be in a
  given family of homotopy intervals, generating for the homotopy
  structure in $M$.
\item[\namedlabel{cond:79.H4}{H~4)}]
  $u$\pspage{252} transforms any contractible object $x$ of $M$ into a
  contractible object of $M'$.
\item[\namedlabel{cond:79.H4prime}{H~4')}]
  Same as \ref{cond:79.H4}, with $x$ being restricted to be in a given
  subcategory $C$ of $M$, generating for the contractibility structure
  of $M$.
\end{description}
\begin{remarknum}\label{rem:79.2}
  It should be noted that the condition upon $C$ stated in
  \ref{cond:79.H4prime}, namely that the (given) $C$ should be
  generating for the contractibility structure $M\subc$ of $M$, means
  exactly two things: (a)\enspace $C\subset M\subc$, and (b)\enspace
  the family of all intervals $\bI=(I,\delta_0,\delta_1)$ made up with
  objects of $C$ (these intervals are necessarily homotopy intervals
  for the homotopy structure of $M$) \emph{generates} the homotopy
  structure of $M$, i.e., two arrows in $M$ are homotopic if{f} they
  can be joined by a chain of arrows, two consecutive among which
  being related by an elementary homotopy involving an interval of
  that family. This reminder being made, it follows that $C$ is
  equally eligible for applying criterion \ref{cond:79.H3prime}, which
  means that we get still another equivalent formulation of
  \ref{cond:79.H4prime}, by demanding merely that for any two sections
  $\delta_0,\delta_1$ of $x$ over $e_M$, the corresponding sections of
  $u(x)$ should be homotopic.
\end{remarknum}

After these preliminaries, we can state at last the following
generalization of the main result of section \ref{sec:65} (th,\
\ref{thm:65.1}, p.\ \ref{p:176}), concerning test functors with values
in \Cat:
\begin{theoremnum}\label{thm:79.1}
  Let $(M,M\subc)$ be a contractibility structure \textup(cf.\ p.\
  \ref{p:248}\textup), $C$ a small full subcategory of $M$ generating
  the contractibility structure \textup(cf.\ remark \ref{rem:79.2}
  above\textup), $A$ a small category, and $i:A\to M$ a functor,
  factoring through $M\subc$. We consider $M$ as endowed equally with
  the asphericity structure $M\subas$ generated by $M\subc$
  \textup(cf.\ prop.\ \ref{prop:79.2}\textup). Then the following
  conditions on $i$ are equivalent:
  \begin{enumerate}[label=(\roman*),font=\normalfont]
  \item\label{it:79.thm1.i}
    The functor $i^*:M\to\Ahat$ deduced from $i$ is compatible with
    the homotopy structures on $M$ and on \Ahat{} \textup(cf.\ pages
    \ref{p:250}--\ref{p:251} for the latter\textup), i.e., $i^*$
    satisfies either one of the six equivalent conditions
    \textup{\ref{cond:79.H1}} to \textup{\ref{cond:79.H4prime}} above.
  \item\label{it:79.thm1.ii}
    The functor $i^*$ is locally $M\subas$-aspheric \textup(def.\
    \ref{def:78.2}, p.\ \ref{p:245}\textup), i.e., for any $x$ in
    $M\subas$, $i^*(x)$ is a locally aspheric object of \Ahat{}
    \textup(cf.\ p.\ \ref{p:250}\textup), i.e., $i^*(x)$ is aspheric
    over $e_\Ahat$.
  \item\label{it:79.thm1.iii}
    Same as \textup{\ref{it:79.thm1.ii}}, but with $x$ restricted to
    be in $C$.
  \end{enumerate}
\end{theoremnum}
\begin{corollarynum}\label{cor:79.1}
  In order for $i$ to be totally $M\subas$-aspheric \textup(def.\
  \ref{def:78.2}, p.\ \ref{p:245}\textup)\pspage{253} it is n.s.\ that
  $A$ be aspheric, and that the equivalent conditions of th.\
  \ref{thm:79.1} be satisfied.
\end{corollarynum}

Indeed, it follows at once from the definitions and from the fact that
the asphericity structure $(M,M\subas)$ is aspheric, i.e., that $e_M$
is aspheric, that $i$ is totally $M\subas$-aspheric if{f} it is
locally $M\subas$-aspheric (i.e., condition \ref{it:79.thm1.ii}), and
if moreover $A$ is aspheric, hence the corollary.

\begin{corollarynum}\label{cor:79.2}
  In order for $i$ to be a test functor, it is n.s.\ that $A$ be a
  test category, that $(M,M\subas)$ be modelizing, and that the
  equivalent conditions of theorem \ref{thm:79.1} be satisfied.
\end{corollarynum}

This follows from cor.\ \ref{cor:79.1} and the reformulation of the
notion of a test-functor, given p.\ \ref{p:245}.

This corollary contains the main result of section \ref{sec:65} as a
particular case, when taking $M=\Cat$ with the usual contractibility
structure, and $C=\{\Simplex_1\}$, except that in loc.\ sit.\ we did
not have to assume beforehand that $A$ be a test category, but only
that $A$ is aspheric: this condition, plus condition
\ref{it:79.thm1.iii} above (namely, $i^*(\Simplex_1)$ locally
aspheric) implies already that $A$ is a test category. In order to get
also this extra result, we state still another corollary:
\begin{corollarynum}\label{cor:79.3}
  Assume
  \[\bI=(I,\delta_0,\delta_1,\mu), \quad\text{with}\quad
  I\in\Ob C,\]
  is a \emph{multiplicative interval} in $M$, i.e., an interval
  endowed with a multiplication $\mu$, admitting $\delta_0$ as a left
  unit and $\delta_1$ as a left zero element \textup(cf.\ section
  \ref{sec:49}, p.\ \ref{p:108} and section \ref{sec:51}, p.\
  \ref{p:120} -- where such an interval was provisionally called a
  ``contractor''\textup). Assume moreover that for any $x$ in
  $M\subc$, the two compositions $x \to $
  \begin{tikzcd}[cramped]
    e_M\ar[r,shift left=1pt,"\ensuremath{\delta_0,\delta_1}"]\ar[r,shift right=2pt] &
    I
  \end{tikzcd} are distinct \textup(which is the case for instance
  if $\Ker(\delta_0,\delta_1)$ exists in $M$ and is a strict initial
  object $\varnothing_M$ of $M$ \textup(i.e., an initial object such
  that any map $x\to\varnothing_M$ in $M$ is an isomorphism\textup), and
  moreover $\varnothing_M\notin M\subc$\textup). Then the conditions of
  th.\ \ref{thm:79.1} imply that $A$ is a local test category, and
  hence a test category provided $A$ is aspheric.
\end{corollarynum}

Indeed, $i^*(\bI)$ is a multiplicative interval in \Ahat{} which is
locally aspheric, and (as follows immediately from the assumptions on
\bI) separating -- hence \Ahat{} is a local test
category.

This\pspage{254} array of immediate corollaries of th.\ \ref{thm:79.1}
do convince me that this statement is indeed ``the'' natural
generalization of the ``awkward'' main result of section
\ref{sec:51}. All we have to do is to prove theorem \ref{thm:79.1}
then.

The three conditions of theorem \ref{thm:79.1} can be rewritten simply
as
\begin{description}
\item[\ref{it:79.thm1.i}]
  $i^*(C) \subset \Ahatc$ (using \hyperref[cond:79.H4prime]{H~4'}),
\item[\ref{it:79.thm1.ii}]
  $i^*(M\subas) \subset \Ahatlocas$,
\item[\ref{it:79.thm1.iii}]
  $i^*(C) \subset \Ahatlocas$,
\end{description}
and because of
\[ C \subset M\subas, \quad \Ahatc\subset\Ahatlocas,\]
it follows tautologically that \ref{it:79.thm1.i} and
\ref{it:79.thm1.ii} both imply \ref{it:79.thm1.iii}. On the other
hand, \ref{it:79.thm1.iii} $\Rightarrow$ \ref{it:79.thm1.i} by
criterion \ref{cond:79.H3prime}, and the definition of the homotopy
interval structure of \Ahat{} in terms of \Ahatlocas. Thus, we are
left with proving \ref{it:79.thm1.i} $\Rightarrow$
\ref{it:79.thm1.ii}. But \ref{it:79.thm1.i} can be rewritten as
\[i^*(M\subc) \subset \Ahatc,\]
which implies
\begin{equation}
  \label{eq:79.star}
  i^*(M\subc) \subset \Ahatlocas.\tag{*}
\end{equation}
That the latter condition implies \ref{it:79.thm1.ii} now follows from
the corollary to the following tautology (with $N=M\subc$), which
should have been stated as a corollary to prop.\ \ref{prop:79.1} above
(p.\ \ref{p:247}):
\begin{lemma}
  Let $(M,M\subas)$ be an asphericity structure, generated by the full
  subcategory $N$ of $M$, let $A$ be a small category, and $i:A\to M$
  a functor \emph{factoring through $N$}. Then $i$ is
  $M\subas$-aspheric if{f} for any $x$ in $N$, $i^*(x)$ is an aspheric
  object of $\Ahat$.
\end{lemma}
\begin{corollary}
  The functor $i$ is locally $M\subas$-aspheric \textup(resp.\ totally
  $M\subas$-aspheric\textup) if{f} $i^*(N)\subset\Ahatlocas$
  \textup(resp.\ $i^*(N)\subset\Ahattotas$\textup).
\end{corollary}
\begin{remarknum}
  Assume\pspage{255} in theorem \ref{thm:79.1} that $A$ is totally aspheric. Then
  condition \ref{it:79.thm1.ii} just means that $i$ is
  $M\subas$-aspheric, and condition \ref{it:79.thm1.iii} that $i^*(x)$
  is aspheric for any $x$ in $C$ (as we got
  $\Ahatlocas=\Ahatas=\Ahattotas$). If moreover $C$ satisfies the
  condition of cor.\ \ref{cor:79.2} above, then these conditions imply
  that $A$ is a strict test category, and if $(M,M\subas)$ is
  modelizing, that $i$ is a test functor as stated in corollary
  \ref{cor:79.2}. These observations sum up the substance of the
  restatement of the main result of section \ref{sec:51}, given in
  theorem \ref{thm:65.2} of p.\ \ref{p:178}, in the present general
  context.
\end{remarknum}

% 80
\hangsection{Reminders and questions around canonical
  modelizers.}\label{sec:80}%
In the preceding section, we associated to any contractibility
structure $M\subc$ on a category $M$, an asphericity structure
$M\subas$ ``generated'' by the former in a natural sense. It is this
possibility of associating (in a topologically meaningful way) an
asphericity structure to a given contractibility structure, which
singles out the latter structure type, among the three essential
distinct ``homotopy flavored'' kind of structures developed at length
in sections \ref{sec:51} and \ref{sec:52}, in preference to the two
weaker notions of a homotopy interval structure, and of a homotopism
structure (or, equivalently, or a ``homotopy relation''). The
association
\begin{equation}
  \label{eq:80.star}
  M\subc\mapsto M\subas \quad(\leftrightarrow \text{corresponding
    notion of weak equivalence $W\suba$ in $M$})
  \tag{*}
\end{equation}
finally carried through in the last section, had been foreshadowed
earlier (cf.\ p.\ \ref{p:110} and p.\ \ref{p:142}), but was pushed off
for quite a while, in order to ``give precedence'' to the other
approach in view by then towards more general test functors than
before, leading up finally to the ``awkward main result'' of section
\ref{sec:65}. In \eqref{eq:80.star}, as the asphericity structure
$M\subas$ is totally aspheric and a fortiori aspheric, $M\subas$ and
the corresponding notion of weak equivalence $W\suba=W_{M\subc}$ (which
of course should not be confused with the notion of homotopism
associated to $M\subc$, giving a considerably smaller set of arrows
$W\subc$) determine each other mutually. Till the writing up of
section \ref{sec:51}, it was the aspect ``weak equivalence'' $W\suba$
which was in the fore, whereas the conceptually more relevant aspect
of ``aspheric objects'' did not appear in full light before I was
through with grinding out the ``awkward approach'' (cf.\ p.\
\ref{p:188}).

It is time now to remember the opposite association
\begin{equation}
  \label{eq:80.starstar}
  W\suba\mapsto\text{homotopy structure $h_{W\suba}$,}\tag{**}
\end{equation}
associating\pspage{256} to a notion of weak equivalence in $M$, i.e.,
to any saturated subset
\[W\suba \subset \Fl(M),\]
a corresponding homotopy structure $h_{W\suba}$, a homotopy interval
structure as a matter of fact (section \ref{sec:54}, p.\
\ref{p:131}). Here we are primarily interested of course in the case
when $W\suba$ is associated to a given asphericity structure $M\subas$
in $M$, which we may as well assume to be aspheric, so to be sure that
$W\suba$ and $M\subas$ determine each other. I recall that the weak
interval structure $h_{W\suba}$ can be described by the generating
family of homotopy intervals, consisting of all intervals
\[ \bI=(I,\delta_0,\delta_1) \]
such that $I\to e_M$ be universally in $W\suba$, i.e., such that for
any object $x$ in $M$, the projection
\[ x\times I\to x\]
be in $W\suba$. For pinning down further the exact relationship
between contractibility structures (which may be viewed as just
special types of homotopy interval structures) and asphericity
structures, we are thinking of course more specifically of
\emph{totally aspheric} asphericity structures, in view of prop.\
\ref{prop:79.2} \ref{it:79.2.c} of the preceding section (p.\
\ref{p:248}). It is immediate in this case that for an object $I$ of
$M$, $I\to e_M$ is in $\mathrm UW\suba$ (i.e., is universally in
$W\suba$) if (and only if, of course) $I$ is aspheric, i.e., $I\to
e_M$ is in $W\suba$. The most relevant questions which come up here,
now seem to me the following:

\namedlabel{q:80.1}{1)}\enspace If $M\subas$ is generated by a
contractibility structure $M\subc$, is the homotopy structure
$h_{W\suba}$ associated to $M\subas$ (indeed, a homotopy interval
structure as recalled above) also the one defined by $M\subc$, using
intervals in $M\subc$ as a generating family of homotopy intervals?

\namedlabel{q:80.2}{2)}\enspace Conversely, what extra conditions on a
given asphericity structure $M\subas$ on $M$ are needed (besides total
asphericity) to ensure that the corresponding homotopy structure
$h_{W\suba}$ on $M$ comes from a contractibility structure $M\subc$
(i.e., admits a generating family of homotopy intervals which are
\emph{contractible}),\pspage{257} and that moreover $M\subc$
\emph{generates} $M\subas$?

Before looking up a little these questions, I would like however to
carry through at once the ``idyllic picture'' of canonical modelizers,
foreshadowed in section \ref{sec:50} (p.\ \ref{p:110}), as I feel that
this should be possible at present at no costs. Taking into account
the reflections of the later sections \ref{sec:57} and \ref{sec:59},
we get the following set-up.

Let $M$ be a \scrU-category, stable under finite products, endowed
with a functor
\begin{equation}
  \label{eq:80.1}
  \pi_0^M\text{ or }\piz: M\to\Sets,\tag{1}
\end{equation}
on which we make no assumptions for the time being. We are thinking of
the example when $M$ is a totally $0$-connected category (cf.\ prop.\
on page \ref{p:142} for this notion) and \piz{} is the ``connected
components'' functor, or when $M=\Spaces$ and \piz{} corresponds to
taking sets of arc-wise connected components. According to section
\ref{sec:54} (p.\ \ref{p:131}), we introduce a corresponding homotopy
interval structure $h$ on $M$, admitting the generating family of
homotopy intervals made up with those intervals
$\bI=(I,\delta_0,\delta_1)$ for which $I\to e_M$ is ``universally in
$W_\piz$'', namly
\[\piz(x\times I)\to\piz(x) \quad\text{bijective for any $x$ in
  $M$.}\]
(Under suitable conditions on \piz, this homotopy interval structure
$h$ on $M$ is the widest one ``compatible with \piz'' in the sense of
page \ref{p:130}, namely such that \piz{} transforms homotopisms into
isomorphisms -- cf.\ proposition p.\ \ref{p:133}.) We are interested
in the case when this homotopy structure on $M$ can be described by a
contractibility structure $M\subc$ on $M$, which is then unique of
course, hence well-defined in terms of \piz. Therefore, the
asphericity structure generated by $M\subc$ is equally well defined in
terms of \piz, and likewise the corresponding notion $W\suba$ of
``weak equivalence''. We then get a canonical functor
\begin{equation}
  \label{eq:80.2}
  \HotOf_M = W\suba^{-1}M \to \Hot\tag{2}
\end{equation}
(section \ref{sec:77}, p.\ \ref{p:239}). We'll have to find still a
suitable extra condition on the functor \piz, implying that this
functor is canonically isomorphic deduced from \eqref{eq:80.2} by
composing with the canonical functor\pspage{258}
\begin{equation}
  \label{eq:80.3}
  \Hot \xrightarrow{\piz} \Sets,
  \tag{3}
\end{equation}
which can be defined using a very mild extra condition on the basic
localizer \scrW{} (namely, $f\in\scrW$ implies $\piz(f)$ bijective,
cf.\ condition \ref{it:64.La} on page \ref{p:165}). Thus, there seems
to be a little work ahead after all -- in order to deduce something
like a one to one correspondence, say, between pairs $(M,\piz)$
satisfying suitable conditions, and certain types of asphericity
structures $(M,M\subas)$ (which will have to be assumed totally
aspheric, and presumably a little more still).

The case of special interest to us is the one when the asphericity
structure we get on $M$ in terms of the functor $\pi_0^M$ is
\emph{modelizing}, hence even strictly modelizing (i.e., $(M,W\suba)$
is a strict modelizer), as $(M,M\subas)$ is totally aspheric. If we
assume moreover that the category $M$ is totally $0$-connected and
that \piz{} is just the ``connected components'' functor, then the
modelizing asphericity structure we got on $M$ is canonically
determined by the mere category structure of $M$, and deserves
therefore to be called \emph{the canonical \textup(modelizing\textup)
  asphericity structure on $M$}. A \emph{canonical modelizer} $(M,W)$
is a modelizer which can be obtained from a canonical asphericity
structure $(M,M\subas)$ by $W=W\suba=$ corresponding set of weak
equivalences (for $M\subas$).

\begin{remark}
  The slight sketchy definition we just gave for the canonical
  modelizing asphericity structures, and accordingly for the canonical
  modelizers, is essentially complete. The point however which
  requires clarification is the relationship between the ``connected
  components'' functor on the corresponding (totally $0$-connected)
  category $M$, and the composition of the canonical functors
  \eqref{eq:80.2} and \eqref{eq:80.3}.
\end{remark}

\bigbreak

\presectionfill\ondate{25.6.}\pspage{259}\par

% 81
\hangsection[Contractibility as the common expression of homotopy,
\dots]{Contractibility as the common expression of homotopy,
  asphericity and \texorpdfstring{$0$}{0}-connectedness
  notions. \texorpdfstring{\textup(}{(}An overall review of the
  notions met with so far.\texorpdfstring{\textup)}{)}}\label{sec:81}%
I didn't find much time since Monday for mathematical pondering -- the
little I got nonetheless has been enough for convincing myself that
things came out more nicely still than I expected by then. One main
point being that, provided the basic localizer satisfies the mild
extra assumption \ref{loc:4} below, any contractibility structure
$M\subc$ on a category $M$ with finite products can be recovered, in
the simplest imaginable way, in terms of the associated asphericity
structure $M\suba$, or equivalently, in terms of the corresponding set
$W\suba$ of ``weak equivalences''; namely, $M\subc$ is the set of
contractible objects in $M$, for the homotopy interval structure
defined in terms of all intervals made up with objects of
$M\suba$. This implies that the canonical map
\begin{equation}
  \label{eq:81.1}
  \Homtp_4(M) = \Cont(M) \hookrightarrow \WAsph(M),\tag{1}
\end{equation}
from the set of contractibility structures on $M$ to the set of
asphericity structures on $M$ relative to the basic localizer \scrW{}
(or ``\emph{\scrW-asphericity structures}''), is \emph{injective}. In
other words, we may view a contractibility structure (on a category
$M$ stable under finite products), which is an \emph{absolute} notion
(namely independent of the choice of a basic localizer \scrW), as a
``\emph{particular case}'' of a \scrW-asphericity structure (depending
on the choice of \scrW), namely as ``equivalent'' to a
\scrW-asphericity structure, satisfying some extra conditions which
we'll have to write down below.

This pleasant fact associates immediately with two related ones. The
first is just a reminder of our reflections of sections \ref{sec:51}
and \ref{sec:52}, namely that the set of contractibility structures on
$M$ can be viewed as one among four similar sets of ``homotopy
structures'' on $M$
\begin{equation}
  \label{eq:81.2}
  \Homtp_4(M) \hookrightarrow \Homtp_3(M) \hookrightarrow \Homtp_2(M)
  \tosim \Homtp_1(M),\tag{2}
\end{equation}
corresponding to the four basic ``homotopy notions'' met with so far,
namely (besides contractibility structure $\Homtp_4$) the
\emph{homotopy interval structures} ($\Homtp_3$), the \emph{homotopism
  structures} ($\Homtp_2$), and the \emph{homotopy relations} between
maps ($\Homtp_1$). In the sequel, if
\begin{equation}
  \label{eq:81.3}
  M\subc \subset \Ob M\tag{3}
\end{equation}
is a given contractibility structure on $M$, we'll denote by
\begin{equation}
  \label{eq:81.4}
  J\subc \subset \Int(M), \quad
  W\subc \subset \Fl(M), \quad
  R\subc \subset \Fl(M)\times\Fl(M)
  \tag{4}
\end{equation}
the corresponding other three homotopy structures on $M$, where
$\Int(M)$ denotes the set of all ``intervals''
$\bI=(I,\delta_0,\delta_1)$ in $M$, i.e., objects of $M$ endowed with
two sections $\delta_0,\delta_1$ over the final object $e_M$ of $M$.

The\pspage{260} second fact alluded to above is concerned with
behavior of the \scrW-asphericity notions, for varying \scrW, more
specifically for a pair
\begin{equation}
  \label{eq:81.5}
  \scrW \subset\scrW'\subset\Fl(\Cat)\tag{5}
\end{equation}
of two basic localizers, \scrW{} and \scrW', such that \scrW{}
``refines'' \scrW'. It then follows that for any small category $A$,
we have
\begin{equation}
  \label{eq:81.6}
  \scrWA \subset \scrW'_A,\tag{6}
\end{equation}
and accordingly, that \emph{for any \scrW-asphericity structure}
\begin{equation}
  \label{eq:81.7}
  M_\scrW \subset \Ob M\tag{7}
\end{equation}
\emph{on $M$, there exists a unique $\scrW'$-asphericity structure
  $M_{\scrW'}$ on $M$,}
\begin{equation}
  \label{eq:81.8}
  M_\scrW \subset M_{\scrW'}, \tag{8}
\end{equation}
\emph{such that for any small category $A$\kern1pt, a functor $A\to M$ which
  is $M_\scrW$-aspheric \textup(with respect to \scrW\textup) is also
  $M_{\scrW'}$-aspheric \textup(with respect to
  $\scrW'$\textup)}. This is merely a tautology, which we didn't state
earlier, because there was no compelling reason before to look at what
happens when \scrW{} is allowed to vary. Thus, we get a canonical map
\begin{equation}
  \label{eq:81.9}
  \WAsph(M) \to \WprimeAsph(M),\tag{9}
\end{equation}
with the evident transitivity property for a triple
\[\scrW \subset \scrW' \subset \scrW'',\]
in other words we get a functor
\[ \scrW \mapsto \WAsph(M)\]
from the category of all basic localizers (the arrows between
localizers being inclusions \eqref{eq:81.5}) to the category of
sets. The relation of the canonical inclusion \eqref{eq:81.1} with
this functorial dependence of $\WAsph(M)$ on \scrW{} is expressed in
the commutativity of
\[\begin{tikzcd}[baseline=(O.base),column sep=-6pt]
  & \Homtp_4(M) = \Cont(M)
  \ar[dl,hook]\ar[dr,hook] & \\
  \WAsph(M)\ar[rr] & & |[alias=O]| \WprimeAsph(M)
\end{tikzcd}.\]

It is time to write down the ``mild extra assumption'' on \scrW{}
needed to ensure injectivity of \eqref{eq:81.1}, namely the familiar
enough condition:
\begin{description}
\item[\namedlabel{loc:4}{Loc~4)}]
  The set $\scrW\subset\Fl(\Cat)$ of weak equivalences in \Cat{} is
  ``compatible'' with the functor $\piz:\Cat\to\Sets$, i.e.,
  \begin{equation}
    \label{eq:81.10}
    f\in\scrW \Rightarrow \text{$\piz(f)$ bijective.}\tag{10}
  \end{equation}
\end{description}
(Cf.\ pages\pspage{261} \ref{p:213}--\ref{p:214} for the conditions
\ref{loc:1} to \ref{loc:3}.)

Among all basic localizers satisfying this extra condition, there is
one coarsest of all, which we'll call \scrWz, defined by the
condition
\begin{equation}
  \label{eq:81.11}
  f\in\scrWz \Leftrightarrow \text{$\piz(f)$ bijective.}\tag{11}
\end{equation}
for any map $f$ in \Cat. It is clear too that among all basic
localizers there is a finest, which we'll call $\scrWoo$, and which
can be described as
\begin{equation}
  \label{eq:81.12}
  \scrWoo=\bigcap\scrW, \quad\parbox[t]{0.5\textwidth}{intersection
    of the set of all basic localizers in \Cat.}\tag{12}
\end{equation}
We'll see in\scrcomment{see also \cite{Cisinski2004}} part \ref{ch:V}
of the notes that \scrWoo{} is none else than just the usual notion of
weak equivalence we started with, at the very beginning of our
reflections (cf.\ section \ref{sec:17}). Thus, functoriality of
\eqref{eq:81.1} with respect to \scrW{} implies that \eqref{eq:81.1}
for arbitrary \scrW{} can be described in terms of the particular case
\scrWoo, as the composition
\begin{equation}
  \label{eq:81.13}
  \Cont(M) \hookrightarrow \WooAsph(M) \to \WAsph(M).\tag{13}
\end{equation}
On the other hand, the strongest version of injectivity of
\eqref{eq:81.1}, for different \scrW's, is obtained for \scrWz,
i.e., taking the map
\begin{equation}
  \label{eq:81.14}
  \Cont(M) \hookrightarrow \WzAsph(M).\tag{14}
\end{equation}

This last map seems to me of special significance, because the two
sets it relates correspond to ``\emph{absolute}'' notions (not
depending on the choice of some \scrW), and which moreover are both
``\emph{elementary}'', in the sense that they do not depend on
anything like consideration of non-trivial homotopy or (co)homology
invariants of objects of \Cat. As a matter of fact, the notion of a
contractibility structure corresponds to the algebraic translation of
one of the most elementary and intuitive topological notions, namely
contractibility; whereas the notion of a \scrWz-asphericity structure
can be expressed just as ``elementarily'' in terms of the functor
\begin{equation}
  \label{eq:81.15}
  \piz:\Cat\to\Sets,\tag{15}
\end{equation}
which we may call the ``basic'' connected components-functor, which is
nothing but the algebraic counterpart of the basic intuitive notion of
connected components of a space. We'll denote by
\begin{equation}
  \label{eq:81.16}
  M_0=M_\scrWz\subset\Ob M,\quad W_0\subset\Fl(M)\tag{16}
\end{equation}
the set of \scrWz-aspheric objects and the set of \scrWz-weak
equivalences, associated to a given contractibility structure $M\subc$
on $M$. The objects of $M_0$\pspage{262} merit the name of
\emph{$0$-connected objects} of $M$ (with respect to $M\subc$), and
the arrows of $M$ in $W_0$ merit the name of \emph{$0$-connected
  maps}\footnote{\alsoondate{26.6.} This name for \emph{maps} is
  improper though, as it rather suggests the property of being
  ``universally in $W_0$''.} (with respect to $M\subc$). These two
notions of $0$-connectedness determine each other in an evident way
(valid for any \scrW-aspheric \scrW-asphericity structure, for any
basic localizer \scrW\ldots). Explicitly, this can be expressed by
\begin{equation}
  \label{eq:81.17}
  \begin{cases}
    x\in M_0 \Leftrightarrow \bigl( (x\to e_M)\in W_0 \bigr) \\
    (f:x\to y)\in W_0 \Leftrightarrow
    \bigl(\text{$\piz({M_0}_{/x})\to\piz({M_0}_{/y})$
      bijective}\bigr).
  \end{cases}
  \tag{17}
\end{equation}

The injectivity of \eqref{eq:81.1} can be restated by saying that
\emph{any contractibility structure $M\subc$ on a category $M$} (with
finite products) \emph{can be recovered in terms of the corresponding
  notion of $0$-connected objected objects of $M$}, or, equivalently,
in terms of the corresponding notion of $0$-connected maps in
$M$. Still another way of phrasing this result, is in terms of the
canonical functor
\[ M\to W_0^{-1}M=\HotOf_{(M,M_0),\scrWz} \to \HotOf_\scrWz =
\scrWz^{-1}\Cat \toequ \Sets,\]
which is a functor
\begin{equation}
  \label{eq:81.18}
  \piz:M\to\Sets\tag{18}
\end{equation}
canonically associated to the contractibility structure. We may say
that \emph{the contractibility structure $M\subc$ can be recovered in
  terms of the corresponding functor \piz}, more accurately still, in
terms of the isomorphism class of the latter. Indeed, in terms of this
functor \piz, we recover $M_0$ and $W_0$ by the relations:
\begin{equation}
  \label{eq:81.19}
  \begin{split}
    M_0 &= \set[\big]{x\in\Ob M}{\text{$\piz(x)$ is a one-point set}},\\
    W_0 &= \set[\big]{f\in\Fl(M)}{\text{$\piz(f)$ is bijective}}.
  \end{split}\tag{19}
\end{equation}

This shows that the ``nice'' main fact mentioned at the beginning of
today's notes, namely (essentially) injectivity of \eqref{eq:81.1}
(and, moreover an explicit description of a way how to recover an
$M\subc$ in terms of the corresponding \scrW-asphericity structure) is
not really dependent on relatively sophisticated notions such as
``basic localizers'' and corresponding ``asphericity structures'', but
can be viewed as an ``elementary'' result (namely independent of any
consideration of ``higher'' homotopy or homology invariants, apart
from \piz) about the relationship between contractibility structures,
and corresponding $0$-connectedness notions; the latter may at will be
expressed in terms of either one of the three structural data
\begin{equation}
  \label{eq:81.20}
  M_0, \quad W_0, \quad\text{or}\quad \piz.\tag{20}
\end{equation}

This\pspage{263} relationship has been ``in the air'' since section \ref{sec:50}
(p.\ \ref{p:109}--\ref{p:110}), and I kind of turned around it
consistently up to section \ref{sec:60}, without really getting to the
core. One reason for this ``turning around'' has been, I guess, that I
let myself be distracted, not to say hypnotized, by the ``canonical''
\piz{} functor on a category $M$ (which makes really good sense only
when $M$ is ``totally $0$-connected'' as a category, a condition which
should mean, more or less I suppose, that the $0$-connected objects of
$M$, defined in terms of the mere category structure of $M$, define a
\scrWz-asphericity structure on $M$). Even after realizing (in section
\ref{sec:59}) that one should generalize the description of a
contractibility structure in terms of a ``connected components
functor'' \piz, to the case of a functor $\piz: M\to\Sets$ given
beforehand, and satisfying suitable restrictions (which I did not try
to elucidate), I still was under the impression that the
contractibility structures one could get this way must be of an
extremely special nature. In order to become aware of the fact that
this is by no means so, namely that any contractibility structure
could be obtained from a suitable functor \piz, it would have been
necessary to notice that such a structure $M\subc$ defines in a
natural way a functor \piz. There was indeed the realization that
$M\subc$ should allow to define a notion of weak equivalence (cf.\
page \ref{p:136}), but it wasn't clearly realized by then that at the
same time as a notion of weak equivalence $W$ in $M$, we should also
get a canonical functor
\[M \to W^{-1}M \to \Hot,\]
namely something a lot more precise still than a functor with values
in \Sets{} merely! But rather than push ahead in this direction, I
then decided (p.\ \ref{p:138}) that it would be ``unreasonable'' to go
on still longer pushing of investigation of test functors with values
in \Cat, following the approach which had been on my mind for quite a
while by then, and finally sketched (with the promise of a
corresponding generalization of the former ``key result'' on test
functors) in section \ref{sec:47}. Retrospectively, the whole
``grinding'' part \ref{ch:III} of these notes now looks as a rather
heavy and long-winded digression, prompted by this approach to still a
particular case of test functors (namely with values in \Cat, and
more stringently still, factoring through $\Cat\subas$).

This particular case has been of no use in the present part
\ref{ch:IV} of the notes, developing the really relevant notions in
terms of asphericity structures. \emph{Technically speaking, it now
  appears that most of the reflections of part \ref{ch:III} are
  superseded by part \ref{ch:IV}} -- the main exception being the
development of the\pspage{264} various homotopy notions in sections
\ref{sec:51} and \ref{sec:52}. On the other hand, it is clear that the
main ideas which are coming to fruition in part \ref{ch:IV} all
originated during the awkward grinding process in part \ref{ch:III}!

The ``modelizing story'' so far has turned out as the interplay of
three main sets of notions. One is made up with the
``\emph{test-notions}'', centering around the notion of a
\emph{test-category}, as one giving rise to the most elementary type
of ``\emph{modelizers}'', namely the so-called ``\emph{elementary
  modelizers}'' $(\Ahat,\scrWA)$. This was developed in part
\ref{ch:II} (while part \ref{ch:I} was concerned with the initial
motivation of the reflections, namely stacks, forgotten for the time
being!). The second set of notions concerns the so-called
``\emph{homotopy notions}'', developed at some length in sections
\ref{sec:51} to \ref{sec:55}, summarized in the diagram
\eqref{eq:81.2} above. They constitute the main technical content of
part \ref{ch:III} of the notes, with however one major shortcoming:
the relationship between these notions, and $0$-connectedness notions
\eqref{eq:81.20}, was only partially understood in part \ref{ch:III},
namely as a one-way relationship merely, associating to suitable
$0$-connectedness notions in a category $M$, a corresponding homotopy
structure in $M$. The third set of notions may be called
``\emph{asphericity notions}'', they center around the notions of
\emph{aspheric objects} and \emph{aspheric maps} (in a category
endowed with a so-called \emph{asphericity structure}), and more
specifically around the notion of an aspheric map in \Cat, whose
formal properties turn out to be the key for the development of a
theory of asphericity structures. The first and the third set of
notions (namely test notions and asphericity notions) depend on the
choice of a ``\emph{basic localizer}'' \scrW{} in \Cat, whereas the
second set, namely homotopy notions, is ``absolute'', i.e., does not
depend on any such choice, nor on any knowledge of homotopy or
(co)homology invariants.

Whereas the test notions are essentially concerned with modelizers,
namely getting descriptions of the category of homotopy types \Hot{}
in terms of elementary modelizers \Ahat{} (as being
$\scrWA^{-1}\Ahat$), it appears that the homotopy notions, as well as
the asphericity notions, are independent of any modelizing notions and
assumptions. In a deductive presentation of the theory, the test
notions would come last, whereas they came first in these notes -- as
an illustration of the general fact that the deductive approach will
present things roughly in opposite order in which they have been
discovered! The test notions, as an outcome of the attempt to get a
picture of modelizers, have kept acting as a constant guideline in the
whole reflection, even though technically speaking they are
``irrelevant'' for the development of\pspage{265} the main properties
of homotopy and asphericity notions and their interplay.

By the end of part \ref{ch:IV}, there has been some floating in my
mind as to whether which among the two structures, namely
contractibility structures or asphericity structures, should be
considered as ``the'' key structure for an understanding of the
modelizing story. There was a (justified) feeling, expressed first at
the beginning of section \ref{sec:67}, that in some sense, asphericity
structures were ``more general'' than contractibility structures,
which caused me for a while to view them as the more ``basic'' ones. I
would be more tempted at present to hold the opposite view. \emph{The
  notion of a contractibility structure now appears as a kind of hinge
  between the two main sets of notions besides the test notions,
  namely between homotopy notions and asphericity notions.} On the one
hand, as displayed in diagram \eqref{eq:81.2}, the notion of a
contractibility structure appears as the most stringent one among the
four main types of homotopy structures. On the other hand, by
\eqref{eq:81.1} it can be equally viewed as being a special case of an
asphericity structure, and as such it gives rise to (and can be
expressed by) either one of the following four asphericity-flavored
data on a category $M$ (for any given basic localizer \scrW{}
satisfying \ref{loc:4}):\scrcomment{also listed, but scratched out in
  the typescript, was $\piz:M\to\Sets$}
\begin{equation}
  \label{eq:81.21}
  \begin{cases}
    M_\scrW\subset\Ob M,  &\text{$\scrW_{M\subc}$ or simply
      $\scrW_M\subset\Fl(M)$} \\
    \text{$\varphi_M^\scrW$ or $\varphi_M:M\to\HotOf_\scrW$.} &
  \end{cases}\tag{21}
\end{equation}
Restricting to the case when \scrW{} is either \scrWz{} or \scrWoo, we
get the corresponding structures on $M$, namely the three structures
$M_0$, $W_0$, \piz{} of \eqref{eq:81.20} plus the three extra
structures:
\begin{equation}
  \label{eq:81.22}
  \begin{gathered}
  M_\oo = M_\scrWoo \subset\Ob M, \quad
  W_\oo = (\scrWoo)_M\subset\Fl(M), \\
  \varphi_M : M\to\Hot = \scrWoo^{-1}\Cat,
  \end{gathered}\tag{22}
\end{equation}
which can be referred to as \emph{\oo-connected objects} of $M$,
\emph{\oo-connected arrows} of $M$, and the ``\emph{canonical
  functor}'' from $M$ to \Hot. Putting together \eqref{eq:81.4},
\eqref{eq:81.20}, \eqref{eq:81.22}, we see that a contractibility
structure $M\subc$ gives rise to ten different structures (including
$M\subc$ itself) and is determined by each one of these,\pspage{266}
\begin{equation}
  \label{eq:81.23}
  \left\{\begin{aligned}
    &M\subc\subset\Ob M, &&J\subc\subset\Int(M), \!\!
    &&W\subc\subset\Fl(M), &&R\subc\subset\Fl(M)\times\Fl(M) \\
    &M_0\subset\Ob M,    &&
    &&W_0\subset\Fl(M),    &&\piz:M\to\Sets \\
    &M_\oo\subset\Ob M,\!\!\!  &&
    &&W_\oo\subset\Fl(M),\!\!\!  &&\varphi_M:M\to\Hot,
  \end{aligned}\right.\tag{23}
\end{equation}
namely: \emph{contractible objects}, \emph{homotopy intervals},
\emph{homotopisms}, \emph{homotopy relation for maps},
\emph{$0$-connected objects}, \emph{$0$-connected maps},\footnote{This
  name is inadequate, cf.\ note p.\ \ref{p:262}.} the \emph{connected
  components functor}, \emph{\oo-connected} (or ``aspheric'', more
accurately \scrWoo-aspheric) \emph{objects},
\emph{\scrWoo-equivalences} (or simply ``weak equivalences''), and the
(would-be ``modelizing'') \emph{canonical functor from $M$ to homotopy
  types}. Moreover, it should be remembered that, just as the
structures $J\subc$, $W\subc$, $R\subc$ are by no means unrestricted
(the fact that they stem from a contractibility structure being a
substantial restriction), the two asphericity structures (namely, the
$0$-asphericity structure $M_0$ and the \oo-asphericity structure
$M_\oo$) are subject to extra conditions which will be written down
below, implying among others that they are totally aspheric (hence
\piz{} and $\varphi_M$ are compatible with finite products).

We may complement the ten structure data \eqref{eq:81.23} above, by
the following two
\begin{equation}
  \label{eq:81.24}
  R_0, R_\oo \subset \Fl(M)\times\Fl(M),\tag{24}
\end{equation}
defined by
\[(f,g)\in R_0 \Leftrightarrow \bigl(\piz(f)=\piz(g)\bigr), \quad
(f,g)\in R_\oo \Leftrightarrow
\bigl(\varphi_M(f)=\varphi_M(g)\bigr).\]
More generally, for any basic localizer \scrW, we may define
\begin{equation}
  \label{eq:81.25}
  R_\scrW\subset\Fl(M)\times\Fl(M),\quad
  (f,g)\in R_\scrW \Leftrightarrow
  \bigl(\varphi_M^\scrW(f)=\varphi_M^\scrW(g)\bigr).
  \tag{25}
\end{equation}
It is not clear however that $M\subc$, or equivalently $M_\scrW$ or
$\scrW_M$, can be recovered in terms of the equivalence relation
$R_\scrW$ among maps of $M$.

Among the three series of structures appearing in \eqref{eq:81.23}, we
have the tautological relations
\begin{equation}
  \label{eq:81.26}
  \begin{cases}
    M\subc \subset M_\oo \subset M_0 & \\
    W\subc \subset W_\oo \subset W_0 & \\
    R\subc \subset R_\oo \subset R_0 & \\
    \begin{tikzcd}[cramped]
      M \ar[r,"\varphi_M"] \ar[rr,bend right, "\pi_0"] &
      \Hot \ar[r,"\overline{\piz}"] & \Sets
    \end{tikzcd} & \text{(commutative).}
  \end{cases}\tag{26}
\end{equation}
Finally, we may also display some of the main functors defined in
terms of a given contractibility structure $M\subc$:
\begin{equation}
  \label{eq:81.27}
  \begin{tabular}{@{}c@{}}
  \begin{tikzcd}[baseline=(O.base),sep=tiny]
    |[alias=O]| M\ar[r] &
    \overline M \ar[r]\ar[d,equal] &
    \HotOf_M \ar[r]\ar[d,equal] &
    \Hot \ar[r]\ar[d,equal] &
    \HotOf_\scrW \ar[r] &
    \Sets\ar[d,phantom,sloped,"\equ" description] \\
    & W\subc^{-1}M & W_\oo^{-1}M & \HotOf_\scrWoo & & \HotOf_\scrWz
  \end{tikzcd}.
  \end{tabular}\tag{27}
\end{equation}

\bigbreak

\presectionfill\ondate{26.6.}\pspage{267}\par

% 82
\hangsection[Proof of injectivity of
$\alpha: \Contr(M)\hookrightarrow\WAsph(M)$. \dots]{Proof of
  injectivity of \texorpdfstring{$\alpha:
    \Contr(M)\hookrightarrow\WAsph(M)$}{alpha:Contr(M)->W-Asph(M)}.
  Application to \texorpdfstring{$\bHom$}{Hom} objects and to products
  of aspheric functors \texorpdfstring{$A\to
    M$}{A->M}.}\label{sec:82}%
Yesterday I have been busy mainly with the readjustment of the overall
perspective on the main notions developed so far, which has sprung
from the new fact stated at the beginning of yesterday's notes: namely
that an arbitrary contractibility structure $M\subc$ (on a category
$M$ stable under finite products) can be recovered in terms of the
associated \scrW-asphericity structure, where \scrW{} is any basic
localizer satisfying (besides the condition \ref{loc:1} to \ref{loc:3}
of p.\ \ref{p:213}--\ref{p:214}) the extra assumption \ref{loc:4} of
p.\ \ref{p:260}. It seems about time now to enter into a little more
technical specifications along the same lines -- and to start with,
give a proof of the ``new fact''! Let's state it again in full:
\begin{theoremnum}\label{thm:82.1}
  Let $M$ be a \scrU-category stable under finite products,
  $M\subc\subset\Ob M$ a contractibility structure on $M$, admitting a
  small full subcategory $C$ which generates the structure. Let
  moreover \scrW{} be any basic localizer satisfying
  \textup{\ref{loc:4}} \textup(compatibility with the \piz-functor
  $\Cat\to\Sets$\textup), and let $M_\scrW\subset\Ob M$ the
  \scrW-asphericity structure generated by $M\subc$\kern1pt,
  $W=\scrW_{M\subc}$ the corresponding set of ``\scrW-equivalences''
  or ``weak equivalences'' in $\Fl(M)$. Consider the homotopy
  structure $h_W$ associated to $W$, i.e.\ \textup(cf.\ section
  \ref{sec:54}\textup), the homotopy structure associated to the
  homotopy interval structure $J$ generated by the set $J_0$ of all
  intervals
  \[\bI=(I,\delta_0,\delta_1)\]
  in $M$ such that
  \[I\in M_\scrW.\]
  Then $h_W$ is the homotopy structure on $M$ associated to the
  contractibility structure $M\subc$\kern1pt, and hence $M\subc$ can be
  described in terms of $M_\scrW$ \textup(or of
  $\scrW_{M\subc}=W$\textup) as the set of objects which are
  contractible for $h_W$, i.e., such that the map $x\to e_M$ is an $h_W$-homotopism.
\end{theoremnum}
\begin{proof}
  Let $M\subc'$ be the set of $h_W$-contractible objects of $M$,
  clearly we have
  \[ M\subc \subset M\subc'\subset M_\scrW \eqdef M\suba.\]
  The theorem amounts to saying that $M\subc$ generates the homotopy
  interval structure $J$ (by which we mean that the set of intervals
  of $M$ made up with objects of $M\subc$ generates the structure
  $J$). Indeed, because of $M\subc\subset M\subc'$, this will imply
  that $J$ is associated to a contractibility structure, namely to
  $M\subc'$. But for an object $x$ of $M$ to be in $M\subc'$, i.e., to
  be contractible for the structure $J$, amounts to be contractible
  for $M\subc$, and hence by saturation of $M\subc$, to be in
  $M\subc$, hence $M\subc'=M\subc$ -- which yields what
  we\pspage{268} want. By the description of $J$ in terms of $J_0$, we
  are now reduced to proving the following
\end{proof}
\begin{lemmanum}\label{lem:82.1}
  Let $I$ be an object of $M\suba=M_\scrW$. Then any two sections of
  $I$ \textup(over $e_M$\textup) are $M\subc$-homotopic.
\end{lemmanum}

Let, as in the previous section, $\scrWz\supset\scrW$ be the largest
of all basic localizers satisfying \ref{loc:4}, i.e.,
\[\scrWz=\set[\big]{f\in\Fl(\Cat)}{\text{$\piz(f)$ a bijection}},\]
therefore, we have
\[ M_\scrW \subset M_\scrWz \eqdef M_0,\]
and we are reduced to proving the lemma for \scrWz{} instead of \scrW,
i.e., for $M_0$ instead of $M_\scrW$. We'll use the small full
subcategory $C$ of $M\subc$ generating the contractibility structure
$M\subc$, we may assume that $C$ is stable under finite
products. Hence \Chat{} is totally \scrWz-connected, i.e., totally
$0$-connected. Moreover, as $C\subset M\subc$, and $e_M$ is in $C$, it
follows that every object of $C$ has a section (over $e_M=e_C$) --
which implies that every non-empty object of \Chat{} has a section,
i.e., \Chat{} is ``strictly totally $0$-connected'' (cf.\ p.\
\ref{p:144} and \ref{p:149}). Note that (by prop.\ \ref{prop:79.2}
\ref{it:79.2.b} of p.\ \ref{p:248}) we have
\[ M_0 = \set[\big]{x\in\Ob M}{\text{$i^*(x)$ is $0$-connected in
    \Chat}},\]
where $i$ is the inclusion functor:
\[i:C\to M.\]
We have to prove that any two sections $\delta_0,\delta_1$ of an
object $I$ of $M_0$ are $M\subc$-homotopic, or what amounts to the
same, $C$-homotopic. This translates readily into the statement that
$i^*(\delta_0)$ and $i^*(\delta_1)$ are $C$-homotopic in \Chat. Thus,
we are reduced to the following lemma (in the case of the topos
\Chat):
\begin{lemmanum}\label{lem:82.2}
  Let \scrC{} be a totally $0$-connected topos such that any non-empty
  object of $C$ has a section, and let $C$ be a small full generating
  subcategory, whose elements are $0$-connected. Then for any
  $0$-connected object $I$ of \scrC, and any two sections
  $\delta_0,\delta_1$ of $I$, these are $C$-homotopic, i.e., they can
  be joined by a finite chain of sections, any two consecutive among
  which can be obtained as the images of two sections $s_i,t_i$ of an
  objects $x_i$ of $C$, by means of a map $h_i:x_i\to I$.
\end{lemmanum}

This lemma is essentially a restatement (cleaned from extraneous
assumptions due to an awkward conceptual background) of the
proposition of page \ref{p:149} (section \ref{sec:60}), and the proof
will be left to the reader.

\subsection*{Application to relation between contractibility and
  objects \texorpdfstring{$\bHom(X,Y)$}{Hom(X,Y)}.}
\label{subsec:82.hom}

I\pspage{269} would like to review here a few things along these lines, which were
somewhat scattered in the notes before (section \ref{sec:51}
\ref{subsec:51.E} p.\ \ref{p:121} and section \ref{sec:57} p.\
\ref{p:143} notably), and at times came out awkwardly because of
inadequate conceptual background. We assume $M$ endowed with a
contractibility structure $M\subc$, and use the notations of the
previous section, especially concerning the subsets of $\Ob M$ and of
$\Fl(M)$
\begin{equation}
  \label{eq:82.1}
  M\subc \subset M_\oo \subset M_\scrW \subset M_0, \quad
  W\subc \subset W_\oo \subset \scrW_M \subset W_0. \tag{1}
\end{equation}
Let $X$ be an object of $M$, such that the object
\begin{equation}
  \label{eq:82.2}
  I = \bHom(X,X)\tag{2}
\end{equation}
exists in $M$ (NB\enspace a priori it is an object in $M\uphat$, we
assume it to be representable). We assume that $X$ is endowed with a
section $c$, hence a section $\delta_1$ of $I$, corresponding to the
constant endomorphism of $X$ with value $c$. We'll denote by
$\delta_0$ the section of $I$ corresponding to the identity of
$X$. Thus,
\begin{equation}
  \label{eq:82.3}
  \bI=(I,\delta_0,\delta_1)\tag{3}
\end{equation}
becomes an interval of $M$, and the composition law of $I=\bHom(X,X)$
turns it into a \emph{multiplicative interval}, admitting respectively
$\delta_0$ and $\delta_1$ as left unit and left zero
element.\scrcomment{in the typescript we have $\pi_0$ and $\pi_1$ as
  the elements; I assume this is a typo?} Then the following
conditions are equivalent:
\begin{description}
\item[\namedlabel{it:82.hom.i}{(i)}]
  $X$ is contractible.
\item[\namedlabel{it:82.hom.ii}{(ii)}]
  $\bHom(X,X)$ is contractible, i.e., $I\in M\subc$.
\item[\namedlabel{it:82.hom.iiprime}{(ii')}]
  $\bHom(X,X)$ is $0$-connected, i.e., $I\in M_0$.
\item[\namedlabel{it:82.hom.iidblprime}{(ii'')}]
  The two sections $\delta_0,\delta_1$ of $I=\bHom(X,X)$ are homotopic.
\item[\namedlabel{it:82.hom.iii}{(iii)}]
  For any object $Y$ in $M$ such that $\bHom(Y,X)$ exists in $M$,
  $\bHom(Y,X)$ is contractible.
\item[\namedlabel{it:82.hom.iiiprime}{(iii')}]
  For any $Y$ as in \ref{it:82.hom.iii}, $\bHom(Y,X)$ is $0$-connected.
\item[\namedlabel{it:82.hom.iv}{(iv)}]
  For any $Y$ in $M$ such that $\bHom(X,Y)$ exists in $M$, the
  canonical map
  \begin{equation}
    \label{eq:82.4}
    Y \to \bHom(X,Y)\tag{4}
  \end{equation}
  is a homotopism (more accurately still, $Y$ as a subobject of
  $\bHom(X,Y)$ is a deformation retract). Moreover, $X$ is $0$-connected.
\item[\namedlabel{it:82.hom.ivprime}{(iv')}]
  For any $Y$ in $M$ as in \ref{it:82.hom.iv}, the map \eqref{eq:82.4}
  induces a bijection
  \[\piz(Y)\to\piz(\bHom(X,Y)),\]
  moreover $X$ is $0$-connected.
\end{description}
NB\enspace Of course, the $0$-connectedness notion and the functor
\piz{} used\pspage{270} here are those associated to the given
structure $M\subc$. The case dealt with in section \ref{sec:57} (p.\
\ref{p:143}) is essentially (it seems) the one when these notions are
the ones canonically defined in terms of the category structure of $M$
alone. In view of the inclusions \eqref{eq:82.1}, we could throw in a
handful more obviously equivalent conditions, using $M_\oo$, $M_\scrW$
or $W_\oo$, $\scrW_M$, instead of $M\subc$, etc.\ but it seems this
would confuse the picture rather than complete it.

Next thing is to look at the canonical map
\begin{equation}
  \label{eq:82.5}
  \overline\Gamma(X) \to \piz(X)\tag{5}
\end{equation}
defined for any object $X$ of $M$, where $\Gamma(X)$ denotes the set
of sections of $X$, and $\overline\Gamma(X)$ denotes the set of
corresponding homotopy classes of sections. (This map was considered
in a slightly different case on page.\ \ref{p:144}) The map
\eqref{eq:82.5} can be viewed as the particular case of
\[ \oHom(Y,X) \to \Hom_{\mathrm{Sets}}(\piz(Y),\piz(X)),\]
obtained when $Y=e_M$, hence $\piz(Y)=$ one-point set. By the standard
description
\begin{equation}
  \label{eq:82.6}
  \piz(X)\simeq \bpiz(C_{/X}) \quad(=\piz(i^*(X)))\tag{6}
\end{equation}
we immediately get that \eqref{eq:82.5} is always surjective, as any
element in $\piz(X)$ is induced by an element $x\to X$ of $C_{/X}$,
hence by a section
\[ e_M\to x\to X,\]
where $e_M\to x$ is a section of the (contractible) object $x$ in
$C$. But \eqref{eq:82.5} is in fact bijective. To see this, let's note
that the map \eqref{eq:82.5} is isomorphic to the corresponding map in
\Sets, with $M$ replaced by \Chat{} and $X$ by $i^*(X)$, $C$ remaining
the same, as follows from \eqref{eq:82.6} and the formula
\[\Gamma(X) \simeq \overline\Gamma(i^*(X)).\]
The latter is a particular case of the
\begin{lemmanum}\label{lem:82.3}
  Let $Y,X$ be two objects of $M$, with $Y$ in $C$. Then the natural
  map
  \begin{equation}
    \label{eq:82.7}
    \oHom(Y,X) \to \oHom(i^*(Y),i^*(X))\tag{7}
  \end{equation}
  between sets of homotopy classes of maps \textup(in $M$ and in
  \Chat{} respectively, the latter being endowed with the
  contractibility structure generated by $C$\textup) is bijective.
\end{lemmanum}

The verification is tautological, due to the fact that for any object
$I$ of $C$ (which is going to play the role of a homotopy interval for
deformations), we got
\[\Hom_M(I\times Y,X) \tosim \Hom_\Chat(I\times Y, i^*(X)).\]

Now,\pspage{271} the map \eqref{eq:82.5} in case of a strictly totally
$0$-connected category (here \Chat) has been dealt with in section
\ref{sec:58} (p.\ \ref{p:114}), it follows easily that the map is
bijective -- hence the
\begin{propositionnum}\label{prop:82.1}
  Let $(M,M\subc)$ be any contractibility structure,
  \[\piz:M\to\Sets,\quad
  \piz(X)\simeq\bpiz({M\subc}_{/X})\]
  the corresponding ``connected components functor'', $X$ any object
  of $M$. Then the canonical map \eqref{eq:82.5} above is bijective.
\end{propositionnum}
\begin{corollary}
  Let $X,Y$ be two objects of $M$, such that $\bHom(X,Y)$ exists in
  $M$. Then the natural map
  \begin{equation}
    \label{eq:82.8}
    \oHom(X,Y) \to \piz(\bHom(X,Y))\tag{8}
  \end{equation}
  is bijective.
\end{corollary}

\bigbreak

To finish these generalities on $\bHom$'s, I would like to generalize
to the context of contractibility structures the results of prop.\
\ref{prop:74.4} p.\ \ref{p:228} (section \ref{sec:74}) on products of
aspheric functors. We start within a context of asphericity
structures:
\begin{propositionnum}\label{prop:82.2}
  Let $(M,M\suba)$ be a \scrW-asphericity structure \textup(for a basic
  localizer \scrW\textup), $C$ a full subcategory of $M$ generating
  this structure, i.e., such that $C\to M$ is $M\suba$-\scrW-aspheric,
  $A$ a small category,
  \[ i : A\to M\]
  a $M\suba$-\scrW-aspheric functor. We assume $M$ stable under binary
  products.
  \begin{enumerate}[label=\alph*),font=\normalfont]
  \item\label{it:82.prop2.a}
    Let $b_0$ in $M$ be such that for any $y$ in $C$, $\bHom(b_0,y)$
    exists in $M$, and for any $x$ in $M\suba$, $x\times b_0$ is in
    $M\suba$. Let $i_{b_0}:A\to M$ be the constant functor with value
    $b_0$, and consider the product functor
    \[ i \times i_{b_0}: a\mapsto i(a)\times b_0 : A\to M.\]
    This functor is $M\suba$-\scrW-aspheric if{f} for any $y$ in $C$,
    $\bHom(b_0,y)$ is in $M\suba$ \textup(a condition which does not
    depend on $i$ nor even on $A$\textup).
  \item\label{it:82.prop2.b}
    Let $B$ be a full subcategory of $M$, such that for any $b_0$ in
    $B$, $x$ in $M\suba$, and $y$ in $C$, we have $x\times b_0\in
    M\suba$ and $\bHom(b_0,y)$ exists in $M$, and is in
    $M\suba$. Assume $A$ totally \scrW-aspheric, let $i':A\to M$ be
    any functor factoring through $B$, then the product functor
    \[ i \times i' : a\mapsto i(a)\times i'(a) : A\to M\]
    is $M\suba$-\scrW-aspheric.
  \end{enumerate}
\end{propositionnum}

The proof is word by word the same as for the analogous statements p.\
\ref{p:228}--\ref{p:229}, and therefore left to the reader. (The
analogous statement to cor.\ \ref{cor:74.4.1} p.\ \ref{p:229} is
equally valid.)

The\pspage{272} case I've in mind is when $M\suba$ is generated by a
contractibility structure $M\subc$, and $C=B=M\subc$. The condition in
\ref{it:82.prop2.b} boils down to existence of $\bHom(b_0,y)$ in $M$,
when $b_0$ and $y$ are both contractible objects of $M$ -- as a matter
of fact, in all cases I'm having in mind, the $\bHom$ exists even
without any contractibility assumption. Thus we get:
\begin{corollary}
  Let $(M,M\subc)$ be a contractibility structure, such that for any
  two contractible objects $x,y$, $\bHom(x,y)$ exists in $M$ \textup(a
  condition which presumably is even
  superfluous\ldots\textup). Consider the \scrW-asphericity structure
  $M_\scrW$ on $M$ generated by $M\subc$. Then the product of a
  $M_\scrW$-\scrW-aspheric functor $i:A\to M$ with a functor $i':A\to
  M$ factoring through $M\subc$ is again $M_\scrW$-\scrW-aspheric. In
  particular, the category of all aspheric functors from $A$ to $M$
  which factor through $M\subc$ is stable under binary products.
\end{corollary}

From this one can deduce as on p.\ \ref{p:230} that if the latter
category is non-empty, i.e, if there does exist an aspheric functor
$A\to M$ through $M\subc$, then one can define a canonical functor
\begin{equation}
  \label{eq:82.9}
  \HotOf_{M,\scrW} \to \HotOf_{A,\scrW},\tag{9}
\end{equation}
defined up to unique isomorphism, via a transitive system of
isomorphisms between the functors deduced from aspheric functors $A\to
M$ factoring through $M\subc$.

% 83
\hangsection{Tautologies on \texorpdfstring{$\Imm\alpha$}{Im alpha}, and
  related questions.}\label{sec:83}%
After stating and proving theorem \ref{thm:82.1} of the previous
section, I forgot to give an answer to the most natural question
arising from it -- namely how to characterize those \scrW-asphericity
structures on a category $M$ stable under finite products, which can
be generated by a contractibility structure. The reason for this is
surely that I have nothing better to offer than a tautology: let
\[M\suba \subset \Ob M\]
be the given \scrW-asphericity structure, we assume beforehand that
this structure is totally aspheric (which is a necessary condition for
$M\suba$ to come from a contractibility structure $M\subc$). If
$W\suba$ is the corresponding set of \scrW-equivalences, it follows
that the homotopy structure $h_{W\suba}$ is also the one defined by
the homotopy interval structure $J$ generated by the set $J_0$ of
intervals of $M$ made up with objects of $M\suba$. Let\pspage{273}
\[M\subc \subset \Ob M\]
be the corresponding set of contractible objects (which may not be a
contractibility structure on $M$). The conditions now are the
following:
\begin{enumerate}[label=\alph*)]
\item\label{it:83.a}
  $M\subc$ generates the homotopy interval structure $J$, or
  equivalently, any two sections of an object of $M\suba$ are $M\subc$-homotopic.
\end{enumerate}
This condition is clearly equivalent to saying that $J$ is indeed
generated by a contractibility structure, and the latter is
necessarily $M\subc$.
\begin{enumerate}[label=\alph*),start=2]
\item\label{it:83.b}
  $M\subc$ generates the \scrW-asphericity structure $M\suba$, i.e.,
  there exists a small full subcategory $B$ of $M\subc$ such that the
  inclusion functor $i:B\to M$ be $M\suba$-\scrW-aspheric, i.e., for
  any $x$ in $M\suba$, $i^*(x)$ in \Bhat{} is \scrW-aspheric (i.e.,
  $B_{/x}$ is \scrW-aspheric in \Cat).
\end{enumerate}

One would like too a n.s.\ condition on a functor
\[\varphi:M\to\HotOf_\scrW,\]
for this functor to be isomorphic to the one associated to a
contractibility structure on $M$ -- with special interest in the case
when $\scrW=\scrWz$, i.e., when the functor reduces to a functor
\[\piz:M\to\Sets.\]
Here again, I have nothing to offer except a tautological statement,
which isn't worth the trouble writing down. Nor do I have at present a
compelling feeling that there should exist such a characterization,
under suitable exactness assumptions on $M$ say, and possibly assuming
too that $M$ is stable under the operation $\bHom$. Here also arises
the question \emph{whether a functor $\varphi:M\to\HotOf_\scrW$
  stemming from a contractibility structure $M\subc$ on $M$ can have
  non-trivial automorphisms} -- a question closely connected to the
``inspiring assumption'' of section \ref{sec:28}.

\bigbreak

\presectionfill\ondate{29.6.}\pspage{274}\par

% 84
\hangsection[A silly (provisional) answer to the ``silly question'' --
and \dots]{A silly
  \texorpdfstring{\textup(\kernifitalic{1pt}}{(}provisional\texorpdfstring{\textup)}{)}
  answer to the ``silly question'' -- and the new perplexity
  \texorpdfstring{$f_!(M\subas)\subset M'\subas$}{f!(Mas) in
    M'as}?}\label{sec:84}%
Two days have passed without writing any notes. Much of the time I
spent on writing mathematical letters -- one pretty long one to Gerd
Faltings,\scrcomment{\textcite{Grothendieck1983b}} who (on my request)
had sent me preprints of his recent work, notably on the Tate
conjectures for abelian varieties and on the Mordell conjecture, and
had expressed interest hearing about some ideas and conjectures on
``anabelian algebraic geometry''. I had been impressed, from a glance
upon the last of his manuscripts, to see three key conjectures proved
in about forty pages, while they were being considered as quite out of
reach by the people supposed to know. Some ``anabelian'' conjectures
of mine are closely related to the Tate and Mordell conjectures just
proved by Faltings -- Deligne had pointed out to me about two years
ago that a certain fixed-point conjecture (which I like to view as the
basic conjecture at present in the anabelian program) implied
Mordell's, so why loose one's time on it! I have the feeling Faltings
is the kind of chap who may become interested in things which are
supposed to be too far off to be worth looking at, that's why I had
written him a few words, under the moment's inspiration. -- Another
letter, not quite as long, was an answer to a very long and patient
letter of Tim Porter, telling me about a number of things which have
been done by homotopy people, and which I was of course wholly
ignorant of! His letter has been the first echo I got from someone who
read part of the notes on ``Pursuing stacks'', and I was glad he could
make some sense of what he read so far, and conversely -- that not all
he was telling me was going wholly ``above my head''!

Apart from this and the (not unpleasant!) daily routine, I spent a
fair bunch of hours on scratchwork, centering around trying to figure
out the right notion of morphism for asphericity structures. It was a
surprise that the notion should be so reticent for revealing itself --
as a matter of fact, I am not quite sure yet if I got the right
notion, in sufficient generality I mean. Time will tell -- for the
time being, the notion of morphism I have to offer, while maybe too
restrictive, looks really seducing, because it parallels so perfectly
the formalism of aspheric functors $i:A\to M$ and the corresponding
(would-be modelizing) functor $i^*:M\to\Ahat$. Again, it has been a
surprise, right now, that after uncounted hours of unconvincing
efforts today and yesterday, and almost wholly unrelated to these,
this pretty set-up would come out within ten minutes reflection!

\begin{proposition}
  Let\pspage{275} $(M,M\suba)$, $(M',M'\suba)$ be two
  \scrW-asphericity structures \textup(\scrW{} a basic
  localizer\textup), $B\subset M\suba$ a full subcategory of $M$
  generating the asphericity structure $M\suba$, and
  \[f^*:M\to M'\]
  a functor, admitting a left adjoint $f_!$ \textup(hence $f^*$
  commutes to inverse limits\textup). We consider the following six
  conditions, paralleling those of prop.\ \ref{prop:75.4} of section
  \ref{sec:75} \textup(p.\ \ref{p:236}\textup):
  \begin{description}
  \item[\namedlabel{it:84.i}{(i)}]
    $M\suba = (f^*)^{-1}(M'\suba)$,
  \item[\namedlabel{it:84.iprime}{(i')}]
    $M\suba \subset (f^*)^{-1}(M'\suba)$, i.e., $f^*(M\suba) \subset M'\suba$,
  \item[\namedlabel{it:84.idblprime}{(i'')}]
    $B\suba \subset (f^*)^{-1}(M'\suba)$, i.e., $f^*(B) \subset M'\suba$,
  \item[\namedlabel{it:84.ii}{(ii)}]
    $W\suba = (f^*)^{-1}(W'\suba)$,
  \item[\namedlabel{it:84.iiprime}{(ii')}] 
    $W\suba \subset (f^*)^{-1}(W'\suba)$, i.e., $f^*(W\suba)\subset W'\suba$,
  \item[\namedlabel{it:84.iidblprime}{(ii'')}] 
    $\Fl(B)\suba \subset (f^*)^{-1}(W'\suba)$, i.e., $f^*(\Fl(B))\subset W'\suba$,
  \end{description}
  where $W\suba$, $W'\suba$ are the sets of \scrW-equivalences in $M$
  and $M'$ respectively. We'll make the extra assumption \textup(I
  nearly forgot, sorry!\textup):
  \begin{description}
  \item[\namedlabel{cond:84.Awk}{(Awk)}]
    There exists a $M'\suba$-\scrW-aspheric functor
    \[i':A\to M'\]
    \textup($A$ a small category\textup), such that $i=f_!i':A\to M$
    factors through $B$.
  \end{description}
  Under these conditions and with these notations, the following
  holds: The conditions \textup{\ref{it:84.i}, \ref{it:84.iprime},
    \ref{it:84.idblprime}} are equivalent and imply the three others,
  which satisfy the tautological implications \textup{\ref{it:84.ii}}
  $\Rightarrow$ \textup{\ref{it:84.iiprime}} $\Rightarrow$
  \textup{\ref{it:84.iidblprime}}. If $M$ and $M'$ admit final objects
  and their asphericity structures are aspheric, then the fiver first
  conditions \textup{\ref{it:84.i}} to \textup{\ref{it:84.iiprime}}
  are equivalent, and if moreover $e_M\in \Ob B$, all six are
  equivalent.
\end{proposition}
\noindent\emph{Proof:} reduction to loc.\ sit., using the commutative diagram
\begin{equation}
  \label{eq:84.1}
  \begin{tikzcd}[column sep=tiny]
    M\ar[rr,"f^*"]\ar[dr,"i^*"'] & & M'\ar[dl,"{i'}^*"] \\
    & \Ahat &
  \end{tikzcd}\tag{1}
\end{equation}
(up to isomorphism), and the relations
\[M'\suba = ({i'}^*)^{-1}(\Ahata), \quad
W'\suba = ({i'}^*)^{-1}(\scrWA).\]
This proof shows moreover that the conditions \ref{it:84.i},
\ref{it:84.iprime}, \ref{it:84.idblprime} are equivalent each to $i$
being $M\suba$-\scrW-aspheric. Thus, if these conditions are satisfied
(let's say then that $f^*$ is a
\emph{morphism}\footnote{\scrcommentinline{I can't read this
    footnote}} for the asphericity structures), we may use $i^*$ and
${i'}^*$ for describing the canonical functors from\pspage{276} the
localizations of $M,M'$ to $\HotOf_\scrW$, and therefore from
\eqref{eq:84.1} get a \emph{commutative} diagram (up to canonical
isomorphism)
\begin{equation}
  \label{eq:84.2}
  \begin{tabular}{@{}c@{}}
    \begin{tikzcd}[baseline=(O.base),column sep=tiny]
      \HotOf_{M,\scrW}\ar[rr,"\overline{f^*}"]
      \ar[dr] & & \HotOf_{M',\scrW}\ar[dl] \\
      & |[alias=O]| \HotOf_\scrW &
    \end{tikzcd}.
  \end{tabular}\tag{2}
\end{equation}
From this follows:
\begin{corollary}
  Let $f^*$ be a morphism for the given asphericity structure, and
  assume these to be \emph{modelizing}. Then the induced functor for
  the localizations
  \[\overline{f^*}: W\suba^{-1}M = \HotOf_{M,\scrW} \to
  {W'\suba}^{-1}M' = \HotOf_{M',\scrW}\]
  is an equivalence of categories.
\end{corollary}

This is the answer, at last, of the ``silly question'' of section
\ref{sec:45} (p.\ \ref{p:95})!

We have still to comment though on the restrictive conditions we had
to make, for getting the equivalences stated in the proposition, and
the result stated in the corollary. These conditions are twofold:
a)\enspace existence of a left adjoint $f_!$, which presumably, in
most circumstances we are going to meet, will be equivalent with $f^*$
commuting to inverse limits. It's a pretty restrictive condition, but
of a rather natural kind, often met with in the modelizing situations;
b)\enspace this is the ``awkward'' condition \ref{cond:84.Awk}, which
can be equally stated as follows: there exists a small full
subcategory $B'$ of $M'$ (NB\enspace we may take the full subcategory
defined by $i'(\Ob A)$), \emph{generating} the asphericity structure
$M'\suba$, and such moreover that
\[f_!(B')\subset M\suba,\]
(so that we can choose $C$, a full subcategory of $M$ generating the
asphericity structure, such that $C$ contains $f_!(B')$. Another way
of phrasing this condition on $f^*$ (preliminary to the choice of $C$)
is that the full subcategory of $M'$
\begin{equation}
  \label{eq:84.3}
  \text{$(f_!)^{-1}(M\suba) \sand M'\suba$ generates the asphericity
    structure of $M'$.}\tag{3}
\end{equation}

If this condition, plus the condition \ref{it:84.i} say, which we
express jointly by saying that $f^*$ is a ``morphism of asphericity
structures'', did imply the condition (stronger than
\eqref{eq:84.3})\pspage{277}
\begin{equation}
  \label{eq:84.4}
  f_!(M'\suba) \subset M\suba,\tag{4}
\end{equation}
we would replace \ref{cond:84.Awk} by this condition \eqref{eq:84.4},
which doesn't look awkward any longer, and does not refer to any $A$
or $i'$ whatsoever (and the six conditions \ref{it:84.i} to
\ref{it:84.iidblprime}, with the exception of \ref{it:84.idblprime}
and \ref{it:84.iidblprime}, do not make any reference to any given
aspheric functor). In any case, one might think of defining a
\emph{morphism} $f^*$ of asphericity structures as a functor admitting
a left adjoint satisfying \eqref{eq:84.4}, and such moreover that
\ref{it:84.i} above, or equivalently \ref{it:84.iprime}, namely the
nice symmetric relation to \eqref{eq:84.4}
\begin{equation}
  \label{eq:84.5}
  f^*(M\suba)\subset M'\suba,\tag{5}
\end{equation}
is satisfied. This notion for a morphism, stricter than the one we
adopted provisionally, looks a lot nicer indeed -- the trouble is that
I could not make up my mind if the restriction \eqref{eq:84.4} is not
an unreasonable one. It would be reasonable indeed, if for those cases
which are the most interesting for us, and primarily for any functor
\[f^*=i^*: M\to\Ahat\]
associated to an $M\suba$-\scrW-aspheric functor
\[ i:A\to M\]
for any given ``nice'' asphericity structure $(M,M\suba)$, this
condition is satisfied. Of course, in this case, $i^*$ does admit a
left adjoint
\[ i_!:\Ahat\to M,\]
provided only $M$ is stable under (small) direct limits, which we'll
assume without any reluctance! Thus, we are led to the following
\begin{questionnum}\label{q:84.1}
  Under which conditions is it true, for an $M\suba$-\scrW-aspheric
  functor $i:A\to M$ (where $M$ is endowed with an asphericity
  structure $M\suba$) that
  \begin{equation}
    \label{eq:84.6}
    i_!(\Ahata) \subset M\suba,\tag{6}
  \end{equation}
  i.e., $i_!$ (the canonical extension of $i$ to \Ahat, commuting with
  direct limits) takes aspheric objects into aspheric objects?
\end{questionnum}

Yesterday and today I pondered mainly about the typical case when
$M=\Bhat$, endowed with its canonical asphericity
structure,\pspage{278} where $B$ is a small category, and moreover $i$
is a functor $i:A\to B\subset \Bhat$. This then brings us to the
related
\setcounter{questionnum}{0}
\renewcommand*{\thequestionnum}{\arabic{questionnum}'}
\begin{questionnum}\label{q:84.1prime}
  Let $i:A\to B$ be an aspherical map in \Cat, under which restrictive
  conditions on $A,B,i$ (if any) does
  \[i_!: \Ahat\to\Bhat\]
  take aspheric objects into aspheric ones, i.e., do we
  have\scrcomment{here in the typescript, AG reverts to putting an
    extra ``s'' in the subscript for asphericity structures, but I'm
    sticking to just ``a'' from now on}
  \begin{equation}
    \label{eq:84.7}
    i_!(\Ahata) \subset B\uphat\suba\text{?}\tag{7}
  \end{equation}
\end{questionnum}
\renewcommand*{\thequestionnum}{\arabic{questionnum}}

The question makes sense even without assuming $i$ to be aspheric. If
we drop this asphericity assumption on $i$, we are reduced, for a
given $B$, (by replacing $A$ by $A_{/F}$, where $F$ is a given
aspheric object of \Ahat) to the case when $A$ is aspheric and
$F=e_\Ahat$, i.e., to looking at whether the element
\begin{equation}
  \label{eq:84.8}
  i_!(e_\Ahat) = \varinjlim_A \prescript{}{(\Bhat)}{i(a)}\tag{8}
\end{equation}
in \Bhat{} is aspheric. This element appears as the direct limit in
\Bhat{} of aspheric elements $i(a)$, the indexing category $A$ being
itself aspheric. Thus, it is rather tempting to hope that the limit
might well be aspheric too. Let's assume $\scrW=\scrWoo=$ usual weak
equivalence, it turns out that when $A$ is $1$-connected, so is
\eqref{eq:84.8}, which would seem to give some support to the hope
that \eqref{eq:84.8} is aspheric if $A$ is. However, this is
definitely not so, unless at the very least we assume $B$ to be
equally aspheric (which is however a rather natural assumption, as it
follows from $A$ aspheric if we assume moreover $i$ to be
aspheric). To see this, we take $i$ to be cofibering with
$0$-connected fibers -- the cofibration assumption implies that the
direct limit over $A$ can be computed first fiberwise and then take a
limit over $B$ (the most general version of associativity for direct
limits!), whereas the $0$-connectedness assumption on the fibers
implies that the limit taken over the fiber $A_b$ is just $b$ itself,
hence \eqref{eq:84.8} is isomorphic to
\[ \varinjlim_b \prescript{}{(\Bhat)}{b} = e_\Bhat,\]
which isn't aspheric except precisely when $B$ is; now for topological
reasons it is easy to find an aspheric $A$ cofibered with
$0$-connected fibers over a non-aspheric $B$.

Thus, in the question of asphericity of \eqref{eq:84.8} we'll better
assume both $A$ \emph{and} $B$ aspheric. If either $A$ has a final
object or $B$ has an initial object, asphericity of \eqref{eq:84.8} is
more or less trivial in any case. For certain categories $B$ even
without initial object, \eqref{eq:84.8} is always aspheric when $A$
is, this I checked at any rate for the ordered category
\[ B= \begin{tikzcd}[row sep=-3pt,column sep=small,baseline=(O.base)]
  \alpha\ar[dr] & \\ & |[alias=O]| \gamma \\ \beta\ar[ur] &
\end{tikzcd},\]
giving rise to a rather interesting computation. This suggested, for
arbitrary $B$ again, to look at the dual of the category above as $A$:
\begin{equation}
  \label{eq:84.9}
A= \begin{tikzcd}[row sep=-3pt,column sep=small,baseline=(O.base)]
  \alpha & \\ & |[alias=O]| \gamma\ar[ul]\ar[dl] \\ \beta &
\end{tikzcd}.\tag{9}
\end{equation}
Here the question then amounts to \emph{whether for a diagram}
\[\begin{tikzcd}[row sep=-3pt,column sep=small]
  & a \\ c\ar[ur]\ar[dr] & \\ & b
\end{tikzcd}\]
\emph{in $B$, the amalgamated sum}
\begin{equation}
  \label{eq:84.10}
  a \amalg_c b \tag{10}
\end{equation}
\emph{in \Bhat{} is aspheric.} This I know to be true if either
$c\to a$ or $c\to b$ are monomorphisms (as stated in section
\ref{sec:70}). Now, I don't really expect this to be true in general,
even if $B$ is such an excellent category as $\Simplex$ say, however,
I didn't push through and make a counterexample. If we now want to
remember the asphericity condition on $i$ which we dropped, this
condition, when $A$ has an initial object $\gamma$, just means that
$c=i(\gamma)$ is an initial object of $B$. Even with this extra
condition, I do not really expect \eqref{eq:84.10} to be necessarily
aspheric. Therefore, I do not expect the inclusion \eqref{eq:84.7} to
hold for any aspheric functor $i$, even when $A$ (and hence $B$) are
assumed to be aspheric, without some extra condition, on $A$ or $B$
say.

The condition that \Ahat{} be \emph{totally aspheric} seems in this
context a rather natural one -- for instance, when reducing the
question whether \eqref{eq:84.7} holds to the question of asphericity
of an object \eqref{eq:84.8} (where $A$ stands for $A_{/F}$), and if
we do not want to loose the asphericity assumption for $i$ when taking
the composition $A_{/F}\to A\xrightarrow i B$, we would like that
asphericity of $F$ imply asphericity of $A_{/F}\to A$ -- which means
precisely that $A$ is totally aspheric. The trouble is that in the
would-be counterexample above with $A$ given by \eqref{eq:84.9}, $A$
is stable under Inf, i.e., under binary products, hence \Ahat{}
\emph{is} totally aspheric indeed -- thus it is doubtful that this
extra condition in the ``question \ref{q:84.1prime}\kern1pt'' above is
quite enough. If it does fail indeed, the next best would be to try
the stronger condition ``$A$ is a strict test category'' (NB\enspace
the category \eqref{eq:84.9} isn't a test category!), which leads to
the question whether \eqref{eq:84.8} is aspheric when $A$ is a test
category (no longer a strict one though!) and $i:A\to B$ an aspheric
functor. But I confess I have no idea at present how to handle this
question, and I am dubious there will come out any positive result
along these lines, even when assuming $A$ and $B$ to be both stable
under binary products say, and to be contractors and what-not!

Another\pspage{280} typical case for ``Question \ref{q:84.1}'' (p.\
\ref{p:277}) is the case when
\[ M= \Cat,\]
endowed with the usual asphericity structure, giving rise to the
notion of ``test functor''
\[i:A\to\Cat\]
we have been working with almost from the very start. The most
important case of all is of course the canonical functor
\[ i_A: a\mapsto A_{/a} : A\to\Cat,\]
and we well know that this functor is aspheric (for the natural
asphericity structure of \Cat) if $A$ is a weak test category, and in
this case it is true indeed that not only $i_A^*$, but equally
${i_A}_!$ is modelizing, and transforms aspheric objects into aspheric
objects. But we know too that for more general test functors, for
instance the standard inclusion
\begin{equation}
  \label{eq:84.11}
  i:\Simplex\hookrightarrow\Cat,\tag{11}
\end{equation}
it is no longer true in general that $i_!$ be modelizing -- so is it
at all reasonable to expect
\begin{equation}
  \label{eq:84.12}
  i_!(\Ahata) \subset\Cat\suba \text{?}\tag{12}
\end{equation}
Definitely, I'll have to find out the answer in the typical case
\eqref{eq:84.11}, whether I like it or not!

\bigbreak
\presectionfill\ondate{30.6.}\par

% 85
\hangsection[Digression on left exactness properties of $f_!$
functors, \dots]{Digression on left exactness properties of
  \texorpdfstring{$f_!$}{f!} functors, application to the inclusion
  \texorpdfstring{\protect\iSimplexIntoCat}{i:Delta->(Cat)}.}\label{sec:85}%
Just got an impressive heap of reprints and preprints by Tim Porter
(announced in his letter two days ago). Many titles have to do with
``shape theory'', ``coherence'' and ``homotopy limits'' (corresponding
to proobjects in various modelizers, as I understand it from his
letter). The one which most attracts my attention though is ``Cat as a
closed model category'' by
R.W.~Thomason\scrcomment{\textcite{Thomason1980}} -- the title comes
quite as a surprise, as my ponderings on the homotopy structure of
\Cat{} had left me with the definite feeling that there wasn't a
closed model structure on \Cat, with the usual notion of weak
equivalence. I'll have to have a closer look at this paper definitely,
before starting on part \ref{ch:V} of these notes!

It was getting prohibitively late yesterday, so I had to stop. It then
occurred to me that the (admittedly provisional) notion of
\emph{morphism} of asphericity structures I had proposed yesterday
(p.\ \ref{p:275}) is definitely stupid, because there is no reason
whatever that it should be stable under composition! This surely was
the reason for the feeling of uneasiness\pspage{281} caused by the
condition \ref{cond:84.Awk}, which surely deserves its name! This flaw
of course disappears, if we strengthen the condition, as suggested on
p.\ \ref{p:277}, by condition \ref{eq:84.4} -- that $f_!$ should take
aspheric objects into aspheric ones. This makes all the more
imperative the task of finding out whether this condition is
reasonable, and otherwise, what to put in as a substitute. The test
case now seems to be really the one when $f^*$ is the usual nerve
functor
\[f^*=i^*:\Cat\to\Simplexhat,\]
associated to the standard inclusion
\[i:\Simplex\to\Cat.\]
I never before had a closer look upon the corresponding functor
\[i_!:\Simplexhat\to\Cat,\]
left adjoint to the nerve functor, except just rectifying a big
blunder (p.\ \ref{p:22}), and convincing myself that $i_!$ did
\emph{not} take weak equivalences into weak equivalences. But why not
aspheric objects into aspheric ones?

Pondering a bit over this matter today, and trying to see whether
$i_!$ takes \emph{contractible} objects into contractible ones, this
brought up the question whether $i_!$ commutes with finite products --
which will allow then to make use of the generalities on morphisms of
contractibility structures, and will imply that $i_!$ is indeed such a
morphism (as it transforms $\Simplex$, which generates the
contractibility structure of $\Simplexhat$, into contractible elements
of \Cat). As a matter of fact, a number of times in my scratchwork
during the last four months I've met with the question of when a
functor of the type $f_!$ commutes to various types of finite inverse
limits, and this now is the occasion for writing down some useful
general facts in this respect, which had remained somewhat in the air
so far. It seems that the relevant facts can be summed up in two
steps.
\begin{propositionnum}\label{prop:85.1}
  Let $M$ be a category stable under \textup(small\textup) direct
  limits, $A$ a small category, $f:A\to M$ a functor, and
  \[f_!:\Ahat\to M, \quad f^*:M\to\Ahat\]
  the corresponding pair of adjoint functors.
  \begin{enumerate}[label=\alph*),font=\normalfont]
  \item\label{it:85.prop1.a}
    Assume $M$ stable under binary products, and that for any $x$ in
    $M$, the product functor $y\mapsto y\times x$ commutes to
    arbitrary direct limits \textup(with small categories of indices
    of course\textup). Then, in order for $f_!$ to commute with binary
    products,\pspage{282} is it n.s.\ that for any two objects $a,b$
    of $A$\kern1pt, the map
    \begin{equation}
      \label{eq:85.star}
      f_!(a\times b) \to f(a) \times f(b)\tag{*}
    \end{equation}
    in $M$ be an isomorphism.
  \item\label{it:85.prop1.b}
    Assume $M$ stable under fibered products, and that base change in
    $M$ commutes to arbitrary direct limits \textup(with small
    indexing categories of course\textup). Then, in order for $f_!$ to
    commute with fibered products, it is n.s.\ that it does so for any
    diagram
    \[\begin{tikzcd}[baseline=(O.base),column sep=tiny,row sep=small]
      b\ar[dr] & & c\ar[dl] \\ & |[alias=O]| F &
    \end{tikzcd},\]
    with $b,c$ objects in $A$\kern1pt, $F$ in \Ahat.
  \end{enumerate}
\end{propositionnum}
\begin{corollary}
  Assume $M$ stable under finite inverse limits, and that base change
  in $M$ commutes with arbitrary direct limits. The $f_!$ is left
  exact if{f} it satisfies the conditions \textup{\ref{it:85.prop1.a}}
  and \textup{\ref{it:85.prop1.b}} above, and moreover transforms
  $e_\Ahat$ into a final object of $M$ \textup(which also means, if
  $A$ has a final object $e_A$, that $f(e_A)$ is a final object of
  $M$\textup).
\end{corollary}
\noindent\emph{Proof of proposition.}
\textup{\ref{it:85.prop1.a}}\enspace As we have, by definition of
$f_!$, for any $F$ and $G$ in \Ahat
\[f_!(F) = \varinjlim_{A_{/F}} f(a) ,\quad
f_!(G) = \varinjlim_{A_{/G}} f(b) ,\]
deduced from the corresponding relations in \Ahat
\[ F = \varinjlim_{A_{/F}} a , \quad
G = \varinjlim_{A_{/G}} b ,\]
the looked for bijectivity of
\[ f_!(F\times G) \to f_!(F)\times f_!(G)\]
will follow from the assumption \eqref{eq:85.star}, and from the
following lemma (applied in both categories \Ahat{} and $M$), the
proof of which is immediate:
\begin{lemma}
  Let\pspage{283} $M$ be a category stable under direct limits and
  under binary products, and such that the product functors $y\mapsto
  y\times x$ commute to direct limits. Then binary products are
  distributive with respect to direct limits, i.e., if
  \[ u : I\to M, \quad v : J\to M\]
  are two functor of small categories with values in $M$, we have
  \[\varinjlim_{I\times J} u(i)\times v(j) \tosim \bigl( \varinjlim_I
  u(i) \bigr) \times \bigl( \varinjlim_J v(j) \bigr).\]
\end{lemma}

\emph{Proof of} \ref{it:85.prop1.b}. Follows from \ref{it:85.prop1.a},
applying it to the induced functor
\[ f_{/F} : A_{/F} \to M_{/f_!(F)},\]
in order to prove commutation of $f_!$ with a fibered product
corresponding to a diagram
\begin{equation}
  \label{eq:85.starstar}
  \begin{tikzcd}[column sep=tiny,row sep=small]
    G\ar[dr] & & H\ar[dl] \\ &  F &
  \end{tikzcd}\tag{**}
\end{equation}
in \Ahat.

\begin{remarknum}\label{rem:85.1}
  Part \ref{it:85.prop1.b} does not look as nice as part
  \ref{it:85.prop1.a}, as we would like to be able to take $F$ equally
  in $A$ -- what I first expected to get out. If we make only this
  weaker assumption, it will follow at once that $f_!$ commutes to
  fibered products corresponding to diagrams \eqref{eq:85.starstar}
  \emph{with $F$ in $A$}. It doesn't seem finally that from this
  follows commutation of $f_!$ to all fibered products, unless making
  some extra assumptions, such as $f_!$ transforming monomorphisms
  into monomorphisms (which is a necessary condition anyhow). We'll
  have to come back upon this -- for the time being we need only to
  deal with products, for which \ref{it:85.prop1.a} is adequate.
\end{remarknum}

The ``second step'' now is a tautology:
\begin{propositionnum}\label{prop:85.2}
  Under the preliminary assumptions of prop.\ \ref{prop:85.1} above,
  assume moreover that $f$ is fully faithful, and that $f(A)$
  ``generates $M$ by strict epimorphisms'', i.e., that for any object
  $P$ in $M$, we have
  \begin{equation}
    \label{eq:85.starstarstar}
    P \xleftarrow{\sim} \varinjlim_{A_{/P}} f(x).\tag{***}
  \end{equation}
  Then $f_!$ commutes to finite products in \Ahat{} of objects of $A$\kern1pt,
  and to fibered products in \Ahat{} of objects of $A$\kern1pt.
\end{propositionnum}
\noindent\emph{Proof.}\pspage{284} That $f_!$ transforms final object
into final object follows from the formula \eqref{eq:85.starstarstar}
above, by taking $P$ to be the final object of $M$. To get commutation
to binary products, we apply the formula for $P=a\times_M b$, and
notice that
\[ A_{/P} \simeq A_{/P'}, \quad\text{with $P'=a\times_\Ahat b$,}\]
because the inclusion $f:A\hookrightarrow M$ is full (I forgot to say
we may assume $f$ to be the inclusion functor of a full
subcategory). But we have too
\[ f_!(P') = \varinjlim_{A_{/P'}} f(x),\]
hence $f_!(P') \simeq P$. The proof for fibered products is similar,
or can be reduced to the case of products by considering the induced
functor
\[ f_{/a} : A_{/a} \to M_{/a}.\]
\begin{remarknum}\label{rem:85.2}
  The various variants of the notion of a generating subcategory $A$
  of a category $M$ have been dealth with in some detail in
  SGA~4 vol.~1\scrcomment{\textcite{SGA4vol1}}
  exp.~I par.~7 (p.\ 45--60) (Springer LN 269), cf.\ more
  specifically prop.\ 7.2 on page 47, summarizing the main
  relationships. The strongest of all notions considered there is
  generation by strict epimorphisms, which can be expressed by the
  formula \eqref{eq:85.starstarstar} above, and is also equivalent to
  the \emph{functor $f^*$ being fully faithful}.
\end{remarknum}

We come back now to the case $M=\Cat$ -- it is immediate that the
preliminary exactness conditions on $M$ of prop.\ \ref{prop:85.1} are
satisfied. Thus, putting together prop.\ \ref{prop:85.1} and
\ref{prop:85.2}, we get:
\begin{propositionnum}\label{prop:85.3}
  Let $A$ be a full subcategory of \Cat, generating \Cat{} by strict
  epimorphisms, $i:A\hookrightarrow\Cat$ the inclusion functor. Then
  the corresponding functor
  \[i_!:\Ahat\to\Cat\]
  commutes to finite products, and also to fibered products in \Ahat{}
  over an object of \Ahat{} coming from $A$\kern1pt.
\end{propositionnum}

The most familiar case when this applies is indeed the case of the
canonical inclusion of $\Simplex$ into \Cat, as it is well-known that
the corresponding $i^*$, namely the nerve functor, is fully
faithful. From this follows that \emph{a fortiori},\marginpar{hasty!}
for any full subcategory $A$ of \Cat{} containing $\Simplex$, $A$
generates \Cat{} by strict epimorphisms, and hence proposition
\ref{prop:85.3} applies.
\begin{remarknum}\label{rem:85.3}
  I doubt however, even for $A=\Simplex$, the most typical
  case,\pspage{285} that $i_!$ is even left exact, because it looks
  unlikely to me, in view of the explicit description of
  $i_!(K_\bullet)$ for a given ss~complex $K_\bullet$, in terms of
  ``generators and relations'' (cf.\ Gabriel-Zisman's
  book),\scrcomment{\textcite{GabrielZisman1967}} that $i_!$ should
  transform monomorphisms into monomorphisms. For what follows, this
  is irrelevant anyhow.\scrcomment{there is an unreadable footnote
    here}
\end{remarknum}

% 86
\hangsection[Bimorphisms of contractibility structure as the (final?)
\dots]{Bimorphisms of contractibility structure as the
  \texorpdfstring{\textup(\kernifitalic{2pt}}{(}final?\texorpdfstring{\textup)}{)}
  answer. Does the notion of a map of asphericity structures exist?}
\label{sec:86}%
After these somewhat painful preliminaries, it seems to me that firm
ground is in sight at last! The main feeling that finally comes out of
these reflections, is that we have a very good hold on checking
whether or not a functor $f_!:\Ahat\to M$ commutes to finite products,
and moreover, that this condition is satisfied in many ``good'' cases,
in more cases at any rate than I suspected. In case of the inclusion
functor of $\Simplex$ into \Cat, more generally of any subcategory $A$
of \Cat{} consisting of contractible objects, containing $\Simplex_1$,
and large enough in order to generate \Cat{} by strict epimorphisms
(or, as we'll say, to \emph{generate strictly} the category \Cat), it
follows that we have the inclusion
\begin{equation}
  \label{eq:86.1}
  i_!(\Ahatc) \subset \Cat\subc,\tag{1}
\end{equation}
i.e., $i_!$ transforms \emph{contractible} objects into contractible
objects, or, equivalently, is a \emph{morphism for the homotopy
  structures on \Ahat{} and \Cat}. (Note that the two first
assumptions on the subcategory $A$ of \Cat{} imply that the homotopy
structure on \Ahat{} generated by the ``intervals'' made up with
elements of $A$, is a contractibility structure, admitting $A$ as a
generating set of contractible objects. The situation is really nice
only when assuming that \Ahat{} is totally aspheric though, which will
imply that $A$ is a test category (even a strict one) and that the
\scrW-asphericity structure on \Ahat{} associated to the
contractibility structure just described is the usual one (cf.\ prop.\
\ref{prop:79.2} \ref{it:79.2.b} p.\ \ref{p:248}, where we take $C$ to
be $A$).)

The inclusion \eqref{eq:86.1} of course strongly resembles the
problematic inclusion \eqref{eq:84.4} (p.\ \ref{p:277}) of yesterday's
notes, namely, here to $i_!$ taking aspheric objects into aspheric
ones. I doubt though this latter is true even for the standard case
$A=\Simplex$? But before trying at all costs to see whether it holds
or not, the point I wish to make is that for the purpose we have in
view, namely defining suitable conditions, stable under composition,
on a pair $(f_!,f^*)$ of adjoint functors between two asphericity
structures $M$ and $M'$, -- for this purpose, a relation of the type
\eqref{eq:86.1}, namely\pspage{286}
\begin{equation}
  \label{eq:86.2}
  f_!(M'\subc) \subset M\subc,\tag{2}
\end{equation}
is just as good as the relation \eqref{eq:84.4} on p.\ \ref{p:277},
provided we make the evident assumption needed for \eqref{eq:86.2} to
make sense, namely \emph{the given asphericity structures on $M,M'$ to
  be associated to contractibility structures}. This condition
\eqref{eq:86.2} is (considerably!) weaker than
\begin{equation}
  \label{eq:86.2prime}
  f_!(M'\suba)\subset M\suba,\tag{2'}
\end{equation}
due to the inclusion $M'\subc\subset M'\suba$ and to the fact that
(with the notations and using the results of section \ref{sec:81}) the
inclusion \eqref{eq:86.2} is implied by the apparently weaker
inclusion (under the assumption \ref{loc:4}):
\[f_!(M'\subc)\subset M_0,\]
where $M\subc\subset M\suba\subset M_0$. On the other hand,
\eqref{eq:86.2} \emph{is evidently stable under composition}.

To be specific about ``as good as'', let's come back to the statement
of the ``pretty'' proposition on p.\ \ref{p:275}, somewhat marred by
the ``awkward'' condition \ref{cond:84.Awk} we had to throw in, which
looks so ugly because of its lack of stability under
compositions. Let's replace this condition by the slightly stronger
one \eqref{eq:86.2}, which is stable under composition. Condition
\eqref{eq:86.2} does imply \ref{cond:84.Awk} indeed, as we see by
taking for $A$ a small full subcategory of $M\subc$, generating the
asphericity structure of $M$, and for $B$ any full subcategory of $M$
contained in $M\suba$ and big enough in order a)\enspace to contain
$f_!(A)$ and b)\enspace to generate the asphericity structure of
$M$. We may take for instance for $A$ any subcategory of $M'\subc$
closed under finite products in $M'$ and generating the
contractibility structure, and accordingly for $B$, so that the pair
$(A,B)$ will match for any choice of the basic localizer \scrW. Thus,
the six conditions \ref{it:84.i} to \ref{it:84.iidblprime} of the
proposition are equivalent, provided in the last
\ref{it:84.iidblprime} we assume moreover that $B$ contain a final
object $e_M$ of $M$. Moreover, it is clear now that the conditions can
be viewed also as a property of the functor
\begin{equation}
  \label{eq:86.3}
  {f_!}\subc : M'\subc \to M\subc\tag{3}
\end{equation}
induced by $f_!$, which we may adequately express by saying that this
functor \eqref{eq:86.3} is \emph{\scrW-aspheric} (where the two sides
of \eqref{eq:86.3} are categories which need not be small, but which
are at any rate \emph{\scrW-aspherators} in the sense of section
\ref{sec:79}, p.\ \ref{p:247}).

Stated\pspage{287} this way, the condition just obtained on the pair
of adjoint functors $(f_!,f^*)$, between the two categories $M,M'$
endowed with contractibility structures, namely \eqref{eq:86.2} and
\begin{equation}
  \label{eq:86.4}
  f^*(M\suba)\subset M'\suba,\tag{4}
\end{equation}
are not quite symmetric, as the condition on $f_!$ is expressed in
terms of contractible objects, whereas the condition on $f^*$ is
expressed in terms of aspheric ones. However, assuming that \scrW{}
satisfies \ref{loc:4} and using theorem \ref{thm:79.1} p.\
\ref{p:252}, we see that \eqref{eq:86.4} is equivalent to the
condition
\begin{equation}
  \label{eq:86.5}
  f^*(M\subc)\subset M'\subc,\tag{5}
\end{equation}
which does not depend any longer on the choice of \scrW!

Finally, we are led to a notion of pure homotopy theory, in terms of
contractibility structures alone, without the intrusion of the choice
of a basic localizer \scrW{} and corresponding asphericity notions. We
may call the pair a ``\emph{bimorphism}'' of contractibility
structures, and introduce it via the following summing-up statement:
\begin{scholie}
  Let\scrcomment{a
    ``\href{https://en.wikipedia.org/wiki/Scholia}{scholium}'' is a
    critical or explanatory comment extracted from preexisting
    propositions} $(M,M\subc)$ and $(M',M'\subc)$ be two
  contractibility structures, each admitting a small generating
  subcategory for the contractibility structure, say $C$ and $C'$
  respectively. Let $(f_!,f^*)$ be a pair of adjoint functors
  \[f_!:M'\to M, \quad f^*:M\to M'.\]
  Let's consider the following inclusion conditions
  \begin{description}
  \item[\namedlabel{it:86.bang}{($!$)}]
    $f_!(M'\subc)\subset M\subc$,
  \item[\namedlabel{it:86.bangprime}{($!$')}] 
    $f_!(C')\subset M_0$,
  \item[\namedlabel{it:86.star}{($*$)}]
    $f^*(M\subc)\subset M'\subc$,
  \item[\namedlabel{it:86.starprime}{($*$')}] 
    $f^*(C)\subset M'_0$,
  \end{description}
  where $M_0$ and $M'_0$ are \textup(as in section
  \ref{sec:81}\textup) the sets of ``$0$-connected'' objects in $M$
  and $M'$ respectively, for the given contractibility structures.
  \begin{enumerate}[label=\alph*),font=\normalfont]
  \item\label{it:86.a}
    Clearly, in view of
    \[C \subset M\subc \subset M_0, \quad C'\subset M'\subc\subset
    M'_0,\]
    condition \textup{\ref{it:86.bang}} implies
    \textup{\ref{it:86.bangprime}} and condition
    \textup{\ref{it:86.star}} implies
    \textup{\ref{it:86.starprime}}. Moreover, it is even true that conditions \textup{\ref{it:86.star}} and
    \textup{\ref{it:86.starprime}} are equivalent, and the same holds
    for \textup{\ref{it:86.bang}} and \textup{\ref{it:86.bangprime}}
    provided $f_!$ commutes with finite products. We'll say that
    $(f_!,f^*)$ is a \emph{bimorphism} for the given contractibility
    structures, if the inclusions \textup{\ref{it:86.bang}} and
    \textup{\ref{it:86.star}} do hold.
  \item\label{it:86.b} Assume\pspage{288} we got a bimorphism
    $(f_!,f^*)$, i.e., \textup{\ref{it:86.bang}} and
    \textup{\ref{it:86.star}} above hold. Let \scrW{} be a basic
    localizer, hence the sets $M_\scrW$ and $M'_\scrW$ of
    \scrW-aspheric objects in $M$ and $M'$ respectively, and the sets
    $\scrW_M$ and $\scrW_{M'}$ of \scrW-equivalences in $M$ and
    $M'$. Then the following relations hold
    \begin{description}
    \item[\namedlabel{it:86.W}{(\scrW)}] $M_\scrW =
      (f^*)^{-1}(M'_\scrW)$,
    \item[\namedlabel{it:86.Wprime}{(\scrW')}] $\scrW_M =
      (f^*)^{-1}(\scrW_{M'})$.      
    \end{description}
  \item\label{it:86.c}
    Under the assumption of \textup{\ref{it:86.b}}, hence
    \textup{\ref{it:86.Wprime}} holds and $f^*$ gives rise to a
    functor $\overline{f^*}$ between the localizations
    \[\HotOf_{M,\scrW} = \scrW_M^{-1}M \quad\text{and}\quad
    \HotOf_{M',\scrW}=\scrW_{M'}^{-1}M',\]
    the following diagram is commutative up to canonical isomorphism:
    \[\begin{tikzcd}[baseline=(O.base),column sep=tiny]
      \HotOf_{M,\scrW}\ar[rr,"\overline{f^*}"]
      \ar[dr] & & \HotOf_{M',\scrW}\ar[dl] \\
      & |[alias=O]| \HotOf_\scrW &
    \end{tikzcd},\]
    where the vertical functors are the canonical functors of section
    \ref{sec:77}. In particular, if the latter are equivalences
    \textup(i.e., $M$ and $M'$ are \scrW-modelizing\textup), then so
    is $\overline{f^*}$.
  \item\label{it:86.d}
    Assume merely that the inclusion \textup{\ref{it:86.bang}} holds,
    and let \scrW{} as in \textup{\ref{it:86.b}}\textup{\ref{it:86.c}}
    be a basic localizer, satisfying moreover
    \textup{\ref{loc:4}}. Then the following conditions on the pair
    $(f_!,f^*)$ are equivalent:
    \begin{enumerate}[label=(\roman*),font=\normalfont]
    \item\label{it:86.d.i}
      The pair is a bimorphism, i.e., \textup{\ref{it:86.star}}
      \textup(or equivalently, \textup{\ref{it:86.starprime})} holds.
    \item\label{it:86.d.ii}
      The inclusion $\subset$ in \textup{\ref{it:86.W}} above holds.
    \item\label{it:86.d.iii}
      The inclusion $\subset$ in \textup{\ref{it:86.Wprime}} above holds.
    \item\label{it:86.d.iv}
      The functor induced by $f_!$\kern1pt,
      \[{f_!}\subc : M'\subc \to M\subc\]
      is a \scrW-aspheric functor between the aspherators $M'\subc,M\subc$.
    \item\label{it:86.d.v}
      \textup(\kern1pt If $A$ is a given small category, and
      \[i':A\to M'\]
      a given functor, \emph{factoring through $M'\subc$}, and
      $M'_\scrW$-\scrW-aspheric.\textup) The composition
      \[i = f_!\,i' : A\to M\]
      is $M_\scrW$-\scrW-aspheric.
    \end{enumerate}
  \end{enumerate}
\end{scholie}
\begin{comments}
  In terms of what is known to us since sections \ref{sec:81} and
  \ref{sec:82}, and notably that any two sections of a $0$-connected
  object of $M$ or of $M'$ are homotopic, the Scholie is just a long
  tautology. I have taken great case however in stating it, so as to
  get as clear a view as possible of exactly what the relevant
  relationships are. In \ref{it:86.a} the conditions \ref{it:86.star}
  and \ref{it:86.starprime} are just different ways of stating that
  $f^*$ is a morphism of contractibility structures, which can still
  be expressed in various other ways, compare p.\
  \ref{p:251}--\ref{p:252}. Similarly, if we assume that $f_!$
  commutes with finite products, then \ref{it:86.bang} or
  \ref{it:86.bangprime} can be viewed as expressing that $f_!$ is a
  morphism of homotopy structures (in opposite direction), which again
  could be expressed in various other ways, for instance in terms of
  the homotopy relation for maps, or in terms of homotopisms. This
  implies that in any case $f^*$ induces a functor between the strict
  localizations
  \[\overline{f^*}^{\mathrm c} : W\subc^{-1}M=\overline M \to
  {W'\subc}^{-1}M' = \overline{M'},\]
  and similarly for $f_!$, when $f_!$ commutes to finite products. I
  doubt however that the latter localized map is of geometric
  relevance, except maybe in the cases when $f_!$ gives rise to
  relations similar to \ref{it:86.W} and \ref{it:86.Wprime} in
  \ref{it:86.b} above, and both $f_!$ and $f^*$ are model-preserving
  with respect to \scrW-equivalences, and define quasi-inverse
  equivalences between the localizations $\HotOf_{M,\scrW}$ and
  $\HotOf_{M',\scrW}$ -- a rather exceptional case indeed.
\end{comments}

After stating the Scholie, there is scarcely a doubt left in my mind
about the notion of bimorphism, which finally peeled out of
reflections, being a relevant one. This Scholie, rather than the
``seducing'' proposition of section \ref{sec:84} (p.\ \ref{p:275}),
now seems to me the adequate ``answer'' of the not-so-silly-after-all
question of section \ref{sec:46} (cf.\ page \ref{p:95} -- nearly 200
pages ago!). The only minor uncertainty remaining in my mind is
whether or not in the notion of a bimorphism of contractibility
structures, we should insist that $f_!$ should commute to finite
products. It seems that it will be hard to check condition
\ref{it:86.bang}, except via an apparently weaker form such as
\ref{it:86.bangprime}, and using commutation of $f_!$ to finite
products. But on the other hand, the Scholie makes good sense without
assuming such commutation property. If the notion of a bimorphism is
going to be useful, time only will tell which terminology use is the
best.

More puzzling however is this facit,\scrcomment{``facit'' comes from
  the Latin verb ``\emph{facere}'', to do, so it means the result. Or
  it could be a typo for ``fact'', which essentially means the
  same\ldots} that we still have not been able to give a satisfactory
definition of a \emph{morphism} of \emph{asphericity} structures,
independently of any restrictive assumption on these, such\pspage{290}
as being generated by a contractibility structure. Even when making
such an assumption, it is not wholly clear yet that there isn't a good
notion of a morphism $f^*$ of contractibility structures, without
having to assume there exists a left adjoint $f_!$. Maybe a little
more pondering on the situation would be useful. If it should become
clear that (except for the obvious notion of \emph{equivalence} of
asphericity structures) there was \emph{not} any reasonable notion of
a morphism of asphericity structures, this would probably mean that
the notion of an asphericity structure should not be viewed as a main
structure type in homotopy theory in its own right, but rather,
mainly, as an important by-product of a contractibility structure. At
any rate, since section \ref{sec:81}, I feel that the main emphasis
has definitely been shifting towards contractibility structures, which
seems now \emph{the} main type of structure dominating in the
modelizing story, as this ``story'' is gradually emerging into light.

%%% Local Variables:
%%% mode: latex
%%% TeX-master: "main.tex"
%%% End:
